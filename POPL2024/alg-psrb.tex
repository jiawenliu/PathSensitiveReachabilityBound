% Based on the \emph{Abstract Transition Graph}, $\absG(c)$, the \emph{Refined Program}, $\rprog$ and the upper bound invariant for the \emph{Ranking Function} computed previously for a program $c$, our algorithm first requires to compute the new quantities, \emph{Path Reachability-bound} and the \emph{Loop Reachability-bound}. This section introduces the algorithm and the following sections describe how we compute them.

Now for a program $c$, we use $\absG(c)$, its refined program $\rprog$, and all the ranking functions to compute a \emph{reachability-bound} for each of its locations.
We first introduce some quantities which are required to compute ahead.

As pre-procedures, we first introduce the \emph{loop bound} for every loop $\rprog_l$ in $\rprog$, and the local \emph{path reachability-bound} for each $\tpath$ w.r.t. its $\kw{enclosed}(\rprog, \tpath)$.
\begin{defn}[Loop Bound]
 % \label{def:loopbound}
 For any program $c$ with it refined program $\rprog$,
 the \emph{loop bound}
 % $BD(\rprog_l, c) \in \scexpr(c)$ 
 for a subprogram, $\rprog_l$ in $\rprog$ is an upper bound on the times that $\rprog_l$ is executed from its entry to the exit point with arbitrary initial trace.
\end{defn}
% 
\begin{defn}[Local Path Reachability-bound]
 % \label{def:pathlocalrb}
 Given program $c$ with its refined program $\rprog$ and a simple transition path $\tpath$ in this program, 
 let $\rprog_l = \kw{enclosed}(\rprog, \tpath)$ be the innermost loop program in $\rprog$ where $\tpath$ is located,
 then $\tpath$'s \emph{path local reachability-bound} w.r.t. $\rprog_l$, 
 % $\outinB(\rprog_l, \tpath, c) \in \scexpr(c)$
 is an upper bound on the times that $\tpath$ is executed when executing program $\rprog_l$ with any initial trace of $c$.
\end{defn}
Intuitively, it bounds the execution times of $\tpath$ when only its innermost loop program is executing.
% and $\rprog$ is the closest loop where $\tpath$ is nested.
For example in the first interleaving pattern $\rprog_1^1$ in Figure~\ref{fig:relatedNestedWhileOdd-overview}, 
$4:\rprepeat(\tpath_3)$ is the innermost loop program of $\tpath_3$.
So we first will compute $m - n$ as its local \emph{path reachability-bound}.

If there isn't a nested loop in a program, we will jump to the Definition~\ref{def:pathrb}.
Otherwise, we need to compute \emph{loop reachability-bound}
% $\lpchB(l: \rprog_l, \tpath, c) \in \scexpr(c)$ 
for each $\tpath$ w.r.t. every level of its outer nested loop.
% This quantity aims to precisely bound the iteration numbers of the outer loop $l$,
% such that,
% during these iterations, the innermost loop $\kw{enclosed(\tpath)}$ is executed, i.e., reached.
\begin{defn}[Loop Reachability-bound]
% \label{def:looprb}
For the program with nested loops, given its refined program $\rprog$, a simple transition path $\tpath$, and an outer loop
program $\rprog_l$ where $\tpath$ is nested in,
$f(\rprog_l, \tpath, c)$ is a \emph{loop reachability-bound} for $\tpath$ w.r.t. $\rprog_l$ if and only if
for every $\trace \in \ftdom_0(c)$, $f(\rprog_l, \tpath, c)(\trace)$ greater or equal to the
iteration numbers of $\rprog_l$,
such that,
during these iterations, the innermost loop $\kw{enclosed(\tpath)}$ is entered.
\end{defn}
In Figure~\ref{fig:relatedNestedWhileOdd-overview}, $\tpath_3$ has an outer loop $\rprog_1^1$. Since $L_4$ will be ``entered'' only in the first iteration of $\rprog_1^1$,
we aim to compute $1$ as the \emph{loop reachability-bound} for $\tpath_3$ w.r.t. $\rprog_1^1$.

Now we want to bound for each $\tpath$ the times it is executed globally.
\begin{defn}[Path Reachability-bound]
% \label{def:pathrb}
For a program $c$ with its refined program $\rprog$ and a simple transition path $\tpath$ in this program, 
the \emph{path reachability-bound},
% $\inoutB(\rprog, \tpath, c) \in \scexpr(c)$ 
is the upper bound on the
times that $\tpath$ is executed when executing $c$ with arbitrary initial trace.
\end{defn}
% Intuitively, $\inoutB(\rprog, \tpath)$ bounds the execution times of $\tpath$ globally during the execution of $c$.
%
In Figure~\ref{fig:relatedNestedWhileOdd-overview}, since there isn't a nested loop for $\tpath_1$ and $\tpath_2$, their local bound $\frac{m}{4}$ is indeed the global \emph{path reachability-bound}.
$\tpath_3$ is nested in two loops, but the inner loop $L_4$ is only entered in the first iteration of the outer loop, we aim to compute $\frac{m}{4}$ using the \emph{loop reachability-bound}.
% by multiplying its local $\inoutB(\rprepeat(\tpath_3), \tpath_3, c)$ with the loop reachability-bound.
% path is $\tpath_3 = \frac{m}{4} \times 1 = \frac{m}{4} $.

Section~\ref{sec:pathrb} computes a sound \emph{path reachability-bound}, $\inoutB(\rprog, \tpath, c)$ for each $\tpath$ in $\rprog$.
Then we use it to compute for each program point, $l \in \lvar(c)$ in a program $c$,
% in a program $c$,
its \emph{reachability-bound}, $\psRB(l, c)$ by summing up all the path reachability-bounds over all $\tpath$ that contains $l$.
%
\begin{defn}[The Reachability-bound Computation]
\label{def:point_psrb}
Given a program $c$ with its {abstract transition graph}, $\absG(c)$ and refined program $\rprog$,
the \emph{reachability-bound} of each program point $l \in \lvar(c)$, $\psRB(l, c)$ is computed as follows,
% sums up all the path reachability-bounds, $\inoutB(\rprog, \tpath, c)$ over all simple transition paths $\tpath$ that contains the program point $l$.
\[ 
 \psRB(l, c) = 
 \sum
 \left\{ \inoutB(\rprog, \tpath, c) ~\vert~ \tpath \in \rprog \land 
 l \in \pathl(\tpath) \right\}
 % \footnotemark
\]
$\tpath \in \rprog$ denotes $\tpath$ is a simple transition path in $\absG(c)$.
% \footnotetext{$l \in \tpath$ and $\tpath \in \rprog$ denote $l$ is a vertex on $\tpath$, and $\tpath$ is a path on $\rprog$ respectively.}
\end{defn}
% $\econfig{\psRB(l, c)}$ is a \emph{reachability-bound} for every program point $l$ in a program $c$.
\begin{thm}[Soundness]
\label{thm:pathsensitive_rb_soundness}
For every program ${c}$ and every label $l$ in this program,
$\econfig{\psRB(l, c)}$ is a \emph{reachability-bound} for $l$ in $c$.
%
{\small
\[
 \begin{array}{l}
 \forall c, c_r \in \cdom, \tpath \in \absG(c), \trace_0 \in \ftdom_0(c), \trace_r \in \ftdom_0(c_r), \trace \in \tdom, l, l' \in \ldom, \rprog \st 
 \\ \quad
 \rprog = REFINE(\algrewrite(c))
 \land 
 \rprog = \algrewrite(c_r)
 \land
 \\ \quad
 \land
 \Big(
 \config{c_r, \trace_0} \rightarrow^* \config{\clabel{\eskip}^{l'}, \trace_0 \tracecat \trace}
 \lor \config{c_r, \trace_0} \uparrow^{\infty} \trace_0 \tracecat \trace 
 \Big)
 \implies \econfig{\psRB(l, c)}(\trace_0) \geq \counter(\trace, l)
 \end{array}.
\]
}
\end{thm}