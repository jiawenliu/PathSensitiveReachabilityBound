
% There are two methods for computing the path bound, $BD(\rprog_l, c)$ for every loop program $\rprog_l$ in $c$'s refined program.
% %
% Based only on the estimated ranking function from Section~\ref{sec:rank}, we can compute a loose $BD(\rprog', c)$ for each $\rprog'$
% using the $\absclr(\absevent, c)$ as follows.
% \begin{equation}
%  \label{eq:absBD}
%  BD(\rprog_l, c) = \min\left\{ \absclr(\absevent, c) \middle\vert \absevent \in \rprog_l \right\}.
% \end{equation}
% It is loose for multipath or nested loops because of the path-insensitivity in the ranking function invariant estimation in Definition~\ref{def:ranking_bound}. 
% For example in Figure~\ref{fig:relatedNestedWhileOdd-overview},
% it computes $BD(\rprog_1^1) = n$ and $BD(\rprog_1^2) = n $ for the two interleaving pattern of loop $L_1$.
% % The bounds for $\rprog_1^1$ and $\rprog_1^2$ are both $n$ while 
% But both of their iteration bounds are expected to be $\lfloor\frac{m}{4}\rfloor$.
As subroutines of the \emph{path reachability-bound} computation, this section gives the \emph{path bound} computation for every subprogram of a program and the local \emph{path reachability-bound} computation for every simple transition path.
\subsection{Path Bound}
\begin{defn}[Path Bound]
 % \label{def:loopbound}
 For any program $c$ with it refined program $\rprog$,
 the \emph{path bound}
 % $BD(\rprog_l, c) \in \scexpr(c)$ 
 for a subprogram, $\rprog_l$ in $\rprog$ is an upper bound on the times that $\rprog_l$ is executed from its entry to the exit point with arbitrary initial trace.
\end{defn}

To compute the path bound for every subprogram, our approach first compute three abstract states for each ranking function in different subprograms of a refined program as below.
% Then we these precise abstract states, and estimate the path bound path sensitively.
\begin{defn}[Rank Abstract States Computation]
 \label{def:alg-absstate}
 Given a program $c$ with its refined program $\rprog_c$ and a sub-program $\rprog$ in $\rprog_c$, the three abstract states, $\rfinit(\rprog, c)$, $\rffinal(\rprog, c)$ and $\rfnext(\rprog, c) \in \scexpr(c)$ w.r.t. the ranking functions in this sub-program as follows.
 % \begin{itemize}
 % \item 
 
 1. We compute the initial execution point and continuation execution points of $\rprog$ denoted by
 $\absinit(\rprog)$
 and 
 $\absfinal(\rprog)$ using standard control flow method.
 
2. The \emph{Initial State}, 
 $\rfinit(\rprog, c)$ is a set of equations $x = e$, where $e \in \scexpr(c)$ is a
 symbolic expression bounding the initial value of $x$ before executing $\rprog$.
 Each $x$ is a ranking function of a transition edge in this program, and $\rfinit(\rprog, c)$ contains the equations for every ranking function in $\rprog$,
 $\rfinit(\rprog, c) \triangleq $
 {\small
 \[
 \bigcup_{x \in \left\{ \rank(\absevent, c) | \tpath \in \rprog \land \absevent \in \tpath \right\} }
 \left \{ 
 x = \max_{l_1}\left\{
 \varinvar(y, c) + v ~\middle\vert~ 
 \begin{array}{l} 
 (l_1, x' \leq y + v, l_2) \in \reset(x, c) 
 \\
 \land l_1 \leq \absinit(\rprog, c)
 \end{array}
 \right\}
 \right\}.
 \]
 }
%  $\rfinit(\tpath, c)$ can also be estimated by $\kw{INIT(c, 0, \absinit(\rprog))}$ in \cite{GulwaniJK09}. 
 %

 3. The \emph{Final State}, $\rffinal(\rprog, c)$ computes the set of minimum values for every ranking functions
 after executing $\rprog$.
 % $\rffinal(\rprog, c) \triangleq $
 % \\
 {\small
\[
 \begin{array}{l} 
 \bigcup\limits_{x \in \left\{ \rank(\absevent, c) | \tpath \in \rprog \land \absevent \in \tpath \right\} }
 \left \{ 
 x = \min\limits_{ \left\{ l_1 ~|~ l_1 \geq \absfinal(\rprog, c) \right\} }
 \max\left\{
 v ~\middle\vert~ 
 \left( (x = v) \land \bigwedge\limits_{b \in \kw{B}(\rprog, c)} \neg b \right) \neq \efalse
 \right\}
 \right\}
 \end{array}
 \]
 }
 $\kw{B}(\rprog, c)$ is the set of all the boolean constraints in $\rprog$.

 4. The \emph{Next State}, $\rfnext(\rprog, c) \in \scexpr(c)$ 
 computes for the ranking functions on $\rprog$ the amount that each of them is changed after the first execution of $\rprog$ and before the second visit. This quantity is only required for $\tpath$.
 {\small
 \[
 \begin{array}{l}
 \rfnext(\tpath, c) \triangleq 
 \\
 \bigcup\limits_{x \in \left\{ \rank(\absevent, c) | \absevent \in \tpath \right\}}
 \left\{ x = 
 \begin{array}{l}
 \sum\limits_{\absevent \in \inc(x, c) }
 \left\{ v ~\middle\vert~ \absevent = (l, x' \leq x + v, \_) \land l \in \tpath\right\}
 \\ \qquad 
 + \max\limits_{l' }
 \left\{ \varinvar(y, c) + v ~\middle\vert~ (l, x' \leq y + v, l') \in \reset(x, c) \land l \in \tpath\right\}
 \\ \qquad 
 - \sum\limits_{ \absevent \in \dec(x, c) }\left\{ 
 v ~\middle\vert~ \absevent = (l, x' \leq x - v, \_) \land l \in \tpath 
 \right\}
 \end{array}
 \right\} 
 \end{array}
 \]
 }
% \end{itemize}
\end{defn}
Now we compute $BD(\rprog, c)$ through a mutual recursion procedure involving the computation of another quantity, the \emph{Variable Gradient Decent} as follows.
\begin{defn}[Path Bound Computation]
\label{def:loopbound}
Given a program $c$ with its refined program $\rprog$ and a subprogram $\rprog' = \rprepeat(\rprog_l) \in \rprog$, we compute a \emph{path bound}, $BD(\rprepeat(\rprog_l), c)$
through a mutual recursion between  $BD(\rprepeat(\rprog_l), c)$ and $\varGD(\rprog', c)$.
$\varGD(\rprog', c)$ is the \emph{Variable Gradient Decent}--an invariant of the minimum amount decreased in one execution of $\rprog_l$ for the ranking functions in $\rprog$.
%  $ BD(\rprepeat(\rprog_l), c) \triangleq$
% and recursively computes the path bound, $BD(\rprog, c)$ in the meantime.

% $BD(\rprepeat(\rprog_l), c)\triangleq$
 \[
  BD(\rprepeat(\rprog_l), c)\triangleq
 \left\{ 
 \begin{array}{ll}
 \max\limits_{x \in \left\{ \rank(\absevent, c) | \absevent \in \tpath \right\}} 
 \Big\{ {\frac{a - b}{\varGD(\tpath, c)}} & \rprog_l = \tpath
 \\ \qquad 
 ~\vert~
 x = a \in \rfinit(\tpath, c)
 \land x = b \in \rffinal(\tpath, c)
 \Big\} 
 \\
 \max\limits_{x \in \left\{ \rank(\absevent, c) | \absevent \in \tpath \right\}} 
 \Big\{ {\frac{\min\{ a, a - \varGD(\rprog_1, c) \} - b}{\varGD(\rprog_1;\rprog_2, c)}} 
 & \rprog_l = \rprog_1;\rprog_2
 \\ \qquad 
 ~\vert~
 x = a \in \rfinit(\rprog_1, c)
 % \\ \qquad 
 \land x = b \in \rffinal(\rprog_2;\rprog_2, c)
 \Big\} \\
 0 & o.w.,
 \end{array} 
 \right.
\]
where the \emph{variable gradient decent},
$\varGD(\rprog, c)$ for the ranking functions of $\rprog$ is computed as follows,
\[
  \varGD(\rprog, c) \triangleq
 \left\{
 \begin{array}{ll}
 BD(\rprog, c) \times {\varGD(\rprog_l', c)} & \rprog = \rprepeat(\rprog_l') \\
 \varGD(\rprog_1, c) + \varGD(\rprog_2, c) & \rprog = \rprog_1;\rprog_2 \\
 \min\left\{v ~|~ x = v \in \rfnext(\tpath, c) \right\} & \rprog = \tpath.
 \end{array}
 \right.
 \]
\end{defn}
 Definition~\ref{def:loopbound} computes a sound path bound for every loop in a refined program, which is formally presented as below.
\begin{lem}[Soundness of Path Bound]
  \label{lem:loopbound_sound}
  For every program $c$ with it refined program $\rprog$,
  % for every loop $\rprepeat(\rprog')$ inside $\rprog$, 
  $BD(\rprog, c)$ is a sound upper bound on the iterating times of this program.
\end{lem}

\paragraph{Example.}
Using our new approach, we compute a more precise path bound $BD(\rprog)$ for each subprogram of our walk through example. 
For example for the subprogram $\rprepeat(\tpath_3)$, we first compute $\rfinit = \{k = n - m\}$, $\rffinal = \{k = 0\}$ and $\rfnext = 1$ and then get $BD(\rprepeat(\tpath_3)) = n - m$.
% This result  tighter than $n$ by Eq.~\ref{eq:absBD}.
For the first iteration pattern $\rprog_1^1$ in Figure~\ref{fig:relatedNestedWhileOdd-overview}(b), we compute 
$\rfinit = \{k = n - m, i = m - 1\}$, $\rffinal = \{k = 0, i = 0\}$, $\varGD(\rprog_1^1) = 4$, and get $BD(\rprog_1^1) = \lfloor \frac{m}{4} \rfloor $ as expected.



For the first iteration pattern $\rprog_1^1$ in Figure~\ref{fig:relatedNestedWhileOdd-overview}(b), we also compute 
$\rfinit = \{k = n - m, i = m\}$, $\rffinal = \{k = 0, i = 0\}$ and $\varGD(\rprog_1^1) = 4$. So we compute $BD(\rprog_1^1) = \lfloor \frac{m}{4} \rfloor $ as expected.

Using the $BD$ computed by our new approach, we compute the \emph{path reachability-bound} locally for
each $\tpath$ w.r.t. the \textbf{innermost loop}, $\kw{enclosed}(\rprog, \tpath)$ where $\tpath$ is located.

\subsection{Local Path Reachability-bound}

\begin{defn}[Local Path Reachability-bound]
  % \label{def:pathlocalrb}
  Given program $c$ with its refined program $\rprog$ and a simple transition path $\tpath$ in this program, 
  let $\rprog_l = \kw{enclosed}(\rprog, \tpath)$ be the innermost loop program in $\rprog$ where $\tpath$ is located,
  then $\tpath$'s \emph{path local reachability-bound} w.r.t. $\rprog_l$, 
  % $\outinB(\rprog_l, \tpath, c) \in \scexpr(c)$
  is an upper bound on the times that $\tpath$ is executed when executing program $\rprog_l$ with any initial trace of $c$.
 \end{defn}
 Intuitively, it bounds the execution times of $\tpath$ when only its innermost loop program is executing.
 % and $\rprog$ is the closest loop where $\tpath$ is nested.
 For example in the first interleaving pattern $\rprog_1^1$ in Figure~\ref{fig:relatedNestedWhileOdd-overview}, 
 $4:\rprepeat(\tpath_3)$ is the innermost loop program of $\tpath_3$.
 So we first will compute $m - n$ as its local \emph{path reachability-bound}.

To compute a sound and accurate local \emph{path reachability-bound},
we first compute the \textbf{innermost loop} for a simple transition path.
\begin{defn}[Innermost Loop]
  \label{def:innermost}
  The innermost loop of a simple transition path is the loop program that is closest to this path where this path is located.
\end{defn}

Then by using the path bound, we compute the \emph{local path reachability-bound} of
the $\tpath$ in its closest nested loop as follows in Definition~\ref{def:pathlocalrb}.
\begin{defn}[Path Local Reachability-bound Computation]
    \label{def:pathlocalrb}
    Given program $c$ with its refined program $\rprog$ and a simple transition path $\tpath$ in this program, 
    let $l: \rprog_l = \kw{enclosed}(\rprog, \tpath)$ be a sub loop program in $\rprog$,
    then $\tpath$'s \emph{local reachability-bound} w.r.t. $l: \rprog_l$,
    $\outinB(\rprog_l, \tpath, c)$
    is computed inductively as follows.

    $\outinB(\rprog_l, \tpath, c) \triangleq$
    \[ 
      \left\{
    \begin{array}{ll}
       1  & \rprog = \tpath\\
      {0} & \rprog = \tpath' \neq \tpath\\
      \outinB(\rprog_1, \tpath, c) + \outinB(\rprog_2, \tpath, c) & \rprog = \rprog_1;\rprog_2 \\
      % \outinB(\rpchoose{\rprog_1, \ldots, \rprog_m }, \tpath) & \triangleq 
      % & 
      \max\left\{ \outinB(\rprog_1, \tpath, c), \ldots, \outinB(\rprog_m, \tpath, c) \right\} 
      & \rprog = \rpchoose{\rprog_1, \ldots, \rprog_m } \\
      % \outinB(\rprepeat(\rprog'), \tpath) & \triangleq 
      % & 
      BD(\rprepeat(\rprog'), c) \times \outinB(\rprog', \tpath, c) & \rprog = \rprepeat(\rprog').
    \end{array}
    \right.
    \]
\end{defn}
The lemma below states that the \emph{path local reachability-bound} of the simple transition path $\tpath$ in a refined program $\rprog$ is a sound upper bound of its execution times when executing its enclosed loop $\kw{enclosed}(\rprog, \tpath)$.
\begin{lem}[Soundness of the Path Local Reachability-bound]
  \label{lem:pathlocalrb-sound}
  For any program $c$ with its refined program $\rprog$ and a simple transition path $\tpath$ in $\rprog$,
  if $l: \rprog_l = \kw{enclosed}(\rprog, \tpath)$ is the closest loop where $\tpath$ is nested in this program,
  then the execution times of $\tpath$ when executing the $\rprog_l$ under initial trace $\trace_l \in \ftdom_0(c_l)$ is bounded by $\econfig{\outinB(\rprog_l, \tpath)}(\trace_0)$ with any initial trace $\trace_0 \in \ftdom_0(c)$.
  \[
    \begin{array}{l}
    \forall c, c_l \in \cdom, \tpath \in \absG(c), 
    \trace_l \in \ftdom_0(c_l), \trace_0 \in \ftdom_0(c), \trace \in \tdom, l, l' \in \ldom, \rprog \st 
    \\ \qquad
    \rprog = REFINE(c)
    \land
    l: \rprog_l = \kw{enclosed}(\rprog, \tpath)
    \land 
    \rprog_l = \algrewrite(c_l)
    \\ \qquad
    \land
    \Big(
    \config{c_l, \trace_l} \rightarrow^* \config{\clabel{\eskip}^{l'}, \trace_l \tracecat \trace}
    \lor \config{c_l, \trace_l} \uparrow^{\infty} \trace_l \tracecat \trace 
    \Big)
    \\ \qquad
    \implies
    \econfig{\outinB(\rprog_l, \tpath)}(\trace_0) \geq \lcounter(\trace, \pathl(\tpath)).
    \end{array}
  \]  
\end{lem}


\paragraph{Example.}
% \todo{Walk through example}
For the running example in Figure~\ref{fig:relatedNestedWhileOdd-overview}, 
we compute local path reachability-bound for every simple transition path with respect to its innermost loop as
$\outinB(\rprog, \tpath_0) = 1$,
$\outinB(\rprog, \tpath_4) = 1$,
$\outinB(1: \rprog_1^1, \tpath_1) = \frac{m}{4}$,
$\outinB(1: \rprog_1^1, \tpath_2) = \frac{m}{4}$,
$\outinB(1: \rprog_1^1, \tpath_4) = \frac{m}{4}$,
$\outinB(1: \rprog_1^2, \tpath_1) = \frac{m}{4}$,
$\outinB(1: \rprog_1^2, \tpath_2) = \frac{m}{4}$,
$\outinB(1: \rprog_1^2, \tpath_4) = \frac{m}{4}$, and
$\outinB(4: \rprepeat(\tpath_3), \tpath_3) = n - m$ 

Specifically, for the program in the first refined execution $\rprog_1^1$ in Example~\ref{ex:relatedNestedWhileOdd-overview}, $\rprog_1^1$ is the innermost loop for $\tpath_1$, $\tpath_2$ and $\tpath_4$, 
so compute $\outinB(\rprog_1^1, \tpath_1, c) = BD(\rprog_1^1, c) \times \outinB(\tpath_1; 4:\rprepeat(\tpath_3); \tpath_2; \tpath_4, \tpath_1, c)
= BD(\rprog_1^1, c) \times (1 + 0) = \lfloor \frac{m}{4} \rfloor $ and the same for $\tpath_2$ and $\tpath_4$.


For the program in the first interleaving pattern $\rprog_1^1$ in Figure~\ref{fig:relatedNestedWhileOdd-overview}, $\rprog_1^1$ is the innermost loop for $\tpath_1$, $\tpath_2$ and $\tpath_4$.
We compute as follows
 $$
 \begin{array}{l}
  \outinB(\rprog_1^1, \tpath_1, c) 
  = BD(\rprog_1^1, c) \times \outinB(\tpath_1; 4:\rprepeat(\tpath_3); \tpath_2; \tpath_4, \tpath_1, c)\\
  = BD(\rprog_1^1, c) \times (1 + 0) = \lfloor \frac{m}{4} \rfloor 
 \end{array}
 $$
  and the same for $\tpath_2$ and $\tpath_4$.

If there does not exist a nested loop, we can skip the Section~\ref{sec:looprb} and use this local bound directly as the global \emph{path reachability-bound}, $\inoutB(\rprog, \tpath, c)$ as in Section~\ref{sec:pathrb}. 
 So we already had the $\psRB$ for every program point $l \in \lvar(c)$ by replacing the equation in Definition~\ref{def:point_psrb} with the following one,
 \begin{equation}
 \label{eq:psrb-sub}
 \psRB(l, c) = 
 \sum
 \left\{ \outinB(\rprog, \tpath, c) ~\vert~ \tpath \in \rprog \land 
 l \in \tpath \right\}.
 \end{equation}
 However, if there are nested loops, 
 by replacing the restriction, $\rprog_l$ in Definition~\ref{def:pathlocalrb} with the global program $\rprog$, $\outinB(\rprog, \tpath, c)$ is actually a loose global \emph{path reachability-bound} for $\tpath$. Though we can still compute a loose $\psRB$ just using $\outinB(\rprog, \tpath, c)$, it is not tight enough. 
It computes
$\outinB(\rprog, \tpath_3, c) = \frac{m}{4} \times (n - m)$ in Figure~\ref{fig:relatedNestedWhileOdd-overview}.
It is loose because Eq.~\ref{eq:psrb-sub} assumes the inner loop $L_4$ is reached in every iteration of the outer loop $L_1$. In this sense, $\psRB(4, c) = \frac{m}{4} \times (n - m)$ is loose as well.
While the inner loop $L_4$ is only reached in the first iteration of the outer loop,
so we will compute a more accurate quantity, the \emph{loop reachability-bound} in the next section.