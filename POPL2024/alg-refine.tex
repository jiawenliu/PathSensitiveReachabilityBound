% This procedure has three sub-procedures in each paragraph below.
\paragraph{The Simple Transition Path}
We first collect the loop headers $\loopl(c) \subseteq \lvar(c)$ from a program $c$, which is the set of all program points corresponding to the loop headers in program $c$.
\begin{defn}[Loop Headers ($\loopl : \cdom \to \mathcal{P}(\ldom)$)]
  \label{def:loopl}
  \[
  \loopl(c) \triangleq 
  \left\{
    \begin{array}{ll}
      \{\}  & {c} = \clabel{\assign x e}^{l} \\
      \loopl({c_1}) \cup \loopl({{c_2}})  & {c} = {c_1};{c_2} \\
      \loopl(c_t) \cup \loopl({{c_f}})   & {c} =\eif(\clabel{\bexpr}^{l}, c_t, c_f) \\
  \loopl(c_w) \cup \{l\}, &  {c}   = \ewhile \clabel{\bexpr}^{l} \edo (c_w)
  \end{array}
\right.
\]
  \end{defn}
% \begin{defn}[Loop Path]
%   \label{def:looppath}
% A simple transition path
% $\tpath \in \paths(\absG(c))$ for the program $c$, is a path on its abstract transition graph $\absG(c) = (\absV(c), \absE(c))$ with 
% \begin{itemize}
% \item a vertices sequence $(l_0, \ldots, l_n)$, where $l_i \in \absV(c)$ for every $i = 0, \ldots, n$ and
% %
% \item an edge sequence $(e_1, \ldots, e_n)$, where $e_i = (l_{i - 1}, dc_i, l_{i}) \in \absE(c)$ for every $i = 1, \ldots, n$,
% \end{itemize}
% %
% satisfying:
% \begin{itemize}
%   \item $l_i \neq l_j$ for every $i = 0, \ldots, n$ and $j = 0, \ldots, {n - 1}$,
%   \item $l_0$ is either the program point of a loop header or the program entrance ($l_0 = 0$),
%   i.e., $l_0 \in \loopl(c) \cup \{ 0 \}$
%   \item and $l_n$ is either the program point of a loop header or the program exit ($l_n = \lex$),
%   i.e., $l_0 \in \loopl(c) \cup \{ \lex \}$.
% \end{itemize}
% \end{defn}

\begin{defn}[Simple Tansition Path]
  \label{def:tpath}
A \emph{simple transition path}
$\tpath \in \paths(\absG(c))$ for the program $c$, is either a simple cyclic path, which has the same start- and end-point
or a simple path has either different while loop headers, the program entrance or exit as its start- and end-point
without visiting any loop header inside the path.
\\
Specifically, a path $l_0 \xrightarrow{dc_0} l_1 \xrightarrow{dc_1} \ldots l_n \in \paths(\absG(c))$ with the
vertices sequence $(l_0, \ldots, l_n)$, where $l_i \in \absV(c)$ for every $i = 0, \ldots, n$ and
%
the edge sequence $(e_1, \ldots, e_n)$, where $e_i = (l_{i - 1}, dc_i, l_{i}) \in \absE(c)$ for every $i = 1, \ldots, n$,
%
is a \emph{simple transition path} if and only if it satisfies,
\begin{itemize}
  \item $l_i \neq l_j$ for every $i = 0, \ldots, n$ and $j = 0, \ldots, {n - 1}$,
  \item $l_0$ is either the program point of a loop header or the program entrance ($l_0 = 0$),
  i.e., $l_0 \in \loopl(c) \cup \{ 0 \}$
  \item and $l_n$ is either the program point of a loop header or the program exit ($l_n = \lex$),
  i.e., $l_0 \in \loopl(c) \cup \{ \lex \}$,
  \item and $l_i \notin \loopl(c) \cup \{ 0, \lex \}$ for every $i = 1, \ldots, n-1$.
\end{itemize}
\end{defn}

\paragraph{Program Rewriting}
\paragraph{Syntax.}

\[
    \rprog := \tpath ~|~ \rprepeat(\rprog) ~|~ l : \rprepeat(\rprog) ~|~ \rprog;\rprog ~|~ \rpchoose{\rprog, \ldots} 
\]
where $l\in \loopl(c)$ is a loop header.

In the new syntax, $\rprepeat(\rprog)$ is a loop statement iterating over the transitions in $\rprog$.
$l: \rprepeat(\rprog)$ represents that the loop $ \rprepeat(\rprog)$
corresponds to a loop of the source program with loop header $l$.
The multiple-paths statement $\rpchoose{\rprog, \ldots} $ contains all the execution paths of a loop from the source program.


\paragraph{Algorithm.}
Algorithm~\ref{alg:alg-refine_rewrite} transforms the syntax of the while language program 
into the new syntax defined above.
% following~\cite{GulwaniJK09} and preserves the semantics.

In this algorithm, we first use a simple depth-first search strategy to collect all the \emph{simple transition path}s satisfying the Definition~\ref{def:tpath}. 
% It guarantees that every $\tpath$ is equivalent to a path $\rho$ in Definition~4.1 of \cite{GulwaniJK09}.
In line:2, we initialize each candidate $c_i$ with a \emph{simple transition path} $\tpath_i$. 
New candidates generated in line:4, 5, and 6 correspond to the loop statement $\rprepeat(c_i)$, multiple-paths statement $\rpchoose{\ldots}$ and sequence statements $c_i; c_j$ respectively.

\begin{algorithm}
 \caption{Program Rewriting Algorithm. $\kw{Rewrite}(c)$}
 \label{alg:alg-refine_rewrite}
 \begin{algorithmic}[1]
 \REQUIRE program $c$
%  collects all $c$'s \emph{simple transition path}s from $\absG(c)$, $\tpath_1, \ldots, \tpath_n \in \paths(\absG(c))$.
 \STATE \textbf{init}: 
 Set of all \emph{simple transition path}s, 
 $\mathcal{P} = \{ \tpath_1, \ldots, \tpath_n \}$.
 \\
 The candidate set $W = \{c_1, \ldots, c_n\}$, where $c_i = \tpath_i$ for $i = 1, \ldots, n$
 \STATE \textbf{while} $W.size()> 1$:
 \STATE
 \quad create $c' = \rprepeat(c)$ s.t. $c_i \in W \land c[0] = c[-1] \land c[0] \in \loopl(c)$
 \\ \quad $W.add(c[0]: c')$, \qquad $W.remove(c)$
 \STATE \quad create $c' = \rpchoose{c_1, \ldots, c_m}$ 
 s.t. $c_i, c_j \in W \land c_i[0] = c_j[0] = c_i[-1] = c_j[-1]$, $i, j = 1, \ldots, m$.
 \\ \quad $W.add(c')$ \qquad $W.remove(c_1, \ldots, c_m)$
 \STATE \quad create $c' = c_1; c_2$ s.t. $c_1, c_2 \in W \land c_1[-1] = c_2[0]$
 \\
 \quad $W.add(c')$ \qquad $W.remove(c_1, c_2)$
 \STATE \textbf{Endwhile}
 \\ $c^T = W[0]$
 \RETURN $c^T$.
\end{algorithmic}
\end{algorithm}
%
%  in paper~\cite{GulwaniJK09} respectively.
\begin{itemize}
\item
Line:4 find the candidate $c'$ that has the same start- and end-point.
Each candidate corresponds to a loop path, and we create for this candidate a loop statement
$\rprepeat(c')$.

\item
 Line:5 
 finds all the candidates $c_1, \ldots, c_m$ that start and end at the same point.
 These candidates are multiple-paths of a same loop, 
 we create for these candidates a multiple-path loop statement,  $\rpchoose{c_1, \ldots, c_m}$.
\item
Line:6 finds all pairs of candidates $c_1, c_2$ such that  $c_2$ starts with the point where $c_1$ ends.
%  label, rewrite them 
For each pair, we create a sequence statement
 $c_1; c_2$.
\end{itemize}

% \todo{soundness of the program write}
\subsection{Program Refinement}

\paragraph{Contextualization}

$cxlE$ is a set of edges between two simple transition paths of program $c$. There is an edge from $\tpath_1$ to $\tpath_2$
if and only if $\tpath_2$ can execute right after execution of $\tpath_1$.
% We adopt the contextualization method from paper~\cite{ZulegerGSV11} Definition~10 and build the edge.

\paragraph{Path Interleaving Refinement}

  \begin{algorithm}
    \caption{
    {Interleaving Refinement}
    \label{alg:prog-refine}
    }
    \begin{algorithmic}[1]
    \REQUIRE a loop program $c$ with multiple paths;
    \\
    set of all the paths in this loop program:
    $P = \{\tpath_0, \ldots, \tpath_m\}$,
    \\
    the contextualization graph $cxlG(c)$ over paths.
    % the target while loop with label $l$.
    \STATE  \textbf{Init} 
    % \STATE \todo{Define algorithm $\kw{dfs(w, I_{l}, FS)}$}
    \STATE \textbf{Define} {$\kw{dfs(wking\_path, inv\_list, done\_set)}$:}
    \STATE \quad {$\kw{unfinished\_set = \{\}}$:}
    \STATE \quad {$\kw{inv0 = inv\_list[-1]}$:}
    \STATE \quad \textbf{For} $(\tpath, \tpath_i) \in cxlE(c)$ and $\tpath \neq \tpath'$:
    \\
    \STATE \quad \quad \textbf{If} {$\kw{visited[\tpath_i]}$} skip the loop
    \STATE \quad \quad $\kw{visited[\tpath_i]} = \etrue$
    \STATE \quad \quad $\kw{w_i = wking\_path.append(\tpath')}$
    \\
    \quad \quad {Compute Invariant $\kw{inv_i = merge(w_i, inv\_list[-1])}$}
    \STATE \quad \quad \textbf{If} {$\kw{inv_i = \bot}$} $\eskip$
    \STATE \quad \quad \textbf{Elif} {$\kw{inv_i = inv\_list[0]}$} $\kw{done\_set.add(\rprepeat(w_i))}$.
    \STATE \quad \quad \textbf{Elif} {$\kw{inv_i \notin inv\_list}$} 
    $\kw{unfinished\_set \cup (dfs(w', inv\_list++[inv_i], done\_set)}$
    \STATE \quad \quad \textbf{Else} 
    Find the index $j$, such that $\kw{inv_i = inv\_list[j] }$,
    \STATE \quad \quad \quad
    $\kw{(dfs(w[:j]++[\rprepeat(w[j:i])]', inv\_list[:j], done\_set)}$.
    \STATE \quad \quad $\kw{visited[\tpath_i]} = \efalse$.
    \STATE \quad \textbf{return} $\kw{unfinished\_set}$
    \STATE Take $\tpath_0$ from $P$ 
    \STATE initialize $\kw{visited} = [\efalse]*m$ and $\kw{done\_set = \{\}}$
    \STATE compute the invariant of the loop program $\kw{I_0}$
    \STATE $\kw{dfs([\tpath_0], I_0, [I_0], done\_set )}$
    \RETURN $\kw{done\_set}$
    \end{algorithmic}
    \end{algorithm}



The interleaving refinement algorithm computes the execution order of each simple transition paths explicitly.
This algorithm is inspired from the algorithm REFINE in paper~\cite{GulwaniJK09}. 

It first computes the execution orders of
the simple transition paths that come from the same loop.
For example, two simple transition paths $\tpath_1 = (1 \to 2 \to 3 \to 4)$ and 
$\tpath_4 = (1 \to 2 \to 8 \to 1)$ in the same loop Figure~\ref{fig:relatedNestedWhileOdd-overview}(b) have two possible execution orders:
either $\tpath_1$ executes after the execution of $\tpath_4$, or vice verse.
% and compute the 

Then we construct the refined program by explicating the execution orders of simple transition paths.
% refined program $\rprog$ for a rewritten program $c$.
Simple transition paths can iterate on themselves, so an execution order could
contain iterations of simple transition paths, and sequence of iterations.
For example, the simple transition path $\tpath_3 = (4 \to 5 \to 4)$
the loop $L_4$ in Figure~\ref{fig:relatedNestedWhileOdd-overview}(b) has only one possible execution order,
which is the iteration on itself, $\rprepeat(\tpath_3)$.

Each loop in a refined program contains several execution orders over simple transition paths.
For example, the loop $L_1$ in Figure~\ref{fig:relatedNestedWhileOdd-overview}(b)
has two execution orders. In the first one,
$\rprepeat(\tpath_4; \tpath_1; 4:\rprepeat(\tpath_3); \tpath_2)$ 
$\tpath_1$ will execute right after the execution of $\tpath_4$.
Then following the execution of $\tpath_1$, loop $L_4$ finishes full iterations and followed by the execution of $\tpath_2$.
In the second execution order,
$\rprepeat(\tpath_1; 4:\rprepeat(\tpath_3); \tpath_2; \tpath_4)$,
$\tpath_4$ is executed after the execution of $\tpath_1; 4:\rprepeat(\tpath_3); \tpath_2$.
Iteration of one execution order is equivalent to one possible full iterations of the source loop program.
In this sense, we construct $1: \rprepeat(\tpath_1; 4:\rprepeat(\tpath_3); \tpath_2; \tpath_4)$ as an execution path of the loop $L_1$ in the source program with loop header $1$.
The refined loop program for $L_1$ is 
\[
\rpchoose{ 
 \begin{array}{l}
 1: \rprepeat(\tpath_1; 4:\rprepeat(\tpath_3); \tpath_2; \tpath_4), \\
 1: \rprepeat(\tpath_4; \tpath_1; 4:\rprepeat(\tpath_3); \tpath_2) 
 \end{array}
 }.
\]

Then we construct the refined program for program $\kw{nestedOdd}(n, m)$ in Figure~\ref{fig:relatedNestedWhileOdd-overview}(a) as follows, which is also presented in Figure~\ref{fig:relatedNestedWhileOdd-overview}(c).
\[
 \tpath_0 ; \rpchoose{ 
 \begin{array}{l}
    1: \rprepeat(\tpath_1; 4:\rprepeat(\tpath_3); \tpath_2; \tpath_4), \\
    1: \rprepeat(\tpath_4; \tpath_1; 4:\rprepeat(\tpath_3); \tpath_2) 
 \end{array}
 }; \tpath_5.
\]


\paragraph{Termination Analysis.}
The visiting array guarantees that every simple transition path is visited at most once.
In this sense, the algorithm will not fall into infinite loop.

