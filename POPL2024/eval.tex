\paragraph{Set up} We implemented a prototype {\PSRB} that takes a program as input 
and outputs the reachability-bound on each program point.
The 
program abstraction is implemented in OCaml and the program refinement, ranking function estimation, and the path-sensitive reachability-bound algorithm are implemented in Python.
We evaluate the prototype on 290 programs: 20 hand-crafted programs and 270 programs extracted from different sets of benchmarks. 
These programs include  37 {\tt c} programs 
from the testing set of the \hyperlink{https://forsyte.at/static/people/sinn/loopusJAR/index.html}{LoopusJAR}~\cite{BenchmarkLoopus} tool,
37 {\tt java} programs from the testing set of the tool~\hyperlink{https://zenodo.org/record/5140586\#.Y5pBoC-B1QI}{Tianhan}\cite{BenchmarkTianhan,LuCT21} 
and 118 {\tt c} programs from the testing set of~\hyperlink{https://github.com/icra-team/icra}{Icra}~\cite{KincaidBCR19,CyphertBKR19}, and 110 representative {\tt c} programs from the testing sets of 7 other tools\footnote{ABC: 15, C4B~: 15, KoAT: 3, Loopus: 10, Rank10: 3, SPEED~\cite{GulwaniJK09,GulwaniZ10,GulwaniMC09}: 27, WTC: 37}.
%
We translate these programs into our syntax.
For comparison, we also report the results of four bound analysis tools, 
\hyperlink{https://forsyte.at/software/loopus/}{LoopusJAR}~\cite{SinnZV17},
\hyperlink{https://github.com/aeflores/CoFloCo/tree/master/src}{CoFloCo}~\cite{ToolCofloco},
% and \hyperlink{https://zenodo.org/record/5140586\#.Y5pBoC-B1QI}{Tianhan}
~\cite{BenchmarkTianhan}
and our own SPEED~\cite{GulwaniJK09} implementation.
% \\

When comparing with tools for worst-case
complexity analysis we aim to answer the following two questions: is our prototype able to compute precise reachability-bound for different program points (\emph{Effectiveness})? If so, can we also have better overall complexity estimation (\emph{Accuracy})?
%  based on the precise reachability-bound?

\paragraph{Effectiveness}
% \newcommand{\pointnum}{\kw{\# l}}
\begin{table}[ht]
    \vspace{-0.3cm}    
    \caption{The Effectiveness Evaluation Table of {\PSRB}}
    \label{tb:effectiveness-eval}
    \vspace{-0.3cm}    
    \centering
    {\scriptsize
    \begin{tabular}{ | >{\scriptsize}c | >{\scriptsize}c | >{\scriptsize}c | >{\scriptsize}c | >{\scriptsize}c | c | c | c | c | c | c | c | c | c | c | c | c | c |}
    \hline
    \hline
    {Bench-} & {theory} & \multirow{2}{*}{P. \# } & \multirow{2}{*}{L. \#} & \multicolumn{6}{c|}{reachability-bound (L. \#)} & \multicolumn{5}{c|}{worst bound (P. \#)} \\
    \cline{5-15}
    -marks & worst &  &  & $1$ & $n$ & $n^2$ & $n^3$ & $n^4$ & $\infty$ & {\tiny \PSRB} & {\tiny Loopus} & {\tiny Cofloco} & {\tiny SPEED} & {\tiny \cite{BenchmarkTianhan}}  \\
    \hline
    \hline
    & $O(1)$   &  4 & 82 & 32 & 0 & 0 & 0 & 0 & 0 & 1 & 2 & 3 & 2 & 1  \\
    \cline{2-2}
    & $O(n)$   & 59 & 373 & 112 & 261 & 0 & 0 & 0 & 0 & 48 & 51 & 45 & 46 & 40 \\
    \cline{2-2}Represen-
    & $O(n^2)$ & 31 & 414 & 70 & 191 & 153 & 0 & 0 & 0  & 37 & 29 & 34 & 37 & 49  \\
    \cline{2-2}
    tatives
    & $n\log(n)$ & 6 & 81 & 0 & 0 & 0 & 0 & 0 & 0 & 0 & 0 & 0 & 0 & 0  \\
    \cline{2-2}
    & $O(n^3)$   & 2 & 91 & 6 & 16 & 13 & 13 & 0 & 0 & 4 & 3 & 2 & 5 & 7 \\
    \cline{2-2}
    & $O(n^{4})$ & 3 & 36 & 4 & 9 & 9 & 12 & 11 & 0 & 4 & 4 & 3 & 5 & 5 \\
    \hline \hline
    \multirow{5}{*}{Loopus} 
    & $O(1)$     & 1 & 154 & 0 & 0 & 0 & 0 & 0 & 0 & 0 & 1 & 0 & 0 & 0  \\
    \cline{2-2}
    & $O(n)$     & 13 & 176 & 49 & 105 & 0 & 0 & 0 & 0 & 12 & 13 & 14 & 14 & 11 \\
    \cline{2-2}
    & $O(n^2)$   & 4 & 64 & 4 & 13 & 15 & 0 & 0 & 0 & 3 & 2 & 5 & 2 & 6 \\
    \cline{2-2}
    & $O(n^3)$   & 4 & 193 & 2 & 7 & 10 & 4 & 0 & 0 & 2 & 1 & 2 & 2 & 3 \\
    \cline{2-2}
    & $O(n^{4})$ & 1 & 21 & 1 & 3 & 6 & 8 & 2 & 0 & 1 & 1 & 1 & 1 & 0 \\
    \hline \hline
    \multirow{6}{*}{Icra} 
    & $O(1)$ & 4 & 93 & 41 & 0 & 0 & 0 & 0 & 0 & 1 & 3 & 2 & 2 & 0  \\
    \cline{2-2}
    & $\log(n)$ & 2 & 16 & 0 & 0 & 0 & 0 & 0 & 0 & 0 & 0 & 0 & 0 & 0   \\
    \cline{2-2}
    & $O(n)$ & 79 & 988 & 267 & 698 & 0 & 0 & 0 & 0 & 78 & 80 & 82 & 78 & 77  \\
    \cline{2-2}
    & $O(n^2)$ & 12 & 151 & 26 & 110 & 26 & 0 & 0 & 0 & 18 & 14 & 11 & 16 & 17  \\
    \cline{2-2}
    & $O(n^3)$ & 1 &  9 & 1 & 3 & 3 & 2 & 0 & 0 & 1 & 1 & 4 & 2 & 4  \\
    \cline{2-2}
    & $\infty$ & 20 & 295 &  50 & 0 & 0 & 0 & 0 & 245 & 9 & 0 & 0 & 0 & 0  \\
    \hline \hline
    \multirow{2}{*}{Tianhan} 
    & $O(n)$ & 35 & 188 & 68 & 120 & 0 & 0 & 0 & 0 & 35 & 35 & 35 & 35 & 35  \\
    \cline{2-2}
    & $O(n^2)$ & 2 & 12 & 4 & 8 & 4 & 0 & 0 & 0 & 2 & 2 & 2 & 2 & 2   \\
    \hline
    \hline
    \end{tabular}
    }
    \vspace{-0.3cm}    
    \end{table}
We summarize the results of the effectiveness evaluation in Table~\ref{tb:effectiveness-eval}.
On each row, a theoretical worst-case complexity is listed in the $2^{nd}$ column and the number of programs (P. \#) with this particular complexity is given in the $3^{rd}$ column.
Each row of the $4^{th}$ column is the total number of locations (L. \#) in the programs with certain complexity.
% The  and $3^{rd}$ columns are the theoretical worst-case complexity and the number of programs that have this complexity.
In the ``worst bound (P. \#)'' column, we present the evaluation result of different tools in sub-columns.
For each tool, the result in each row is the number of programs that this tool identified with the corresponding ``theory worst'' complexity. For example, the number $1$ in the column of our tool for the row $O(1)$ says that our tool infer the right complexity for only one of the programs, for the other three programs it infers a higher complexity.
To show \textit{effectiveness}, 
we present in the ``{reachability-bound (L. \#)}'' column our results w.r.t. the program locations. 
Because different program points usually have different reachability-bounds,
we divide this column into sub-columns corresponding to different reachability-bounds.
Results in the same row are from the programs of the same estimated worst-case complexity.
For these programs of the same estimated worst-case complexity in the same row, each sub-column contains the computed number of locations that has certain reachability-bound.
% their reachability-bounds that has this 
% Each sub-column is the number of locations (L. \#) with certain reachability-bound.
% Though having the same worst-case complexity, the results show that different points have significantly different reachability-bounds.
The results show that although these locations are from the programs of the same worst-case complexity, they have significantly different reachability-bounds. This is because of the existence of multipaths loops, where locations on different paths have different reachability-bound.
We identify a number of loop patterns with large worst-case complexity but only reachability-bounds of $O(1)$ on points inside the loop.
For example in the row of $O(n)$ worst complexity in Icra benchmark, we observe that
almost $\frac{1}{3}$ of the program points inside the loops are evaluated only once.
Given this, if we use the worst-case bound to estimate the reachability-bounds on different locations, these locations that are only evaluated once will be over-approximated significantly. 
% under the same worst complexity.
While our tool computes for almost all the multipath loops precise \emph{reachability-bound}s on locations in different paths.
%  that have significantly different \emph{reachability-bound}s.
Even for the unbounded loops, we can still compute finite \emph{reachability-bound}s on certain program points that are not involved with the unbounded loop paths.
None of other tools compute bounds for these programs.

\paragraph{Accuracy}
Based on the worst-case bound and reachability-bound, we present the accuracy evaluation result in
Table~\ref{tb:accuracy-eval}.
In the last column we show the ``overall complexity bound'' estimation.
In the $1^{st}$ sub-column, formula in each row is the sum of the reachability-bounds multiply the corresponding (L. \#) that has certain reachability-bound.
For comparison, we show the worst-case bound times $\sum(L. \#)$ in the $2^{nd}$ sub-column by other tools. Given the precise reachability-bounds on every program locations, we can compute tighter overall complexity bound than the best worst-case analysis tools.
\begin{table}[ht]
    \vspace{-0.3cm}    
    \caption{The Accuracy Evaluation Table of {\PSRB}}
    \label{tb:accuracy-eval}
\vspace{-0.3cm}    
\centering
    {\scriptsize
    \begin{tabular}{| >{\scriptsize}c | >{\scriptsize}c | >{\scriptsize}c | >{\scriptsize}c | >{\scriptsize}c | c | c | c | c | c | c |}
        \hline
        \hline
    {Bench-} & {theory} & \multirow{2}{*}{P. \#} & \multirow{2}{*}{L. \#} & \multicolumn{2}{c|}{overall complexity bound}\\
    \cline{5-6}
    -marks & worst &  &   & {\PSRB} & worst \\
    \hline
    \hline
    & $O(1)$   &  4 & 82  & 32 & 64 \\
    \cline{2-2}
    & $O(n)$   & 59 & 373 & $112 + 261n$  & $373n$\\
    \cline{2-2}Represen-
    & $O(n^2)$ & 31 & 414  & $70 + 191n + 153n^2$ & $ 414n^2 $ \\
    \cline{2-2}
    tatives
    & $n\log(n)$ & 6 & 81  & 0 & 0 \\
    \cline{2-2}
    & $O(n^3)$   & 2 & 91  & $6 + 16n + 13n^2 + 13n^3$ & $48 n^3$\\
    \cline{2-2}
    & $O(n^{4})$ & 3 & 36 & $4 + 9n + 9n^2 + 12n^3 + 11n^4$ & $45n^4$\\
    \hline \hline
    \multirow{5}{*}{Loopus} 
    & $O(1)$     & 1 & 154 & 0 & - \\
    \cline{2-2}
    & $O(n)$     & 13 & 176  & $49 + 105n$ & $162n$\\
    \cline{2-2}
    & $O(n^2)$   & 4 & 64 & $4 + 13n + 15n^2$ & $32n^2$ \\
    \cline{2-2}
    & $O(n^3)$   & 4 & 193  & $2 + 7n + 10n^2 + 4n^3 $ & $23n^3$\\
    \cline{2-2}
    & $O(n^{4})$ & 1 & 21  & $1+3n+6n^2+8n^3+2n^4$ & $20n^4$\\
    \hline \hline
    \multirow{6}{*}{Icra} 
    & $O(1)$ & 4 & 93 & 6 & 64 \\
    \cline{2-2}
    & $\log(n)$ & 2 & 16 & 0 & 0 \\
    \cline{2-2}
    & $O(n)$ & 79 & 988  & $267 + 698n$ & $ 965n $\\
    \cline{2-2}
    & $O(n^2)$ & 12 & 151  & $26+110n+26n^2$ & $162n^2$\\
    \cline{2-2}
    & $O(n^3)$ & 1 &  9 & $1+3n+3n^2+2n^3$ & $9n^3$\\
    \cline{2-2}
    & $\infty$ & 20 & 295  & $50$ & $ \infty$\\
    \hline \hline
    \multirow{2}{*}{Tianhan} 
    & $O(n)$ & 35 & 188 & $68+120n$ & $188n$\\
    \cline{2-2}
    & $O(n^2)$ & 2 & 12  & $4+8n+4n^2$ & $16n^2$ \\
    \hline
    \hline
    \end{tabular}
    }
    \vspace{-0.3cm}
    \end{table}


%
%  \begin{table}[ht]
%  \caption{The Effectiveness And Accuracy Evaluation of {\PSRB}}
%  \label{tb:accuracy-eval}
%  \centering
%  {\scriptsize
%  \begin{tabular}{ >{\scriptsize}c | >{\scriptsize}c | >{\scriptsize}c | >{\scriptsize}c | >{\scriptsize}c | c | c | c | c | c | c | c | c | c | c | c | c |}
%         &  &  &  & \multicolumn{5}{c|}{reachability-bound} & \multicolumn{5}{c|}{worst bound } &  \multicolumn{2}{c|}{} \\
% {Bench-} & {theory} & ($\# p$) & ($\# l$) & \multicolumn{5}{c|}{ (L. \#)} & \multicolumn{5}{c|}{($\# p$)} & \multicolumn{2}{c|}{overall complexity bound}\\
%  \cline{5-16}
%  -marks & worst &  &  & $1$ & $n$ & $n^2$ & $n^3$ & $\infty$ & {\tiny \PSRB} & {\tiny Loopus} & {\tiny Cofloco} & {\tiny SPEED} & {\tiny \cite{BenchmarkTianhan}} & {\PSRB} & worst \\
%  \hline
%  & $O(1)$   &  4 & 82 & 32 & 0 & 0 & 0 &  0 & 1 & 2 & 3 & 2 & 1 & 32 & 64 \\
%  \cline{2-2}
%  & $O(n)$   & 59 & 373 & 112 & 261 & 0 & 0 &  0 & 48 & 51 & 45 & 46 & 40 & $112 + 261n$  & $373n$\\
%  \cline{2-2}Other
%  & $O(n^2)$ & 31 & 414 & 70 & 191 & 153 & 0 &  0  & 37 & 29 & 34 & 37 & 49 & $70 + 191n + 153n^2$ & $ 414n^2 $ \\
%  \cline{2-2}
%  Tools
%  & $n\log(n)$ & 6 & 81 & 0 & 0 & 0 & 0 & 0 &  0 & 0 & 0 & 0 & 0 & 0 & 0 \\
%  \cline{2-2}
%  & $O(n^3)$   & 2 & 91 & 6 & 16 & 13 & 13 &  0 & 4 & 3 & 2 & 5 & 7 & $6 + 16n + 13n^2 + 13n^3$ & $48 n^3$\\
%  \cline{2-2}
%  & $O(n^{4})$ & 3 & 36 & 4 & 9 & 9 & 12 & 0 & 4 & 4 & 3 & 5 & 5 & $4 + 9n + 9n^2 + 12n^3 $ & $45n^4$\\
%  \hline \hline
%  \multirow{5}{*}{Loopus} 
%  & $O(1)$     & 1 & 154 & 0 & 0 & 0 & 0  & 0 & 0 & 1 & 0 & 0 & 0 & 0 & - \\
%  \cline{2-2}
%  & $O(n)$     & 13 & 176 & 49 & 105 & 0  & 0 & 0 & 12 & 13 & 14 & 14 & 11 & $49 + 105n$ & $162n$\\
%  \cline{2-2}
%  & $O(n^2)$   & 4 & 64 & 4 & 13 & 15 & 0  & 0 & 3 & 2 & 5 & 2 & 6 & $4 + 13n + 15n^2$ & $32n^2$ \\
%  \cline{2-2}
%  & $O(n^3)$   & 4 & 193 & 2 & 7 & 10 & 4  & 0 & 2 & 1 & 2 & 2 & 3 & $2 + 7n + 10n^2 + 4n^3 $ & $23n^3$\\
%  \cline{2-2}
%  & $O(n^{4})$ & 1 & 21 & 1 & 3 & 6 & 8 & 2  & 1 & 1 & 1 & 1 & 0 & $1+3n+6n^2+8n^3$ & $20n^4$\\
%  \hline \hline
%  \multirow{6}{*}{Icra} 
%  & $O(1)$ & 4 & 93 & 41 & 0 & 0 & 0 & 0  & 1 & 3 & 2 & 2 & 0 & 6 & 64 \\
%  \cline{2-2}
%  & $\log(n)$ & 2 & 16 & 0 & 0 & 0 & 0 & 0  & 0 & 0 & 0 & 0 & 0 & 0 & 0 \\
%  \cline{2-2}
%  & $O(n)$ & 79 & 988 & 267 & 698 & 0 & 0 & 0  & 78 & 80 & 82 & 78 & 77 & $267 + 698n$ & $ 965n $\\
%  \cline{2-2}
%  & $O(n^2)$ & 12 & 151 & 26 & 110 & 26 & 0 & 0  & 18 & 14 & 11 & 16 & 17 & $26+110n+26n^2$ & $162n^2$\\
%  \cline{2-2}
%  & $O(n^3)$ & 1 &  9 & 1 & 3 & 3 & 2 & 0  & 1 & 1 & 4 & 2 & 4 & $1+3n+3n^2+2n^3$ & $9n^3$\\
%  \cline{2-2}
%  & $\infty$ & 20 & 295 &  50 & 0 & 0 & 0  & 245 & 9 & 0 & 0 & 0 & 0 & $50$ & $ \infty$\\
%  \hline \hline
%  \multirow{2}{*}{Tianhan} 
%  & $O(n)$ & 35 & 188 & 68 & 120 & 0 & 0 &  0 & 35 & 35 & 35 & 35 & 35 & $68+120n$ & $188n$\\
%  \cline{2-2}
%  & $O(n^2)$ & 2 & 12 & 4 & 8 & 4 & 0 & 0 & 2 & 2 & 2 & 2 & 2 & $4+8n+4n^2$ & $16n^2$ \\
%  \hline
%  \end{tabular}
%  }
%  \end{table}
%
\paragraph{Performance}
The performance evaluation is in Table~\ref{tb:performance-eval}.
The columns from $2$ to $6$ are the numbers of the programs, program with multipath loops, program with nested loops, and program with bounded loops and the lines of code in each benchmark set respectively.
The $7$ to $9$ columns are our success rate, the number of programs we failed and time outs. The last column is the total runtime w.r.t. the programs that we successfully computed bounds.
We are not doing well in terms of efficiency. Because of the possible non-terminating in program refining and the ranking function estimating, our time outs are higher than other tools.
Since our prototype consists of the implementations in OCaml and Python, the inter-procedures calls also reduce the efficiency.
%
%

\begin{table}[H]
    \vspace{-0.3cm}    
    \caption{The Performance Evaluation Table of {\PSRB}}
    \label{tb:performance-eval}
    \vspace{-0.3cm}    
    \centering
    {\scriptsize
    \begin{tabular}{ | c | c | c | c | c | c | c | c | c | c | c |}
    \hline
    \hline
    \multirow{2}{*}{Benchmarks} & \multirow{2}{*}{\# P}  & {M.P.L} & Nested  & {Bounded} 
    & \multirow{2}{*}{L.O.C} & {Success} & \multirow{2}{*}{Failed} & Time  & Total
    \\
    &  & \# & Loop \# & \# & & Rate &  & Outs &   Runtime \\
    \hline
    \hline 
    {Representatives} & {110}  & 53  & 52  & 107 & 954 & 85.5\% & 4 & 12 & 7min42sec \\
    \hline
    Challenges & 23  & 20 & 20 & {20} & 454 & {78.2\%}  & 1 & 4 & {12min39sec} \\
    \hline
    {Icra} & 118 & 111 & 21 & 98 & 1434  & 90.7\% & 1 & 10 & {4min48sec} \\
    \hline
    Tianhan & 37 & 2 & 3 & 37 & 204 & 100.0\% & 0 & 0 & 1min03sec \\
    \hline
    \hline
    \end{tabular}    
    }
    \vspace{-0.3cm}    
    \end{table}
\paragraph{Failed Examples}
Because of the limitation of our program language model, we are only able to handle the integer program with
linear computation.
For the set of example programs that have loop of logarithm complexity in the ``representative'' benchmark set in Tab.~\ref{tb:effectiveness-eval}, we
only over-estimate them as $O(n^2)$ complexity.

Another kind of failed cases come from our refinement algorithm.
% \begin{itemize}
%     \item Logarithm time loops, the binary search
%     \item 
% \end{itemize}
