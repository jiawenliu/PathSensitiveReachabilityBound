Now we want to bound for each $\tpath$ the number of times it is executed globally.
This section introduces formally the \emph{path reachability-bound} and gives the computation algorithm.

\begin{defn}[Path Reachability-bound]
% \label{def:pathrb}
For a program $c$ with its refined program $\rprog$ and a simple transition path $\tpath$ in this program, 
the \emph{path reachability-bound},
% $\inoutB(\rprog, \tpath, c) \in \scexpr(c)$ 
is the upper bound on the
times that $\tpath$ is executed when executing $c$ with arbitrary initial trace.
\end{defn}
% Intuitively, $\inoutB(\rprog, \tpath)$ bounds the execution times of $\tpath$ globally during the execution of $c$.
%
In Figure~\ref{fig:relatedNestedWhileOdd-overview}, since there isn't a nested loop for $\tpath_1$ and $\tpath_2$, their local bound $\frac{m}{4}$ is indeed the global \emph{path reachability-bound}.
$\tpath_3$ is nested in two loops, but the inner loop $L_4$ is only entered in the first iteration of the outer loop, we aim to compute $\frac{m}{4}$ using the \emph{loop reachability-bound}.

To compute a sound \emph{path reachability-bound}, we simply use the production of the $\outinB$ and $\lpchB$ for every simple transition path $\tpath$ as below.
%
\begin{defn}[Path Reachability-bound Computation]
 \label{def:pathrb}
 For a program $c$ with its refined program $\rprog$ and a simple transition path $\tpath$ in $\rprog$, 
 we compute for $\tpath$ its \emph{path reachability-bound}, $\inoutB(\rprog, \tpath, c)$
 as follows. 
 % The input $c$ is omitted in the following equations for concise given the context is clear.
{\small 
\[
 \left\{ 
 \begin{array}{ll}
 1 & \rprog = \tpath\\
 \highlight{0} & \rprog = \tpath' \neq \tpath\\
 \inoutB(\rprog_1, \tpath, c) + \inoutB(\rprog_2, \tpath, c) & \rprog = \rprog_1;\rprog_2 \\
 % & 
 \max\left\{ \inoutB(\rprog_1, \tpath, c), \ldots, \inoutB(\rprog_m, \tpath, c) \right\} 
 & \rprog = \rpchoose{\rprog_1, \ldots, \rprog_m } \\
 \outinB(\rprog_l, \tpath, c), & l: \rprog_l = \kw{enclosed}(\tpath) \\
 \lpchB(\rprog_l, \tpath , c)
 \times \max\limits_{l:\rprog_l = \kw{enclosed}(\rprog')}
 \{\inoutB(\rprog', \tpath, c)\} , & l: \rprog_l \neq \kw{enclosed}(\tpath) \\
 % \outinB(\rprepeat(\rprog'), \rprog) \times \inoutB(\rprog', \tpath) & \rprog = \rprepeat(\rprog')\\
 \end{array}
 \right.
 \]
 }
 \end{defn}
% \footnotetext{In a refined program, there is only one non-zero $\inoutB(l':\rprog'', \tpath)$ where $\tpath \in \rprog''$.}

$\inoutB(\rprog, \tpath, c)$ is a sound \emph{path reachability-bound}.
%

For the example program in Figure~\ref{fig:relatedNestedWhileOdd-overview}(a),
with $\lpchB(\rprog_1^1, \tpath_3, c) = 1$ and
$\outinB(\rprepeat(\tpath_3), \tpath_3, c) = n - m$ computed before,
we obtain $\inoutB(\rprog, \tpath, c) = 1 \times (n - m)$ as tight as expected.

For the other example in Figure~\ref{fig:threeWhile-looprb}(a), with $\lpchB(\rprog_1, \tpath_3, c) = 1$,
$\lpchB(\rprog_3, \tpath_3, c) = 1$ and
$\outinB(\rprog_6, \tpath_3, c) = N$,
we also obtain a tight reachability-bound $1 \times 1 \times N$.


We show the \emph{path reachability-bound} of the simple transition path $\tpath$ in a refined program $\rprog$ is a sound upper bound of its execution times when executing the program in Lemma~\ref{lem:pathrb-sound} below.
% , with the proof in \highlight{Appendix~\ref{apdx:pathrb-sound}}.
\begin{lem}[Soundness of the Path Reachability-bound]
  \label{lem:pathrb-sound}
  For any program with its refined program $\rprog$ and a simple transition path $\tpath$ in this program,
  the execution times of $\tpath$ when executing the $\rprog$ under initial trace $\trace_0 \in \tdom$ is bounded by $\econfig{\inoutB(\rprog, \tpath)}(\trace_0)$.
  \[
    \begin{array}{l}
    \forall c, c_r \in \cdom, \tpath \in \absG(c), \trace_0 \in \ftdom_0(c), \trace \in \tdom, l \in \ldom, \rprog \st 
    \\ \qquad
    \kw{IRefine}(c) = \rprog
    \land 
    \algrewrite(c_r) = \rprog
    \land
    % \\ \qquad
    % \land 
    \Big(
      \config{c_r, \trace_0} \rightarrow^* \config{\clabel{\eskip}^{l}, \trace_0 \tracecat \trace}
      \lor \config{c_r, \trace_0} \uparrow^{\infty} \trace_0 \tracecat \trace 
      \Big)
  \\ \qquad
    \implies
    \econfig{\inoutB(\rprog, \tpath)}(\trace_0) \geq \lcounter(\trace, \pathl(\tpath)).
    \end{array}
  \]  
\end{lem}
This lemma is similar to Lemma~\ref{lem:pathlocalrb-sound}, except we require the program that is executing is the entire program rather than part of the program.