Similar to our new approach of computing loop bound,
we also compute two refined abstract states for each loop w.r.t. a simple transition path inside it,
$\lpinit(l: \rprog, \tpath, c)$, and $\lpnext(l: \rprog, \tpath, c)$ based on the refined program
as in Definition~\ref{def:alg-absstate}.
Then we use the two states and the $\rffinal$ state to compute the
\emph{loop reachability-bound} as in Definition~\ref{def:looprb}.
\begin{defn}[Loop Abstract States Computation]
\label{def:alg-loopabsstate}
Given a program $c$, with $l:\rprog$ as a sub-program (a loop) in $c$'s refined program, 
where $l: \rprog \subseteq REFINE(\kw{Rewrite(c)})$, 
we compute four two abstract states for the ranking functions of this loop, 
   $\lpinit(l: \rprog, \tpath, c)$, and $\lpnext(l: \rprog, \tpath, c) \in \scexpr(c)$
   are formally presented as follows.
   \begin{itemize}%
   \item 
The loop initial state 
$\lpinit(l: \rprog, \tpath, c) \in \scexpr(c)$ is symbolic expression as well. 
It describes the abstract initial value of $\tpath$'s ranking function before
any visit of $\tpath$ and during the first execution of $l: \rprog$.
\[
  \lpinit(l: \rprog, \tpath, c) \triangleq 
  \max_{l_1}\left\{
       \varinvar(y, c) + v ~\middle\vert~ 
       \begin{array}{l} 
         (l_1, x' \leq y + v, l_2) \in \reset(x, c) 
         \\
         \land \absinit(\rprog) \leq l_1 \leq \absinit(\tpath)
         \land
         x \in \left\{ \locbound(\absevent, c) | \absevent \in \tpath \right\}
       \end{array}
     \right\}
  \]
\item
The loop next state 
$\lpnext(l: \rprog, \tpath, c) \in \scexpr(c)$ 
describes how much $\tpath$'s ranking function
is modified before
the second visit of $\tpath$ but during the second execution of $l: \rprog$.
\footnote{$l' \in \rprog$: the $\in$ notation is abused to denote
the program point $l'$ is a vertex on a path in the program $\rprog$.}
%
\[
  \begin{array}{l}
  \lpnext(l: \rprog, \tpath, c) \triangleq 
  \\
  \max\limits_{x \in \left\{ \locbound(\absevent, c) | \absevent \in \tpath \right\}}
  \left\{
    \begin{array}{l}
  \sum\limits_{\absevent \in \inc(x, c) }
  \left\{ 
      v ~\middle\vert~ \absevent = (l', x' \leq x + v, \_) \land  l' \in \rprog 
      \land l' \notin \tpath \right\}
      \\ \qquad 
      + \max\limits_{l_2 }
         \left\{ \varinvar(y, c) + v ~\middle\vert~ 
         (l_1, x' \leq y + v, l_2) \in \reset(x, c) \land l_1 \in \rprog \land l_1 \notin \tpath\right\}
     \\ \qquad 
      - \sum\limits_{ \absevent \in \dec(x, c) }\left\{ 
      v 
      ~\middle\vert~ \absevent = (l', x' \leq x + v, \_) \land l' \in \rprog \land l' \notin \tpath \right\}
      \\ \qquad 
      + BD(\kw{enclosed}(\tpath), c) \times \rfnext(\tpath, c)
    \end{array}
    \right\}
  \end{array}
  \]
    \end{itemize}
\end{defn}
%
\begin{defn}[Loop Reachability-bound Computation]
  \label{def:looprb}
  Given a refined program $\rprog$ and a simple transition path $\tpath$ in this program, 
  let $l: \rprog_l$ be a loop program in $\rprog$,
  then $l: \rprog_l$'s \emph{loop reachability-bound} w.r.t. $\tpath$, $\lpchB(l: \rprog, \tpath, c)$
  is computed interactively with the abstract loop states as follows. 
  \[
    \lpchB(l: \rprog, \tpath, c) \triangleq
    \max\limits_{x = a \in \rffinal(\tpath, c)}
    \frac{\lpinit(\rprog, \tpath, c) - a }{\lpnext(\rprog, \tpath, c)}
  \]
\end{defn}
%
The computation $\lpchB(l: \rprog, \tpath)$ returns
a sound upper bound on iteration numbers of the outside loop $l$,
such that,
during these iterations, the nested loop $l' = \kw{enclosed(\tpath)}$ is executed, i.e., reached,
which is formally stated in Lemma~\ref{lem:looprb-sound}.
%  with proof in \highlight{Appendix~\ref{apdx:looprb-sound}}.
\begin{lem}[Soundness of the Loop Reachability-Bound]
  \label{lem:looprb-sound}
  For any loop $l: \rprog$ and a simple transition path $\tpath$ in a refined program, if $l_t: \rprog_t$ is the closest loop enclosing the $\tpath$, then the entering times of $l_t: \rprog_t$ when executing the $l: \rprog$ under initial trace $\trace_0 \in \ftdom_0(c)$, is bounded by $\econfig{\lpchB(l: \rprog, \tpath, c)}(\trace_0)$.
  \[
    \begin{array}{l}
    \forall c, c_l \in \cdom, \tpath \in \absG(c), \trace_0 \in \ftdom_0(c_l), \trace \in \tdom, l, l' \in \ldom, \rprog \st 
    \rprog = REFINE(\algrewrite(c))
    \\ \qquad
    \land l_t: \rprog_t = \kw{enclosed}(\rprog, \tpath)
    \land 
    \rprog = \algrewrite(c_l)
    \land
    \Big(
    \config{c_l, \trace_0} \to^* \config{\clabel{\eskip}^{l'}, \trace_0 \tracecat \trace}
    \lor \config{c_l, \trace_0} \uparrow^{\infty} \trace_0 \tracecat \trace 
    \Big)
    \\ \qquad
    \implies
    \econfig{\lpchB(l: \rprog, \tpath, c)}(\trace_0) 
    \\ \qquad \geq 
      \sum\Big\{
      \lcounter(\trace, \tracel(\trace_{l \to l_t})) ~\vert~ \trace_{l \to l_t} \in 
      \big\{\trace_{l \to l_t} ~\vert~ 
      % \\ \qquad \qquad \qquad \qquad
      \trace_{l \to l_t}, \trace' \in \ftdom, \trace'' \in \tdom
      \land \trace = \trace_{l \to l_t} \tracecat \trace''
      \\ \qquad \qquad \qquad \qquad
      \land \trace_{l \to l_t} = [(\_, l, \etrue)] \tracecat \trace' \tracecat [(\_, l_t, \etrue)]
      \land \counter(\trace', l) = \counter(\trace', l_t) = 0 
      \big\}
      \Big\}.
\end{array}
  \]
\end{lem}
% In Figure~\ref{fig:relatedNestedWhileOdd-overview}, $\tpath_3$ has an outer loop program $\rprog_1^1$.
% We first compute $\lpinit(\rprog_1^1, \tpath_3, c) = n - m $ and $\lpnext(\rprog_1^1, \tpath_3, c) = n - m - 1$ and then derive $\lpchB(\rprog_1^1, \tpath_3, c) = 1$ as tight as expected.

There is only one nested loop in Figure~\ref{fig:relatedNestedWhileOdd-overview}(a),
the only interesting \emph{loop reachability-bound} are $\lpchB(\rprog_1^1, \tpath_3)$ and $\lpchB(1: \rprog_1^1, \tpath_3)$ 
w.r.t. $\tpath_3$ and its outside loop $L_1$.
The outer loop program of $\tpath_3$ is $\rprog_1^1$.
We first compute $\lpinit(\rprog_1^1, \tpath_3, c) = n - m $ and $\lpnext(\rprog_1^1, \tpath_3, c) = n - m - 1$ and then derive $\lpchB(\rprog_1^1, \tpath_3, c) = 1$ as tight as expected.

To illustrate clearer the improvement of the \emph{loop reachability-bound},
we show the example program in Figure~\ref{fig:threeWhile-looprb} with three nested loops.
%  to better show the improvement of the \emph{loop reachability-bound}.
\input{examples/threeNestedWhile-looprb}
This example is similar to the loop $L_4$ nested in the second branch in Figure~\ref{fig:relatedNestedWhileOdd-overview}.
$w$ is reset by $j$ in command 5 and $i$ is then reset by $w$, so $L_6$ is only executed in the first iteration of loops $L_1$ and $L_3$.
% \\
Then the total iterations times are
$n + m^2 - m \times N$,
and the expected \emph{reachability-bound} for location $7$ inside the $L_6$ is $N$,
for locations $4, 5$ and $8$ between the $L_3$ and $L_6$ are $(n-N) \times (m - N)$,
and $n - N$ for locations $2$ and $9$.
% \\
The challenge here is that the locations inside the loop $L_6$ have the same
\emph{reachability-bound} as the local iteration times of $L_6$.
% , as well as our \emph{path reachability-bound}.
While for the locations between $L_3$ and $L_6$, their bounds are the product of the inner and outer loop iteration bounds.
To compute the precise bound, it is critical to know
the numbers of iterations of the outer loops $L_3$ and $L_1$ such that,
during these iterations, the loop $L_6$ is ``entered'', which is exactly our \emph{loop reachability-bound}.
Figure~\ref{fig:threeWhile-overview}(a) shows its abstract transition graph,
and we compute its refined program in the bottom. 
$\rprog_1$, $\rprog_3$ and $\rprog_6$ denote the body of the loop $L_1$, $L_3$ and $L_6$ respectively.
We first compute the $\outinB(\rprog_6, \tpath_3, c) = N $ for $\tpath_3$ w.r.t. its innermost loop.
Then for its two outer loops, $L_1$ and $L_3$,
% it is nested.
we compute $\lpinit( \rprog_1, \tpath_3, c) = 0$,
$\lpnext( \rprog_1, \tpath_3, c) = N + 1 $, and
$\rffinal(\rprog_1, c) = \{ i = n, k = N \}$ and obtain
$\lpchB(\rprog_1, \tpath_3, c) = \max\{ \frac{n - 0}{n - N - 1}, \frac{N - 0}{N + 1}\} = 1$, as expected.
In the same way, we also compute $\lpchB(\rprog_3, \tpath_3, c) = 1$.

