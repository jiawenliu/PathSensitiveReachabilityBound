
\textbf{\emph{Step1: Program abstraction.}}
In Section~\ref{sec:progabs},
we first 
generate the \emph{Abstract Transition Graph} as in Figure~\ref{fig:relatedNestedWhileOdd-overview}(b).
Each edge $l \xrightarrow{dc} l'$ is an abstract transition $\absevent = (l, dc, l')$ annotated with a constraint $dc$ corresponding to the command of label $l$.

Then we abstract the program in the form of paths.
$$
\tpath_0 ; \rpchoose{ 1: \rprepeat(\tpath_1; 4:\rprepeat(\tpath_3); \tpath_2), 1:\rprepeat(\tpath_4) }; \tpath_5
$$
$;$ concatenates sequence of execution paths,
$\rprepeat(\tpath_3)$ represents looping on the path $\tpath_3$ and
$\rpchoose{ 1: \rprog_1, 1: \rprog_2 }$ represents the loop $L_1$ that contains two possible execution paths,
$\rprog_1 = \tpath_1; 4:\rprepeat(\tpath_3);\tpath_2$ and $\rprog_2 =\tpath_4$.

\textbf{\emph{Step 2: ranking function estimation.}}
Algorithms in Section~\ref{sec:rank} identifies the ranking function for each transition edge, which is a symbol whose number of decreasing times can represent the number of execution of this edge.
For example for edge $4 \to 5$, its ranking function is $k$ and edges on $\tpath_1$, $\tpath_2$ and $\tpath_4$ all have $i$ as their ranking functions.


% \textbf{Step 2: Program Refinement}
\textbf{\emph{Step 3: Path interleaving refinement.}} 
Two execution paths are not simply iterating on themselves during the program execution,
they could interleave each other at certain iteration.
We summarize each execution path into conjunctions of transition relations on this path.
% , which is a  of transition relations.
\begin{equation}
    \begin{array}{l}
        \rprog_1 \models i > 0 \land  i \%2 \neq 0 \land k' \leq \emin(0, i - m) 
        \land i' \leq \emin(0, i - m) + m - 1\\
    \rprog_2 \models i > 0 \land i \%2 = 0 \land  i' \leq i - 3
    \end{array}
\end{equation}

However, $\rpchoose{1: \rprepeat(\rprog_1), 1:\rprepeat(\rprog_2)}$ represents a larger set of paths than the actual program execution paths because $\rprog_1 $ and $\rprog_2$ cannot always executable during the program execution.
To know how they interleave between each other, computes the interleaving orders
by exhaustively checking the compositions of their transition relations.
\begin{equation}
    \begin{array}{l}
        \rprog_1 ; \rprog_1 \models 
        \exists m \st  i > 0 \land  i \%2 \neq 0 \land k' \leq \emin(0, i - m - 1) \land \emin(0, i - m) + m - 1 > 0 \\
            \qquad \qquad \qquad \land \emin(0, i - m) + m - 1 \% 2 \neq 0 \land i' \leq  \emin(0, i - m) + m - 2 \\
        \rprog_2 ; \rprog_2 \models \exists m \st  i > 0 \land  i \%2 = 0 \land i - 3 > 0 \land (i - 3) \% 2 = 0 \land i' \leq i - 6  \\
        \rprog_2 ; \rprog_1 \models \exists m \st  i > 0 \land  i \%2 = 0 \land i - 3 > 0 \land (i - 3) \%2 \neq 0 \land k' \leq \emin(0, i - m - 3) \\
        \qquad \qquad \qquad 
        \land i' \leq \emin(0, i - m) + m - 4 \\
        \rprog_1 ; \rprog_2 \models \exists m \st  i > 0 \land  i \%2 \neq 0 \land k' \leq \emin(0, i - m - 1) \land \emin(0, i - m) + m - 1 > 0 \\
        \qquad \qquad \qquad \land \emin(0, i - m) + m - 1 \% 2 = 0 \land i' \leq  \emin(0, i - m) + m - 4
    \end{array}
\end{equation}
Then we eliminate the interleaving orders that are infeasible and preserve these compositions (interleaving orders) that are equivalent
representations of the original loop.
For each equivalent representations, we create a new loop with the corresponding interleaving order as the only one execution path.
We compute two new loops for this example that
either $\rprog_1$ executes after one iteration of $\rprog_2$ or vice versa.
% Then, loop $L_1$ in the source program is generates new execution paths as follows,
\[
    \rprog^1 = \rprog_1 ; \rprog_2 = \tpath_1; 4:\rprepeat(\tpath_3); \tpath_2; \tpath_4
    \qquad
    \rprog^2 = \rprog_2 ; \rprog_1 = \tpath_4; \tpath_1; 4:\rprepeat(\tpath_3); \tpath_2
\]
% The second step in Section~\ref{sec:refine}
Then, the multiple-paths loop $L_1$ in the source program is refined
into multiple loops where each one can only iterate following the specified interleaving order.
% the interleaving of paths is explicit.
As in the bottom of Figure~\ref{fig:relatedNestedWhileOdd-overview}(c),
the program is transformed into 
\[
    \tpath_0 ; \rpchoose{ 1: \rprepeat(\tpath_1; 4:\rprepeat(\tpath_3); \tpath_2; \tpath_4), 
1: \rprepeat(\tpath_4; \tpath_1; 4:\rprepeat(\tpath_3); \tpath_2) }; \tpath_5
\]
In this refined program, 
each new execution path is equivalent to the execution of the original loop. 
% denoted as $\rprog_1^1$ and $\rprog_1^2$.

% \textbf{Step 3: Ranking Function Estimation}
% \textbf{Step 4: Path-sensitive Reachability-bound Computation.}
\textbf{\emph{Step 4: local path reachability-bound.}}
For $\tpath_3$ in the program in Figure~\ref{fig:relatedNestedWhileOdd-overview}, we want to know how many times it is ``reached'' during the program execution.
From the refined program, $\tpath_3$ shows up in both newly generated execution paths $\rprog^1$ and $\rprog^2$  and nested in two level loops.
The algorithm in Section~\ref{sec:pathlocalrb} first
% computes a local \emph{path reachability-bound} for it w.r.t. its innermost loop $L_4$ by computing 
computes three abstract states for the ranking functions on $\tpath_3$ when first, second and last visiting during execution of $L_4$,
\begin{equation*}
    \rfinit(\rprog^1, \tpath_3, c) = \{k = n - m\} \quad
    \rfnext(\rprog^1, \tpath_3, c) = \{k = 1\} \quad
    \rffinal(\tpath_3, c) = \{ k = 0 \}.
\end{equation*}
Then  the maximal value of the following formula provides   
an upper bound on the number of execution times of $\tpath_3$ when executing only the innermost loop where $\tpath_3$ is nested. 
\[
    \max
    \left\{ 
        {\frac{a - b}{1}} 
        ~\vert~
        x = a \in \{k = n - m\}
        \land x = b \in \{ k = 0 \}
    \right\}  = n - m
\]
% The algorithm in Section~\ref{sec:pathlocalrb}
% computes $\outinB(4:\rprepeat(\tpath_3), \tpath_3, c) = n - m$ by computing
% the initial state, next state and final state of ranking functions on $\tpath_3$ during the execution of $\rprepeat(\tpath_3)$.

\textbf{\emph{Step 5: loop reachability-bound.}}
Previous step only provides the path reachability-bound for a simple transition path w.r.t. the innermost loop.
For nested loops, we need to compute the \emph{loop reachability-bound} for each simple transition path with respect to every level of the outer loop.
Since $\tpath_3$ is located in two level nested loops, we compute 
with respect to its only outer loop $L_1$ a \emph{loop reachability-bound}.
% It is expected to be $1$ because the inner loop $L_4$ is reached only in the first iteration of the outer loop $L_1$.
% , we aim to compute $1$ as the \emph{loop reachability-bound} of $\tpath_3$ w.r.t. $L_1$.
In the first new execution path, $\rprog^1 = \rprepeat(\tpath_1; 4:\rprepeat(\tpath_3); \tpath_2; \tpath_4)$,
Definition~\ref{def:alg-loopabsstate} in Section~\ref{def:looprb} computes three abstract states when visiting $L_4$ the first, second and last time during the execution of loop $\rprog^1$,
\begin{equation*}
\lpinit(\rprog^1, \tpath_3, c) = \max\{ n - m\} \quad
\lpnext(\rprog^1, \tpath_3, c) = \max\{n - m\} \quad
\rffinal(\tpath_3, c) = \{k = 0\}.
\end{equation*}
Then we compute \emph{loop reachability-bound} as the maximal value of the formula,
\[
    \max\limits_{x = a \in \{k = 0\}}
    \left\{ \frac{\max\{n - m\} - a }{\max\{n - m\}} \right\} = 1
  \]
For the second new execution path $\rprog^2$, algorithm in Section~\ref{def:looprb} computes the same result.
% $\outinB(4:\rprepeat(\tpath_3), \tpath, c) = n - m - 3$ and the same $\lpchB(\rprog_1^2, \tpath_3, c)$.
% So 

\textbf{\emph{Step 6: path reachability-bound.}}
For each simple transition path in every refined execution path where it shows up, we take the production of the local \emph{path reachability-bound} and the  \emph{loop reachability-bound} w.r.t. each level of its outer loop.
For $\tpath_3$,
% $\rprepeat(\tpath_1; 4:\rprepeat(\tpath_3); \tpath_2; \tpath_4)$,
we compute its \emph{path reachability-bound} globally
$$
\inoutB(\rprog, \tpath_3, c) = \max\{ 1 \times (n - m), 1 \times (n - m - 3) \} = n - m
$$
% , and $1 \times (n - m - 3)$ in the second execution path.
% Then we
by
taking the maximal value over all newly generated execution paths.
This maximization operation does not produce over-approximation because there does not exist interleave
between the refined execution paths and each refined execution path is equivalent to the original loop, and each other as well.

\textbf{\emph{Step 7: reachability-bound.}}
Now for every program point $l$, we sum up the $\inoutB(\rprog, \tpath)$ over all $\tpath$ that contains $l$ to get $\psRB(l, c)$.
In Figure~\ref{fig:relatedNestedWhileOdd-overview}(c), program point $5$ only shows up on $\tpath_3$, so we have
$$\psRB(5, c) = \inoutB(\rprog, \tpath_3, c) = n - m.$$
The points $0$ and $\lex$ are not in any loop, so we have $\psRB(0, c) = \psRB(\lex, c) = 1$.
The points $3, 6, 7$ and $8$ which only show up once on $\tpath_2$ and $\tpath_4$ has reachability-bound as
$$\inoutB(\rprog, \tpath_2, c) = \frac{m}{4}, \inoutB(\rprog, \tpath_4, c) = \frac{m}{4}$$
% $\lfloor\frac{m}{4}\rfloor$ the same as their $\inoutB$.
For the loop headers $1$ and $4$, we only count the $\tpath$ where they show up as a start-point,
$$\psRB(4, c) = \lfloor\frac{m}{4}\rfloor + n - m + 1, \qquad \psRB(1,c) = 2 \times \lfloor\frac{m}{4}\rfloor + 1$$ 
all as expected.