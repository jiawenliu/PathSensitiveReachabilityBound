
\textbf{\emph{Step1: Program abstraction.}}
In Section~\ref{sec:progabs},
we first 
generate the \emph{Abstract Transition Graph} as in Figure~\ref{fig:relatedNestedWhileOdd-overview}(b).
Each edge $l \xrightarrow{dc} l'$ is an abstract transition $\absevent = (l, dc, l')$ annotated with a constraint $dc$ corresponding to the command of label $l$.

Then we abstract the program in the form of paths.
$$
\tpath_0 ; \rpchoose{ 1: \rprepeat(\tpath_1; 4:\rprepeat(\tpath_3); \tpath_2), 1:\rprepeat(\tpath_4) }; \tpath_5
$$
$;$ concatenates sequence of execution paths,
$\rprepeat(\tpath_3)$ represents looping on the path $\tpath_3$ and
$\rpchoose{ \ldots}$ represents the loop $L_1$ which contains two possible execution paths,
$\rprog_1 = \tpath_1; 4:\rprepeat(\tpath_3);\tpath_2$ and $\rprog_2 =\tpath_4$.

% \textbf{Step 2: Program Refinement}
\textbf{\emph{Step 2: Path interleaving refinement.}} 
Two execution paths are not simply iterating on themselves during the program execution,
they could interleave each other at certain iteration.
We summarize each execution path into conjunctions of transition relations.
\begin{equation}
    \begin{array}{l}
        \rprog_1 \models \phi_1 = \\
    \rprog_2 \models \phi_2 = 
    \end{array}
\end{equation}
  
In this sense, Algorithm~\ref{alg:prog-refine} in Section~\ref{sec:refine} computes the interleaving orders
by exhaustively checking the compositions of transition relations of different execution paths,
\begin{equation}
    \begin{array}{l}
        \rprog_1 ; \rprog_1 \models \exists i, k \st \phi_1 \circ \phi_1 = ... \implies \efalse\\
        \rprog_2 ; \rprog_2 \models \exists i, k \st \phi_2 \circ \phi_2 = ... \implies \efalse \\
        \rprog_2 ; \rprog_1 \models \exists i, k \st \phi_2 \circ \phi_1 = ...  \\
        \rprog_1 ; \rprog_2 \models \exists i, k \st \phi_1 \circ \phi_2 = ... 
    \end{array}
\end{equation}
Only two execution paths are feasible, so we identify two unique interleaving orders --
either $\rprog_1$ executes after one iteration of $\rprog_2$ or vice versa.
% Then, loop $L_1$ in the source program is generates new execution paths as follows,
\[
    \rprog_1 ; \rprog_2 = \tpath_1; 4:\rprepeat(\tpath_3); \tpath_2; \tpath_4
    \qquad
    \rprog_2 ; \rprog_1 = \tpath_4; \tpath_1; 4:\rprepeat(\tpath_3); \tpath_2
\]
% The second step in Section~\ref{sec:refine}
Then, the multiple-paths loop $L_1$ in the source program is refined
into multiple loops where each one can only iterate following the specified interleaving order.
% the interleaving of paths is explicit.
As in the bottom of Figure~\ref{fig:relatedNestedWhileOdd-overview}(c),
the program is transformed into 
\[
    \tpath_0 ; \rpchoose{ 1: \rprepeat(\tpath_1; 4:\rprepeat(\tpath_3); \tpath_2; \tpath_4), 
1: \rprepeat(\tpath_4; \tpath_1; 4:\rprepeat(\tpath_3); \tpath_2) }; \tpath_5
\]
In this refined program, 
each new execution path is equivalent to the execution of the original loop. 
% denoted as $\rprog_1^1$ and $\rprog_1^2$.

% \textbf{Step 3: Ranking Function Estimation}
\textbf{\emph{Step 3: ranking function estimation.}}
Algorithms in Section~\ref{sec:rank} identifies the ranking function for each transition edge, which is a symbol whose number of decreasing times can represent the number of execution of this edge.
For example for edge $4 \to 5$, its ranking function is $k$ and edges on $\tpath_1$, $\tpath_2$ and $\tpath_4$ all have $i$ as their ranking functions.

% \textbf{Step 4: Path-sensitive Reachability-bound Computation.}
\textbf{\emph{Step 4: local path reachability-bound}}
For $\tpath_3$ in the program in Figure~\ref{fig:relatedNestedWhileOdd-overview}, we want to know how many times it is ``reached'' during the program execution.
From the refined program, it shows up in both the new generated execution paths and nested in two level loops.
We first compute a local \emph{path reachability-bound} for it w.r.t. its innermost loop $L_4$.
This is an upper bound on the number of execution times of $\tpath_3$ when executing only the innermost loop where $\tpath_3$ is nested. The algorithm in Section~\ref{sec:pathlocalrb}
computes $\outinB(4:\rprepeat(\tpath_3), \tpath_3, c) = n - m$ by computing
the initial state, next state and final state of ranking functions on $\tpath_3$ during the execution of $\rprepeat(\tpath_3)$.

\textbf{\emph{Step 5: loop reachability-bound}}
For nested loops, we need to compute the \emph{loop reachability-bound} for each simple transition path with respect to every level of the outer loop.
Since $\tpath_3$ is nested in two level loops, we compute its \emph{loop reachability-bound}
with respect to the outer loop $L_1$. 
It is expected to be $1$ because the inner loop $L_4$ is reached only in the first iteration of the outer loop $L_1$.
% , we aim to compute $1$ as the \emph{loop reachability-bound} of $\tpath_3$ w.r.t. $L_1$.
In the first refined execution path, $\rprog^1 = \rprepeat(\tpath_1; 4:\rprepeat(\tpath_3); \tpath_2; \tpath_4)$,
we compute the initial state, next state of ranking function $i$ when visiting $L_4$
as $\{ k = n - m\}$, $\{n - m\}$ and the final state $\{k = 0\}$
during the execution of loop $\rprog^1$ and get
$\frac{n-m - 0}{n - m} = 1$ as its \emph{loop reachability-bound}.
We also compute in the second refined execution path $\rprog_1^2$ the same number.
% $\outinB(4:\rprepeat(\tpath_3), \tpath, c) = n - m - 3$ and the same $\lpchB(\rprog_1^2, \tpath_3, c)$.
% So 

\textbf{\emph{Step 6: path reachability-bound.}}
For each simple transition path in every refined execution path where it shows up, we take the production of the \emph{loop reachability-bound}
and local \emph{path reachability-bound}.
For example for $\tpath_3$ in the first refined execution path 
$\rprepeat(\tpath_1; 4:\rprepeat(\tpath_3); \tpath_2; \tpath_4)$,
we compute $1 \times (n - m)$ and $1 \times (n - m - 3)$ in the second execution path.
Then we
take the maximal value over all refined execution order and
$\inoutB(\rprog, \tpath_3, c) = \max\{ 1 \times (n - m), 1 \times (n - m - 3) \} = n - m$.
This maximization operation does not produce over-approximation because there does not exist interleave
between the refined execution paths and each refined execution path is equivalent to the original loop, and each other as well.

\textbf{\emph{Step 7: reachability-bound.}}
Now for every program point $l$, we sum up the $\inoutB(\rprog, \tpath)$ over all $\tpath$ that contains $l$ and get $\psRB(l, c)$.
Since point $5$ only shows up on $\tpath_3$, we compute \highlight{$\psRB(5, c) = n - m$}.
The points $0$ and $\lex$ are not in any loop, so we have $\psRB(0, c) = \psRB(\lex, c) = 1$.
The points $3, 6, 7$ and $8$ which only show up once on $\tpath_2$ and $\tpath_4$ are all equal to $\lfloor\frac{m}{4}\rfloor$ the same as their $\inoutB$.
For the loop headers $1$ and $4$, we only count the $\tpath$ where they show up as a start-point.
So $\psRB(4, c) = \lfloor\frac{m}{4}\rfloor + n - m + 1$ and $\psRB(1,c) = 2 \times \lfloor\frac{m}{4}\rfloor + 1$ all as expected.