We compute the ranking function and estimate its upper bound invariant in three steps as follows.

\begin{enumerate}
 \item For each variable, we collect three sets 
 in which the variable increases, decreases, and is reset respectively.
 \item
 Then, we assign a symbol (a variable or a normalized counter expression that is viewed as variable)  as the ranking function to the edge on which this symbol decreases.
 \item
 In the last step, we compute the upper bound invariant on the maximum value of each ranking function recursively.
 In the meantime, it also computes an upper bound on the iteration times of each loop in a path-insensitive manner.
 \end{enumerate}

  %
 \subsection{Collecting Variable Modifications}
 For each variable $x$ in a program $c$, this step computes three edge sets, $\inc(x, c)$, $\dec(x, c)$,
 and $\reset(x, c)$ for $x$.
 Every edge in a set corresponds to a transition in which $x$ is increased,
 % $\inc(x, c)$,
 decreased
 % $\dec(x, c)$ and 
 or reset
 respectively.

 \begin{itemize}
 \item
 $\inc: \cdom \to \vardom \to \mathcal{P}(\absevent) $
 is the set of edges where the variable increase, 
 %\\
 \[ \inc(x, c) = \left\{ \absevent | \absevent = (l, x' \leq x + v, l') \land \absevent \in \absflow(c) \right\} \]
 %\\
 \item
 $\dec: \vardom \to \mathcal{P}(\absevent) $
 is the set of abstract events where the variable decrease,
 %\\
 \[\dec(x, c) = \left\{\absevent| \absevent = (l, x' \leq x - v, l') \land \absevent \in \absflow(c) \right\}\]
 %\\
 \item
 $\reset: \cdom \to \vardom \to \mathcal{P}(\absevent) $ is the set of the abstract events where the variable is reset,
%
 \[\reset(x, c) = \left\{ \absevent| \absevent = (l, x' \leq y \pm v, l') \land x \neq y \land \absevent \in \absflow(c) \right\}\]
 \item
 Additionally,
 we also compute the reset graph $\resetG(c)$ and the reset chain, $\resetchain(x, c) \in \mathcal{P}(\mathcal{P}(\absevent))$ for every rank $x$.
 The $\resetchain(x, c)$ for every rank $x$ contains all the paths in $\resetG(c)$ that end at $x$.
 The computation of $\resetG(c)$ and $\resetchain(x, c)$ follows the Definition~20 in~\cite{SinnZV17}.
 \begin{defn}
    \label{def:resetgraph}
    For a program $c$, its reset graph $\resetG(c) = (\resetV(c), \resetE(c))$ has two components,
    each of them is computed as follows,
%  \[\resetG(c) = (\resetV(c), \resetE(c))\]
 \[
    \resetE(c) = \left\{ (x, \absevent, y) ~\vert~ \absevent \in \reset(x, c) \land \absevent = (l, x' \leq y \pm v, l') \right\} 
 \]
 \[
    \resetV(c) = \left\{ x ~\vert~ (x, \_, \_) \in \resetE(c) \lor (\_, \_, x) \in \resetE(c) \right\} 
    \]
 \end{defn}
 In a variable $x$'s reset chain set, $\resetchain(x, c)$, in each chain $(e_0, \ldots, e_m) \in \resetchain(x, c)$
 a variable $x_i$ is reset by another variable $x_{i + 1}$ on edge $e_{i}$
 and $x_{i + 1}$ is reset on edge $e_{i + 1}$ recursively
 for every $i = 0, \ldots, m - 1$.
 $x$ is reset on the first edge $e_0$ of every sequence in $\resetchain(x, c)$.
 {Each edge $e_i$ in a sequence $(e_0, \ldots, e_m) \in \resetchain(x, c)$
 resets a variable $x_i$ by another variable $x_{i + 1}$ such that $x_{i + 1}$
 is reset on edge $e_{i + 1}$ recursively. The first edge $e_0$ of each sequence resets the variable $x$.}
 % 
 \end{itemize}
 In the following steps, $c$ is omitted in $\inc(x, c)$,
 $\dec(x, c)$ and $\reset(x, c)$ for concise when the reference of a program $c$ is clear in the context.

 \subsection{Assigning The Ranking Function to An Edge}
 For each edge in the transition graph $\absG(c)$ of a program $c$,
 this step assigns the variable that decreases on this edge as the ranking function of this edge.
 This step adopts the local bound computation method in Section 4 of~\cite{SinnZV17} to assign the local bound to each edge,
 formally as follows.
 \begin{defn}[Ranking Function Generatation]
 \label{def:ranking_gen}
 For every edge $\absevent$ in the transition graph $\absG(c)$ of a program $c$,
 its \emph{ranking function/local bound}, $\rank(\absevent, c)$
 is the variable that decreases on this edge, computed as follows,
 %
 \[ 
\begin{array}{ll}
 \rank(\absevent, c) \triangleq 1 
 & \absevent \notin \kw{SCC}(\absG(c))
 \\
 \rank(\absevent, c) \triangleq x
 & \absevent \in \kw{SCC}(\absG(c)) \land \absevent \in \dec(x, c) \land \absevent = (\_, \_ , x' \leq x - v) \\
 \rank(\absevent, c) \triangleq \rank(\absevent', c)
 & \exists \absevent' \in \absG(c), \tpath \in \paths( \absG(c)) \st \absevent, \absevent' \in \tpath \\
 \rank(\absevent, c) \triangleq x
 & \absevent \in \kw{SCC}(\absG(c)) \land 
 \absevent \notin \bigcup_{x \in \vardom} \dec(x, c) \\
 & \qquad \land \absevent \notin \kw{SCC}(\absG(c) \setminus \dec(x, c))\\
 \rank(\absevent, c) \triangleq \infty
 & o.w..
\end{array}
\]
 $\kw{SCC}(\absG(c))$ is the set of all the non-trivial strongly connected components of $\absG(c)$.
 \end{defn}
 The first case is straightforward. 
 For the label $l$ which is not in any while loop, 
 the labeled command with the label $l$ will be 
 evaluated at most once. 
%  The soundness is formalized in Lemma~\ref{lem:local_bound_sound} with proof in Appendix~\ref{apdx:pathinsensitive_rb_soundness}.
 %

 \paragraph{Example}
 In Figure~\ref{fig:relatedNestedWhileOdd-overview}(b), we assign variable $i$ to edge $7 \xrightarrow{i' \leq i - 1} 1$ as ranking.
 If we remove this edge, the edges on $\tpath_2$ and $\tpath_1$ are not in any SCC, so they also have the same ranking.
In the same way, ranking for edges on $\tpath_4$ is $i$ as well as $k$ for them on $\tpath_3$.


 \subsection{Ranking Function Estimation}
 This step estimates the upper bound, $\varinvar(x, c) \in \scexpr(c)$
 on the maximum value for each ranking function $x \in \vardom \cup \scvar(c)$.
 \\
 For a program $c$, the upper bound invariant,
 $\varinvar(\rank(\absevent, c), c)$ for a ranking function $\rank(\absevent, c)$ 
 is an invariant of the maximum value of $\rank(\absevent, c)$ during the program execution.

 In order to estimate the invariant of the maximum value of the ranking function $\rank(\absevent, c)$
 of the edge $\absevent \in \absE(c)$, we use a mutual recursion and recursively compute
 the bound on the iteration times of the edge, $\absclr(\absevent, c)$ in the meantime.
%  is computed interactively in a path-insensitive manner.
 % \\ 
 % $\varinvar (x, c) \in \scexpr(c)$
 % \\
 % $\absclr(\absevent, c) \in \scexpr(c)$
 \begin{defn}[Ranking Function Estimation]
 \label{def:ranking_bound}
 For a program $c$ and an edge $\absevent \in \absE(c)$,
 the \emph{ranking function bound}, 
 $\varinvar(\rank(\absevent, c), c)$ for the ranking function $x = \rank(\absevent, c)$
 of this edge
 is computed through a mutual recursion between $\varinvar(x, c)$ and $\absclr(\absevent, c)$ as follows,
 \[ 
 \begin{array}{lll}
 \varinvar(x, c) & \triangleq x & \text{if} ~ x \in \scvar(c) \\
 \varinvar(x, c) & \triangleq \incrs(x, c) & \text{if} ~ x \notin \scvar(c) \\
 & \qquad + \max\left\{\varinvar(y, c) + v ~\mid~ (l, x' \leq y + v, l') \in \reset(x, c) \right\} &
 \end{array}
 \]
 %
 where $\incrs(x, c) \triangleq \sum\limits_{\absevent \in \inc(x, c)}\{\absclr(\absevent, c) \times v ~\mid~ \absevent = (l, x' \leq x + v, l')\}$
$\absclr(\absevent, c) \in \scexpr(c)$ is an upper bound on the execution times of the transition $\absevent$, which is recursively computed in a path-insensitive manner as below,
\[ 
\begin{array}{lll}
 \absclr(\absevent, c) 
 & \triangleq \varinvar(\rank(\absevent, c), c) & \\
 & \quad \text{if} ~ \rank(\absevent, c) \in \scvar(c) & \\
 \absclr(\absevent, c) 
 & \triangleq
 \sum \left\{ \incrs(y, c) | ch \in \resetchain(x, c) \land y \in ch \right\} & \\
 & \quad + 
 \sum\limits_{ch \in \resetchain(x, c)}
 \min \left\{\absclr(\absevent', c) ~\mid~ \absevent' \in ch\right\} \times 
 \big(\varinvar(in(ch), c) 
 + \sum\limits_{(\_, (\_, x' \leq y + v, \_), \_) \in ch} v \big) & \\
 & \quad \text{if} ~\rank(\absevent, c) = x \land x \notin \scvar(c) & ,
\end{array}
 \]
 where $in(ch)$ is the first vertex of the reset chain $ch$.
\end{defn}
 %
 For every edge, if the ranking function of this edge computed at the previous step is a symbol, specifically an input variable, then this is already the upper bound invariant. 

 If instead, the ranking function of the edge is a variable $y$ which is not an input variable, this step will eliminate it and replace it with a symbolic expression. 
 In order to do this, these steps will compute two quantities: first, it will recursively sum over the TB of all the edges whose constraint may increment the variable $y$, multiplying the corresponding increment; second, it will recursively compute the upper bound invariant of all the other edges whose difference constraint may reset the variable $y$ and take the maximum value. The sum of these two quantities provides the symbolic expression that is an upper-bound invariant on the maximum value of this ranking function. To compute these two quantities we use two mutually recursive procedures.
 
 If a ranking function is increased in an edge of its nested loop, then this mutual recursion is non-terminating. In this case, we identify a cycle and produce infinity as the bound.
 
%
\paragraph{Example.}
% \todo{The w example}
For the example program in Figure~\ref{fig:relatedNestedWhileOdd-overview}, we first find the ranks for every transition,
$\rank(0 \to 1) = 1$,
$\rank(1 \to \lex) = 1$,
$\rank(1 \to 2) = i$,
$\rank(2 \to 3) = i$,
$\rank(3 \to 4) = i$,
$\rank(4 \to 5) = k$,
$\rank(5 \to 4) = k$,
$\rank(4\to 6) = \rank(6 \to 7) = \rank(7 \to 1) = i$,
and $\rank(2 \to 8) = \rank(8 \to 1) = i$.
Then we need to estimate the upper bound invariant for the ranking function $i$ and $k$.
The ranking function $i$ is reset by $n$ at edge $0 \to 1$ and 
$k + m$ at edge $6 \to 7$ and $k$ is assigned value $i - m$ at edge $3 \to 4$. 
$i$ is reset by $k+m$, and $k$ is reset by $i - m$.
 We compute a variable reset graph, and we use a variable renaming method to unify the variable name of $k$ and $i$ into the same new variable $z$.
 Then this new variable $z$ is the ranking function of all edges where $k$ and $i$ were, and $z$ is only reset on the edge $0 \to 1$ by input variable $n$. $\varinvar(n) = n$ because $n$ is an input variable.
 So we compute the upper bound invariant for $z$ is $n$, and
 the upper bound invariant for both $i$ and $k$ is $n$ as well.
