
Gulwani et al.~\cite{GulwaniZ10} first introduced \emph{reachability-bound analysis} as a way to provide upper bounds on the number of times a given program location 
is visited during program execution.
Obtaining tight reachability-bounds is useful in many applications.
For example from a privacy and security perspective,
how much secret information is leaked by a program may depend on the number of times a certain operation that leaks the data
is executed~\cite{Malacaria07};
from an efficiency perspective, when different program locations consume different resources, a precise reachability-bound for each location can help to estimate the resource cost more accurately than just computing the overall complexity.

%Existing works all have different limitations when inferring the \emph{reachability-bound}.
Gulwani et. al~\cite{GulwaniZ10}
give a two-step solution combining program abstraction with a set of proof-rules that are used to compute the actual reachability-bounds. This combination is effective at computing reachability-bounds for a rich class of programs, however, it also has limitations. This method 
computes complexity bounds for loops and over-approximate the reachability-bounds of the different locations inside the loop by using this bound. This result in an over-approximation of the reachability-bounds of several locations, especially when there are loops with multiple paths.

A similar approach to reachability-bounds can be also computed, in principle, using techniques from \emph{amortized program complexity analysis}~\cite{BradleyMS05,CookSZ13,Zuleger18,SinnZV14,SinnZV17,LuCT21,AliasDFG10}. Indeed, using techniques like ranking functions, counter increments, difference constraints, etc.
one can estimate 
precise bounds on loop iterations and execution time, and in turn obtain reachability bounds for complex programs. However, these techniques are designed to capture the worst-case or overall complexity, and again they may over-approximate the reachability-bounds of certain locations. This is again particularly evident when a loop contains multiple paths, since in this situation one usually obtains an overall complexity bound which is greater than the reachability-bounds on some locations of some of the paths. 

To overcome these limitations, 
we introduce a new path-sensitive reachability-bound analysis that combines ideas from amortized complexity analysis with ideas from 
\emph{path refinement complexity analysis}~\cite{GustafssonEL05,ManoliosV06,BalakrishnanSIG09,SharmaDDA11,Flores-MontoyaH14,HumenbergerJK18,CyphertBKR19,GulwaniJK09,ZulegerGSV11}.
% can help to summarize the loop paths and compute the program complexity path-sensitively. However, they still do not compute the  reachability-bound for every program location.
Specifically, we propose a new algorithm which by combining amortized  analysis with loop summarization-based multipath refinement mitigates the limitations faced by each technique individually. 
% \end{itemize}
%

More in details, our algorithm first uses the difference constraints program abstraction model~\cite{SinnZV17,SinnZV14} enriched with boolean expressions to generate an abstract transition graph.
 %in Section~\ref{sec:progabs}
Then the algorithm performs both path refinement and amortized complexity analysis through ranking estimation over this graph.
%in Section~\ref{sec:refine} and  Section~\ref{sec:rank}.
By computing the ranking function, we effectively alternate computation for the loop bound with the bound invariant of the ranking function. 
% $\locbound(\absevent, c)$ for each transition edge $\absevent$, 
% and estimating its bound invariant in Section~\ref{sec:rank}.
We also leverage the path-insensitivity of ranking function estimation through a lightweight path refinement algorithm adopted from~\cite{GulwaniJK09}.
Building on this combination, we then focus on two useful quantities which we call \emph{path reachability-bound} and \emph{loop reachability-bound}.
The first one bounds the evaluation times of each loop-free and interleave-free path in a refined program, and the second one bounds the number of iterations of an outer loop w.r.t. an inner loop, counting only the iterations when the inner loop is ``entered''. 
We present algorithms for estimating these two quantities. By using them we can then 
%Section~\ref{sec:looprb} and~\ref{sec:pathrb} 
compute the \emph{reachability-bound} for each program point in a path-sensitive way.
%in Section~\ref{sec:psrb}.
We have implemented our algorithm in a prototype and evaluated it on 290 program extracted from different benchmarks. Our evaluation %Section~\ref{sec:eval} 
shows that we can accurately estimate different bounds for different locations in programs with loops containing multiple paths. Based on this, we can also compute a tighter overall complexity compared to the state-of-art worst-case complexity analysis tools.

To summarize, our contributions are as follows.
\begin{itemize}
\item We introduce a path-sensitive reachability-bound algorithm 
combining \emph{amortized bound analysis}  and \emph{path refinement}. The algorithm computes tight bounds on the times that each program point is evaluated.

\item 
 We identify two quantities,  \emph{path reachability-bound} and \emph{loop reachability-bound}, which are useful to measure precise information about the program execution.  We also propose two algorithms for soundly estimating these two quantities.
\item 
 We develop a prototype implementation with evaluation over 290 programs.
 The evaluation shows improvement in precision compared to other tools for reachability-bound analysis and the worst-case complexity analysis.
\end{itemize}

\highlight{
    This paper is organized as follows.
Section~\ref{sec:preliminary} recalls the standard while language syntax, semantics and the reachability-bound
definition under this language model.
Section~\ref{sec:progabs} generates the abstract transition graph for a program and then abstracts the program in the form of paths.
Section~\ref{sec:refine} presents the path interleaving refinement algorithm.
Section~\ref{sec:pathlocalrb} gives the local path reachability-bound computation, which is a subroutine of the path reachability-bound algorithm.
In Section~\ref{sec:looprb}, we introduce the loop reachability-bound and give the computation algorithm, which is used in Section~\ref{sec:pathrb} to compute a global path reachability-bound.
Then in Section~\ref{sec:alg-rb}, we show how to compute a sound reachability-bound for every program location.
Section~\ref{sec:eval} reports the evaluation results of the {\PSRB} and the comparison result with other state-of-art complexity analysis tools.
Section~\ref{sec:relatedwork} presents the related works and Section~\ref{sec:conlusion}
gives the conclusion and future works.
All the soundness proofs for algorithms are given in the Appendices.}