As first steps, we transform a program in its abstract transition graph and we do some program rewriting to prepare it for the main algorithm.

\subsection{Abstraction Transition Graph}
We consider the \emph{Abstract Transition Graph}, $\absG(c)$
% $ =(\absV(c), \absE(c))$ 
of a program $c$.
We first introduce some components of this graph.
\begin{defn}[Symbolic Constants, Symbolic Expressions]
 \label{def:symbolic_expr}
A symbolic constant is either a natural number, $\infty$, or a program's input variable. We denote by $\scvardom \subseteq \mathbb{N} \cup \vardom \cup \{\infty \}$ the universe of the \emph{symbolic constant},
and $ \scvar(c) = \mathbb{N} \cup \inpvar(c) \cup \{\infty \}$ the set of symbolic constant of
program $c$.\\
% and $\scexpr$
 The set of the \emph{symbolic expression} for a program $c$,
$\scexpr(c)$ is the set of all the expressions over $\scvar(c)$ that are
formed using the arithmetical operators' addition ($+$), multiplication
($\times $),  integer division ($/$) maximum ($\emax $), minimum ($\emin$).
\[
  \begin{array}{rl}
  \scexpr(c) := \sexpr \in \mathbb{N} \cup \inpvar(c) \cup \{\infty \} ~|~ \sexpr \times \sexpr ~|~ \sexpr / \sexpr ~|~
  \sexpr + \sexpr ~|~ \emax(\sexpr, \sexpr) ~|~ \emin(\sexpr, \sexpr)
  \end{array}
\]
\end{defn}
% \mg{This language of expression is different from the language of expression we use in programs, e.g. in programs we don’t have division, max and min, but we have mod for a scalar, subtraction, multiplication for a scalar. Why do we use the same notation? I think it would be better to differentiate it, and also to give explicitly the semantics, since there are non-trivial cases. E.g. is 0*infinity =0 or infinity?}
\highlight{
\begin{defn}[Symbolic Expression Semantics]
\label{def:symbolic_expr_semantics}
  The semantics function for symbolic expressions, $\sceconfig{}: \scexpr(c) \to (\tdom \to \mathbb{N} \cup \{\infty \})$ evaluates a \emph{symbolic expression}
  $e \in \scexpr(c)$ over a trace $\trace \in \tdom$ using the usual operator semantics extend with $\infty$ as follows,
  \[
    \begin{array}{llll}
    \sceconfig{\sexpr + \infty} (\trace) = \infty
   &
   \sceconfig{\sexpr - \infty} (\trace) = 0
   &
   \sceconfig{\infty - \sexpr} (\trace) = \infty
    &
   \sceconfig{\sexpr / \infty} (\trace) = 0
   \\
   \sceconfig{0 \times \infty} (\trace) = 0
   &
   \sceconfig{\sexpr \times \infty} (\trace) = \infty
   &
   \sceconfig{\emin(\sexpr, \infty)} (\trace) = \sexpr
   &
   \sceconfig{\emax(\sexpr, \infty)} (\trace) = \infty
    \end{array}
   \]
\end{defn}
}
In the next definition we will need to refer to the current state and the previous state of a computation. We will use ``primed'' variables  to refer to the current state and ``non-primed'' to refer to their value in the previous state. For example $x'$ will refer to the value of the variable $x$ in the current state, while $x$ will denote its value in the previous state. We will denote by $\vardom'$ the set of primed copies of variables in $\vardom$.
\mg{is this the convention we use?}
\begin{defn}[Difference Constraints]
A difference constraint $d$ is an inequality of
form $x' \leq y + v$ or $x' \leq v$ where $x, y \in \vardom $, $v \in \scvardom$ and $x' \in \vardom'$. The difference constraint
$x' \leq y + v$ (rsp. $x' \leq v$) states that the value of $x$ in the current state is
at most the value of $y$ in the previous state plus the symbolic constant $v$ (rsp. the value of $v$).
We use $DC(\vardom \cup \scvardom)$ to denote the set of all difference constraints over $\vardom \cup \scvardom$.
\end{defn}

\mg{I am confused by why in the next definition we need the symbol $\top$. In the set of boolean expressions we have the expression $\kw{true}$ which is always true. What is the difference between them?}

\begin{defn}[Constraints]
A constraint $dc$
is either a
difference constraint $d \in DC(\vardom \cup \scvardom)$, a boolean expression $\bexpr \in \booldom$
or the symbol $\top$ denoting ``true''. We use $\dcdom^{\top}$ to denote the set of constraints. 
\end{defn}

When the constraint in a transition is a difference constrain, $l \xrightarrow{x' \leq y + v} l'$,
% Then $x'$ 
it denotes that
the value of variable $x$
after executing the command at $l$ is at most
% and the right-hand side describes 
the value of variable $y$ plus $v$ before the execution,
and $l \xrightarrow{x' \leq v} l'$ respectively denotes
% value of variable $x$
% after executing the command at $l$ is 
$x$ has at most
% and the right-hand side describes 
the value of the symbolic constant $v$.
% before the execution.
For example in Figure~\ref{fig:relatedNestedWhileOdd-overview}(b), constraint $i' \leq i - 1$ on the edge $7 \xrightarrow{i' \leq i - 1} 1$
describes the execution of
 the command at line $7$, 
$\clabel{\assign{i}{i - 1}}^{7}$. 

% For every expression in each of the label commands, it is computed in three steps via the program abstraction method adopted from Section~6 in~\cite{SinnZV17}. 

The boolean constraint, $\bexpr$ on an edge $l \xrightarrow{b} l'$ describes
that after evaluating the guard with label $l$,
$\bexpr$ holds and the command with label $l$ will execute right after.
In Figure~\ref{fig:relatedNestedWhileOdd-overview}, $i \leq 0 $ on the edge $1 \xrightarrow{i \leq 0} \lex$, 
represents the negation of the testing guard $\clabel{i > 0}^1$
in the $\ewhile$ command, and $i \leq 0$ must hold in order to perform this transition from program point $1$ to
the program exit. 
We also have $\top$, which is preserved for $\eskip$ command or the commands that don't interfere with any guard variable.


% For example in Figure~\ref{fig:relatedNestedWhileOdd-overview}(b), the constraint $i' \leq i - 1$ on the edge $7 \xrightarrow{i' \leq i - 1} 1$
% describes after the execution of
%  the command
% $\clabel{\assign{i}{i - 1}}^{7}$, the value of $i$ is at most the previous $i$'s value minus $1$.
% %
% % For every expression in each of the label commands, it is computed in three steps via the program abstraction method adopted from Section~6 in~\cite{SinnZV17}. 
% %
% The boolean constraint
% % $\bexpr$ on an edge $l \xrightarrow{b} l'$ describes
% % that after evaluating the guard with label $l$,
% % $\bexpr$ holds to execute the command with label $l'$.
% % For example 
% $i \leq 0 $ on the edge $1 \xrightarrow{i \leq 0} \lex$ 
% represents the negation of the testing guard $\clabel{i > 0}^1$
% in the $\ewhile$ command, and $i \leq 0$ must hold to perform the transition from location $1$ to
% the exit $\lex$. 
% $\top$ is preserved for the $\eskip$ command or the commands that don't interfere with any couter.

\begin{defn}[Abstract Transition Graph]
 \label{def:abs_cfg}
 The \emph{abstract transition graph} of a program $c$ is $\absG(c) \triangleq (\absV(c), \absE(c))$, where
 $\absV(c) \triangleq \lvar(c)\cup\{\lex\}$
 and 
 % $\absE(c) \triangleq \{(l_1, dc, l_2) | (l_1, dc, l_2) \in \absflow(c)\}$,
 % and
 % .
 each edge $(l, dc, l') \in \absE(c)$ is an abstract transition
between two program points $l, l'$ if and only if
the command with label $l'$ can execute right after the execution of the command with label $l$.
% if and only if there is a control flow between two program points.
The constraint $dc \in \dcdom$ on each edge
describes the abstract execution of the command with label $l$.
\end{defn}
The edge $(0 \xrightarrow{i' \leq n} 1)$ on the top of Figure~\ref{fig:relatedNestedWhileOdd-overview}(b) tells us the command 
$\clabel{\assign{i}{n}}^0$ is executed with a continuation point $1$, and the while loop with header at location $1$, $\ewhile \clabel{i > 0}^1 \edo \{\ldots\}$ will be executed right after.

\begin{defn}[Path]
 \label{def:abs_cfgpath} 
 A path $p$ on a program's abstract transition graph $\absG(c)$ is a sequence, $ l_0 \xrightarrow{dc_0} l_1 \xrightarrow{dc_1} \ldots $ with
 \begin{itemize}
 \item the vertices $(l_0, \ldots)$, where $l_i \in \absV(c)$ for every $i = 0, 1, \ldots$ and
 %
 \item the edges $(e_0, \ldots)$, where $e_i = (l_{i}, dc_i, l_{i + 1}) \in \absE(c)$ for every $i = 0, 1, \ldots$.
 \end{itemize}
  $(e_0, \ldots)$ is the edge sequence and $(l_0, \ldots)$ is the vertices sequence
 of this path, $v_i \in p$ (rsp. $e_i \in p$) represent that $v_i$ (rsp. $e_i$ )is a vertex (rsp. edge) in this sequence respectively.
 A path is cyclic if it has the same start- and end-point. A path is simple if it does not visit a location twice except for the start- and end-location. We use $\paths(\absG(c))$ to denote the set of all the paths on $\absG(c)$,
 and $\pathl(p)$ to represent the list of program points corresponding to the vertices sequence of this path $p \in \paths(\absG(c))$,
 where $\pathl: \paths(\absG(c)) \to \mathcal{P}{(\ldom)}$.
 \end{defn}
 In Figure~\ref{fig:relatedNestedWhileOdd-overview}(b), $1 \to 2 \to 3 \to 4 \to 5 \to 4$ is a \emph{path} but not simple (the program points $4$ is visited twice). The path $4 \to 5 \to 4$ is both cyclic and simple.


\subsection{Program Rewriting}
This step rewrite the program in the form of combinations of simple transition paths.



\paragraph{Loop Header.}
We first collect the loop headers $\loopl(c) \subseteq \lvar(c)$ from a program $c$, which is the set of all program points corresponding to the loop headers in program $c$.
\begin{defn}[Loop Headers ($\loopl : \cdom \to \mathcal{P}(\ldom)$)]
  \label{def:loopl}
  \[
  \loopl(c) \triangleq 
  \left\{
    \begin{array}{ll}
      \{\}  & {c} = \clabel{\assign x e}^{l} \\
      \loopl({c_1}) \cup \loopl({{c_2}})  & {c} = {c_1};{c_2} \\
      \loopl(c_t) \cup \loopl({{c_f}})   & {c} =\eif(\clabel{\bexpr}^{l}, c_t, c_f) \\
  \loopl(c_w) \cup \{l\}, &  {c}   = \ewhile \clabel{\bexpr}^{l} \edo (c_w)
  \end{array}
\right.
\]
  \end{defn}

  \paragraph{The Simple Transition Path.}
  \begin{defn}[Simple Tansition Path]
  \label{def:tpath}
A \emph{simple transition path}
$\tpath \in \paths(\absG(c))$ for the program $c$, is either a simple cyclic path, which has the same start- and end-point
or a simple path has either different while loop headers, the program entrance or exit as its start- and end-point
without visiting any loop header inside the path.
\\
Specifically, a path $l_0 \xrightarrow{dc_0} l_1 \xrightarrow{dc_1} \ldots l_n \in \paths(\absG(c))$ with the
vertices sequence $(l_0, \ldots, l_n)$, where $l_i \in \absV(c)$ for every $i = 0, \ldots, n$ and
%
the edge sequence $(e_1, \ldots, e_n)$, where $e_i = (l_{i - 1}, dc_i, l_{i}) \in \absE(c)$ for every $i = 1, \ldots, n$,
%
is a \emph{simple transition path} if and only if it satisfies,
\begin{itemize}
  \item $l_i \neq l_j$ for every $i = 0, \ldots, n$ and $j = 0, \ldots, {n - 1}$,
  \item $l_0$ is either the program point of a loop header or the program entrance ($l_0 = 0$),
  i.e., $l_0 \in \loopl(c) \cup \{ 0 \}$
  \item and $l_n$ is either the program point of a loop header or the program exit ($l_n = \lex$),
  i.e., $l_0 \in \loopl(c) \cup \{ \lex \}$,
  \item and $l_i \notin \loopl(c) \cup \{ 0, \lex \}$ for every $i = 1, \ldots, n-1$.
\end{itemize}
\end{defn}
A simple transition path is also an execution path.

\paragraph{Syntax.}
We define the syntax of abstract program and the transition relation based on graph paths as below. 
The syntax of the refined program is combination of the execution paths on this program and the logic transition relation of is a logic formula over constraints.
\[
  \begin{array}{rll}
   \mbox{Abstract Program} & \rprog &:= \tpath ~|~ \rprepeat(\rprog) ~|~ l : \rprepeat(\rprog) ~|~ \rprog;\rprog ~|~ \rpchoose{\rprog, \ldots} \\
   \mbox{Transition Relation} & \phi & := dc ~|~ \phi \land \phi ~|~ \phi \lor \phi ~|~ (\phi)^* ~|~ \exists v \in \mathbb{Z} \st \phi
  \end{array}
\]
where $l\in \loopl(c)$ is a loop header.

In the abstract program syntax, $\rprepeat(\rprog)$ is a loop statement iterating over the transitions in $\rprog$.
$l: \rprepeat(\rprog)$ represents that the loop $ \rprepeat(\rprog)$
corresponds to a loop of the source program with loop header $l$.
The multiple-paths statement $\rpchoose{\rprog, \ldots} $ contains all the execution paths of a loop from the source program.

The transition relation can be either a constraint, conjunctions or disjunctions of itself.
$(\phi)^*$ corresponds to a loop path in the program and denotes unknown times of compositions of
$\phi$, $\exists i \in \mathbb{N} \st \underbrace{\phi \lcompose \phi \lcompose \ldots}_{i} $.
Composition operation $\lcompose$ of two transition relations is computed as below.
\[
  \phi_1 \lcompose \phi_2 = \phi_1 \land \phi_2[\varvector{x''}/\varvector{x'}][\varvector{x'}/{\varvector{x}}]
\]
It can also be a formula quantified by existential of set of symbols $v \in \scvardom$.
% \begin{defn}
%   \label{def:lcompose}
% \end{defn}
% Program represented in the above syntax
% \begin{defn}[Transition Relation]
%   The transition relation of is a logic formula over constraints.
%   \[
%     \begin{array}{rll}
%      \mbox{Transition Relation} & \phi & := dc ~|~ \phi \land \phi ~|~ \phi \lor \phi ~|~ (\phi)^* ~|~ \exists x \in \vardom \st \phi
%     \end{array}
%   \]
% \end{defn}
The derivation relations between an abstract program and the transition relations are in Figure~\ref{fig:transition-relation-rule}.
In the rule \rname{base}, a simple transition path is translated into conjunctions of constraints on this path.
Transition relation from sequence statements is conjunction of two transition relations respectively and the multiple-path statement $\rpchoose{\rprog, \ldots} $ is translated into disjunctions of 
transition relation of each path.

\begin{figure}
  \begin{mathpar}
    %
    \inferrule
    {
    \empty
    }
    {
      \tpath \models \bigwedge \left\{ dc \vert (l, dc, l') \in \tpath \right\}
    }
    ~\rname{base}
    \and
    %
    \inferrule
    {
    \rprog_1 \models \phi_1
     \and 
     \rprog_2 \models \phi_2
    }
    {
      \rprog_1; \rprog_2 \models \phi_1 \lcompose \phi_2
    }
    ~\rname{seq}
    %
    %
    \and
    %
    \inferrule
    {
      \rprog \models \phi 
    }
    {
      \rprepeat(\rprog) \models (\phi)^*
    }
    ~\rname{loop}
    \and
    %
    %
    \inferrule
    {
    \rprog \models \phi 
    }
    {
      l:\rprepeat(\rprog) \models 
      \bigwedge\limits_{v \in \left\{ v \vert (l, x \leq y' + v, l') \in \phi  \land v \notin \mathbb{N} \cup \{\infty\}\right\}} \exists {v} \in \mathbb{Z} \st (\phi)^*
    }
    ~\rname{loop-l}
    %
    \and
    %
    \inferrule
    {
      \rprog_1 \models \phi_1 \ldots
    }
    {
      \rpchoose{\rprog_1, \ldots} \models \lor \left\{ \phi_1 \ldots \right\}
    }
    ~\rname{choose}
    %
    %
    %
    %
    \end{mathpar}
    \caption{Transition Relation Generation}
    \label{fig:transition-relation-rule}
\end{figure}


\paragraph{Algorithm.}
Algorithm~\ref{alg:alg-refine_rewrite} transforms the syntax of the while language program 
into the new syntax defined above.
% following~\cite{GulwaniJK09} and preserves the semantics.

In this algorithm, we first use a simple depth-first search strategy to collect all the \emph{simple transition path}s satisfying the Definition~\ref{def:tpath}. 
% It guarantees that every $\tpath$ is equivalent to a path $\rho$ in Definition~4.1 of \cite{GulwaniJK09}.
In line:2, we initialize each candidate $c_i$ with a \emph{simple transition path} $\tpath_i$. 
New candidates generated in line:4, 5, and 6 correspond to the loop statement $\rprepeat(c_i)$, multiple-paths statement $\rpchoose{\ldots}$ and sequence statements $c_i; c_j$ respectively.

\begin{algorithm}
 \caption{Program Rewriting Algorithm. $\kw{Rewrite}(c)$}
 \label{alg:alg-refine_rewrite}
 \begin{algorithmic}[1]
 \REQUIRE program $c$
%  collects all $c$'s \emph{simple transition path}s from $\absG(c)$, $\tpath_1, \ldots, \tpath_n \in \paths(\absG(c))$.
 \STATE \textbf{init}: 
 Set of all \emph{simple transition path}s, 
 $\mathcal{P} = \{ \tpath_1, \ldots, \tpath_n \}$.
 \\
 The candidate set $W = \{c_1, \ldots, c_n\}$, where $c_i = \tpath_i$ for $i = 1, \ldots, n$
 \STATE \textbf{while} $W.size()> 1$:
 \STATE
 \quad create $c' = \rprepeat(c)$ s.t. $c_i \in W \land c[0] = c[-1] \land c[0] \in \loopl(c)$
 \\ \quad $W.add(c[0]: c')$, \qquad $W.remove(c)$
 \STATE \quad create $c' = \rpchoose{c_1, \ldots, c_m}$ 
 s.t. $c_i, c_j \in W \land c_i[0] = c_j[0] = c_i[-1] = c_j[-1]$, $i, j = 1, \ldots, m$.
 \\ \quad $W.add(c')$ \qquad $W.remove(c_1, \ldots, c_m)$
 \STATE \quad create $c' = c_1; c_2$ s.t. $c_1, c_2 \in W \land c_1[-1] = c_2[0]$
 \\
 \quad $W.add(c')$ \qquad $W.remove(c_1, c_2)$
 \STATE \textbf{Endwhile}
 \\ $c^T = W[0]$
 \RETURN $c^T$.
\end{algorithmic}
\end{algorithm}
%
%  in paper~\cite{GulwaniJK09} respectively.
\begin{itemize}
\item
Line:4 find the candidate $c'$ that has the same start- and end-point.
Each candidate corresponds to a loop path, and we create for this candidate a loop statement
$\rprepeat(c')$.

\item
 Line:5 
 finds all the candidates $c_1, \ldots, c_m$ that start and end at the same point.
 These candidates are multiple-paths of a same loop, 
 we create for these candidates a multiple-path loop statement,  $\rpchoose{c_1, \ldots, c_m}$.
\item
Line:6 finds all pairs of candidates $c_1, c_2$ such that  $c_2$ starts with the point where $c_1$ ends.
%  label, rewrite them 
For each pair, we create a sequence statement
 $c_1; c_2$.
\end{itemize}
