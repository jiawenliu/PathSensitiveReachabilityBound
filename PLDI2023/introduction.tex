Skeleton and Plan: Background and Importance of analyzing the reachability bound.
\\
\textbf{Where it is required? How useful it can be? What impact it can bring?}
\\
\textbf{In one area: area-I}
"One of the fundamental questions that needs to be answered for computing such resource bounds is: 
How many times is a given control location inside the program that consumes these resources executed?"
\\
\textbf{In some other areas, II, III or ...}
How many times is a given control location inside the program that performs certain operations executed?
\\
\textbf{Short Summary} of existing works and limitations
\\
\textbf{Short Summary} of the new technique/algorithm: major steps/technique used,  major outcome
\\
\textbf{Introduce} each step of the new technique/algorithm:
\\
\textbf{Introduce} new technique/ results or experimental results. 
Summary of comparison with existing works. \cite{GulwaniJK09} \cite{Sumit2010rechability}, \cite{sinn2017complexity}
\\
\paragraph{Contributions}
\begin{itemize}
  \item New path-sensitive reachability-bound analysis algorithm, 
  solve the \emph{Reachability-Bound Program} more precise than existing methods.
  \item More efficient bound computation method than existing methods.
  \item Efficient implementation and good evaluation results.
\end{itemize}
Outlines