%
A trace $\trace \in \tdom $ is a list of events, 
collecting the events generated along the program execution. 
\[
\begin{array}{llll}
\mbox{Trace} & \trace
& ::= & [] ~|~ \trace :: \event
\end{array}
\]
A trace can be regarded as the program history, 
which records all the evaluations for assignment commands and guards in $\eif$ and $\ewhile$ command.
\\
\highlight{If a program doesn't terminate when executing under some initial trace, it produces an infinite trace $\trace \in \tdom^{\infty}$.
$\tdom^{\infty}$ is the set of all finite or infinite traces.}
\\
Below are some useful operators on trace.
\highlight{
\begin{defn}[Trace Concatenation, $\tracecat: \tdominf \to \tdom \to \tdominf $]
Given two traces $\trace_1 \in \tdominf, \trace_2 \in \tdom$, the trace concatenation operator 
$\tracecat$ is defined as:
\[
\trace_1 \tracecat \trace_2 \triangleq
\left\{
\begin{array}{ll} 
  \trace_1 & \trace_2 = [] \\
  (\trace_1  \tracecat \trace_2'):: \event & \trace_2 = \trace_2' :: \event
\end{array}
\right.
\]
\end{defn}
}
\begin{defn}(An Event Belongs to A Trace)
  An event $\event \in \eventset$ belongs to a trace $\trace$, i.e., $\event \in \trace$ are defined as follows:
%
\begin{equation*}
  \event \in \trace  
  \triangleq \left\{
  \begin{array}{ll} 
    \etrue                  & \trace =  \trace' :: \event'
     \land \event = \event'
                              \\
    \event \in \trace' & \trace =  \trace' :: \event'
    \land \event \neq \event' \\ 
    \efalse                 & \trace = []
  \end{array}
  \right.
\end{equation*}
As usual, we denote by $\event \notin \trace$ that the event $\event$ doesn't belong to the trace $\trace$.
\end{defn}
%
The counting operator $\vcounter : \tdominf \to \ldom \to (\mathbb{N} \cup \{\infty\})$
counts the appearance of a label in a trace.
\[
\begin{array}{lll}
\vcounter(\trace :: (x, l, v), l ) \triangleq \vcounter(\trace, l) + 1
&
\vcounter(\trace  ::(b, l, v), l) \triangleq \vcounter(\trace, l) + 1
&
\vcounter({[]}, l) \triangleq 0
\\
\vcounter(\trace  :: (x, l', v), l) \triangleq \vcounter(\trace, l), l' \neq l
&
\vcounter(\trace  :: (b, l', v), l) \triangleq \vcounter(\trace, l), l' \neq l
\end{array}
\]
\highlight{When the input trace is infinite, it returns $\infty$.}
The counter operator is abused as follows when the input is a sequence of labels $L = (l_1, \cdots, l_n)$:
\[
  \vcounter(\trace :: (\_, l_1, \_) :: \cdots :: (\_, l_n, \_), L ) 
  \triangleq \vcounter(\trace, L) + 1
  \qquad 
  \vcounter(\trace :: (\_, l, \_), L ) 
  \triangleq \vcounter(\trace, L) ~ l \neq l_n
  \]
%
% The Latest Label $\llabel : \tdom \to \mathcal{VAR} \to \mathbb{N}$ 
% The label of the latest assignment event which assigns value to variable $x$.
% \[
%   \begin{array}{lll}
% \llabel((x, l, v):: \trace) x \triangleq l
% &
% \llabel((b, l, v)):: \trace x \triangleq \llabel(\trace) x
% &
% \llabel((x, l, \qval, v):: \trace) x \triangleq l
% \\
% \llabel((y, l, v):: \trace) x \triangleq \llabel(\trace ) x
% &
% \llabel((y, l, \qval, v):: \trace) x \triangleq \llabel(\trace ) x
% \\
% \llabel({[]}) x \triangleq \bot
% &&
% \end{array}
% \]
%
% \todo{wording}
% \mg{This wording needs to be fixed. Also notice that the type is wrong, a label is not always returned.}
%  The Latest Label $\llabel : \tdom \to \mathcal{VAR} \to \mathbb{N}$ 
% The label of the latest assignment event which assigns value to variable $x$.
% \[
%   \begin{array}{lll}
% \llabel(\trace  \tracecat [(x, l, v)]) x \triangleq l
% &
% \llabel(\trace  \tracecat [(b, l, v)]) x \triangleq \llabel(\trace) x
% &
% \llabel(\trace  \tracecat [(x, l, \qval, v)]) x \triangleq l
% \\
% \llabel(\trace  \tracecat [(y, l, v)]) x \triangleq \llabel(\trace ) x
% &
% \llabel(\trace  \tracecat [(y, l, \qval, v)]) x \triangleq \llabel(\trace ) x
% \\
% \llabel({[]}) x \triangleq \bot
% &&
% \end{array}
% \]
% We introduce an operator $\llabel : \tdom \to \mathcal{VAR} \to \ldom \cup \{\bot\}$, which 
% takes a trace and a variable and returns the label of the latest assignment event which assigns value to that variable.
% Its behavior is defined as follows,
% % \begin{defn}[Latest Label]
%   \[
%     % \begin{array}{lll}
%   \llabel(\trace  :: (x, l, \_, \_)) x \triangleq l
%   ~~~
%   \llabel(\trace  :: (y, l, \_, \_)) x \triangleq \llabel(\trace ) x, y \neq x
%   % &
%   ~~~
%   \llabel(\trace :: (b, l, v, \bullet)) x \triangleq \llabel(\trace) x
%   % &
%   % \\
%   % \llabel(\trace  :: (y, l, v, \bullet)) x \triangleq \llabel(\trace ) x
%   % &
%   % \llabel(\trace :: (y, l, v, \qval)) x \triangleq \llabel(\trace ) x
%   % &
%   ~~~
%   \llabel({[]}) x \triangleq \bot
%   % \end{array}
%   \]
% \end{defn}
%
% \mg{This wording needs to be fixed but also the description does not make sense. This operator seems to just collect all the labels in a trace. Again, this definition would be shorter with a more uniform definition of events.}
% The Trace Label Set $\tlabel : \tdom \to \mathcal{P}{(\mathbb{N})}$ 
% The label of the latest assignment event which assigns value to variable $x$.
% \[
%   \begin{array}{llll}
% \tlabel_{(\trace  \tracecat [(x, l, v)])} \triangleq \{l\} \cup \tlabel_{(\trace )}
% &
% \tlabel_{(\trace  \tracecat [(b, l, v)])} \triangleq \{l\} \cup \tlabel_{(\trace)}
% &
% \tlabel_{(\trace  \tracecat [(x, l, \qval, v)])} \triangleq \{l\} \cup \tlabel_{(\trace)}
% &
% \tlabel_{[]} \triangleq \{\}
% \end{array}
% \]
% \begin{defn}
  The operator $\tlabel : \tdom \to \mathcal{P}{(\ldom)}$ gives the set of labels in every event belonging to 
  a trace,
  % , whoes behavior is 
defined as follows,
\[
  % \begin{array}{llll}
\tlabel{(\trace  :: (\_, l, \_)])} \triangleq \{l\} \cup \tlabel{(\trace )}
~~~
\tlabel({[ ]}) \triangleq \{\}
% \end{array}
\]
% \end{defn}%
% Given a trace $\trace$, its -processed trace $\trace$ is computed by a function $p : \trace \to \trace$ as follows:
% \[
%   \trace \triangleq
%   \left\{
%   \begin{array}{ll} 
%   p(\trace' \tracecate (x, l, v)) & = p(\trace') \tracecate (x, l, \vcounter(\trace') l + 1, v) \\
%   p(\trace' \tracecate (b, l, v)) & = p(\trace') \tracecate (b, l, \vcounter(\trace') l + 1, v) \\
%   p(\trace' \tracecate (x, l, \qval, v)) & = p(\trace') \tracecate (x, l, \vcounter(\trace') l + 1, \qval, v) \\
%   p([]) & = []
%   \end{array}
%   \right.
% \]
%
%
% $\tdom$ : Set of Well-formed Traces (in Definition~\ref{def:wf_trace})
%
%
% \\
%

%

% \begin{proof}
%   Taking arbitrary trace $\trace \in \tdom$, by induction on program $c$, we have the following cases:
%   \caseL{$c = [\assign{x}{\expr}]^{l}$}
%   By the evaluation rule $\rname{assn}$, we have
%   $
%   {
%   \config{[\assign{{x}}{\aexpr}]^{l},  \trace } 
%   \xrightarrow{} 
%   \config{\eskip, \trace :: ({x}, l, v, \bullet)}
%   }$, for some $v \in \mathbb{N}$.
%   \\
%   Picking $\trace' = \trace ::({x}, l, v, \bullet)$ and $\trace'' =  [({x}, l, v, \bullet) ]$,
%   it is obvious that $\trace \tracecat \trace'' = \trace'$.
%   % \\
%   % There are 2 cases, where $l' = l$ and $l' \neq l$.
%   % \\
%   % In case of $l' \neq l$, we know $\event \not\eventin \trace$, then this Lemma is vacuously true.
%   %   \\
%   %   In case of $l' = l$, by the abstract Execution Trace computation, we know 
%   %   $\absflow(c) = \absflow'([x := \expr]^{l}; \clabel{\eskip}^{l_e}) = \{(l, \absexpr(\expr), l_e)\}$  
%   %   \\
%   % Then we have $\absevent = (l, \absexpr(\expr), l_e) $ and $\absevent \in \absflow(c)$.
%   \\
%   This case is proved.
%   \caseL{$c = [\assign{x}{\query(\qexpr)}]^{l'}$}
%   This case is proved in the same way as \textbf{case: $c = [\assign{x}{\expr}]^l$}.
%   \caseL{$\ewhile [b]^{l_w} \edo c$}
%   By the first rule applied to $c$, there are two cases:
%   \subcaseL{$\textbf{while-t}$}
%   If the first rule applied to is $\rname{while-t}$, we have
%   \\
%   $\config{{\ewhile [b]^{l_w} \edo c_w, \trace}}
%     \xrightarrow{} 
%     \config{{
%     c_w; \ewhile [b]^{l_w} \edo c_w,
%     \trace :: (b, l_w, \etrue, \bullet)}}~ (1)
%   $.
%   \\
%   Let $\trace_w' \in \tdom$ be the trace satisfying following execution:
%   \\
%   $
%   \config{{
%   c_w,
%   \trace :: (b, l_w, \etrue, \bullet)}}
%   \xrightarrow{*} 
%   \config{{
%   \eskip, \trace_w'}}
% $
% \\
% By induction hypothesis on sub program $c_w$ with starting trace 
% $\trace :: (b, l_w, \etrue, \bullet)$ and ending trace $\trace_w'$, 
% we know there exist
% $\trace_w \in \tdom$ such that $\trace_w' = \trace :: (b, l_w, \etrue, \bullet) \tracecat \trace_w$.
% \\
% Then we have the following execution continued from $(1)$:
% \\
% $
% \config{{\ewhile [b]^{l_w} \edo c_w, \trace}}
%     \xrightarrow{} 
%     \config{{
%     c_w; \ewhile [b]^{l_w} \edo c_w,
%     \trace :: (b, l_w, \etrue, \bullet)}}
%     \xrightarrow{*} 
%     \config{\ewhile [b]^{l_w} \edo c_w, \trace :: (b, l_w, \etrue, \bullet) \tracecat \trace_w}
%     ~(2)
% $
% By repeating the execution (1) and (2) until the program is evaluated into $\eskip$,
% with trace $\trace_w^{i'} $ for $i = 1, \cdots, n n \geq 1$ in each iteration, we know 
% in the $i-th$ iteration,
%  there exists  $\trace_w^i \in \tdom$ such that  
% $\trace_w^{i'} = \trace_w^{(i-1)'} :: (b, l_w, \etrue, \bullet) ++ \trace_w^{i'}$
% \\
% Then we have the following execution:
% \\
% $
% \config{{\ewhile [b]^{l_w} \edo c_w, \trace}}
%     \xrightarrow{} 
%     \config{{
%     c_w; \ewhile [b]^{l_w} \edo c_w,
%     \trace :: (b, l_w, \etrue, \bullet)}}
%     \xrightarrow{*} 
%     \config{\ewhile [b]^{l_w} \edo c_w, \trace_w^{n'}}
%     \xrightarrow{}^\rname{{while-f}}
%     \config{\eskip, \trace_w^{n'}:: (b, l_w, \efalse, \bullet)}
% $ and $\trace_w^{n'} = \trace :: (b, l_w, \etrue, \bullet) \tracecat \trace_w^{1} :: \cdots :: (b, l_w, \etrue, \bullet) \tracecat \trace_w^{n} $.
% \\
% Picking $\trace' = \trace_w^{n'} :: (b, l_w, \efalse, \bullet)$ and $\trace'' = [(b, l_w, \etrue, \bullet)] \tracecat \trace_w^{1} :: \cdots :: (b, l_w, \etrue, \bullet) \tracecat \trace_w^{n}$,
% we have 
% $\trace ++ \trace'' = \trace'$.
% \\
% This case is proved.
%   \subcaseL{$\textbf{while-f}$}
%   If the first rule applied to $c$ is $\rname{while-f}$, we have
%   \\
%   $
%   {
%     \config{{\ewhile [b]^{l_w} \edo c_w, \trace}}
%     \xrightarrow{}^\rname{while-f}
%     \config{{
%     \eskip,
%     \trace :: (b, l_w, \efalse, \bullet)}}
%   }$,
%   In this case, picking $\trace' = \trace ::(b, l_w, \efalse, \bullet)$ and $\trace'' =  [(b, l_w, \efalse, \bullet) ]$,
%   it is obvious that $\trace \tracecat \trace'' = \trace'$.
%   \\
%   This case is proved.
%   \caseL{$\eif([b]^l, c_t, c_f)$}
%   This case is proved in the same way as \textbf{case: $c = \ewhile [b]^{l} \edo c$}.
%   \caseL{$c = c_{s1};c_{s2}$}
%  By the induction hypothesis on $c_{s1}$ and $c_{s2}$ separately,
%  we have this case proved.
% \end{proof}
%
% \todo{more explanation}
% \mg{This corollary needs some explanation. In particular, we should stress that $\event$ and $\event'$ may differ in the query value.}
% Since the equivalence over two events is defined over the query value equivalence, 
% when there is an event belonging to a trace, 
% if this event is a query assignment event, 
% it is possible that 
% the event showing up in this trace has a different form of query value, 
% but they are equivalent by Definition~\ref{def:query_equal}.
% So we have the following Corollary~\ref{coro:aqintrace} with proof in Appendix~\ref{apdx:lemma_sec123}.
% \begin{coro}
% \label{coro:aqintrace}
% For every event and a trace $\trace \in \tdom$,
% if $\event \in \trace$, 
% then there exist another event $\event' \in \eventset$ and traces $\trace_1, \trace_2 \in \tdom$
% such that $\trace_1 \tracecat [\event'] \tracecat \trace_2 = \trace $
% with 
% $\event$ and $\event'$ equivalent but may differ in their query value.
% \[
%   \forall \event \in \eventset, \trace \in \tdom \st
% \event \in \trace \implies \exists \trace_1, \trace_2 \in \tdom, 
% \event' \in \eventset \st (\event \in \event') \land \trace_1 \tracecat [\event'] \tracecat \trace_2 = \trace  
% \]
% \end{coro}
% \begin{subproof}
% Proof in File: {\tt ``coro\_aqintrace.tex''}
% % \input{coro_aqintrace}
% %
% \end{subproof}
% \\
% %
% %
% \mg{I'll skip the rest of this page because it seems not relevant.}
% \todo{Not Necessary but keep it for now}
% \\
% Given a trace $\trace$, its post-processed trace $\trace_p$ is computed by a function $p : \trace \to \trace_p$ as follows:
% \[
%   p(\trace) \triangleq
%   \left\{
%   \begin{array}{ll} 
%   p(\trace' \tracecat [(x, l, v)]) & = p(\trace') \tracecat [(x, l, \vcounter(\trace') l + 1, v)] \\
%   p(\trace' \tracecat [(b, l, v)]) & = p(\trace') \tracecat [(b, l, \vcounter(\trace') l + 1, v)] \\
%   p(\trace' \tracecat [(x, l, \qval, v)]) & = p(\trace') \tracecat [(x, l, \vcounter(\trace') l + 1, \qval, v)] \\
%   p({[]}) & = []
%   \end{array}
%   \right.
% \]
% \\
% \begin{defn}[Well-formed Post-Processed Trace $\mathcal{T_p}$]
% \label{def:wf_trace}
% A post-processed trace $\trace_p$ is well formed, i.e., $\trace_p \in \mathcal{T_p}$ if and only if it preserves the following two properties:
% \begin{itemize}
% \item{\emph{(Uniqueness)}} 
% $\forall \event_1, \event_2 \eventin \trace_p \st \event_1 \signeq \event_2$
% %
% \item{\emph{(Ordering)}} $\forall \event_1, \event_2 \eventin \trace_p \st 
% (\event_1 \eventlt \event_2) \Longleftrightarrow
% \exists \trace_1, \trace_2, \trace_3 \in \mathbb{T_p},
%  \event_1', \event_2' \in \eventset \st
% (\event_1 \eventeq \event_1') \land (\event_2 \eventeq \event_2')
% \land \trace_1 \tracecat [\event_1'] \tracecat \trace_2 \tracecat [\event_2'] \tracecat \trace_3 = \trace_p$
% \end{itemize}
% \end{defn}
% %
% %
% \begin{thm}[Trace Generated from Operational Semantics after Post Processing is Well-formed $c \vDash \trace$].
% \label{thm:os_wf_trace}
% \\
% \[
% \forall \trace \in \tdom, c \st
% \config{c, \vtrace} \to^{*} \config{\eskip, \trace \tracecat \trace'} \land \trace_p = p(\trace')
% \implies
% \trace_p\in \tdom
% \] 
% % \wqside{ is $p(\tau)$ defined some where? I guess it means post-processing.}
% %
% \end{thm}
% \begin{proof}
% Proof in File: {\tt ``thm\_os\_wf\_trace.tex''}.
% % \input{thm_os_wf_trace}
% \end{proof}
%
%
% \todo{
% \begin{lem}[While Map Remains Unchanged (Invariant)]
% \label{lem:wunchange}
% Given a program $c$ with a starting memory $m$, trace $t$ and while map $w$, s.t.,
% $\config{m, c, t, w} \to^{*} \config{m', \eskip, t', w'}$ and $Labels(c) \cap Keys(w) = \emptyset$, then 
% \[
%   w = w'
% \]
% \end{lem}
% \begin{subproof}[Proof of Lemma~\ref{lem:wunchange}]
% %
% Proof in File: {\tt ``lem\_wunchange.tex''}
% % \input{lem_wunchange}
% %
% \end{subproof}
% }
%
% \todo{
% \begin{lem}[Trace is Written Only]
% \label{lem:twriteonly}
% Given a program $c$ with starting trace $t_1$ and $t_2$,
% for arbitrary starting memory $m$ and while map $w$,
% if there exist evaluations
% $$\config{m, c, t_1, w} \to^{*} \config{m_1', \eskip, t_1', w_1'}$$
% % 
% $$\config{m, c, t_2, w} \to^{*} \config{m_2', \eskip, t_2', w_2'}$$
% %
% then:
% %
% \[
%   m_1' = m_2' \land w_1' = w_2'
% \]
% \end{lem}
% %
% \begin{subproof}[Proof of Lemma~\ref{lem:twriteonly}]
% %
% Proof in File: {\tt ``lem\_twriteonly.tex''}
% % \input{lem_twriteonly}
% \end{subproof}
% }
%
% \todo{
% \begin{lem}[Trace Uniqueness]
% \label{lem:tunique}
% Given a program $c$ with a starting memory $m$, \wq{a while map w,}
% for any starting trace $t_1$ and $t_2$, if there exist evaluations
% $$\config{m, c, t_1, w} \to^{*} \config{m_1', \eskip, t_1', w_1'}$$
% % 
% $$\config{m, c, t_2, w} \to^{*} \config{m_2', \eskip, t_2', w_2'}$$
% %
% then:
% %
% \[
% t_1' - t_1 = t_2' - t_2
% \]
% \end{lem}
% %
% \begin{subproof}[Proof of Lemma~\ref{lem:tunique}]
% %
% Proof in File: {\tt ``lem\_tunique.tex''}
% % \input{lem_tunique}
% \end{subproof}
% }
%

%
%
