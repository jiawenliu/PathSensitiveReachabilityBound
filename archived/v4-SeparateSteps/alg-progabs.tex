This path-sensitive reachability-bound algorithm
is performed on basis of an \emph{Abstract Transition Graph} for the program $c$.
% We first show 
This step shows how to generate the abstract transition graph $\absG(c)$ of a
program $c$ through constructing its vertices and edges.
%  before the introduction of the edge and weight estimation.  
% We discuss the vertices and edge of the
% abstract transition graph for a program $c$, $\absG(c)$.

\subsubsection{Vertices Construction}
\label{sec:abs_prog-vertex}
Every 
vertex corresponds to a program execution point, which is a unique
label of a command in this program.
Specifically,
the vertices of this graph is the set of $c$'s labels with the exit label ${\lex}$, 
\[ 
  \absV(c) = \lvar(c)\cup\{{\lex}\}
  \]
%  corresponding to a label command in the program.

\subsubsection{Edge Construction}
\label{sec:abs_prog-edge}
  The vertices can be easily collected and the key point of the abstract
  transition graph for a program is constructing the edge set, $\absE(c)$ for a program $c$.
  It relies on the control flow analysis and the program abstraction of each command.
  %  and abstract transition (we also call it abstract event).
  To make it easy to understand, it
  is an enriched control flow graph with an annotation on each edge.
  The edge set is constructed by a program abstraction method in three steps.
  \\
  In the first step, \textbf{Constraint Computation} generates the constraint
  for the expression in each program command,
  which is used as the annotation of an edge.
  \\
  In the second step, \textbf{Initial and Final State Computation} generates two sets for each command. 
  The initial state is a set that contains the
  program point where this command \todo{starts} executing, 
  and the final state is a set
  that contains the constraint of this command
  and the continuation program points after the execution of this command.
  \\ 
  In the third step, \textbf{Abstract Event Computation} generates a set of edges for the program.
  Each edge is a pair of initial and finial state.
  % The annotation of each edge is a constraint generated by a program abstraction method.
  % by adopting the program abstraction.
  %  method in Section 6 in~\cite{sinn2017complexity}.
  %  the program's every command.
% It is computed as follows.
  % The edge in the abstract transition graph comes from the abstract execution trace of the program. 
  % The abstract execution trace, an abstract representation of the execution, consists of a set of abstract events. 
%   The edge in the abstract transition graph comes from the program's abstract events set $\absflow(c)$.
%   Each abstract event $(l_1, dc, l_2)$ in this set represents an edge in $\absE(c)$.
%   % Then, every abstract transition in the abstraction execution trace corresponds to an edge in the abstract transition graph. In another word, the edge $(l_1, dc, l_2)$ in the abstract transition graph, represents an abstract transition 
% %  from $l_1$ to $l_2$, with a set of difference constraints $dc$. 
%  Also notice, the difference constraints generated during the abstract transition appears in the edge as annotation.
%
\paragraph{Constraint Computation}
In this step, we first show how to compute the constraints for expressions in a program $c$,
by a program abstraction method adopted from the
algorithm in Section 6 in~\cite{sinn2017complexity}.
\\
Given a program $c$,
every arithmetic expression in an assignment command with label $l$,
or boolean expression in the guard of a $\eif$ or $\ewhile$ command with label $l$
is transformed into a constraint.
\\
This constraint describes the abstract execution of the assignment command with label $l$,
or abstract evaluation of the boolean expression in the guard with label $l$.

\highlight{Notations / Formal Definitions:}
\begin{itemize}
\item Operator: $\absexpr : \mathcal{A} \cup \mathcal{B} \to DC(\mathcal{VAR}  \cup \constdom)\cup \booldom \cup \{\top\}$
%
\item Constraints $\dcdom^{\top}: DC(\mathcal{VAR}  \cup \constdom) \cup \booldom\cup \{\top\}$  contains:
%
\begin{itemize}
\item Difference Constraints $DC(\mathcal{VAR}  \cup \constdom)$ is the set of all the inequality of
form $x' \leq y + v$ where $x \in \mathcal{VAR} $, 
$y \in \mathcal{VAR}$ and $v \in \constdom$.
The \emph{Symbolic Constant} set $\constdom = \mathbb{N} \cup \inpvar \cup{\infty}$
is the set of natural numbers with $\infty$ and input variables.
An inequality $x' \leq y + v$ describes that the value of $x$ in the current state is
at most the value of $y$ in the previous state plus some constant $v$.
$x' \leq y + v$ on an edge describes $l \xrightarrow{x' \leq y + v} l'$ describes
that after evaluating the assignment command with label $l$, the value of $x$ is
at most the value of $y$ before executing this command plus some constant $v$.
%
\item The Boolean Expressions $b$ from the set $\booldom$.
$b$ on an edge $l \xrightarrow{b} l'$ describes
that after evaluating the guard with label $l$,
$b$ holds and the command with label $l$ will execute right after.
%
\item The top constraint, $\top$ denotes always true. It is preserved for $\eskip$ command.
\end{itemize}
\end{itemize}

\highlight{Computation Steps:}
\begin{defn}[Constraint Computation]
  \label{def:constraint_compute}
  For a program $c$, a boolean expression $\bexpr$ in the guard of a $\eif$ or $\ewhile$ command
  or an expression $\expr$ and a variable $x$
  in an assignment command $\assign{x}{\expr}$,
  % or 
  % For a boolean expression $\bexpr$ or an arithmetic expression $\aexpr$ and a variable $x$,
  the constraint $\absexpr(\bexpr, \_)$ or $\absexpr(x - v, x)$ is computed as follows,
  \[
    \begin{array}{ll} 
      \absexpr(x - v, x)  = x' \leq x - v  & x \in \grdvar \land v \in \mathbb{N} \\
      \absexpr(y + v, x)  = x' \leq y + v  & x \in \grdvar \land v \in \mathbb{Z} \land y \in (\grdvar \cup \constdom) \\
      \absexpr(v, x)  = x' \leq v + 0  & x \in \grdvar \land v \in (\grdvar \cup \constdom) \\
      \absexpr(y + v, x)  = x' \leq y + v & \\
      \grdvar = \grdvar \cup \{y\} & x \in \grdvar \land v \in \mathbb{Z} \land y \notin (\grdvar \cup \constdom)  \\
      % \absexpr(\qexpr, x)  = x' \leq 0 + Q_m & x \in \grdvar \land \qexpr \text{ is a query expression}  \\
      \absexpr(\expr, x) = x' \leq \infty  &  x \in \grdvar \land \expr \text{ doesn't have any of the forms as above} \\
      \absexpr(\expr, x) = \etrue  &  x \notin \grdvar \\
      \absexpr(\bexpr, \_) = \bexpr   & \\
      % \absexpr(y + v, x)  = x' \leq y + v & \\
      \grdvar = \grdvar \cup FV(\bexpr) &  x \in \grdvar \land \bexpr \text{ is a boolean expression} \\
    \end{array}
    \]
  \end{defn}
%
  $\grdvar$ is the set of variables used in the guard expression of every while command in the program $c$. 
  In the case 4, if a variable $x$, belonging to the set 
  $\grdvar$ is updated by a variable $y$, which isn't in this set, 
  we add $y$ into the set $\grdvar$ and repeat 
  above procedure  until $\grdvar$ and $\absexpr(\expr, x)$ is stabilized. 
  % \wq{I do not understand this sentence:-(}
  \\
Specifically 
% understanding the intuition, 
we handle a 
% simplified 
normalized expression, $x > 0$
in guards of while loop headers, and 
%  \wq{I do not understand this sentence:-(}
%  .
% \\
% The counter variables only increase, decrease or reset by expression in the form of arithmetic minus and plus (able to extend to max and min.)
the counter variable $x$ only increase, decrease or reset by 
% expression in the form of 
simple arithmetic expression (mainly multiplication, division, minus and plus (able to extend to max and min)). 
The counter variable $x$ is generalized into norm when the boolean expression $x > 0$
in $\ewhile$ doesn't have the form $x > 0$.
The way of normalizing the guards and computing the norms is adopted from the computation step 1 in Section 6.1 in paper \cite{sinn2017complexity}. 
% \\
% For more complex expression assignments, where the counter reset, or calculated from $\elog$, 
% multiplication or division, and expressions involving multiple variables, the constraint is approximated as reset of $\infty$.
% \\
% % This simplification \wq{which part we simplify here?} 
% This approximation strategy
% doesn't affect our analysis results in our examples. It is easy to extend the normalized expression 
% into more complex forms as in \cite{sinn2017complexity}, as well as the 
% counter variable manipulation with more advanced expressions.
% \\ 
% The boolean expression in the guard of $\ewhile$ command is normalized into form of $ x > 0$ where $x^l \in \lvar_c$ for some $l$.
\begin{defn}[Symbolic Expression ($\mathcal{A}_{S}$)]
  $\mathcal{A}_{S}$ is the set of all the symbolic expressions 
over $\constdom$.
% For concise, $\mathcal{A}_{\lin}$ is used as the same meaning of $EXPR(\constdom)$ in the follows, to denote the arithmetic expression 
% over the symbolic variables, (i.e., $\mathbb{N}$ with input variables).
\end{defn}
The symbolic expression set is a subset of arithmetic expressions over $\mathbb{N}$ with input variables, 
i.e., $\mathcal{A}_{S} \subseteq \mathcal{A}_{\lin}$.
% \subsubsection{Abstract Transition Graph through an Example}
\paragraph{Abstract Initial and Final State Computation}
This step computes two sets for each command. 
The initial state is a set that contains the
program points before executing this command, which is computed by the standard initial state generation method from control flow analysis.
The final state is a set
that contains the constraint of this command and the program points after the execution of this command.
This set is enriched 
% program's initial and final states 
from the standard control flow analysis.

%
\highlight{Notations / Formal Definitions:}
\begin{itemize}
\item The abstract initial state: $\absinit(c) \in \ldom$.
%
\item The abstract Final State: $\absfinal(c) \in \mathcal{P}(\ldom \times \dcdom^{\top})$
\end{itemize}

\highlight{Computation Steps:}
\begin{itemize}
  \item The \emph{abstract initial state}, $\absinit(c) \in \mathcal{P}(\ldom)$
  for a command $c$ is the set of the initial program points.
Each point in this set is a unique program label corresponds to the command before executing this command. 
% when executing this program.
\\
Given a program $c$, its abstract initial state, $\absinit(c)$ is computed as follows,
%
\[
  \begin{array}{ll}
    \absinit(\clabel{\assign{x}{\expr}}{}^l)  & = \{l\}  \\
    \absinit(\clabel{\eskip}^{l})  & = l \\
    \absinit(\eif [b]^l \ethen c_1 \eelse c_2)  & = \{l\} \\
    \absinit(\ewhile [b]^l \edo c)  & = \{l\} \\
    \absinit(c_1 ; c_2)  & = \absinit(c_1) \\
 \end{array}
 \]
%
%
\item The \emph{abstract final state} of the program $c$, 
$\absfinal(c) \in \mathcal{P}(\ldom \times \dcdom^{\top})$
is a set of pairs, $(l, dc)$ with a
program point (i.e., a label), $l$ as the first component and a constraint, 
$dc$ as the second component.
% Every pair in $\absfinal(c)$ 
The program point $l$ corresponds to the labeled command after the execution of $c$,
and the constraint $dc$ in this pair is computed by $\absexpr$ for the expression in $c$.
%  in the first step.
\\
Given a program $c$, its final state, $\absfinal(c)$ is computed as follows,
% $\absfinal(c) \in\mathcal{P}(\ldom \times \dcdom^{\top})$,
% computes the set of Abstract Final State for the command. 
 \[
  \begin{array}{ll}
    \absfinal(\clabel{\assign{x}{\expr}}{}^l)  & = \{(l, \absexpr\eapp (\expr, x))\}  \\
    %  \absfinal(\clabel{\assign{x}{\query(\qexpr)}}{}^l)  & = \{
    %   (l, x' \leq 0 + Q_m )\}  \\
     \absfinal(\clabel{\eskip}^{l})  
     & = \{(l, \top)\} \\
     \absfinal(\eif [b]^l \ethen c_1 \eelse c_2)  & = \absfinal(c_1) \cup \absfinal(c_2) \\
     \absfinal(\ewhile [b]^l \edo c)  & = \{(l, \absexpr(\bexpr, \top))\} \\
     \absfinal(c_1 ; c_2)  & =  \absfinal(c_2) \\
 \end{array}
 \]
 %
\end{itemize}
 \paragraph{Abstract Event Computation} Each abstract event is an edge between two vertices in the abstract transition graph.
 It is generated by computing the initial state and finial state interactively and recursively for a program $c$.
 
 \highlight{Notations / Formal Definitions:}
 \begin{itemize}
  \item \emph{Abstract Event}: 
  $\absevent \in $
  $\ldom \times \dcdom^{\top} \times \ldom$
  \item \emph{Abstract Event Computation}: $\absflow \in \cdom \to \mathcal{P}( \ldom \times \dcdom^{\top} \times \ldom )$
 \end{itemize}
 Its type is defined as follows,
 \begin{defn}[Abstract Event]
   \label{def:abs_event}
   Abstract Event: 
   $\absevent \in $
   $\ldom \times \dcdom^{\top} \times \ldom$
   is a 
   % pair of abstract initial state and final state.
   triple where the first and third components are labels,
   second component is a constraint from $\dcdom^{\top}$.
   % the thrid % computed from program's abstract final and initial state, $\absfinal(c)$ and $\absinit(c)$ with formal definition, and algorithm detail in Appendix.
   %  the constraint and the third corresponds to a final state.
   \end{defn}
   In an abstract event $(l, dc, l')$ of a program $c$, 
   the first label $l \in \ldom$ corresponds to an initial state of $c$, and 
   the second label $l' \in \ldom$ with the constraint $dc\dcdom^{\top}$ correspond to an abstract final state of $c$.
  The abstract initial state is a label from $\ldom$.
%  The abstract final state is a pair from $\ldom \times \dcdom^{\top}$,  
%  where first component is a label from $\ldom$ and the second component is a constraint from $\dcdom^{\top}$.
 %
We abuse the notation $\mathcal{P}(\absevent)$ for the power set of all abstract events.

 \highlight{Computation Steps:}
\\
% The abstract event is computed for w.r.t the program in Definition~\ref{def:absevent_compute}, 
%  generated during computing its abstract execution trace, 
%  , and we have $\absflow(c) \in \mathcal{P}(\absevent)$.
%  Now, we  extract the abstract execution trace  $\absflow(c)$ for a program, which computes the 
%  The \emph{Abstract Execution Trace} for program $c$ is a s
The set of the abstract events $\absflow(c)$ for a program $c$
% .
%  Its type is formally defined 
is computed as follows in Definition~\ref{def:absevent_compute}.
 %
 \begin{defn}[Abstract Event Computation]
 \label{def:absevent_compute}
  $\absflow \in \cdom \to \mathcal{P}( \ldom \times \dcdom^{\top} \times \ldom )$
  \end{defn}
 %
%  The \emph{Abstract Execution Trace} for program $c$ is computed as follows.
%  \\
  % We now show how to compute the abstract execution trace. 
 We first append a $\eskip$ command with 
%  a symbolic label $l_e$, i.e., $\clabel{\eskip}^{l_e}$ at the end of the program $c$, and compute the $\absflow(c) = \absflow'(c')$ for $c'$, where $c' = c;\clabel{\eskip}^{l_e}$ as follows,
the label $\lex$, i.e., $\clabel{\eskip}^{{\lex}}$ at the end of the program $c$, and construct 
the program $c' = c;\clabel{\eskip}^{{\lex}}$.
Then, we compute the $\absflow(c) = \absflow'(c')$ for $c'$ as follows,
 %
 {\footnotesize
 \[
   \begin{array}{ll}
      \absflow'(\clabel{\assign{x}{\expr}}{}^l)  & = \emptyset  \\
      \absflow'([\eskip]^{l})  & = \emptyset \\
      \absflow'(\eif [b]^l \ethen c_t \eelse c_f)  & =  \absflow'(c_t) \cup \absflow'(c_f)
        \\ & \quad 
        \cup \{(l, \absexpr(\bexpr, \top),  \absinit(c_t) ) ,  (l, \absexpr(\neg\bexpr, \top), \absinit(c_f)) \} \\
       \absflow'(\ewhile [b]^l \edo c_w)  & =  \absflow'(c_w) \cup \{(l, \absexpr(\bexpr, \top), \absinit(c_w)) \} 
       \\ & \quad 
       \cup \{(l', dc, l)| (l', dc) \in \absfinal(c_w) \} \\
       \absflow'(c_1 ; c_2)  & = \absflow'(c_1) \cup  \absflow'(c_2) 
       \\ & \quad 
       \cup \{ (l, dc, \absinit(c_2)) | (l, dc) \in \absfinal(c_1) \} \\
   \end{array}
   \]
   }
   Notice $\absflow'([x := \expr]^{l})$ and $\absflow'([\eskip]^{l})$ are both empty set. 
   For every event $\event$ with label $l$ in an execution trace $\trace$ of program $c$, 
   there is an abstract event in program's abstract execution trace of form $(l, \_, \_)$.  
   We also show the soundness of the \emph{abstract events computation} in Appendix.

 \highlight{Theorem Guarantee:}
   \begin{lem}[Soundness of the Abstract Events Computation]
     \label{lem:abscfg_sound}
     For every program $c$ and
     an execution trace $\trace \in \tdom$ that is generated w.r.t.
     an initial trace  $\vtrace_0 \in \tdom_0(c)$,
     there is an abstract event $\absevent = (l, \_, \_) \in \absflow(c)$ 
     % with initial label $l$
     for every event $\event \in \trace$ having the label $l$, i.e., $\event = (\_, l, \_)$.
      %
   \[
     \begin{array}{l}
       \forall c \in \cdom, \vtrace_0 \in \tdom_0(c), \trace \in \tdom ,  \event = (\_, l, \_) \in \eventset \st
   \config{{c}, \trace_0} \to^{*} \config{\eskip, \trace_0 \tracecat \vtrace} 
   \land \event \in \trace 
   \\
   \qquad \implies \exists \absevent = (l, \_, \_) \in (\ldom\times \dcdom^{\top} \times \ldom) \st 
   \absevent \in \absflow(c)
   \end{array}
   \]
   \end{lem}
%    This lemma is proved formally in Appendix~\ref{apdx:pathinsensitive_rb_soundness}.
% For every event $\event$ with label $l$ in an execution trace $\trace$ of program $c$, 
% there is an abstract event in program's abstract events computation of form $(l, \_, \_)$. 
This lemma is proved formally in Lemma~\ref{lem:abscfg_sound} in Appendix~\ref{apdx:pathinsensitive_rb_soundness}.
\\
For every program point $l$ corresponding to an assignment command in a program $c$,
%  $x^l \in \lvar_c$, 
there is a unique abstract event in the program's abstract events set $\absevent \in \absflow(c)$ of form $(l, \_, \_)$. 
\begin{lem}[Uniqueness of the Abstract Events Computation]
  \label{lem:abscfg_unique}
  For every program $c$ and
  an execution trace $\trace \in \tdom$ that is generated w.r.t.
  an initial trace  $\vtrace_0 \in \tdom_0(c)$,
  there is a unique abstract event $\absevent = (l, \_, \_) \in \absflow(c)$ 
  % with initial label $l$
  for every assignment event $\event \in \eventset^{\asn}$ in the
  execution trace having the label $l$, i.e., $\event = (\_, l, \_)$ and  $\event \in \trace$.
%
\[
  \begin{array}{l}
    \forall c \in \cdom, \vtrace_0 \in \tdom_0(c), \trace \in \tdom ,  \event = (\_, l, \_) \in \eventset^{\asn} \st
\config{{c}, \trace_0} \to^{*} \config{\eskip, \trace_0 \tracecat \vtrace} 
\land \event \in \trace 
\\
\qquad \implies \exists! \absevent = (l, \_, \_) \in (\ldom\times \dcdom^{\top} \times \ldom) \st 
\absevent \in \absflow(c)
\end{array}
\]
\end{lem}
This lemma and proof is also 
formalized in Lemma~\ref{lem:absevent_unique} in Appendix~\ref{apdx:pathinsensitive_rb_soundness}.

  \paragraph{Edge Construction}
The edge for $c$'s abstract transition graph is constructed simply by computing the program's abstract events set, $\absflow(c)$ as follows,
  \[
    \absE(c) = \{(l_1, dc, l_2) | (l_1, dc, l_2) \in \absflow(c)\}
  \]
  For each edge $(l, dc, l') \in \absE(c)$, $dc$ describes an abstract execution of the assignment command with label $l$,
  of evaluation of the guard with label $l$.
% The edge set is constructed simply by computing the program's abstract events set, $\absflow(c)$.
%   Each abstract event $(l_1, dc, l_2)$ in this set represents an edge in $\absE(c)$.
%   % Then, every abstract transition in the abstraction execution trace corresponds to an edge in the abstract transition graph. In another word, the edge $(l_1, dc, l_2)$ in the abstract transition graph, represents an abstract transition 
% %  from $l_1$ to $l_2$, with a set of difference constraints $dc$. 
%  The constraints generated in the abstract event appears in the edge as annotation.
% We have a pre-processing algorithm to go through the programs and returns the list of labels associating with a loop and whose visiting times need to be analyzed.
%
\subsubsection{Abstract Transition Graph Construction} 
With the vertices $\absV(c)$ and edges $\absE(c)$ ready, we construct the abstract transition graph, formally in
% Through a program $c$'s abstract events computation, its abstract transition graph is computed in 
Definition~\ref{def:abs_cfg}.
%
\begin{defn}[Abstract Transition Graph]
\label{def:abs_cfg}
Given a program $c$, 
its \emph{abstract transition graph} $\absG(c) =(\absV(c), \absE(c))$ is computed as follows,
\\
$\absE(c) = \{(l_1, dc, l_2) | (l_1, dc, l_2) \in \absflow(c)\}$,
\\
$\absV(c) = \lvar(c)\cup\{\lex\}$
% \\
%  $\absW(c) 
% \triangleq \left\{ (l, w) \in \mathbb{L} \times EXPR(\constdom) \right\}$.
\end{defn}
% \\
% Notice we also define the $\absW(c)$ in this graph without giving an actual value.
% This $\absW(c)$ is the set of weight for every 
% % vertex 
% label. 
% The weight $w \in EXPR(\constdom)$ is a symbolic expression over the symbolic constant, 
% which is the estimated upper bound on the number of visiting time for every program point
% through the reachability bound analysis as follows.
%
% $EXPR(\constdom)$ is the set of all the symbolic expressions 
% over $\constdom$, which is a subset of arithmetic expressions over $\mathbb{N}$ with input variables.
% For concise, $\mathcal{A}_{\lin}$ is used as the same meaning of $EXPR(\constdom)$ in the follows, to denote the arithmetic expression 
% over the symbolic variables, (i.e., $\mathbb{N}$ with input variables).
\subsubsection{Abstract Transition Graph through An Example \todo{Change the Walk-through Example}}
\label{sec:abs_prog_example}
% 
% Look at the two-round example again, its generated abstract control is shown as in Figure~\ref{fig:adapfun_tworound}(a).
% In this abstract transition graph, every vertex is a label,
% corresponding to a label command in the program.
% Each directed 
% edge represents an abstract transition 
% between two program points, 
% i.e., the labels of two commands (we call the labels also program point and they refer to the same thing), 
% where the second labeled command will be executed after execution of the command with first label.
% For example, the edge $0, a \leq 0, 1$ on the top, represents,
% from location $0$, the command 
% $\clabel{\assign{a}{0}}^0$ is executed with next continuation location $1$,
% where the 
% command $\clabel{\assign{j}{k}}^1$ will be executed next.
% The constraint $a \leq 0$ is generated by abstracting from the assignment command $\assign{a}{0}$,
% representing that value of $a$ is less than or equals to $0$ after 
% location $0$ before executing command at line $1$.
% %
% The same way for the rest edges' constructions.
%
\begin{example}[The Abstract Control Flow Graph for A Simple While Loop Program]
  \label{ex:whileSim_abscfg}
    For the simple while loop example program, 
its abstract control flow graph is shown as in Figure~\ref{fig:whileSim_abscfg}(b).
For example, the edge $(0, a' \leq 0, 1)$ on the top, tells us the command 
$\clabel{\assign{a}{0}}^0$ is executed with next continuation point $1$,
where the 
command $\clabel{\assign{j}{k}}^1$ will be executed next.
The constraint $a' \leq 0$ is a difference constraint, generated by abstracting from the assignment command $\assign{a}{0}$.
It represents that the value of $a$ is less than or equals to $0$ after 
execution of $\clabel{\assign{a}{0}}^0$ and before executing $\clabel{\assign{j}{k}}^1$.
The difference constraint is an inequality relation, 
the left-hand side of the inequality $x'$ denotes the variable $x$
after executing the command at $l$
and the right-hand side describes the variable $x$ in the state before the execution. 
Look at the $a' < a+x $ on the edge $5$ to $2$, which describes the execution of the command at line $5$, 
which is an assignment $a' = a+x$. The $a'$ on the left side of $a' < a+x$ represents the value of $a$ after the assignment,
while the right-hand side $a$ stores the value before the assignment. 
% $top$ means there is no assignment executed, for example, 
I have 
The boolean constraint $j \leq 0 $ on the edge $2 \to 6$, 
represents the negation of the testing guard $j > 0$
in the $\ewhile$ command with loop header at line $2$.
%
% The same way for the rest edges' constructions.
\begin{figure} 
  \centering
  \begin{subfigure}{.7\textwidth}
  \begin{centering}
  {\small
  $
  \kw{whileSim(k)} \triangleq
    \begin{array}{l}
        \clabel{ \assign{a}{0}}^{0} ;   
              \clabel{\assign{j}{k} }^{1} ;\\
              \ewhile ~ \clabel{j > 0}^{2} ~ \edo ~ 
              \Big(
               \clabel{\assign{x}{j} }^{3}  ;
               \clabel{\assign{j}{j-1}}^{4} ;
              \clabel{\assign{a}{x + a}}^{5}  \Big);\\
              \clabel{\assign{l}{k * a} }^{6}
          \end{array}
  $
  }
  \caption{}
  \end{centering}
  \end{subfigure}
    \begin{subfigure}{.45\textwidth}
    \begin{centering}
  \begin{tikzpicture}[scale=\textwidth/20cm,samples=200]
  \draw[] (-7, 10) circle (0pt) node{{ $0$}};
  \draw[] (0, 10) circle (0pt) node{{ $1$}};
  \draw[] (0, 7) circle (0pt) node{\textbf{$2$}};
  \draw[] (0, 4) circle (0pt) node{{ $3$}};
  \draw[] (0, 1) circle (0pt) node{{ $4$}};
  \draw[] (-7, 1) circle (0pt) node{{ $5$}};
  % Counter Variables
  \draw[] (6, 7) circle (0pt) node {\textbf{$6$}};
  \draw[] (6, 4) circle (0pt) node {{ $\lex$}};
  %
  % Control Flow Edges:
  \draw[ thick, -latex] (-6, 10)  -- node [above] {$a' \leq 0$}(-0.5, 10);
  \draw[ thick, -latex] (0, 9.5)  -- node [left] {$j' \leq k$} (0, 7.5) ;
  \draw[ thick, -latex] (0, 6.5)  -- node [right] {$j > 0$}  (0, 4.5);
  \draw[ thick, -latex] (0, 3.5)  -- node [right] {$x' \leq j$} (0, 1.5) ;
  \draw[ thick, -latex] (-0.5, 1)  -- node [above] {$j' \leq j - 1$} (-6, 1) ;
  \draw[ thick, -latex] (-6, 1.5)  -- node [left] {$a' \leq x + a$} (-0.5, 7)  ;
  \draw[ thick, -latex] (0.5, 7)  -- node [above] {$ j \leq 0 $}  (5.5, 7);
  \draw[ thick, -latex] (6, 6.5)  -- node [right] {$l' \leq k * a$} (6, 4.5) ;
  \end{tikzpicture}
  \caption{}
    \end{centering}
    \end{subfigure}
    \begin{subfigure}{.45\textwidth}
      \begin{centering}
    %   \todo{abstract-cfg for two round}
    \begin{tikzpicture}[scale=\textwidth/20cm,samples=200]
    \draw[] (-10, 10) circle (0pt) node{{ $0: 1$}};
    \draw[] (0, 10) circle (0pt) node{{ $1: 1$}};
    \draw[] (0, 7) circle (0pt) node{\textbf{$2: k$}};
    \draw[] (0, 4) circle (0pt) node{{ $3: k$}};
    \draw[] (0, 1) circle (0pt) node{{ $4: k$}};
    \draw[] (-10, 1) circle (0pt) node{{ $5: k$}};
    % Counter Variables
    \draw[] (6, 7) circle (0pt) node {\textbf{$6: 1$}};
    \draw[] (6, 4) circle (0pt) node {{ $\lex: 1$}};
    %
    % Control Flow Edges:
  \draw[ thick, -latex] (-8, 10)  -- node [above] {$a' \leq 0$}(-1.5, 10);
  \draw[ thick, -latex] (0, 9.5)  -- node [left] {$j' \leq k$} (0, 7.5) ;
  \draw[ thick, -latex] (0, 6.5)  -- node [right] {$j > 0 $}  (0, 4.5);
  \draw[ thick, -latex] (0, 3.5)  -- node [right] {$x' \leq j$} (0, 1.5) ;
  \draw[ thick, -latex] (-1.5, 1)  -- node [above] {$j' \leq j - 1$} (-8, 1) ;
  \draw[ thick, -latex] (-8, 1.5)  -- node [left] {$a' \leq x + a$} (-1.5, 7)  ;
  \draw[ thick, -latex] (1.5, 7)  -- node [above] {$j \leq 0 $}  (4.5, 7);
  \draw[ thick, -latex] (6, 6.5)  -- node [right] {$l' \leq k * a$} (6, 4.5) ;
    \end{tikzpicture}
    \caption{}
      \end{centering}
      \end{subfigure}
    \caption{(a) The Simple While Loop Example Program $\kw{whileSim(k)}$
    (b) The abstract control flow graph for $\kw{whileSim(k)}$  
    (c) The abstract control flow graph with the path insensitive reachability bound for $\kw{whileSim(k)}$.}
    \label{fig:whileSim_abscfg}
  \end{figure}
\end{example}