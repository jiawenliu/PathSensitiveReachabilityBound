\subsection{Constraint Program}
\label{sec:abs_prog}
% \textbf{Step 1: Program Abstract Execution Control Flow Graph}
Given a program $c$, the \emph{constraint program} of $c$ is an \emph{Abstract Transition Graph},
% For a program $c$, this analysis first generates its abstract execution control flow graph notated as follows,
\[\absG(c) =(\absV(c), \absE(c)).\]
%
$\absV(c)$ is the set of program $c$'s control location, each edge in $\absE(c)$ is a transition
between two control locations if and only if there is a control flow between the two locations.
Each edge is annotated by a constraint
from the set $\dcdom^{\top}$.
\subsection{Constraint Program Refinement}
\label{sec:refine}
Three sub-steps:
\\
\textbf{Step1: Simple Transition Path}. $\tpath \in \paths(\absG(c))$.
\\
% For a constraint program $\absG(c)$,
A \emph{simple transition path} of a constraint program is a path in the graph $\absG(c)$, $\tpath \in \paths(\absG(c))$.
It either
\begin{itemize}
  \item contains only one loop (without any nested loop) starting from a loop header at location $l$ and go back to the same $l$;
  \item or doesn't contain any loop, and starts from a loop header $l$ (or the program entrance $l_0$),
and ends with different loop header $l'$ (or the program exist $\lex$).
\end{itemize}
%
\textbf{Step2: Repeat Pattern}.
\\
A \emph{Repeat Pattern} ($\rprog \in \mathcal{P}({\absG(c)})$) is either a simple path or sequence of repeat patterns of this program $c$. 
\[
  \rpattern := \tpath ~|~ \rprepeat(\rpattern) ~|~ \rpattern; \rpattern
\]
Every $\rprog'$ with the annotation $\rprepeat$, (for example, $\rpattern = \rprepeat(\rpattern')$)
can consecutively execute at least twice.
Every two sub-repeat patterns following each other in a $\rprog$ can execute in sequence, for example in $\rpattern = \rpattern_1; \rpattern_2$,
$\rpattern_2$ can execute after $\rpattern_1$.
Every sub-repeat patterns in the sequence are distinct.
% \begin{defn}[Repeat Pattern of A Program]
%   \label{def:repeat-pattern}
%   Given a constraint program $\absG(c)$,
%   $\rpattern(c)$ is a \emph{Repeat Pattern} of $\absG(c)$ if and only if, it is a simple path or
%   a sequence of \emph{repeat pattern}
%   has the following syntax
%   \[
%     \rpattern := \tpath ~|~ \rprepeat(\rpattern) ~|~ \rpattern; \rpattern
%   \] 
%   and satisfying,
%   \begin{itemize}
%   \item every sub-repeat pattern $\rpattern' \in \rpattern(c)$ with the annotation $\rpattern$
%   can consecutively execute twice
%   %  when executing this program $c$ 
%   w.r.t. some initial trace,
%   \item every sub-repeat patter in a sequence is distinct, i.e., $\rpattern(c) = \rpattern_1; \cdots; \rpattern_n$ and 
%   % $\rpattern_i, \rpattern_j \in \rpattern_1; \cdots; \rpattern_n$, 
%   $\rpattern_i \neq \rpattern_j$ for every $i, j = 1, \cdots, n$,
%   \item and every two continuous sub-repeat patters in a sequence can execute after each other w.r.t some initial trace,
%   i.e., $\rpattern(c) = \rpattern_1; \cdots; \rpattern_n$ and for every $i = 1, \cdots, n$,
%   $\rpattern_i$ can execute after $\rpattern_{i - 1}$ under some initial trace.
%   \end{itemize}
% \end{defn}
%
\\
\textbf{Step3: Refined Program}.
\\
A \emph{Refined Program} for a program $c$, ($\rprog \in \mathcal{P}(\mathcal{P}(\absG(c)))$),
is either a repeat pattern, or a set of refined programs, or sequence of refined program
% ($\rprog \in \mathcal{P}({\absG(c)})$) is either a simple path or sequence of repeat patterns. 
\[
  \rprog :=  \rpattern ~|~ \rpchoose{\rprog} ~|~ \rprog; \rprog.
\]
It satisfies,
\begin{itemize}
  \item every sub-refined program in a same set with annotation $\rpchoose{\cdots}$ starts and ends with the same control location,
  \item and for every two sub-refined programs following each other in a sequence,
    the first refined program's ending control location is the same 
    as the starting control location of the second refined program.
    For example if
    $\rprog = \rprog_0; \cdots; \rprog_n$,
    $\rpattern_1 \in \rprog_i$
    and $\rpattern_2 \in \rprog_{i + 1}$,
    then $\rpattern_1$ ends with the same control location as $\rpattern_2$'s starting location.
\end{itemize}
% \\
% \begin{defn}[Refined Program of A Program]
%   Given a constraint program $\absG(c)$,
%   its \emph{Refined Program} is either a repeat pattern, or a set of refined program, or sequence of refined program has
%   the following syntax,
%   \[
%     \rprog :=  \rprepeat ~|~ \rpchoose\left\{\rprog\right\} ~|~ \rprog; \rprog.
%   \]
%   It satisfies that for every sub refined program $\rprog' \in \rprog$
%   % $\rprog = \rprog_1, \cdots; \rprog_n$ and
%   % $\rprog_i = \{\rpattern_{0}; \cdots; \rpattern_{i_n}\}$, $n, i_n \in \mathbb{N}$ and
%   \begin{itemize}
%   \item if
%    $\rprog' = \rpchoose\left\{\rprog_1, \cdots, \rprog_m \right\}$,
%    then all the $\rprog_1, \cdots, \rprog_m$ have the same starting and ending label,
%   \item if $\rprog' = \rprog_1; \rprog_2$, then
%   % and for every $\rpattern_1 \in \rprog_i$ and $\rpattern_2 \in \rprog_{i + 1}$,
%   $\rprog_1$'s ending label is the same label as $\rprog_2$'s starting label.
%   \end{itemize}
% \end{defn}
%
%  annotation for two branches of $\eif$.
% them into
%  of this loop header
%s
% \textbf{Theorem Guarantee.}
% Soundness of the refinement.
\subsection{Ranking / Local Bound Computation}
\label{sec:ranking}
It estimates the bounds on the maximum value of counter variables, and
the iteration times
% of the ranks / local bounds 
for each edge in the constraint program $\absG(c)$, using the method in \cite{sinn2017complexity}.

% \textbf{Step1: Variable Constraint Collection:}
% Identify the abstract events where each variable is increased, decreased and reset:
% \\
% $\inc: \mathcal{VAR} \to \mathcal{P}(\absE(c)) $
% the set of the abstract events where the variable increase.
% \\
% $\inc(x) = \{(e, c) | e = (l, x' \leq x + c, l') \land e \in \absE(c)\}$
% \\
% $\reset: \mathcal{VAR} \to \mathcal{P}(\absE(c)) $
% The set of the abstract events where the variable is reset.
% \\
% $\dec: \mathcal{VAR} \to \mathcal{P}(\absE(c)) $
% The set of abstract events where the variable decrease.
% \\
% \textbf{{Step2: Assign Ranks / Local Bound to Edges}}
% $\locbound: \absE(c) \to \mathcal{VAR} \cup \constdom$.
%  \\
% \textbf{Step3: Ranks / Local Bounds Estimation}
% Estimating the bounds on the (ranks'/ local bounds') maximum value.
% % , the 
% \\ 
% $ \varinvar: \mathcal{VAR} \cup \constdom \to \mathcal{A}_{\lin}$
% % \\
% % $\absclr: \absE(c) \to \mathcal{A}_{\lin}$
% \\
% $Incr(x) \triangleq \sum\limits_{(e, c) \in \inc(x)}\{\absclr(\absevent) \times v\}$
% \\
% It estimates the bounds on the maximum value of counter variables, and
% the iteration times
% % of the ranks / local bounds 
% for each edge in the constraint program $\absG(c)$, using the method in \cite{sinn2017complexity}.
% % \\
% % computing the loop bound in a path-insensitive way as the base step.
% % \\ 
% % $ \varinvar: \mathcal{VAR} \cup \constdom \to \mathcal{A}_{\lin}$
% \\
% $\absclr: \absevent \to \mathcal{A}_{\lin}$
% \\
% \textbf{Theorem Guarantee.}
% Soundness of the Path-Insensitive Local Bound / Rank Estimation.
%  Require Variable Bound Computation.
\subsection{Outside-In Algorithm}
\label{sec:outinalg}
For the refined program $\rprog \in \mathcal{RP}$, the \textbf{Outside-In Algorithm} ($\outinB: \rprog \to \mathcal{A}_{in}$)
computes the \emph{OutIn} bound on the iteration numbers of every repeat pattern in this program.
\\
It is called \emph{OutIn} bound because for a repeat pattern $\rpattern$ nested
in another repeat pattern $\rpattern' = \rprepeat(\rpattern)$,
this value only bounds the iteration number of $\rpattern$ inside $\rpattern'$ by assuming $\rpattern'$ executes once.

% \highlight{\textbf{Notations / Operators:}}
% \\
% The \emph{State}: 
% $\absstate \in \mathcal{P}(\dcdom^{\top})$ : conjunctions of the edge annotations.
% %  constraints.
% \\
% For a refined program ($\rprog$), there are the 
% \\
% \emph{Initial State} ($\rfinit(\rprog)$), 
% \emph{Final State} ($\rffinal(\rprog)$), and \emph{Next State} ($\rfnext(\rprog)$)  $\in \absstate$.
% % The \emph{Initial State}: $\rfinit : \rprog \to \absstate $.
% \\
% The \emph{Variable Grade Decedent}: $\varGD : \rprog \to \mathcal{A}_{in}$, is the set of the variables' variation in one iteration;
% % The \emph{Initial State}: $\rfinit : \rprog \to \absstate $.

\subsection{Inside-Out Algorithm}
\label{sec:inoutalg}
For the refined program $\rprog \in \mathcal{RP}$, the \textbf{Inside-Out Algorithm}
computes the reachability-bound on the execution numbers for every simple transition path $\tpath$ in this program $\rprog$,
through the following steps.
%
\begin{enumerate}
  \item \emph{Repeat Chain Bound:} $\rpchB(l, \tpath, \rprog) \in \mathcal{A}_{in}$.
  \\
  For every transition path $\tpath$ in a refined program $\rprog$,
  its \emph{Repeat Chain Bound} is the
  bound on its execution numbers inside its closet while loop.
  \item \emph{Relative Loop Bound:} $\lpchB(l, \tpath, \rprog) \in \mathcal{A}_{\lin}$.
  \\
  For every simple transition path $\tpath$
and a loop header at location $l$ in a refined program $\rprog$,
the \emph{Relative Loop Bound} $\lpchB(l, \tpath, \rprog) \in \mathcal{A}_{\lin}$ is a symbolic expression in $\mathcal{A}_{\lin}$.
\\
This expression is a sound bound on the iteration numbers of loop $l$ relative to the simple transition path $\tpath$.
This $\tpath$'s closest enclosing loop has the loop header at $l'$ and $l'$ is nested inside the loop $l$.
\\
It estimates the iteration numbers of loop $l$ such that during these iterations, the nested loop $l'$ is executed.
%
\item The \emph{Inside-Out Bound} ($\inoutB(\tpath, \rprog) \in \mathcal{A}_{in}$).
\\
For every simple transition path $\tpath \in \rprog$,
its \emph{Inside-Out Loop Bound}
 $\inoutB(\tpath, \rprog) \in \mathcal{A}_{in}$ is 
% Compute 
the path sensitive reachability-bound on $\tpath$'s execution numbers.

% \highlight{\textbf{Omitted Computations:}}
\begin{defn}[{Inside-Out Loop Bound}]
  \label{def:outin_bound}
  Given a refined program $\rprog$, for every transition path $\tpath \in \rprog$, 
  its \emph{Inside-Out Loop Bound}
  $\inoutB(\tpath, \rprog)$ is 
 % Compute 
\[
  \prod\limits_{L \in \lpch(\tpath, \rprog)} \rpchB(l, \tpath, \rprog).
% lpRB(\rprog_i, \tpath) 
% ~ \middle\vert~ l \in lp\mathcal{C}(\tpath) \right
\]
\end{defn}
% For $chain \in rpchain(\tpath)$:
% where $lpchains(\tpath)$ is set of $lpchains(\tpath)$ containing all the loop chains of $\tpath$.
\end{enumerate}
\subsection{Path Sensitive Reachability Bound Computation}

For every program control location $l \in \lvar(c)$, with $\rprog$ as its refined program,
%  in a program $c$,
its path-sensitive reachability-bound ($\psRB(l, \rprog)$) is a symbolic sound bound on the executing times of $l$.
 \begin{defn}
  \label{def:label_psrb}
Given a program $c$ with its refined program $\rprog \in \mathcal{RP}$
%  with 
% \emph{Global Loop Bound} $\inoutB(\tpath)$
% computed for its every transition path $\tpath \in \rprog$  notated by $\inoutB(\tpath)$,
%  for each of its transition path $\tpath \in \rprog$ 
% with the \emph{Global Loop Bound}
% computed as above, notated by $\inoutB(\tpath)$.
the $\psRB(c, l)$ for every label $l \in \lvar(c)$ is computed as follows,
\\
\[ \psRB(c, l) = \sum\limits_{\tpath \in \rprog \land 
l \in \tpath} \inoutB(\tpath)\]
 \end{defn}