\begin{example}[Nested Loop with Related Iterator]
    \label{ex:threeNestedWhile}
    %
    %
    { \small
  \begin{figure}
  \centering
  \begin{subfigure}{.4\textwidth}
    \begin{centering}
    {\footnotesize
    $
    \begin{array}{l}
        \kw{threeNestedWhile}(k, m, N) \triangleq \\
        \clabel{ \assign{i}{0} }^{0} ; \\
            \ewhile ~ \clabel{i < k}^{1} ~ \edo ~ \\
            \qquad \Big(
             \clabel{\assign{j}{0}}^{2} ;\\
             \qquad \ewhile ~ \clabel{j < m}^{3} ~ \edo ~ \\
             \qquad \qquad \Big(
              \clabel{\assign{j}{j + 1}}^{4};
              \clabel{\assign{w}{i}}^{5};\\
              \qquad \qquad \ewhile ~ \clabel{w < N}^{6} ~ \edo ~
              \Big(
                \clabel{\assign{w}{w + 1}}^{7}
                  \Big); \\
                  \qquad \qquad \clabel{\assign{i}{w}}^{8}
                  \Big); \\
                  \qquad \clabel{\assign{i}{i+1}}^{9}
              \Big)
        \end{array}
    $
    }
    \caption{}
    \end{centering}
    \end{subfigure}
  \begin{subfigure}{.5\textwidth}
    \begin{centering}
  %   \todo{abstract-cfg for two round}
  \begin{tikzpicture}[scale=\textwidth/20cm,samples=200]
  \draw[] (-7, 10) circle (0pt) node{{ $0$}};
  \draw[] (0, 10) circle (0pt) node{{ $1$}};
  \draw[] (6, 10) circle (0pt) node {{$\lex$}};
  \draw[] (0, 7) circle (0pt) node{{$2$}};
  \draw[] (0, 4) circle (0pt) node{{ $3$}};
  \draw[] (-7, 4) circle (0pt) node{{ $9$}};
  \draw[] (0, 1) circle (0pt) node{{ $4$}};
  \draw[] (3, 1) circle (0pt) node{{ $5$}};
  \draw[] (6, 1) circle (0pt) node{{ $6$}};
  \draw[] (9, 1) circle (0pt) node{{ $7$}};
  \draw[] (5, 6) circle (0pt) node{{ $8$}};
  % Counter Variables
  %
  % Control Flow Edges:
  \draw[ thick, -latex] (-6, 10)  -- (-0.5, 10);
  \draw[ thick, -latex] (0, 9.5)  -- (0, 7.5) ;
  \draw[ thick, -latex] (0, 6.5)  --  (0, 4.5) ;
  \draw[ thick, -latex] (0, 3.5)  -- (0, 1.5) ;
  \draw[ thick, -latex] (-0.5, 4)  -- (-6.5, 4) ;
  \draw[ thick, -latex] (-6.5, 4.5)  to  [out=90,in=180]  (-0.5, 9.5);
  \draw[ thick, -latex] (0.5, 10)  --  (5.5, 10);
  \draw[ thick, -latex] (0.5, 1)  --  (2.5, 1);
  \draw[ thick, -latex] (3.5, 1)  --  (5.5, 1);
  \draw[ thick, -latex] (6.5, 1)  --  (8.5, 1);
  \draw[ thick, -latex] (6, 1.5)  to [out=90,in=-60]  (5, 5.5);
  \draw[ thick, -latex] (9, 1.5)  to  [out=120,in=30]  (6, 1.5);
  \draw[ thick, -latex] (5, 6.5)  to  [out=120,in=0]  (0.5, 9.5);
  % \draw[ thick, -latex] (-6, 10)  -- node [above] {$i \leq 0$}(-0.5, 10);
  % \draw[ thick, -latex] (0, 9.5)  -- node [left] {$i < k$} (0, 7.5) ;
  % \draw[ thick, -latex] (0, 6.5)  -- node [left] {$j \leq m$} (0, 4.5) ;
  % \draw[ thick, -latex] (0, 3.5)  -- node [left] {$j > 0$} (0, 1.5) ;
  % \draw[ thick, -latex] (-0.5, 4)  -- node [above] {$j \leq 0$} (-6.5, 4) ;
  % \draw[ thick, -latex] (-6.5, 4.5)  to  [out=90,in=180]  node [left] {$i \leq i + 1$ }(-0.5, 9.5);
  % \draw[ thick, -latex] (0.5, 10)  -- node [above] {$i \leq 0$}  (5.5, 10);
  % \draw[ thick, -latex] (0.5, 1)  -- node [below] {$j \leq j - 1$}  (2.5, 1);
  % \draw[ thick, -latex] (3.5, 1)  -- node [above] {$w \leq i$}  (5.5, 1);
  % \draw[ thick, -latex] (6.5, 1)  -- node [below] {$w < N$}  (8.5, 1);
  % \draw[ thick, -latex] (6, 1.5)  to [out=90,in=-60] node [left] {$w \geq N$}  (5, 5.5);
  % \draw[ thick, -latex] (9, 1.5)  to  [out=120,in=30] node [above] {$w \leq w + 1$}  (6, 1.5);
  % \draw[ thick, -latex] (5, 6.5)  to  [out=120,in=0]  node [right] {$i \leq i - 1$ }(0.5, 9.5);
  \end{tikzpicture}
  \caption{}
    \end{centering}
    \end{subfigure}
  \caption{
  (a) The Example of Nested Loop with Related Iterator
    (b) The Standard Control Flow Graph}
      \label{fig:threeNestedWhile}
  \end{figure}
  }
\end{example}
\newpage
\begin{enumerate}
    % \item  \textbf{The Abstract Execution Control Flow Graph} is generated in Figure~\ref{fig:threeNestedWhile}(b).
    \item \textbf{Rewrite The Program into The Language Model in~\cite{GulwaniJK09}}
    \[
      \begin{array}{l}
        \kw{threeNestedWhile}(k, m, N) \triangleq \\
        \clabel{ \assign{i}{0} }^{0}; \\
        \rprepeat(\rpasum(\clabel{i < k}^{1}); \clabel{\assign{j}{0}}^{2};\\
        \qquad \rprepeat(\rpasum(\clabel{j < m}^{3}); \clabel{\assign{j}{j + 1}}^{4}; \clabel{\assign{w}{i}}^{5};\\
        \qquad \qquad \rprepeat(\rpasum(\clabel{w < N}^{6}); \clabel{\assign{w}{w + 1}}^{7});\\
        \qquad \qquad \rpasum(\clabel{w \geq N}^{6}); \clabel{\assign{i}{w}}^{8});\\
        \qquad \rpasum(\clabel{j \geq m}^{3}); \clabel{\assign{i}{i+1}}^{9});\\
        \rpasum(\clabel{i \geq k}^{1})
        \end{array}
      \]

    \item \textbf{Program Refinement}.
    \\
    The refined program is as follows.
    \[
      \begin{array}{l}
        \kw{threeNestedWhile}(k, m, N) \triangleq \\
        \clabel{ \assign{i}{0} }^{0}; \\
        \rpchoose\Big\{ \eskip, \\
        \qquad \qquad \rprepeat(\rpasum(\clabel{i < k}^{1}); \clabel{\assign{j}{0}}^{2};\\
        \qquad \qquad \qquad \qquad \rpchoose\Big\{ \eskip, 
        \\ \qquad \qquad \qquad \qquad \qquad \qquad 
        \rprepeat(\rpasum(\clabel{j < m}^{3}); \clabel{\assign{j}{j + 1}}^{4}; \clabel{\assign{w}{i}}^{5};
        \\ \qquad \qquad \qquad \qquad \qquad \qquad \qquad \qquad 
        \rpchoose\Big\{ \eskip,  \rprepeat(\rpasum(\clabel{w < N}^{6}); \clabel{\assign{w}{w + 1}}^{7})
        \Big\}; 
        \\ \qquad \qquad \qquad \qquad \qquad \qquad \qquad \qquad 
          \rpasum(\clabel{w \geq N}^{6}); \clabel{\assign{i}{w}}^{8}) \Big\};
          \\ \qquad \qquad \qquad \qquad 
          \rpasum(\clabel{j \geq m}^{3}); \clabel{\assign{i}{i+1}}^{9}) 
        \Big\};\\
        \rpasum(\clabel{i \geq k}^{1})
        \end{array}
      \]
    % \[
    % \tpath_0 ; L_1: \rprepeat(\tpath_1; LOOP2: \rprepeat(\tpath_2; LOOP3 : \rprepeat(\tpath_3); \tpath_4); \tpath_5); \tpath_6
    % \]
Let
$\begin{array}{l}
 \text{ body1} = \rpasum(\clabel{i < k}^{1}); \clabel{\assign{j}{0}}^{2};
% \qquad \qquad \qquad \qquad 
\rpchoose\Big\{ \eskip, 
% \\ 
% \qquad \qquad \qquad \qquad \qquad \qquad 
\rprepeat(\rpasum(\clabel{j < m}^{3}); \clabel{\assign{j}{j + 1}}^{4}; \clabel{\assign{w}{i}}^{5};
\\ \qquad
% \qquad \qquad \qquad \qquad \qquad \qquad \qquad \qquad 
\rpchoose\Big\{ \eskip,  \rprepeat(\rpasum(\clabel{w < N}^{6}); \clabel{\assign{w}{w + 1}}^{7})
\Big\}; 
% \\ \qquad \qquad \qquad \qquad \qquad \qquad \qquad \qquad 
  \rpasum(\clabel{w \geq N}^{6}); \clabel{\assign{i}{w}}^{8}) \Big\};
  \\ \qquad
  % \qquad \qquad \qquad \qquad 
  \rpasum(\clabel{j \geq m}^{3}); \clabel{\assign{i}{i+1}}^{9}
\end{array}
$
  \\
, and $\begin{array}{l}
    \text{ body2} = \rpasum(\clabel{j < m}^{3}); \clabel{\assign{j}{j + 1}}^{4}; \clabel{\assign{w}{i}}^{5};
  %  \\ \qquad
   % \qquad \qquad \qquad \qquad \qquad \qquad \qquad \qquad 
   \rpchoose\Big\{ \eskip,  \rprepeat(\rpasum(\clabel{w < N}^{6}); 
   \\ \qquad \qquad \clabel{\assign{w}{w + 1}}^{7})
   \Big\}; 
   % \\ \qquad \qquad \qquad \qquad \qquad \qquad \qquad \qquad 
     \rpasum(\clabel{w \geq N}^{6}); \clabel{\assign{i}{w}}^{8}
   \end{array}
   $
   \\
, and $
    \text{ body3} = \rpasum(\clabel{w < N}^{6}); \clabel{\assign{w}{w + 1}}^{7}
   $
       \item \textbf{Bound Computation} :
    \\

    $L_1$, $L_2$ and $L_3$ denote the while loop at location $1$, $3$ and $6$ respectively.
    \\
    Step-by-Step of the BOUND computation in Figure~5 in \cite{GulwaniJK09}.
    \\
    \newcommand{\BD}{\mathcal{B}}
$\BD(\kw{threeNestedWhile})$
    \[
      \begin{array}{l}
        \BD(\kw{threeNestedWhile})  \triangleq  (c_1 + c_2 + c_3, Z_1 \cup Z_2 \cup Z_3) 
        \\ \qquad
    \textbf{where} ~(c_1, Z_1) \triangleq  \BD(\clabel{ \assign{i}{0} }^{0})
        \\ \qquad
    \textbf{and} ~(c_2, Z_2) \triangleq  \BD( \rpchoose\Big\{\eskip, L_1: \rprepeat(\text{body1})\Big\})
        \\ \qquad
    \textbf{and} ~(c_3, Z_3) \triangleq  \BD(\rpasum(\clabel{i \geq k}^{1})) 
        \\ \qquad = (3, \{(5 + (6 + 2 \times N) \times m, L_1)\})
      \end{array}
  \]
  \[
    \begin{array}{l}
      \BD( \rpchoose\Big\{\eskip, L_1: \rprepeat(\text{body1}) \Big\})  \triangleq 
      (\max\{c', c_1\}, Z' \cup Z_1) \\ \qquad
      \textbf{where} ~ (c', Z')  \triangleq  \BD(\eskip)
      \\ \qquad
    \textbf{and} ~ (c_1, Z_1) \triangleq  \BD(L_1: \rprepeat(\text{body1}))
      \\ \qquad = (1, \{(5 + (6 + 2 \times N) \times m, L_1)\})
    \end{array}
\]
\[
  \begin{array}{l}
    \BD(L_1: \rprepeat(\text{body1}))  \triangleq (0, Z \cup (c_1, L_1)) \\ \qquad
    \textbf{where} ~ c_1 = c' + \sum\limits_{(c'', L'') \in Z' \land L_1 = Parent(L'')}
    % \text{Definition}
    \\ \qquad
    \textbf{and} ~
    Z = \{(c'', L'') | (c'', L'') \in Z' \land L_1 \neq Parent(L'')\}
    \\
    \qquad
    \textbf{and} ~ (c', Z') = \BD(\text{body1})
    \\ \qquad = (0, \{(5 + (6 + 2 \times N) \times m, L_1)\})
\end{array}
\]
This step calls the external 
\\
$BOUNDFINDERD(INIT(\kw{threeNestedWhile}, 1, 3), NEXT(\kw{threeNestedWhile}, 1, 3), \{k, m, n\})$.
\\
It computes the $I(L_2, L_1)$.
$I(L_2, L_1)$ is the bound on total number of iterations of $L_2$ for each iteration of $L_1$.
\\
Assuming a perfect $BOUNDFINDERD$ computes this bound precisely, then we have $I(L_2, L_1) = m$.

\[
  \begin{array}{l}
    \BD(\text{body1} )  \triangleq (1 + 1 + \max\{1, c_2\} + 1 + 1, \emptyset \cup Z_2) 
    \\ \qquad
    \textbf{where} ~ (c_2, Z_2) \land \BD(L_2 : \rprepeat(\text{body2}))
    \\ \qquad = (5, \{( (6 + 2 \times N), L_2)\})
\end{array}
\]
\[
  \begin{array}{l}
    \BD(L_2: \rprepeat(\text{body2}))  \triangleq (0, Z \cup (c_2, L_2)) \\ \qquad
    \textbf{where} ~ c_2 = c' + \sum\limits_{(c'', L'') \in Z' \land L_2 = Parent(L'')}
    % \text{Definition}
    \\ \qquad
    \textbf{and} ~
    Z = \{(c'', L'') | (c'', L'') \in Z' \land L_2 \neq Parent(L'')\}
    \\
    \qquad
    \textbf{and} ~ (c', Z') = \BD(\text{body2})
    \\ \qquad = (0, \{(6 + 2 \times N), L_2\})
\end{array}
\]
This step calls the external 
\\
$BOUNDFINDERD(INIT(\kw{threeNestedWhile}, 3, 6), NEXT(\kw{threeNestedWhile}, 3, 6), \{k, m, n\})$.
\\
$I(L_3, L_2)$ is the bound on total number of iterations of $L_3$ for each iteration of $L_2$.
\\
Assuming a perfect $BOUNDFINDERD$ computes this bound precisely, then we have the $I(L_3, L_2) = N$.

\[
  \begin{array}{l}
    \BD(\text{body2} )  \triangleq (1 + 1 + 1 + \max\{1, c_3\} + 1 + 1, \emptyset \cup Z_3) 
    \\ \qquad  
    \textbf{where} ~  (c_3, Z_3) = \BD(L_3 : \rprepeat(\text{body3}))
    \\ \qquad = (6, \{( 2, L_3)\})
\end{array}
\]
\[
  \begin{array}{l}
    \BD(L_3 : \rprepeat(\text{body3}))  
    \triangleq (0, Z \cup (c_3, L_3)) 
    \\ \qquad
    \textbf{where} ~ c_3 = c' + \sum\limits_{(c'', L'') \in Z' \land L_3 = Parent(L'')}
    % \text{Definition}
    \\ \qquad
    \textbf{and} ~
    Z = \{(c'', L'') | (c'', L'') \in Z' \land L_3 \neq Parent(L'')\}
    \\
    \qquad
    \textbf{and} ~ (c', Z') = \BD(\text{body3})
    \\ \qquad = (0, \{(2, L_3)\})
\end{array}
\]
\[
  \begin{array}{l}
    \BD(\text{body3} = \rpasum(\clabel{w < N}^{6}); \clabel{\assign{w}{w + 1}}^{7})
    \triangleq (c_1 + c_2, Z_1 \cup Z_2 ) 
    \\ \qquad
    \textbf{where} (c_1, Z_1) \triangleq  \BD( \rpasum(\clabel{w < N}^{6})) =  (1, \emptyset)
    \\ \qquad
    \textbf{and} (c_2, Z_2) \triangleq  \BD(\clabel{\assign{w}{w + 1}}^{7})  = (1, \emptyset)
    \\ \qquad = (2, \emptyset)
\end{array}
\]
BOUND($\kw{threeNestedWhile}$):
\[
  \begin{array}{l}
    BOUND(\kw{threeNestedWhile})
    \triangleq c + (c' \times T(L_1))
    \\ \qquad  \textbf{where} ~
    (c', L_1) \in \{(5 + (6 + 2 \times N) \times m, L_1)\}
    \land c = 3 \land 
    T(L_1) = n \land c' = 5 + (6 + 2 \times N) \times m
    \\ \qquad = 3 + 5 \times n + 6 \times m \times n + 2 \times m \times n \times N
\end{array}
\]
This step calls the perfect external
\\
$BOUNDFINDERD(INIT(L_1, 0, 1), NEXT(L_1, 0, 1), \{k, m, n\})$.
It computes $T(L_1)$, which is the number of iterations for the while loop at location $1$.
We assume that it perfectly computes the $T(L_1) = n$.
\end{enumerate}
