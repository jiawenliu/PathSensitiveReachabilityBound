\documentclass[a4paper,11pt]{article}
\usepackage[table]{xcolor}




\usepackage[T1]{fontenc}
\usepackage[normalem]{ulem}
\usepackage{mathtools}
\usepackage{blkarray,bigstrut} 
\usepackage{graphicx,wrapfig,lipsum}
\usepackage{tcolorbox}
\usepackage{enumitem}
\usepackage{array}
\usepackage{algorithm}
\usepackage{algorithmic}
\usepackage{mathpartir}
\usepackage{multirow}
\usepackage{hyperref}
\usepackage{amssymb}
\usepackage{subcaption}
\usepackage{stmaryrd}
\usepackage{color} 


\usepackage{tikz}
\usetikzlibrary{snakes}
\usetikzlibrary{svg.path} 
\usetikzlibrary{calc} 
\usetikzlibrary{shapes}
\usetikzlibrary{shapes.geometric}
\usetikzlibrary{arrows.meta}
\usetikzlibrary{arrows}
\usetikzlibrary{decorations.text,decorations.markings}
% % % % 

%%%%%%%%%%Packages for adaoption%%%

% \usepackage{amsthm} 

%Packages

%%%%%%%%%%%%%%%%%%%%%%%%%%%%%%%%%%%%%%%%%%%%%%%%%%%%%%%%%%%%%%%%%%%%%%%%%%%%%%%%%%%%%%%%%%%%%%%%%%%%%%%%%%%%%%%%%%%%%%%%%%%%%%%%%%%%%%%%%
%%%%%%%%%%%%%%%%%%%%%%%%%%%%%%%%%%%%%%%%%%%%%%%%%%%%% COMMANDS FOR GENERAL PAPER WRITING %%%%%%%%%%%%%%%%%%%%%%%%%%%%%%%%%%%%%%%%%%%%%%%%%%%%
%%%%%%%%%%%%%%%%%%%%%%%%%%%%%%%%%%%%%%%%%%%%%%%%%%%%%%%%%%%%%%%%%%%%%%%%%%%%%%%%%%%%%%%%%%%%%%%%%%%%%%%%%%%%%%%%%%%%%%%%%%%%%%%%%%%%%%%%%

%%%%%%%%%%%%%%%%%% LINK STYLE:
\hypersetup{
  colorlinks=true,
  linkcolor=blue!50!red,
  urlcolor=green!70!black
}

%%%%%%%%%%%%%%%%%%%%%%%%%%%% Theorem, Definition and Proof
\newtheorem{lem}{Lemma}[section]
\newtheorem{thm}{Theorem}[section]
\newtheorem{defn}{Definition}
\newtheorem{coro}{Corollary}[thm]

\newtheorem{example}{Example}[section]

\newcommand{\todo}[1]{{\color{red}\textbf{[[ #1 ]]}}}
\newcommand{\todomath}[1]{{\scriptstyle \color{red}\mathbf{[[ #1 ]]}}}
\newcommand{\completeness}[1]{{\color{blue}\textbf{[[ #1 ]]}}}
\newcommand{\caseL}[1]{\item \textbf{case: #1}\newline}
\newcommand{\subcaseL}[1]{\item \textbf{sub-case: #1}\newline}
\newcommand{\subsubcaseL}[1]{\item \textbf{subsub-case: #1}\newline}
\newcommand{\subsubsubcaseL}[1]{\item \textbf{subsubsub-case: \boldmath{#1}}\newline}

\newcommand{\blue}[1]{{\tiny \color{blue}{ #1 }}}


\let\originalleft\left
\let\originalright\right
\renewcommand{\left}{\mathopen{}\mathclose\bgroup\originalleft}
\renewcommand{\right}{\aftergroup\egroup\originalright}
\newcommand{\ts}[1]{ \llparenthesis {#1} \rrparenthesis }

\theoremstyle{definition}

\newtheorem{case}{Case}
\newtheorem{subcase}{Case}
\numberwithin{subcase}{case}
\newtheorem{subsubcase}{Case}
\numberwithin{subsubcase}{subcase}

\newtheorem{subsubsubcase}{Case}
\numberwithin{subsubsubcase}{subsubcase}
\newcommand{\st}{~.~}

%%%%COLORS
\definecolor{periwinkle}{rgb}{0.8, 0.8, 1.0}
\definecolor{powderblue}{rgb}{0.69, 0.88, 0.9}
\definecolor{sandstorm}{rgb}{0.93, 0.84, 0.25}
\definecolor{trueblue}{rgb}{0.0, 0.45, 0.81}

\newlength\Origarrayrulewidth
% horizontal rule equivalent to \cline but with 2pt width
\newcommand{\Cline}[1]{%
 \noalign{\global\setlength\Origarrayrulewidth{\arrayrulewidth}}%
 \noalign{\global\setlength\arrayrulewidth{2pt}}\cline{#1}%
 \noalign{\global\setlength\arrayrulewidth{\Origarrayrulewidth}}%
}

% draw a vertical rule of width 2pt on both sides of a cell
\newcommand\Thickvrule[1]{%
  \multicolumn{1}{!{\vrule width 2pt}c!{\vrule width 2pt}}{#1}%
}

% draw a vertical rule of width 2pt on the left side of a cell
\newcommand\Thickvrulel[1]{%
  \multicolumn{1}{!{\vrule width 2pt}c|}{#1}%
}

% draw a vertical rule of width 2pt on the right side of a cell
\newcommand\Thickvruler[1]{%
  \multicolumn{1}{|c!{\vrule width 2pt}}{#1}%
}

\newenvironment{subproof}[1][\proofname]{%
  \renewcommand{\qedsymbol}{$\blacksquare$}%
  \begin{proof}[#1]%
}{%
  \end{proof}%
}
%%%%%%%%%%%%%%%%%%%%%%%%%%%%%%% Fonts Definition %%%%%%%%%%%%%%%%%%%%%%%%%%%
\newcommand{\omitthis}[1]{}

% Misc.
\newcommand{\etal}{\textit{et al.}}
\newcommand{\bump}{\hspace{3.5pt}}

% Text fonts
\newcommand{\tbf}[1]{\textbf{#1}}

% Math fonts
\newcommand{\mbb}[1]{\mathbb{#1}}
\newcommand{\mbf}[1]{\mathbf{#1}}
\newcommand{\mrm}[1]{\mathrm{#1}}
\newcommand{\mtt}[1]{\mathtt{#1}}
\newcommand{\mcal}[1]{\mathcal{#1}}
\newcommand{\mfrak}[1]{\mathfrak{#1}}
\newcommand{\msf}[1]{\mathsf{#1}}
\newcommand{\mscr}[1]{\mathscr{#1}}

\newcommand{\diam}{{\color{red}\diamond}}
\newcommand{\dagg}{{\color{blue}\dagger}}
\let\oldstar\star
\renewcommand{\star}{\oldstar}

\newcommand{\im}[1]{\ensuremath{#1}}

\newcommand{\kw}[1]{\im{\mathtt{#1}}}
\newcommand{\set}[1]{\im{\{{#1}\}}}

\newcommand{\mmax}{\ensuremath{\mathsf{max}}}

\lstnewenvironment{ocaml}[2][]%
  {\lstset{language=ocaml,style=ocaml-pretty,captionpos=t,abovecaptionskip=-\medskipamount,caption={#2},#1}}
  %
  {}

\makeatletter
\newcommand{\mysmallishfont}{\@setfontsize\mysmallishfont{8.7pt}{9.7pt}}
\makeatother

\makeatletter
\newcommand{\myecfont}{\@setfontsize\myecfont{9.7pt}{10.7pt}}
\makeatother

\makeatletter
\newcommand{\myecsmfont}{\@setfontsize\myecfont{8.7pt}{9.7pt}}
\makeatother

\lstdefinelanguage{ocaml}{
  style=ocaml-default,
  keywordsprefix={'},
  morekeywords=[1]{},
  morekeywords=[2]{type,op,axiom,lemma,module,pred,const,declare},
  morekeywords=[3]{var,proc},
  morekeywords=[4]{while,if,then,else,elif,return,proof,qed,realize,rec, match},
}

\lstdefinestyle{ocaml-default}{
  escapechar=\#,
  upquote=true,
  columns=fullflexible,
  captionpos=b,
  frame=tb,
  xleftmargin=0pt,
  xrightmargin=0pt,
  rangebeginprefix={(**\ begin\ },
  rangeendprefix={(**\ end\ },
  rangesuffix={\ *)},
  includerangemarker=false,
  basicstyle=\mysmallishfont\sffamily,
  identifierstyle={},
  keywordstyle=[1]{\itshape},
  keywordstyle=[2]{\bfseries},
  keywordstyle=[3]{\bfseries},
  keywordstyle=[4]{\bfseries},
  keywordstyle=[5]{\bfseries},
  keywordstyle=[6]{\bfseries},
  keywordstyle=[7]{},
  keywordstyle=[8]{\bfseries},
  keywordstyle=[9]{\bfseries},
  literate={phi}{{$\!\phi\,$}}1
           {phi1}{{$\!\phi_1$}}1
           {phi2}{{$\!\phi_2$}}1
           {phi3}{{$\!\phi_3$}}1
           {phin}{{$\!\phi_n$}}1
}

\lstdefinestyle{ocaml-pretty}{
    basicstyle=\mysmallishfont\sffamily,
    literate={:=}{{$\mathrel{\gets}\;$}}1
              {<=}{{$\mathrel{\leq}\;$}}1
              {>=}{{$\mathrel{\geq}\;$}}1
              {<>}{{$\mathrel{\neq}\;$}}1
              {=\$}{{$\stackrel{\$}{\gets}\;$}}1
              {->}{{$\rightarrow\;$}}1
              {<-}{{$\leftarrow\;$}}1
              {<->}{{$\leftrightarrow\;$}}1
              {<=>}{{$\Leftrightarrow\;$}}1
              {=>}{{$\Rightarrow\;$}}1
              {==>}{{$\Longrightarrow\;$}}1
              {\/\\}{{$\wedge\;$}}1
              {\\\/}{{$\vee\;$}}1
              {\^}{{\textasciicircum}}1
              {procx}{{proc}}1
}


%%%%%%%%%%%%%%%%%%%%%%%%%%%%%%%%%%%%%%%%%%%%%%%%%%%%%%%%%%%%%%%%%%%%%%%%%%%%%%%%%%%%%%%%%%%%%%%%%%%%%%%%%%%%%%%%%%%%%%%%%%%%%%%%%%%%%%%%%%%%%%%%%%%%%%%%%%
%%%%%%%%%%%%%%%%%%%%%%%%%%%%%%%%%%%%%%%%%%%%%%%%%%%%%%%%%%%% Query While Language %%%%%%%%%%%%%%%%%%%%%%%%%%%%%%%%%%%%%%%%%%%%%%%%%%%%%%%%%%%%%%%%
%%%%%%%%%%%%%%%%%%%%%%%%%%%%%%%%%%%%%%%%%%%%%%%%%%%%%%%%%%%%%%%%%%%%%%%%%%%%%%%%%%%%%%%%%%%%%%%%%%%%%%%%%%%%%%%%%%%%%%%%%%%%%%%%%%%%%%%%%%%%%%%%%%%%%%%%%%
% Language
\newcommand{\command}{c}
%Label
\newcommand{\lin}{\kw{in}}
\newcommand{\lex}{\kw{ex}}
% expression
\newcommand{\expr}{e}
\newcommand{\aexpr}{a}
\newcommand{\bexpr}{b}
\newcommand{\sexpr}{\ssa{\expr} }
\newcommand{\qexpr}{\psi}
\newcommand{\qval}{\alpha}
\newcommand{\query}{{\tt query}}
\newcommand{\eif}{\;\kw{if}\;}
\newcommand{\ethen}{\kw{\;then\;}}
\newcommand{\eelse}{\kw{\;else\;}} 
\newcommand{\eapp}{\;}
\newcommand{\eprojl}{\kw{fst}}
\newcommand{\eprojr}{\kw{snd}}
\newcommand{\eifvar}{\kw{ifvar}}
\newcommand{\ewhile}{\;\kw{while}\;}
\newcommand{\bop}{\;*\;}
\newcommand{\uop}{\;\circ\;}
\newcommand{\eskip}{\kw{skip}}
\newcommand{\edo}{\;\kw{do}\;}
% More unary expression operators:
\newcommand{\esign}{~\kw{sign}~}
\newcommand{\elog}{~\kw{log}~}
% More binary expression operators:
\newcommand{\emax}{~\kw{max}~}
\newcommand{\emin}{~\kw{min}~}

%%%%%%%%%% Extended
\newcommand{\efun}{~\kw{fun}~}
\newcommand{\ecall}{~\kw{call}~}


% Domains
\newcommand{\qdom}{\mathcal{QD}}
\newcommand{\memdom}{\mathcal{M}}
\newcommand{\dbdom}{\mathcal{DB}}
\newcommand{\cdom}{\mathcal{C}}
\newcommand{\ldom}{\mathcal{L}}

\newcommand{\emap}{~\kw{map}~}
\newcommand{\efilter}{~\kw{filter}~}

%configuration
\newcommand{\config}[1]{\langle #1 \rangle}
\newcommand{\ematch}{\kw{match}}
\newcommand{\clabel}[1]{\left[ #1 \right]}

\newcommand{\etrue}{\kw{true}}
\newcommand{\efalse}{\kw{false}}
\newcommand{\econst}{c}
\newcommand{\eop}{\delta}
\newcommand{\efix}{\mathop{\kw{fix}}}
\newcommand{\elet}{\mathop{\kw{let}}}
\newcommand{\ein}{\mathop{ \kw{in}} }
\newcommand{\eas}{\mathop{ \kw{as}} }
\newcommand{\enil}{\kw{nil}}
\newcommand{\econs}{\mathop{\kw{cons}}}
\newcommand{\term}{t}
\newcommand{\return}{\kw{return}}
\newcommand{\bernoulli}{\kw{bernoulli}}
\newcommand{\uniform}{\kw{uniform}}
\newcommand{\app}[2]{\mathrel{ {#1} \, {#2} }}


% Operational Semantics
\newcommand{\env}{\rho}
\newcommand{\rname}[1]{\textsf{\small{#1}}}
\newcommand{\aarrow}{\Downarrow_a}
\newcommand{\barrow}{\Downarrow_b}
\newcommand{\earrow}{\Downarrow_e}
\newcommand{\qarrow}{\Downarrow_q}
\newcommand{\cmd}{c}
\newcommand{\node}{N}
\newcommand{\assign}[2]{ \mathrel{ #1  \leftarrow #2 } }


%%%%%%%%%%%%%%%%%%%%%%%%%%%%%%%%%%%%%%%%%%%%%%%%%%%%%%%%%%%%%%%%%%%%%%%% Trace and Events %%%%%%%%%%%%%%%%%%%%%%%%%%%%%%%%%%%%%%%%
%%%%%%%%%%%%%%%%%%%%%%%%%%%%%%%%%%%%%%%%%%%%%%%%%%%%%%%%%%%%%%%%%%%%%% Trace 
%%%%%%%% annotated query
\newcommand{\aq}{\kw{aq}}
\newcommand{\qtrace}{\kw{qt}}
%annotated variables
\newcommand{\av}{\kw{av}}
\newcommand{\vtrace}{\kw{\tau}}
\newcommand{\ostrace}{{\kw{\tau}}}
\newcommand{\posttrace}{{\kw{\tau}}}

\newcommand{\trace}{\kw{\tau}}

% \newcommand{\vcounter}{\kw{\zeta}}
\newcommand{\vcounter}{\kw{cnt}}

\newcommand{\postevent}{{\kw{\epsilon}}}

% \newcommand{\event}{\kw{\epsilon}}
% \newcommand{\eventset}{\mathcal{E}}
% \newcommand{\eventin}{\in_{\kw{e}}}
% \newcommand{\eventeq}{=_{\kw{e}}}
% \newcommand{\eventneq}{\neq_{\kw{e}}}
% \newcommand{\eventgeq}{\geq_{\kw{e}}}
% \newcommand{\eventlt}{<_{\kw{e}}}
% \newcommand{\eventleq}{\leq_{\kw{e}}}
% \newcommand{\eventdep}{\mathsf{DEP_{\kw{e}}}}
% \newcommand{\asn}{\kw{{asn}}}
% \newcommand{\test}{\kw{{test}}}
% \newcommand{\ctl}{\kw{{ctl}}}
\newcommand{\event}{\kw{\epsilon}}
\newcommand{\eventset}{\mathcal{E}}
\newcommand{\eventin}{\in_{\kw{e}}}
\newcommand{\eventeq}{=_{\kw{e}}}
\newcommand{\eventneq}{\neq_{\kw{e}}}
\newcommand{\eventgeq}{\geq_{\kw{e}}}
\newcommand{\eventlt}{<_{\kw{e}}}
\newcommand{\eventleq}{\leq_{\kw{e}}}
\newcommand{\eventdep}{\mathsf{DEP_{\kw{e}}}}
\newcommand{\asn}{\kw{{asn}}}
\newcommand{\test}{\kw{{test}}}
\newcommand{\ctl}{\kw{{ctl}}}

\newcommand{\sig}{\kw{sig}}
\newcommand{\sigeq}{=_{\sig}}
\newcommand{\signeq}{\neq_{\sig}}
\newcommand{\notsigin}{\notin_{\sig}}
\newcommand{\sigin}{\in_{\sig}}
\newcommand{\sigdiff}{\kw{Diff}_{\sig}}
\newcommand{\action}{\kw{act}}
\newcommand{\diff}{\kw{Diff}}
\newcommand{\seq}{\kw{seq}}
\newcommand{\sdiff}{\kw{Diff}_{\seq}}

\newcommand{\tracecat}{{\scriptscriptstyle ++}}
\newcommand{\traceadd}{{\small ::}}

\newcommand{\ism}{\kw{ism}}
\newcommand{\ismdiff}{\kw{Diff}_{\sig}}
\newcommand{\ismeq}{=_{\ism}}
\newcommand{\ismneq}{\neq_{\ism}}
\newcommand{\notismin}{\notin_{\ism}}
\newcommand{\ismin}{\in_{\ism}}

%operations on the trace and Annotated Query
\newcommand{\projl}[1]{\kw{\pi_{l}(#1)}}
\newcommand{\projr}[1]{\kw{\pi_{r}(#1)}}

% operations on annotated query, i.e., aq
\newcommand{\aqin}{\in_{\kw{aq}}}
\newcommand{\aqeq}{=_{\kw{aq}}}
\newcommand{\aqneq}{\neq_{\kw{aq}}}
\newcommand{\aqgeq}{\geq_{\kw{aq}}}

% operations on annotated variables, i.e., av
\newcommand{\avin}{\in_{\kw{av}}}
\newcommand{\aveq}{=_{\kw{av}}}
\newcommand{\avneq}{\neq_{\kw{av}}}
\newcommand{\avgeq}{\geq_{\kw{av}}}
\newcommand{\avlt}{<_{\kw{av}}}

% adaptivity
\newcommand{\adap}{\kw{adap}}
\newcommand{\ddep}[1]{\kw{depth}_{#1}}
\newcommand{\nat}{\mathbb{N}}
\newcommand{\natb}{\nat_{\bot}}
\newcommand{\natbi}{\natb^\infty}
\newcommand{\nnatA}{Z}
\newcommand{\nnatB}{m}
\newcommand{\nnatbA}{s}
\newcommand{\nnatbB}{t}
\newcommand{\nnatbiA}{q}
\newcommand{\nnatbiB}{r}


%%%%%%%%%%%%%%%%%%%%%%%%%%%%%%%%%%%%%%%%%%%%%%%%%%%%%%%%%%%%%%%%%%%%%%%%%%%%%%%%%%%%%%%%%%%%%%%%%%%%%%%%%%%%%%%%%%%%%%%%%%%%%%%%%%%%%%%%%%%%%%%%%%%%%%%%%%
%%%%%%%%%%%%%%%%%%%%%%%%%%%%%%%%%%%%%%%%%%%%%%%%%%%%%%%%%%%% Dynamic Program Analysis %%%%%%%%%%%%%%%%%%%%%%%%%%%%%%%%%%%%%%%%%%%%%%%%%%%%%%%%%%%%%%%%
%%%%%%%%%%%%%%%%%%%%%%%%%%%%%%%%%%%%%%%%%%%%%%%%%%%%%%%%%%%%%%%%%%%%%%%%%%%%%%%%%%%%%%%%%%%%%%%%%%%%%%%%%%%%%%%%%%%%%%%%%%%%%%%%%%%%%%%%%%%%%%%%%%%%%%%%%%

%%%%%%%%%%%%%%%%%%%%%%%%%%%%%%%%% Execution Based Dependency, Graph and Adaptivity 
\newcommand{\paths}{\mathcal{PATH}}
\newcommand{\walks}{\mathcal{WK}}
\newcommand{\progwalks}{\mathcal{WK}^{\kw{prog}}}

\newcommand{\len}{\kw{len}}
% \newcommand{\lvar}{\mathbb{LV}}
\newcommand{\lvar}{\mathbb{L}}
\newcommand{\avar}{\mathbb{AV}}
\newcommand{\qvar}{\mathbb{QV}}
\newcommand{\qdep}{\mathsf{DEP_{q}}}
\newcommand{\vardep}{\mathsf{DEP_{var}}}
\newcommand{\avdep}{\mathsf{DEP_{\av}}}
\newcommand{\finitewalk}{\kw{fw}}
\newcommand{\pfinitewalk}{\kw{fwp}}
\newcommand{\dep}{\mathsf{DEP}}

\newcommand{\llabel}{\iota}
\newcommand{\entry}{\kw{entry}}
\newcommand{\tlabel}{\mathbb{TL}}

\newcommand{\traceG}{\kw{{G_{trace}}}}
\newcommand{\traceV}{\kw{{V_{trace}}}}
\newcommand{\traceE}{\kw{{E_{trace}}}}
\newcommand{\traceF}{\kw{{Q_{trace}}}}
\newcommand{\traceW}{\kw{{W_{trace}}}}
\newcommand{\exeRB}{\kw{RB_{exe}}}
%%%%%%%%%%%%%%%%%%%%%%%%%%%%%%%%%%%%%%%%%%%%%%%%%%%%%%%%%%%%%%%%%%%%%%%%%%%%%%%%%%%%%%%%%%%%%%%%%%%%%%%%%%%%%%%%%%%%%%%%%%%%%%%%%%%%%%%%%%%%%%%%%%%%%%%%%%
%%%%%%%%%%%%%%%%%%%%%%%%%%%%%%%%%%%%%%%%%%%%%%%%%%%%%%%%%%%% Static Program Analysis %%%%%%%%%%%%%%%%%%%%%%%%%%%%%%%%%%%%%%%%%%%%%%%%%%%%%%%%%%%%%%%%
%%%%%%%%%%%%%%%%%%%%%%%%%%%%%%%%%%%%%%%%%%%%%%%%%%%%%%%%%%%%%%%%%%%%%%%%%%%%%%%%%%%%%%%%%%%%%%%%%%%%%%%%%%%%%%%%%%%%%%%%%%%%%%%%%%%%%%%%%%%%%%%%%%%%%%%%%%

%Static Adaptivity Definition:
\newcommand{\flowsto}{\kw{flowsTo}}
\newcommand{\live}{\kw{RD}}

%Analysis Algorithms and Graphs
\newcommand{\weight}{\mathsf{W}}
\newcommand{\green}[1]{{ \color{green} #1 } }

\newcommand{\func}[2]{\mathsf{AD}(#1) \to (#2)}
\newcommand{\varCol}{\bf{VetxCol}}
\newcommand{\graphGen}{\bf{FlowGen}}
\newcommand{\progG}{\kw{{G_{prog}}}}
\newcommand{\progV}{\kw{{V_{prog}}}}
\newcommand{\progE}{\kw{{E_{prog}}}}
\newcommand{\progF}{\kw{{Q_{prog}}}}
\newcommand{\progW}{\kw{{W_{prog}}}}
\newcommand{\progA}{A_{\kw{prog}}}

\newcommand{\midG}{\kw{{G_{mid}}}}
\newcommand{\midV}{\kw{{V_{mid}}}}
\newcommand{\midE}{\kw{{E_{mid}}}}
\newcommand{\midF}{\kw{{Q_{mid}}}}



\newcommand{\sccgraph}{\kw{G^{SCC}}}
\newcommand{\sccG}{\kw{{G_{scc}}}}
\newcommand{\sccV}{\kw{{V_{scc}}}}
\newcommand{\sccE}{\kw{{E_{scc}}}}
\newcommand{\sccF}{\kw{{Q_{scc}}}}
\newcommand{\sccW}{\kw{{W_{scc}}}}


\newcommand{\visit}{\kw{visit}}

\newcommand{\flag}{\kw{F}}
\newcommand{\Mtrix}{\kw{M}}
\newcommand{\rMtrix}{\kw{RM}}
\newcommand{\lMtrix}{\kw{LM}}
\newcommand{\vertxs}{\kw{V}}
\newcommand{\qvertxs}{\kw{QV}}
\newcommand{\qflag}{\kw{Q}}
\newcommand{\edges}{\kw{E}}
\newcommand{\weights}{\kw{W}}
\newcommand{\qlen}{\len^{\tt q}}
\newcommand{\pwalks}{\mathcal{WK}_{\kw{p}}}

\newcommand{\rb}{\mathsf{RechBound}}
\newcommand{\pathsearch}{\mathsf{AdaptSearch}}

%program abstraction
\newcommand{\abst}[1]{\kw{abs}{#1}}
\newcommand{\absexpr}{\abst{\kw{expr}}}
\newcommand{\absevent}{\stackrel{\scriptscriptstyle{\alpha}}{\event{}}}
\newcommand{\absfinal}{\abst{\kw{final}}}
\newcommand{\absinit}{\abst{\kw{init}}}
\newcommand{\absflow}{\abst{\kw{trace}}}
\newcommand{\absG}{\abst{\kw{G}}}
\newcommand{\absV}{\abst{\kw{V}}}
\newcommand{\absE}{\abst{\kw{E}}}
\newcommand{\absF}{\abst{\kw{F}}}
\newcommand{\absW}{\abst{\kw{W}}}
\newcommand{\locbound}{\kw{locb}}
\newcommand{\absdom}{\mathcal{ADOM}}
\newcommand{\inpvar}{\mathcal{VAR}_{\kw{in}}}
\newcommand{\grdvar}{\mathcal{VAR}_{\kw{guard}}}


\newcommand{\absclr}{{\kw{Tclosure}}}
\newcommand{\varinvar}{{\kw{Vinvar}}}
\newcommand{\init}{\kw{init}}
\newcommand{\constdom}{\mathcal{SMBCST}}
\newcommand{\dcdom}{\mathcal{DC}}
\newcommand{\reset}{\kw{re}}
\newcommand{\resetchain}{\kw{rechain}}
\newcommand{\inc}{\kw{inc}}
\newcommand{\dec}{\kw{dec}}

%%%%%%%%%%%%%%%%%%%%%%%%%%%%%%%%%%%%%%%%%%%%%%%%%%%%%%%%%%%%%%%%%%%%%%%%%%%%%%%%%%%%%%%%%%%%%%%%%%%%%%%%%%%%%%%%%%%%%%%%%%%%%%%%%%%%%%%%%%%%%%%%%%%%%%%%%%
%%%%%%%%%%%%%%%%%%%%%%%%%%%%%%%%%%%%%%%%%%%%%%%%%%%%%%%%%%%% Path Sensitive Reachability Bound Analysis %%%%%%%%%%%%%%%%%%%%%%%%%%%%%%%%%%%%%%%%%%%%%%%%%%%%%%%%%%%%%%%%
%%%%%%%%%%%%%%%%%%%%%%%%%%%%%%%%%%%%%%%%%%%%%%%%%%%%%%%%%%%%%%%%%%%%%%%%%%%%%%%%%%%%%%%%%%%%%%%%%%%%%%%%%%%%%%%%%%%%%%%%%%%%%%%%%%%%%%%%%%%%%%%%%%%%%%%%%%

\newcommand{\tpath}{\kw{tp}}
\newcommand{\rprog}{\kw{rp}}
\newcommand{\absstate}{\kw{rstate}}
\newcommand{\rpchoose}{\kw{choose}}
\newcommand{\rprepeat}{\kw{repreat}}
\newcommand{\rpseq}{\kw{seq}}


%%%%%%%%%%%%%%%%%%%%%%%%%%%%%%%%%%%%%%%%%%%%%%%%%%%%%%%%%%%%%%%%%%%%%%%%%%%%%%%%%%%%%%%%%%%%%%%%%%%%%%%%%%%%%%%%%%%%%%%%%%%%%%%%%%%%%%%%%%%%%%%%%%%%%%%%%%
%%%%%%%%%%%%%%%%%%%%%%%%%%%%%%%%%%%%%%%%%%%%%%%%%%%%%%%%%%%%%%%%%%%%%%% author comments in draft mode %%%%%%%%%%%%%%%%%%%%%%%%%%%%%%%%%%%%%%%%%%%%%%%%%%%%%%%%%%%%%%%%%%%%%%
%%%%%%%%%%%%%%%%%%%%%%%%%%%%%%%%%%%%%%%%%%%%%%%%%%%%%%%%%%%%%%%%%%%%%%%%%%%%%%%%%%%%%%%%%%%%%%%%%%%%%%%%%%%%%%%%%%%%%%%%%%%%%%%%%%%%%%%%%%%%%%%%%%%%%%%%%%

\newif\ifdraft
%\draftfalse
\drafttrue

\ifdraft
% Jiawen
\newcommand{\jl}[1]{\textcolor[rgb]{.00,0.00,1.00}{[JL: #1]}}
\newcommand{\jlside}[1]{\marginpar{\tiny \sf \textcolor[rgb]{.00,0.80,0.00}{[jl: #1]}}}
% Deepak
\newcommand{\dg}[1]{\textcolor[rgb]{.00,0.80,0.00}{[DG: #1]}}
\newcommand{\dgside}[1]{\marginpar{\tiny \sf \textcolor[rgb]{.00,0.80,0.00}{[DG: #1]}}}
% Marco
\newcommand{\mg}[1]{\textcolor[rgb]{.80,0.00,0.00}{[MG: #1]}}
\newcommand{\mgside}[1]{\marginpar{\tiny \sf \textcolor[rgb]{.80,0.00,0.00}{[MG: #1]}}}
% Weihao
\newcommand{\wq}[1]{\textcolor[rgb]{.00,0.80,0.00}{[WQ: #1]}}
\newcommand{\wqside}[1]{\marginpar{\tiny \sf \textcolor[rgb]{.00,0.80,0.00}{[WQ: #1]}}}
\else
\newcommand{\mg}[1]{}
\newcommand{\mgside}[1]{}
\newcommand{\wq}[1]{}
\newcommand{\wqside}[1]{}
\newcommand{\rname}[1]{$\textbf{#1}$}
\fi

\newcommand{\highlight}[1]{\textcolor[rgb]{.0,0.0,1.0}{ #1}}

\usetikzlibrary{shapes,arrows}
\newcommand{\THESYSTEM}{\textsf{PsRB}}

% Define block styles
\tikzstyle{decision} = [diamond, draw, fill=blue!20, 
    text width=4.5em, text badly centered, node distance=3cm, inner sep=0pt]
\tikzstyle{block} = [rectangle, draw, fill=blue!20, 
    text width=5em, text centered, rounded corners, minimum height=4em]
\tikzstyle{line} = [draw, -latex']
\tikzstyle{cloud} = [draw, ellipse,fill=red!20, node distance=3cm,
    minimum height=2em]

\begin{document}
\title{Technique Limitation in PLDI 2009}

\author{}

\date{}

\maketitle
%
% \input{example_cousot}
% \clearpage
% %
% % 
\section{Step-by-Step Computations}

In summary, the algorithm in the Figure~5 in \cite{GulwaniJK09} first summarize the constraints of all nested loop. 
Then they rely on external $BOUNDFINER$ to compute the bound for the outermost loop with the summarized constraints. 

\subsection{Example of The Nested Loop with Related Iterator}
\begin{example}[Nested Loop with Related Iterators]
  \label{ex:threeNestedWhile}
  %
  %
  { \small
\begin{figure}
\centering
\begin{subfigure}{.4\textwidth}
  \begin{centering}
  {\footnotesize
  $
  \begin{array}{l}
      \kw{relatedNestedWhile}(n, m, N) \triangleq \\
      \clabel{ \assign{i}{0} }^{0} ; \\
          \ewhile ~ \clabel{i < n}^{1} ~ \edo ~ \\
          \qquad \Big(
           \clabel{\assign{j}{m}}^{2} ;\\
           \qquad \ewhile ~ \clabel{j > 0}^{3} ~ \edo ~ \\
           \qquad \qquad \Big(
            \clabel{\assign{j}{j-1}}^{4};
            \clabel{\assign{w}{i}}^{5};\\
            \qquad \qquad \ewhile ~ \clabel{w < N}^{6} ~ \edo ~
            \Big(
              \clabel{\assign{w}{w + 1}}^{7}
                \Big); \\
                \qquad \qquad \clabel{\assign{i}{w}}^{8}
                \Big); \\
                \qquad \clabel{\assign{i}{i+1}}^{9}
            \Big)
      \end{array}
  $
  }
  \caption{}
  \end{centering}
  \end{subfigure}
\begin{subfigure}{.5\textwidth}
  \begin{centering}
%   \todo{abstract-cfg for two round}
\begin{tikzpicture}[scale=\textwidth/20cm,samples=200]
\draw[] (-7, 10) circle (0pt) node{{ $0$}};
\draw[] (0, 10) circle (0pt) node{{ $1$}};
\draw[] (6, 10) circle (0pt) node {{$\lex$}};
\draw[] (0, 7) circle (0pt) node{{$2$}};
\draw[] (0, 4) circle (0pt) node{{ $3$}};
\draw[] (-7, 4) circle (0pt) node{{ $9$}};
\draw[] (0, 1) circle (0pt) node{{ $4$}};
\draw[] (3, 1) circle (0pt) node{{ $5$}};
\draw[] (6, 1) circle (0pt) node{{ $6$}};
\draw[] (9, 1) circle (0pt) node{{ $7$}};
\draw[] (5, 6) circle (0pt) node{{ $8$}};
% Counter Variables
%
% Control Flow Edges:
\draw[ thick, -latex] (-6, 10)  -- node [above] {$i' \leq 0$}(-0.5, 10);
\draw[ thick, -latex] (0, 9.5)  -- node [left] {$i < n$} (0, 7.5) ;
\draw[ thick, -latex] (0, 6.5)  -- node [left] {$j' \leq m$} (0, 4.5) ;
\draw[ thick, -latex] (0, 3.5)  -- node [left] {$j > 0$} (0, 1.5) ;
\draw[ thick, -latex] (-0.5, 4)  -- node [below] {$j \leq 0$} (-6.5, 4) ;
\draw[ thick, -latex] (-6.5, 4.5)  to  [out=90,in=180]  node [left] {$i' \leq i + 1$ }(-0.5, 9.5);
\draw[ thick, -latex] (0.5, 10)  -- node [above] {$i \geq n$}  (5, 10);
\draw[ thick, -latex] (0.5, 1)  -- node [below] {$j' \leq j - 1$}  (2.5, 1);
\draw[ thick, -latex] (3.5, 1)  -- node [above] {$w' \leq i$}  (5.5, 1);
\draw[ thick, -latex] (6.5, 1)  -- node [below] {$w < N$}  (8.5, 1);
\draw[ thick, -latex] (6, 1.5)  to [out=90,in=-60] node [right] {$w \geq N$}  (5, 5.5);
\draw[ thick, -latex] (9, 1.5)  to  [out=120,in=30] node [right] {$w' \leq w + 1$}  (6, 1.5);
\draw[ thick, -latex] (5, 5.5)  to  node [above] {$i' \leq k$ }(0.5, 4);
\end{tikzpicture}
\caption{}
  \end{centering}
  \end{subfigure}
\caption{
(a) The Example of Nested Loop with Related Iterators
  (b) The Abstract Execution Control Flow Graph}
    \label{fig:threeNestedWhile}
\end{figure}
}
\end{example}

\begin{enumerate}
  \item  \textbf{The Constraint Program (Abstract Control Flow Graph)} is generated in Figure~\ref{fig:threeNestedWhile}(b).

  \item \textbf{Program Refinement}
  \\
  The loop free simple transition paths are computed as follows,
  \[
      \begin{array}{llll}
          \tpath_0 = (0 \to 1)
          &
          \tpath_1 = (1 \to 2), (2 \to 3)
          &           
          \tpath_2 = (3 \to 4), (4 \to 5), (5 \to 6)
          &
          \tpath_3 = (6 \to 7), (7 \to 6)
          \\
          \tpath_6 = (1 \to \lex)
          &
          \tpath_4 = (6 \to 8), (8 \to 3)
          &
          \tpath_5 = (3 \to 9), (9 \to 1)
      \end{array}
      \]
  \textbf{Refined Program}:
  \[
  \rprog = \tpath_0 ; 1: \rprepeat(\tpath_1; 3: \rprepeat(\tpath_2; 6 : \rprepeat(\tpath_3); \tpath_4); \tpath_5); \tpath_6
  \]
  \item \textbf{Outside-In Algorithm} :The \emph{OutIn} bound for the $\rprog$ and every nested repeat patterns.
  \\
$\outinB(\tpath_0) = 1$
\quad
$\outinB(\tpath_6) = 1$
\quad
$\outinB(6 : \rprepeat(\tpath_3)) = N $
\\
$\outinB(3: \rprepeat(\tpath_2; 6 : \rprepeat(\tpath_3); \tpath_4)) = m $
\\
$\outinB(1: \rprepeat(\tpath_1; 3: \rprepeat(\tpath_2; 6 : \rprepeat(\tpath_3); \tpath_4); \tpath_5)) = n - N $
\item \textbf{Inside-Out Algorithm}
\begin{itemize}
  \item \textbf{Repeat Chain Set}
  \\
  $\rpchset(1, \tpath_1) = \{1: \rprepeat(\tpath_1; 3: \rprepeat(\tpath_2; 6 : \rprepeat(\tpath_3); \tpath_4); \tpath_5)\}$
  \\
  $\rpchset(1, \tpath_5) = \{1: \rprepeat(\tpath_1; 3: \rprepeat(\tpath_2; 6 : \rprepeat(\tpath_3); \tpath_4); \tpath_5)\}$
  \\
  $\rpchset(3, \tpath_2) = \{3: \rprepeat(\tpath_2; 6 : \rprepeat(\tpath_3); \tpath_4)\}$
  \\
  $\rpchset(3, \tpath_4) = \{3: \rprepeat(\tpath_2; 6 : \rprepeat(\tpath_3); \tpath_4)\}$
  \\
  $\rpchset(6, \tpath_3) = \{6: \rprepeat(\tpath_3)\}$
  \\
  $\rpchset(_, \_) = \emptyset$ 
  % \\
  \item \textbf{Repeat Chain Bound} for every simple transition path $\tpath$ through its \emph{Repeat Chain}s
  \\
  $\rpchB(1, \tpath_1) = n - N$ \quad
  $\rpchB(1, \tpath_5) = n - N$ \quad
  $\rpchB(3, \tpath_2) = m$ \\
  $\rpchB(3, \tpath_4) = m$ \quad
  $\rpchB(6, \tpath_3) = N$ \quad \quad 
  $\rpchB(_, \_) = \bot $ 
  %
  \item \textbf{Loop Chain}
  \\
  $\lpch(\tpath_1) = 1\to \tpath_1$ \quad
  $\lpch(\tpath_2) = 1 \to 3 \to \tpath_2$ \\
  $\lpch(\tpath_5) = 1\to \tpath_5$ \quad
  $\lpch(\tpath_4) = 1 \to 3 \to \tpath_4$ \\
  \highlight
  {$\lpch(\tpath_3) = 1 \to 3 \to 5 \to \tpath_3$ }\\
  $\lpch(\tpath_0) = \tpath_0$ \quad
  $\lpch(\tpath_6) = \tpath_6$ 
  \item \textbf{{Relative Loop Bound}} for every simple transition path $\tpath$ through its \emph{Loop Chain}
  \\
  $\rpchB(1, \tpath_1) = n - N$ \\
  $\rpchB(1, \tpath_5) = n - N$ \\
  $\rpchB(1, \tpath_2) = n$; \quad $\rpchB(3, \tpath_2) = m$; \\
  $\rpchB(1, \tpath_4) = n$; \quad $\rpchB(3, \tpath_4) = m$ \\
  \highlight{$\rpchB(1, \tpath_3) = 1$; \quad $\rpchB(3, \tpath_3) = 1$; \quad $\rpchB(5, \tpath_3) = N$} \\
  $\rpchB(_, \_) = \bot $ 
  \item \textbf{Path-Sensitive Reachability-Bound} for every simple transition path $\tpath$
  \\
  $\inoutB(\tpath_1) = n - N$ \quad
  $\inoutB(\tpath_2) = n \times m$ \quad
  $\inoutB(\tpath_0) = 1$ 
  \\
  $\inoutB(\tpath_5) = n - N$ \quad
  $\inoutB(\tpath_4) = n \times m$ \quad
  $\inoutB(\tpath_6) = 1$ 
  \\
  $\inoutB(\tpath_3) = N$ \quad
\end{itemize}
\item \textbf{Path Sensitive Reachability-Bound} on every program control location
\\
$\psRB(\{0, \lex\}) = 1$ \quad
$\psRB(\{1\}) = n - N + 1$ \quad
$\psRB(\{2, 9\}) = n - N$ \quad
$\psRB(\{3\}) = n - N + n \times m$ \quad
$\psRB(\{4, 5, 8\}) = n \times m$ \quad
$\psRB(\{7\}) = N$ \quad
$\psRB(\{6\}) = N + n \times m$ 
\end{enumerate}
\section{Example of The Two Paths While Loop}
% \begin{example}[While with Two Paths]
  \label{ex:twoPathsWhile}
  %
  { \small
  \begin{figure}
  \centering
  \begin{subfigure}{.4\textwidth}
    \begin{centering}
    {\small
    $
    \begin{array}{l}
      \kw{twoPathsWhile}(n, m) \triangleq \\
    \clabel{ \assign{i}{n} }^{0} ; \\
    \clabel{ \assign{j}{0} }^{1} ; \\
        \ewhile ~ \clabel{i > 0}^{2} ~ \edo ~ \\
        \qquad \Big(
          \eif(\clabel{j < m}^{3}, \\
          \qquad \qquad \clabel{\assign{j}{j + 1}}^{4}; 
          \clabel{\assign{i}{i - 1}}^{5},\\
          \qquad \qquad \clabel{\assign{j}{0}}^{6});
          \Big)
        \end{array}
        $
    }
    \caption{}
    \end{centering}
    \end{subfigure}
  \begin{subfigure}{.5\textwidth}
    \begin{centering}
  %   \todo{abstract-cfg for two round}
  \begin{tikzpicture}[scale=\textwidth/20cm,samples=200]
  \draw[] (-8, 10) circle (0pt) node{{ $0$}};
  \draw[] (-4, 10) circle (0pt) node{{ $1$}};
  \draw[] (0, 10) circle (0pt) node{{ $2$}};
  \draw[] (0, 7) circle (0pt) node{{$3$}};
  \draw[] (-3, 4) circle (0pt) node{{ $4$}};
  \draw[] (-8, 4) circle (0pt) node{{ $5$}};
  \draw[] (4, 4) circle (0pt) node{{ $6$}};
  % Counter Variables
  \draw[] (6, 10) circle (0pt) node {\textbf{$\lex$}};
  % \draw[] (6, 4) circle (0pt) node {{ $ex$}};
  %
  % Control Flow Edges:
  \draw[ thick, -latex] (-7, 10)  -- node [above] {$i' \leq n$}(-4.5, 10);
  \draw[ thick, -latex] (-3, 10)  -- node [above] {$j' \leq 0$}(-0.5, 10);
  \draw[ thick, -latex] (0, 9.5)  -- node [left] {$i > 0$} (0, 7.5) ;
  \draw[ thick, -latex] (0.5, 7)  -- node [below] {$ j \geq m $}  (4, 4.5);
  \draw[ thick, -latex] (-7.5, 4.5)  to  [out=90,in=180]  node [left] {$i' \leq i - 1$ }(-0.5, 9.5);
  \draw[ thick, -latex] (4.5, 4)  to  [out=70,in=0]   node [right] {$j' \leq 0 $}(0.5, 9.5);
  \draw[ thick, -latex]  (-0.5, 7) -- node  {$j < m$}  (-3, 4.5) ;
  \draw[ thick, -latex]  (-3.5, 4) -- node [above] {$j' \leq j + 1$}  (-7, 4) ;
  \draw[ thick, -latex] (0.5, 10)  -- node [above] {$i \leq 0$}  (5.5, 10);
  % \draw[ thick, -latex] (6, 6.5)  -- node [right] {$\top$} (6, 4.5) ;
  \end{tikzpicture}
  \caption{}
    \end{centering}
    \end{subfigure}
  \caption{
  (a) The Two Paths While Loop Example
    (b) The Abstract Execution Control Flow Graph}
      \label{fig:twoPathsWhile}
  \end{figure}
  }
\end{example}

\begin{enumerate}
  \item  \textbf{The Abstract Execution Control Flow Graph} is generated in Figure~\ref{fig:twoPathsWhile}(b).

  \item \textbf{Program Rephrase and Refinement}. 
  \\
  The loop free simple transition paths are computed as follows,
  \[
    \begin{array}{ll}
\tpath_0 = (0 \to 1), (1 \to 2)
&
\tpath_2 = (2 \to 3), (3 \to 6), (6 \to 2)
\\
\tpath_1 = (2 \to 3), (3 \to 4), (4 \to 5), (5 \to 2)
&
\tpath_3 = (2 \to \lex)
\end{array}
\]
\textbf{Refined Program}:
\[
  \tpath_0 ; \rpchoose{2: \rprepeat_2(\rprepeat_1(\tpath_1); \tpath_2) , 
  2: \rprepeat_1(\tpath_1) }; \tpath_3
  \]
  \item \textbf{Ranks}:
  The ranking bounds for every simple transition path:
  \\
  $\absclr(\tpath_0) = 1$ 
  $\absclr(\tpath_1) = n $ \quad
  $\absclr(\tpath_2) = n $ \quad
  $\absclr(\tpath_3) = 1$
  % \\
  % $\bot$ means the algorithm fails in inferring the ranking bound.
    \item \textbf{Outside-In Algorithm} :The \emph{OutIn} bound for the $\rprog$ and every nested repeat patterns.
  \[
    \begin{array}{l}
        \outinB(\tpath_0) = 1
        \\
        \outinB(2: \rprepeat_1(\tpath_1)) = m 
        \\
        \outinB(2: \rprepeat_2(\rprepeat_1(\tpath_1); \tpath_2)) = \lfloor\frac{n}{m}\rfloor
        \\
        \outinB(\rpchoose{\rprepeat_2(\rprepeat_1(\tpath_1); \tpath_2), \rprepeat_1(\tpath_1) })
        = \max\{m, (m  + 1)\times \lfloor\frac{n}{m}\rfloor\}
\end{array}
\]
\item \textbf{Inside-Out Algorithm}
\begin{itemize}
  \item \textbf{Repeat Chain Set}
  \\
  $\rpchset(2, \tpath_1) = \{\rprepeat_1(\tpath_1), \rprepeat_2(\rprepeat_1(\tpath_1); \tpath_2) \to \rprepeat_1(\tpath_1)\}$ \\
  $\rpchset(2, \tpath_2) = \{\rprepeat_2(\rprepeat_1(\tpath_1); \tpath_2) \to \rprepeat_1(\tpath_1)\}$ \\
  $\rpchset(\_, \_) = \emptyset$ 
  % \\
  \item \textbf{Repeat Chain Bound} for every simple transition path $\tpath$ through its \emph{Repeat Chain}s
  \\
  $\rpchB(2, \tpath_1) = \max\{m, m \times \lfloor\frac{n}{m}\rfloor\}$ \\
  $\rpchB(2, \tpath_2) = \lfloor\frac{n}{m}\rfloor$ 
  %
  \item \textbf{Loop Chain}
  \\
  $\lpch(\tpath_0) = \tpath_0$ \qquad
  $\lpch(\tpath_1) = 2\to \tpath_1$ \\
  $\lpch(\tpath_3) = \tpath_3$ \qquad
  $\lpch(\tpath_2) = 2\to \tpath_2$ 
  \item \textbf{{Relative Loop Bound}} for every simple transition path $\tpath$ through its \emph{Loop Chain}
  \\
  $\rpchB(2, \tpath_1) = \max\{m, m \times \lfloor\frac{n}{m}\rfloor\}$ \quad
  $\rpchB(2, \tpath_2) = \lfloor\frac{n}{m}\rfloor$  \\
  $\rpchB(\bot, \tpath_0) = 1$ \quad
  $\rpchB(\bot, \tpath_3) = 1$ 
  \item \textbf{Path-Sensitive Reachability-Bound} for every simple transition path $\tpath$
  \\
  $\inoutB(\tpath_1) = n$ \quad
  $\inoutB(\tpath_2) = \lfloor\frac{n}{m}\rfloor$ \quad
  $\inoutB(\tpath_0) = 1$ \quad
  $\inoutB(\tpath_3) = 1$ 
\end{itemize}
\item \textbf{Path Sensitive Reachability-Bound} on every program control location
\\
$\psRB(\{0, 1, \lex\}) = 1$ \qquad
$\psRB(\{6 \}) = \lfloor\frac{n}{m}\rfloor$ \\
$\psRB(\{4, 5 \}) = \max\{m, m \times \lfloor\frac{n}{m}\rfloor\}$ \quad
$\psRB(\{3, 2 \}) = \max\{m, m \times \lfloor\frac{n}{m}\rfloor\} + \lfloor\frac{n}{m}\rfloor + 1 $ \\
\end{enumerate}
\begin{example}[The Example of While Loop with Two Interleaved Paths]
  \label{ex:twoCountersWhile}
  %
  { \small
  \begin{figure}
  \centering
  \begin{subfigure}{.4\textwidth}
    \begin{centering}
    {\small
    $
    \begin{array}{l}
      \rpasum(n > 0 \land m > 0)\\
      \kw{twoPathsWhile}(n, m) \triangleq \\
    \clabel{ \assign{i}{n} }^{0} ; \\
    \clabel{ \assign{j}{0} }^{1} ; \\
        \ewhile ~ \clabel{i > 0}^{2} ~ \edo ~ \\
        \qquad \Big(
          \eif(\clabel{j < m}^{3}, \\
          \qquad \qquad \clabel{\assign{j}{j + 1}}^{4}; 
          \clabel{\assign{i}{i - 1}}^{5},\\
          \qquad \qquad \clabel{\assign{j}{0}}^{6});
          \Big)
        \end{array}
        $
    }
    \caption{}
    \end{centering}
    \end{subfigure}
  \begin{subfigure}{.5\textwidth}
    \begin{centering}
  %   \todo{abstract-cfg for two round}
  \begin{tikzpicture}[scale=\textwidth/20cm,samples=200]
  \draw[] (-8, 10) circle (0pt) node{{ $0$}};
  \draw[] (-4, 10) circle (0pt) node{{ $1$}};
  \draw[] (0, 10) circle (0pt) node{{ $2$}};
  \draw[] (0, 7) circle (0pt) node{{$3$}};
  \draw[] (-3, 4) circle (0pt) node{{ $4$}};
  \draw[] (-8, 4) circle (0pt) node{{ $5$}};
  \draw[] (4, 4) circle (0pt) node{{ $6$}};
  % Counter Variables
  \draw[] (6, 10) circle (0pt) node {\textbf{$\lex$}};
  % \draw[] (6, 4) circle (0pt) node {{ $ex$}};
  %
  % Control Flow Edges:
  \draw[ thick, -latex] (-7, 10)  -- (-4.5, 10);
  \draw[ thick, -latex] (-3, 10)  -- (-0.5, 10);
  \draw[ thick, -latex] (0, 9.5)  --  (0, 7.5) ;
  \draw[ thick, -latex] (0.5, 7)  --  (4, 4.5);
  \draw[ thick, -latex] (-7.5, 4.5)  to  [out=90,in=180]  (-0.5, 9.5);
  \draw[ thick, -latex] (4.5, 4)  to  [out=70,in=0] (0.5, 9.5);
  \draw[ thick, -latex]  (-0.5, 7) --   (-3, 4.5) ;
  \draw[ thick, -latex]  (-3.5, 4) --   (-7, 4) ;
  \draw[ thick, -latex] (0.5, 10)  --  (5.5, 10);
  % \draw[ thick, -latex] (6, 6.5)  -- node [right] {$\top$} (6, 4.5) ;
  \end{tikzpicture}
  \caption{}
    \end{centering}
    \end{subfigure}
  \caption{
  (a) The Two Paths While Loop Example
    (b) The Standard Execution Control Flow Graph}
      \label{fig:twoCountersWhile}
  \end{figure}
  }
\end{example}
\begin{enumerate}
  % \item  \textbf{The Abstract Execution Control Flow Graph} is generated in Figure~\ref{fig:threeNestedWhile}(b).
  \item \textbf{Rewrite The Program into The Language Model in~\cite{GulwaniJK09}}
  \[
    \begin{array}{l}
      \kw{twoPathsWhile}(k, m, N) \triangleq \\
      \rpasum(n > 0 \land m > 0);
      \clabel{ \assign{i}{n} }^{0} ; 
      \clabel{ \assign{j}{0} }^{1} ; \\
      \rprepeat(\rpasum(\clabel{i > 0}^{2}); \\
      \qquad \qquad \rpchoose\Big\{ 
        (\rpasum(\clabel{j < m}^{3}); \clabel{\assign{j}{j + 1}}^{4}; 
      \clabel{\assign{i}{i - 1}}^{5}),\\
      \qquad \qquad \qquad \qquad(\rpasum(\clabel{j \geq m}^{3}); \clabel{\assign{j}{0}}^{6})\Big\}
      );\\
      \rpasum(\clabel{i \leq 0}^{1})
      \end{array}
    \]

  \item \textbf{Program Refinement}
  \\
  % The loop free transition paths are computed as follows,
  \[
    \begin{array}{l}
      \kw{twoPathsWhile}(k, m, N) \triangleq \\
      \clabel{ \assign{i}{n} }^{0} ; 
      \clabel{ \assign{j}{0} }^{1} ; \\
      \rpchoose\Big\{ 
        \rprepeat(\rprepeat(\rpasum(\clabel{i > 0}^{2} \land \clabel{j < m}^{3}); \clabel{\assign{j}{j + 1}}^{4}; \clabel{\assign{i}{i - 1}}^{5});\\
      \qquad \qquad \qquad \qquad \rpasum(\clabel{i > 0}^{2} \land \clabel{j \geq m}^{3}); \clabel{\assign{j}{0}}^{6}),
      \\ \qquad \qquad 
      \rprepeat(\rpasum(\clabel{i > 0}^{2} \land \clabel{j < m}^{3}); \clabel{\assign{j}{j + 1}}^{4}; \clabel{\assign{i}{i - 1}}^{5}),\\
      \\ \qquad \qquad  \eskip
      \Big\};\\
      \rpasum(\clabel{i \leq 0}^{1})
      \end{array}
    \]
  % \[
  % \tpath_0 ; L_1: \rprepeat(\tpath_1; LOOP2: \rprepeat(\tpath_2; LOOP3 : \rprepeat(\tpath_3); \tpath_4); \tpath_5); \tpath_6
  % \]
Let $\rho_1 = \rpasum(\clabel{i > 0}^{2} \land \clabel{j < m}^{3}); \clabel{\assign{j}{j + 1}}^{4}; \clabel{\assign{i}{i - 1}}^{5}$
\\
and $\rho_2 = \rpasum(\clabel{i > 0}^{2} \land \clabel{j \geq m}^{3}); \clabel{\assign{j}{0}}^{6}$
  \item \textbf{Bound Computation}:
  \\
  % $L_1$, $L_2$ and 
  Let the $L$ denote the while loop at location $2$.
  % , $3$ and $6$ respectively.
  \\
  Step-by-Step of the BOUND computation in Figure~5 in \cite{GulwaniJK09}.
  \\
  \newcommand{\BD}{\mathcal{B}}
  $\BD(\kw{twoPathsWhile})$:
  \[
    \begin{array}{l}
      \BD(\kw{twoPathsWhile})  \triangleq  
      (c' + \max\{c'', c_1, c_2\} + c''', \emptyset \cup Z_1 \cup Z_2) 
      \\ \qquad
  \textbf{where} ~(c', \emptyset) \triangleq  \BD(\clabel{ \assign{i}{n} }^{0} ; \clabel{ \assign{j}{0} }^{1})
  \\ \qquad
  \textbf{and} ~(c'', \emptyset) \triangleq  \BD(\eskip)
  \\ \qquad
  \textbf{and} ~(c''', \emptyset) \triangleq  \BD( \rpasum(\clabel{i \leq 0}^{1}))
  \\ \qquad
  \textbf{and} ~(c_1, Z_1) \triangleq 
   \BD(L:  \rprepeat(\rprepeat(\rho_1); \rho_2))
      \\ \qquad
  \textbf{and} ~(c_2, Z_2) \triangleq  \BD(L: \rprepeat(\rho_1))
  %     \\ \qquad
  % \textbf{and} ~(c_3, Z_3) \triangleq  \BD(\rpasum(\clabel{i \geq k}^{1})) 
      \\ \qquad = (4, \{(2, L), (3, L)\})
    \end{array}
\]
\[
\begin{array}{l}
  \BD(L:  \rprepeat(\rprepeat(\rho_1); \rho_2)) \triangleq (0, Z \cup (c, L)) \\ \qquad
  \textbf{where} ~ c = c' + \sum\limits_{(c'', L'') \in Z' \land L = Parent(L'')}
  % \text{Definition}
  \\ \qquad
  \textbf{and} ~
  Z = \{(c'', L'') | (c'', L'') \in Z' \land L \neq Parent(L'')\}
  \\
  \qquad
  \textbf{and} ~ (c', Z') = \BD(\rprepeat(\rho_1); \rho_2))
  \\ \qquad = (0, \{(2, L), (3, L)\})
\end{array}
\]
\[
\begin{array}{l}
  \BD(\rprepeat(\rho_1); \rho_2))  \triangleq (c + 2, Z \cup \emptyset) 
  \\ \qquad
  \textbf{where} ~ (c, Z) \land \BD(L : \rprepeat(\rho_1))
  \\ \qquad = (2, \{( 3, L)\})
\end{array}
\]
\[
\begin{array}{l}
  \BD(L : \rprepeat(\rho_1))  \triangleq (0, Z \cup (c, L)) 
  \\ \qquad
  \textbf{where} ~ c = c' + \sum\limits_{(c'', L'') \in Z' \land L = Parent(L'')}
  % \text{Definition}
  \\ \qquad
  \textbf{and} ~
  Z = \{(c'', L'') | (c'', L'') \in Z' \land L \neq Parent(L'')\}
  \\
  \qquad
  \textbf{and} ~ (c', Z') = \BD(\rho_1)
  \\ \qquad = (0, \{(3, L)\})
\end{array}
\]
\[
\begin{array}{l}
  \BD(\rho_1)  \triangleq (3, \emptyset) 
\end{array}
\]
BOUND($\kw{twoPathsWhile}$):
\[
\begin{array}{l}
  BOUND(\kw{twoPathsWhile})
  \triangleq c + \sum\limits_{(c', L) \in Z}(c' \times T(L)) \textbf{where} ~ (c, Z) = (4, \{(2, L), (3, L)\})
  \\ \qquad 
  = 4 + 2 \times T(L) + 3 * T(L)
  \\ \qquad
  T(L) = \lfloor \frac{n}{m} \rfloor + m \times \lfloor \frac{n}{m} \rfloor
  %  \land c' = 5 + (6 + 2 \times N) \times m
  % \\ \qquad = 3 + 5 \times n + 6 \times m \times n + 2 \times m \times n \times N
\end{array}
\]
This step calls the perfect external
\\
$BOUNDFINDERD(INIT(L, 0, 2), NEXT(L, 0, 2), \{m, n\})$.
It computes $T(L)$, which is the number of iterations for the while loop at location $1$.
We assume that it perfectly computes the $T(L) = \lfloor \frac{n}{m} \rfloor + m \times \lfloor \frac{n}{m} \rfloor$.
\end{enumerate}
%
% \clearpage
% \appendix
% \addcontentsline{toc}{section}{Appendices}
% \section*{Appendices}


\clearpage
\bibliographystyle{plain}
\bibliography{main.bib}
\end{document}



