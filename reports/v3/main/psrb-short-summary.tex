\subsection{Abstract Transition Graph}
\label{sec:abs_prog}
% \textbf{Step 1: Program Abstract Execution Control Flow Graph}
An \emph{Abstract Transition Graph}, $\absG(c) =(\absV(c), \absE(c))$ for a program $c$ is composed of
a vertex set $\absV(c)$ and an edge set $\absE(c)$.
% For a program $c$, this analysis first generates its abstract execution control flow graph notated as follows,
% \[\absG(c) =(\absV(c), \absE(c))\]
%
\\
Every 
vertex $l \in \absV(c)$ corresponds to a program point $l$, which is a unique
label of a command in this program.
% $\absV(c)$ is the set of $c$'s all program points,
\\
Each edge $(l, dc, l') \in \absE(c)$ is an abstract transition
between two program points $l, l'$. There is an edge from $l$ to $l'$ if and only if
the command with label $l'$ can execute right after the execution of the command with label $l$.
% if and only if there is a control flow between two program points.
Each edge is annotated by a constraint $dc$ generated from the command with label $l$.
% from the set $\dcdom^{\top}$.
% The constraint set contains the different constraints and the boolean expressions. 
% A different constraint is an inequality of form $x \leq y + v$ where 
% $x, y$ are program variables and $v$ is either a 
% % $y \in \mathcal{VAR}$ and $v \in \constdom$.
% % The \emph{Symbolic Variables} $\constdom = \mathbb{N} \cup \inpvar$ is the 
% % set of 
% natural number, the infinity or an input variable. 
\subsection{Program Refinement}
\label{sec:refine}
Two steps:
\begin{enumerate}
\item \textbf{Rewrite the Program}
This step transforms the program into the program model in paper~\cite{GulwaniJK09}. 
\\
It first collects all \emph{transition paths} $\tpath$ for the program $c$ from its abstract transition graph $\absG(c)$.
%
Each transition path $\tpath$ corresponds to a path in \cite{GulwaniJK09}.
% %
% \begin{defn}[Simple Tansition Path]
% A path in the program $c$ abstract transition graph $\absG(c) = (\absV(c), \absE(c))$ is a transition path
% $\tpath \in \paths(\absG(c))$ if and only if,
%   \begin{itemize}
%   \item a vertices sequence $(l_0, \cdots, l_n)$, where $l_i \in \absV(c)$ for every $i = 0, \cdots, n$ and
%   %
%   \item an edge sequence $(e_1, \cdots, e_n)$, where $e_i = (l_{i - 1}, dc_i, l_{i}) \in \absE(c)$ for every $i = 1, \cdots, n$,
%   \end{itemize}
%   %
%   satisfying:
%   \begin{itemize}
%     \item $l_i \neq l_j$ for every $i = 0, \cdots, n$ and $j = 0, \cdots, {n - 1}$,
%     \item $l_0$ is either the program point of a loop header or the program entrance ($l_0 = 0$),
%     \item and $l_n$ is either the program point of a loop header or the program exit ($l_n = \lex$).
%   \end{itemize}
%   \end{defn}
%
\\
Then it rewrites these paths by the syntax in \cite{GulwaniJK09} and preserving the same semantics as follows,
\\
1. For each while loop with header at program point $l$, we
collect all the paths $\tpath_1, \cdots , \tpath_n$ with $n \geq 1$ that start and end both at $l$, 
and then rewrite the while into  $\rprepeat(\rpchoose{\tpath_1, \cdots, \tpath_n})$ or $\rprepeat(\tpath_1)$ if $n = 1$.
\\
2. Each if statement that is not in any loop with two branches $\tpath_1, \tpath_2$
is rewritten into $\rpchoose{\tpath_1, \tpath_2 }$.
\\
3. Then, every $\rpchoose{\tpath_1, \tpath_2 }$,  $\rprepeat(\rpchoose{\tpath_1, \cdots, \tpath_n})$ 
and $\tpath$ which isn't in if and while statement 
are rewritten into program sequence  by the order of their starting and ending points,
$\cdots; \tpath;\cdots;  \rpchoose{\tpath_1, \tpath_2 }; \cdots; \rprepeat(\rpchoose{\tpath_1, \cdots, \tpath_n})$.
\\
Then, $\cdots; \tpath;\cdots;  \rpchoose{\tpath_1, \tpath_2 }; \cdots; \rprepeat(\rpchoose{\tpath_1, \cdots, \tpath_n})$
is the transformed program $\rprog^{T}$ for $c$.
%  for its two branches.
% %
% \\
% For each $\tpath$ or $\rpchoose{\tpath_1, \cdots}$, if their starting point and ending point are the same, 
% then creating $\rprepeat(\tpath)$ or $\rprepeat(\rpchoose{\tpath_1, \cdots})$ for them.
%
\item \textbf{Refined Program}
This step invokes the external algorithm REFINE from paper~\cite{GulwaniJK09} and compute the 
refined program $\rprog$ for a program $c$ given the input $\rprog^{T}$.
\end{enumerate}
\subsection{Ranking Function Computation}
\label{sec:ranking}
% It estimates the bounds on the maximum value of counter variables, and
% the iteration times
% % of the ranks / local bounds 
% for each edge in the constraint program $\absG(c)$, using the method in \cite{sinn2017complexity}.
% There are three sub-steps as follows.
This step computes a 
\textbf{ranking function / local bound
\footnote{\textbf{ranking function} is the named used in \cite{SinnZV14}
and \textbf{local bound} is the name used in \cite{ZulegerGSV11}, \cite{sinn2017complexity}.
We choose the name \emph{ranking function} in formal definition and use both names to refer to the same meaning.
We choose the name \emph{ranking function} in formal definition and use both names to refer to the same meaning.
% refer to the two names as the same meaning in this paper.
}}
Then it estimates a bound on the maximum value of each ranking function as well as
% , and
the bound on the execution times of the corresponding edge in a path-insensitive manner.
\\
For each edge $\absevent \in \absE(c)$,
$\locbound(\absevent, c)$ is the \textbf{Ranking Function} assigned to $\absevent$.
\\
For each ranking function $\locbound(\absevent, c)$, $\varinvar(\locbound(\absevent, c))$ is
the \textbf{Ranking Function Bound} on its maximum value.
\\
For each edge $\absevent \in \absE(c)$, the $\absclr(\absevent, c)$ is the \textbf{Path-insensitive Transition Bound}
% $\absclr(\absevent) \in \mathcal{A}_{\lin}$ 
% for the edge $\absevent \in \absG(c)$,
% is the bound 
on the execution times of this edge in $c$.
%
\subsection{Outside-In Algorithm}
\label{sec:outinalg}
For each refined loop $\rprog_l$ with header at $l$ in a refined program $\rprog$, 
this algorithm
% computes a bound, $\outinB(\rprog, \rpattern) \in \mathcal{A}_{in}$ on its iteration numbers locally named \textbf{Outside-In} bound.
% It recursively computes the iteration numbers for every
% sub-program (a sub-structure) nested in this refined loop locally as well.
%  of every repeat pattern 
computes a bound, $\outinB(\rprog, \rpattern) \in \mathcal{A}_{in}$ on its iteration numbers locally named \textbf{Outside-In} bound.
It recursively computes the iteration numbers for every
sub-program (a sub-structure) nested in this refined loop locally as well.
\\
$\outinB(\rprog, \rprog_l)$
computes the bound locally
% is the \textbf{Outside-In} bound for a repeat pattern $\rpattern$ 
in the sense that
% for a repeat pattern 
if $\rprog_l$ is nested
in another while loop $\rprog_{l'}$,
i.e., $\rprog_{l'} = \rprepeat(\rprog_l)$,
it only estimates
$\rprog_{l}$'s iteration number within
$\rprepeat(\rprog_l)$ by assuming that the loop $l'$ executes once.
% \\% \highlight{\textbf{Notations / Operators:}}
\\ 
\highlight{Computation:}
\\
The \textbf{Outside-In} bound, 
$\outinB(\rprog, \rprog_l) \in \mathcal{A}_{in}$
for a refined loop $\rprog_l$in a refined program $\rprog$
is computed recursively as follows,
% for every nested repeat patterns in $\rprog$,
% \\
% $\outinB(\rpchoose{\rprog_1, \cdots, \rprog_m}) =  \max\{\outinB(\rprog_1), \cdots, \outinB(\rprog_m)\}$
\\
$\outinB(\rprog, \rprepeat({\rprog'})) =  \frac{\rfinit(\rprog, \rprog') - \rffinal(\rprog, \rprog')}{\varGD(\rprog, \rprog')}$
\\
$\outinB(\rprog, \rprog_1; \rprog_2) =  \outinB(\rprog, \rprog_1)+ \outinB(\rprog,  \rprog_2)$
\\
$\outinB(\rprog, \tpath) =  1$.
\\
% The \emph{State}: 
% $\absstate \in \mathcal{P}(\dcdom^{\top})$ : conjunctions of the edge annotations.
% %  constraints.
% \\
% For a refined program ($\rprog$), there are the 
% \\
% \emph{Initial State} ($\rfinit(\rprog)$), 
% \emph{Final State} ($\rffinal(\rprog)$), and \emph{Next State} ($\rfnext(\rprog)$)  $\in \absstate$.
% % The \emph{Initial State}: $\rfinit : \rprog \to \absstate $.
% \\
% The \emph{Variable Grade Decedent}: $\varGD : \rprog \to \mathcal{A}_{in}$, is the set of the variables' variation in one iteration;
% % The \emph{Initial State}: $\rfinit : \rprog \to \absstate $.
The computations for $\rfinit(\rprog, \rprog)$,
$\rffinal(\rprog, \rprog)$, $\rfnext(\rprog, \rprog)$ and $\varGD(\rprog, \rprog) \in \mathcal{A}_{in}$
are in the full version.
% $\outinB(\rprog, \rprog)$
% computes the bound locally
% % is the \textbf{Outside-In} bound for a repeat pattern $\rprog$ 
% in the sense that
% % for a repeat pattern 
% if $\rprog'$ that is nested
% in another repeat pattern $\rpattern$, i.e., $\rpattern = \rprepeat(\rpattern')$,
% it only estimates
% $\rpattern'$'s iteration number within $\rpattern$ by assuming that $\rprepeat(\rpattern')$ executes once.
% % \\% \highlight{\textbf{Notations / Operators:}}
% \\ 
% \highlight{Computation:}
% \\
% The \textbf{Outside-In} bound, $\outinB(\rprog, \rpattern) \in \mathcal{A}_{in}$
% for a repeat pattern $\rpattern$ in a refined program $\rprog$
% is computed recursively as follows,
% % for every nested repeat patterns in $\rprog$,
% % \\
% % $\outinB(\rpchoose{\rprog_1, \cdots, \rprog_m}) =  \max\{\outinB(\rprog_1), \cdots, \outinB(\rprog_m)\}$
% \\
% $\outinB(\rprog, \rprepeat({\rpattern})) =  \frac{\rfinit(\rprog, \rpattern) - \rffinal(\rprog, \rpattern)}{\varGD(\rprog, \rpattern)}$
% \\
% $\outinB(\rprog, \rprog_1; \rprog_2) =  \outinB(\rprog, \rprog_1)+ \outinB(\rprog,  \rprog_2)$
% \\
% $\outinB(\rprog, \tpath) =  1$.
% \\
% % The \emph{State}: 
% % $\absstate \in \mathcal{P}(\dcdom^{\top})$ : conjunctions of the edge annotations.
% % %  constraints.
% % \\
% % For a refined program ($\rprog$), there are the 
% % \\
% % \emph{Initial State} ($\rfinit(\rprog)$), 
% % \emph{Final State} ($\rffinal(\rprog)$), and \emph{Next State} ($\rfnext(\rprog)$)  $\in \absstate$.
% % % The \emph{Initial State}: $\rfinit : \rprog \to \absstate $.
% % \\
% % The \emph{Variable Grade Decedent}: $\varGD : \rprog \to \mathcal{A}_{in}$, is the set of the variables' variation in one iteration;
% % % The \emph{Initial State}: $\rfinit : \rprog \to \absstate $.
% The computations for $\rfinit(\rprog, \rpattern)$,
% $\rffinal(\rprog, \rpattern)$, $\rfnext(\rprog, \rpattern)$ and $\varGD(\rprog, \rpattern) \in \mathcal{A}_{in}$
% are in the full version.

\subsection{Inside-Out Algorithm}
\label{sec:inoutalg}
% For the refined program $\rprog \in \mathcal{RP}$, the \textbf{Inside-Out Algorithm}
% computes the reachability-bound on the execution numbers for every transition path $\tpath$ in this program $\rprog$,
% through the following steps.
For every transition path $\tpath$ in a refined program $\rprog$, 
this algorithm
computes a bound, $\outinB(\rprog, \tpath) \in \mathcal{A}_{in}$
%  on its iteration numbers locally.
% computes the bound
on its execution numbers globally and path-sensitively.
This bound is called \textbf{Inside-Out} bound and computed
% for every transition path $\tpath$ in this program $\rprog$,
through the following steps.
%
\begin{enumerate}
  \item \emph{Repeat Chain Bound:} $\rpchB(l, \tpath, \rprog) \in \mathcal{A}_{in}$.
  \\
  For every transition path $\tpath$ in a refined program $\rprog$,
  this step computes a path-sensitive
  bound on $\tpath$'s iteration times within its closest enclosing while loop, named \textbf{Repeat Chain} Bound.
  \highlight{
    It first computes the repeat chains for $\tpath$, which
    filter out the execution of other transition paths that are nested in the same loop as $\tpath$.
    In this way, the repeat chain bound is path-sensitive
    because it only deals with $\tpath$'s execution over these chains.
    }
  \item \emph{Relative Loop Bound:} $\lpchB(l, \tpath, \rprog) \in \mathcal{A}_{\lin}$.
  \\
%   For every transition path $\tpath$
% and a loop header at location $l$ in a refined program $\rprog$,
% the \emph{Relative Loop Bound} $\lpchB(l, \tpath, \rprog) \in \mathcal{A}_{\lin}$ is a symbolic expression in $\mathcal{A}_{\lin}$.
% \\
% This expression is a sound bound on the iteration numbers of loop $l$ relative to the transition path $\tpath$.
% This $\tpath$'s closest enclosing loop has the loop header at $l'$ and $l'$ is nested inside the loop $l$.
% % \\
% It estimates the iteration numbers of loop $l$ such that during these iterations, the nested loop $l'$ is executed.
%
For every transition path $\tpath$
and a loop whose header is at program point $l$ in a refined program $\rprog$,
 $\lpchB(l, \tpath, \rprog) \in \mathcal{A}_{\lin}$ computes a symbolic expression
as the \emph{Relative Loop Bound} for the $\tpath$ and loop $l$.
\\
It bounds the iteration numbers of the loop $l$ w.r.t.
the inner loop $l'$ that is $\tpath$'s closest enclosing loop,
such that,
% the transition path $\tpath$.
% This $\tpath$'s closest enclosing loop has the loop header at $l'$ and $l'$ is nested inside the loop $l$.
% \\
% It estimates the iteration numbers of loop $l$ such that 
during these iterations of loop $l$, the nested loop $l'$ is executed.
\highlight{
% the for every program control location,
% how many times the innermost loop of this control location will be touched w.r.t. every
% outside loop it is nested in.
This is distinguished from the traditional methods, which only compute the bound on the iteration number
for the inner loop w.r.t one iteration of the outside loop where it is nested.
\emph{Relative Loop Bound} for $\tpath$ and $l$ bounds the number of the iterations for
the outside loop $l$,
% w.r.t. an inner loop.
%  of every control location.
such that during these iterations of the outside loop $l$, the inner loop $l'$ is entered. 
}
\item The \emph{Inside-Out Bound} ($\inoutB(\tpath, \rprog) \in \mathcal{A}_{in}$).
\\
For every transition path $\tpath \in \rprog$,
its \emph{Inside-Out Loop Bound}
 $\inoutB(\tpath, \rprog) \in \mathcal{A}_{in}$ is 
% Compute 
the path sensitive reachability-bound on $\tpath$'s execution numbers.

% \highlight{\textbf{Omitted Computations:}}
\begin{defn}[{Inside-Out Loop Bound}]
  \label{def:outin_bound}
  Given a refined program $\rprog$, for every transition path $\tpath \in \rprog$, 
  its \emph{Inside-Out Loop Bound}
  $\inoutB(\tpath, \rprog)$ is 
 % Compute 
$
  \prod\limits_{l \in \lpch(\tpath, \rprog)} \rpchB(l, \tpath, \rprog).
% lpRB(\rprog_i, \tpath) 
% ~ \middle\vert~ l \in lp\mathcal{C}(\tpath) \right
$
\end{defn}
% For $chain \in rpchain(\tpath)$:
% where $lpchains(\tpath)$ is set of $lpchains(\tpath)$ containing all the loop chains of $\tpath$.
\end{enumerate}
\subsection{Path-sensitive Reachability-Bound Computation}

For every program control location $l \in \lvar(c)$, with $\rprog$ as its refined program,
%  in a program $c$,
its path-sensitive reachability-bound ($\psRB(l, \rprog)$) is a symbolic sound bound on the executing times of $l$.
 \begin{defn}
  \label{def:label_psrb}
  Given a program $c$ with $\rprog$ as its refined program,
  %  with 
  % \emph{Global Loop Bound} $\inoutB(\tpath)$
  % computed for its every transition path $\tpath \in \rprog$  notated by $\inoutB(\tpath)$,
  %  for each of its transition path $\tpath \in \rprog$ 
  % with the \emph{Global Loop Bound}
  % computed as above, notated by $\inoutB(\tpath)$.
  for every label $l \in \lvar(c)$, $\psRB(c, l)$ is $\sum\limits_{\tpath \in \rprog \land 
  l \in \tpath} \inoutB(\tpath, \rprog)$.
  % \\
 \end{defn}