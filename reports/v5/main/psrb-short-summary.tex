\subsection{Abstract Transition Graph}
\label{sec:abs_prog}
% \textbf{Step 1: Program Abstract Execution Control Flow Graph}
An \emph{Abstract Transition Graph}, $\absG(c) =(\absV(c), \absE(c))$ for a program $c$ is composed of
a vertex set $\absV(c)$ and an edge set $\absE(c)$.
% For a program $c$, this analysis first generates its abstract execution control flow graph notated as follows,
% \[\absG(c) =(\absV(c), \absE(c))\]
%
\\
Every 
vertex $l \in \absV(c)$ corresponds to a program point $l$, which is a unique
label of a command in this program.
% $\absV(c)$ is the set of $c$'s all program points,
\\
Each edge $(l, dc, l') \in \absE(c)$ is an abstract transition
between two program points $l, l'$. 
There is an edge from $l$ to $l'$ if and only if
the command with label $l'$ can execute right after the execution of the command with label $l$.
% if and only if there is a control flow between two program points.
The constraint $dc$ in each edge is generated from the command with label $l$, which describes the abstract execution of this command.
% from the set $\dcdom^{\top}$.
% The constraint set contains the different constraints and the boolean expressions. 
% A different constraint is an inequality of form $x \leq y + v$ where 
% $x, y$ are program variables and $v$ is either a 
% % $y \in \mathcal{VAR}$ and $v \in \constdom$.
% % The \emph{Symbolic Variables} $\constdom = \mathbb{N} \cup \inpvar$ is the 
% % set of 
% natural number, the infinity or an input variable. 
\subsection{Program Refinement}
\label{sec:refine}
1. It first rewrites the program $c$ by syntax in \cite{GulwaniJK09} and preserves the same semantics, based on $\absG(c)$.
\\
Each \emph{simple transition paths}, $\tpath \in \paths(\absG(c))$ in the rewritten program
corresponds to a path $\rho$ in the flatten program by Definition~4.1 in \cite{GulwaniJK09}.
%
It contains only the edges of atomic assignment or guard transition.
\\
2. Then it computes the 
refined program, $\rprog$ by Algorithm~1 in paper~\cite{GulwaniJK09}.
%
\subsection{Path-sensitive Reachability-Bound Analysis}
Given a program $c$ with its \emph{Abstract Transition Graph}, $\absG(c)$ and refined program $\rprog$,
\begin{enumerate}
  \item $\outinB(\rprog, \tpath)$, \emph{Path Local Reachability-Bound}:
  \\
  The local reachability-bound of a \emph{simple transition path}. 
  \\
  $\outinB(\rprog, \tpath)$ bounds the execution times of $\tpath$ when executing program $\rprog$,
  and $\rprog$ is the closest loop where $\tpath$ is nested.
  The local reachability-bound  considers only the influence of $\tpath$'s closest enclosing loop, i.e., $\kw{enclosed}(\tpath)$.
  %
  \[
    \begin{array}{rcl}
      \outinB(\tpath, \tpath) & \triangleq & 1 \\
      \outinB(\tpath', \tpath) & \triangleq & \highlight{0} \qquad \text{if } \tpath' \neq \tpath\\
      \outinB(\rprog_1;\rprog_2, \tpath) & \triangleq & \outinB(\rprog_1, \tpath) + \outinB(\rprog_2, \tpath) \\
      \outinB(\rpchoose{\rprog_1, \cdots, \rprog_m }, \tpath) & \triangleq 
      & \max\left\{ \outinB(\rprog_1, \tpath), \cdots, \outinB(\rprog_m, \tpath) \right\} \\
      \outinB(\rprepeat(\rprog'), \tpath) & \triangleq 
      & BD(\rprepeat(\rprog'))
       \times \outinB(\rprog', \tpath)
       \\
      \outinB(l: \rprog', \tpath) & \triangleq & \outinB(\rprog', \tpath) \\
      &  & \text{this case always equals to } 0 \\
    \end{array}
    \]
  
    The loop bound $BD(\rprepeat(\rprog')) = BOUND(\rprepeat(\rprog'))$ by paper\cite{GulwaniJK09}, which is precise for simple loop without nested loops. 
    We provide an alternative computation method based on \cite{sinn2017complexity} in full version.
    \item $\lpchB(l: \rprog, \tpath)$,
    {Loop Reachability-Bound}:
    \highlight{
      \[
        \lpchB(l: \rprog, \tpath) \triangleq
        \frac{\lpinit(\rprog, \tpath) - \rffinal(\rprog, \tpath)}{\lpinit(\rprog, \tpath) - \lpnext(\rprog, \tpath)}
      \]    
    }
      $\lpchB(l: \rprog, \tpath)$
      is the bound on iteration numbers of the outside loop $l$,
      such that,
      during these iterations, the nested loop $l' = \kw{enclosed(\tpath)}$ is executed, i.e., reached.
      
      $\lpchB(l:\rprog, \tpath)$ is over-approximated by
      $I(l, l') = \kw{BOUNDFINDERD(INIT(c, i, \absinit(\tpath)), NEXT(c, i, \absinit(\tpath)), V_{\ln})}$ in paper\cite{GulwaniJK09}.
      $I(l, l')$ computes the iteration numbers of $l'$ in one iteration of $l$.
      \item $\inoutB(\rprog, \tpath) $, Path Reachability-Bound:  
      \\
      The reachability-bound of a \emph{simple transition path}.
      \\
      $\inoutB(\rprog, \tpath) $ bounds the execution times of $\tpath$ during the execution of $\rprog$.
  %
  \[
    \begin{array}{rcl}
      \inoutB(\tpath, \tpath) & \triangleq & 1  \\
      \inoutB(\tpath', \tpath) & \triangleq & \highlight{0} \qquad \text{if } \tpath' \neq \tpath\\
      \inoutB(\rprog_1;\rprog_2, \tpath) & \triangleq & \inoutB(\rprog_1, \tpath) + \inoutB(\rprog_2, \tpath) \\
      \inoutB(l: \rprog_l, \tpath) & \triangleq & 
      \highlight{\outinB(\rprog', \tpath), \qquad \text{if } l = \kw{enclosed}(\tpath)}
      \\
      &  & 
      \highlight{
        \lpchB(l:\rprog_l, \tpath ) }
      \highlight{\times \max\limits_{l:\rprog_l = \kw{enclosed}(l':\rprog')}
     \{\inoutB(l':\rprog', \tpath)\} } \footnotemark, o.w. \\
      \inoutB(\rpchoose{\rprog_1, \ldots, \rprog_m }, \tpath) & \triangleq 
      & \max\left\{ \inoutB(\rprog_1, \tpath), \ldots, \inoutB(\rprog_m, \tpath) \right\} 
      \\
      \inoutB(\rprepeat(\rprog'), \tpath) & \triangleq & \outinB(\rprepeat(\rprog'), \rprog) \times \inoutB(\rprog', \tpath)\\
      &  & \text{this case will never be matched}
      \end{array}
    \]
        \footnotetext{There is only one non-zero $\inoutB(l':\rprog'', \tpath)$ where $\tpath \in \rprog''$, 
        all the others $\inoutB(l':\rprog'', \tpath)$ where $\tpath \notin \rprog''$ equal to $0$}%
        $\kw{enclosed}(\rprog)$: \text{the closest loop where } $\rprog$ is nested
        \\
        $\kw{enclosing}(\rprog)$: \text{all the loops which are nested inside } $\rprog$
      \item The \emph{Reachability-Bound} for program points.
    
    The reachability-bound of each program point $l \in \lvar(c)$, $\psRB(c, l)$
    \footnote{$l \in \tpath$: the $\in$ notation is abused to denote
    the program point $l$ is a vertex on this path.}
   \[ 
     \psRB(c, l) = 
     \sum\limits_{\tpath \in \rprog \land 
   l \in \tpath} 
   \inoutB(\rprog, \tpath)
   \]
   % 
   This bound sums up all the path reachability-bounds, $\inoutB(\rprog, \tpath)$ over the $\tpath$ which contains the program point $l$.
%
\end{enumerate}