\paragraph{Step 1: Program Abstract Execution Control Flow Graph}
For a program $c$, this analysis first generates its abstract execution control flow graph notated as follows,
\[\absG(c) =(\absV(c), \absE(c)).\]
%
$\absV(c)$ is the set of program $c$'s control location, each edge in $\absE(c)$ is a transition
between two control locations, annotated by a difference constraint \cite{sinn2017complexity} or a boolean expression.
\paragraph{Step 2: Program Rephrase and Refinement}
In order to analyze the reachability bound path-sensitively, we first rewrite this program through two steps.
\begin{enumerate}
  \item \emph{Program Rephrase through Path Collection on Abstract CFG}.
  \\
  The transition path
  $\tpath \in \paths(\absG(c))$, is a simple path on this program's abstract control flow graph $\absG(c)$.
  \\
  % For the part of the graph not in any SCC:
  \\
  $p \triangleq \tpath $ if $\tpath \in \paths(\absG(c))$ and $\tpath \not\in SCC(\absG(c))$;
  \\
  $p \triangleq \rpchoose\{p_1, p_2 \}$ if $p_1$ and $p_2$ has the same head and end;
  \\
  $p \triangleq p_1; p_2$ if head of $p_1$ is the same as head of $p_2$ and either $p_1$ or $p_2$ isn't a simple path. 
  % \\
  % For the part of the graph not in some SCCs:
  % \\
  % For every while loop with guard label $l$:
  % \\
  % $p \triangleq \tpath $ if $\tpath \in \paths(\absG(c))$
  \\
  $p \triangleq LOOP_l : \rprepeat(\rpchoose\{p_1, \cdots, p_m\})$ if head and end of $p'$ are both $l$;
  % and end of $p'$ are both $l$;
  % \\
  %
\item \emph{Program Refinement}.
\\
  Refined program:
  $\rprog \in \mathcal{RP}$.
  % For a Loop $L$,
  % computes all the transition Paths : $\tpath \in \absG(c)$
  % ->  $\rprog \in \mathcal{RP}$
  % in this loop and generate the refined statement.
  \\
  Given a rephrased program $p$, its refined program is computed as follows,
  \\
  $\rprog \triangleq \tpath $ if $p = \tpath$\\
  $\rprog \triangleq \rpchoose\{\rprog_1, \rprog_2 \}$ where $p \triangleq \rpchoose\{p_1, p_2 \}$ and 
    $\rprog_1$ and $\rprog_2$ are refined $p_1$ and $p_2$. 
    \\
  $\rprog \triangleq LOOP_l : \kw{REFINE(p_w)}$  if $p = \rprepeat(p_w)$ and  $\kw{REFINE(p_w)}$ is the algorithm in \\
  $\rprog \triangleq \kw{REFINE(p_1)}; \kw{REFINE(p_2)}$  if $p = p_1; p_2$ 
  \cite{sinn2017complexity}.
\end{enumerate}
% \\
\paragraph{Step 3: Outside-In Algorithm}
For the refined program $\rprog \in \mathcal{RP}$, the \emph{Outside-In Algorithm}
computes the path-sensitive local bound 
% for the refined program $\rprog$ recursively 
$rLB: \rprog \to \mathcal{A}_{in}$ as follows,
% \\
%  $rLB: \rprog \to \mathcal{A}_{in}$
\\
The \emph{Refined Abstract State}: 
$\absstate \in \mathcal{P}(\dcdom^{\top})$ : conjunctions of difference constraints.
\\
The \emph{Initial Refined Abstract State}: $rInit : \rprog \to \absstate $.
% % , computed as follows,
% % \[
% %   rInit(\rprog) \triangleq 
% %   \bigwedge_{x \in VAR(\rprog)}
% %   \left\{ 
% %   x = v ~\middle\vert~ 
% %   v = arg\max_{l_1}\left\{ 
% %     v~\middle\vert~ 
% %     (v, \absevent) \in \reset(x) 
% %     \land \absevent = (l_1, dc, l_2)
% %     \land l_1 \leq \absinit(rprog)
% %     \right\}
% %   \right\}
% %   \]
% \\
% The $rInit : \rprog \to \absstate $  can also be computed using $INIT(c, l_0, \absinit(\rprog))$ algorithm in \cite{GulwaniJK09}. 
% $INIT(c, l_0, \absinit(\rprog))$ computes the equivalent results.
\\
The \emph{Final Refined Abstract State} $rFinal : \rprog \to \absstate $.
% \[
%   rFinal(\rprog) \triangleq 
%   \bigwedge_{x \in VAR(\rprog)}
%   \left\{ 
%     \varinvar(x)
%   \right\} \land \neg Invariant(\rprog)
%   \]
\\
The \emph{Variable Grade Decedent}: $varGD : \rprog \to \mathcal{A}_{in}$, is the set of the variables' variation in one iteration;
\\
Given a refined program $\rprog$, its path-sensitive local reachability bound is computed as follows,
\\
$rLB(\rpchoose\{\rprog_i\}) =  \max\{rLB(\rprog_i)\}$
\\
$rLB(\rprepeat\{\rprog\}) =  \frac{rInit(\rprog) - rFinal(\rprog)}{varGD(\rprog)}$
\\
$rLB(\rprepeat\{\rprog\})$ can also be computed using the algorithm 
$\kw{BOUNDFINDERD(INITD(c, l_0, \absinit(\rprog)), NEXTD(c, l_0, \absinit(\rprog), V_{\lin} )}$ 
in paper~\cite{GulwaniJK09} with equivalent results.
\\
$rLB(\rpseq(\rprog_1, \rprog_2)) =  rLB(\rprog_1)+ rLB( \rprog_2)$
\\
$rLB(\tpath) =  1$.
\\
% Formally defined as follows,
% \begin{defn}[Path-Sensitive Local Reachability Bound]
%   \label{def:pathsensitive_lrb}
% \end{defn}
The computations of the operations: $rInit(\rprog)$,
$rFinal(\rprog)$ and $rNext(\tpath)$
are formally defined as follows,
\begin{itemize}
\item For a refined program $\rprog$, its initial refined abstract $rInit(\rprog) \in \mathcal{P}(\absstate)$ is computed as follows,
\[
  rInit(\rprog) \triangleq 
  \bigwedge_{x \in VAR(\rprog)}
  \left\{ 
  x = v ~\middle\vert~ 
  v = arg\max_{l_1}\left\{ 
    v~\middle\vert~ 
    (v, \absevent) \in \reset(x) 
    \land \absevent = (l_1, dc, l_2)
    \land l_1 \leq \absinit(rprog)
    \right\}
  \right\}
  \]
% \\
The $rInit : \rprog \to \absstate $  can also be computed using $INIT(c, l_0, \absinit(\rprog))$ algorithm in \cite{GulwaniJK09}. 
$INIT(c, l_0, \absinit(\rprog))$ computes the equivalent results.
%
\item  The program $\rprog$'s final refined abstract state $rFinal(\rprog) \in \mathcal{P}(\absstate) $ is computed as follows, 
\[
  rFinal(\rprog) \triangleq 
  \bigwedge_{x \in VAR(\rprog)}
  \left\{ 
    \varinvar(x)
  \right\} \land \neg Invariant(\rprog)
  \]
  % \\
  \item  The program $\rprog$'s Variable Gradient Decent set is computed as follows,
\\
$varGD(\rpchoose\{\rprog_i\}) =  \max\{varGD(\rprog_i)\}$
\\
$varGD(\rprepeat\{\rprog\}) =  rLB(\rprepeat\{\rprog\}) \times {varGD(\rprog)}$
\\
$varGD(\rpseq(\rprog_1, \rprog_2)) =  varGD(\rprog_1)+ varGD( \rprog_2)$
\\
$varGD(\tpath) =  rInit(\tpath) - rNext(\tpath)$
\\
\item  For a transition path $\tpath \in \paths(\absG(c))$, its $rNext(\tpath)$ is computed as follows,
%
\[
  \begin{array}{l}
  rNext(\tpath) \triangleq 
  \bigwedge\limits_{x \in VAR(\rprog)}
  \left\{ 
    \begin{array}{l}
  x =   \sum\limits_{(x, \absevent) \in \inc(x) }\left\{ 
    \varinvar(y) + c ~\middle\vert~ \absevent = (l, (y, c), \_) \land l \in \lvar(\rprog)\right\}
    \\ \qquad 
    - \sum\limits_{(x, \absevent) \in dec(x) }\left\{ 
      \varinvar(y) + c ~\middle\vert~ \absevent = (l, (y, c), \_) \land l \in \lvar(\rprog) \right\}
    \end{array}
  \right\}
  \end{array}
\]
\end{itemize}
%
\paragraph{Step 4: Inside-Out Algorithm}
For the refined program $\rprog \in \mathcal{RP}$, the \emph{Inside-Out Algorithm}
computes the path-sensitive global bound for every transition path in this program $\rprog$ through following steps.

\begin{enumerate}
  \item \emph{Local Repeat Chain Bound}.
  \\
  For every transition path $\tpath$, its \emph{Local Repeat Chain Bound} is the 
  path sensitive local bound for it, inside its closet while loop.
  It is computed as follows,
  \begin{enumerate}
\item \emph{Repeat Chain Set}
% For a refined while loop program $\rprog_{l} = LOOP_l : \rprog \in \mathcal{RP}$, 
% with its refined statement $\rprog \in \mathcal{RP}$,
\\
For each transition path in the refined program $\tpath \in \rprog$, 
its repeat chain set $rp\mathcal{C}(LOOP, \tpath) \in \mathcal{P}(\mathcal{P}(\rprog))$
 is a set of repeat chains. 
 Every repeat chain is a list of refined program contained inside the same while loop $LOOP$ as the $\tpath$. It is computed as follows,
\\
% 1. collect the repeat chain: 
$rp\mathcal{C}(LOOP_l, \tpath) \in \mathcal{P}(\mathcal{P}(\rprog))$
\\
 $rp\mathcal{C}(LOOP_l, \tpath) \in \mathcal{P}(rpch(LOOP_l, \tpath))$
  \\
  $rpch(LOOP_l, \tpath) \in \mathcal{P}(\rprog)$\\
  $rpch(LOOP_l, \tpath) \triangleq \rprog_n \to \rprog_{n-1} \to \cdots \to \tpath $
 such that \\
%  $\rprog_{n}= \rprepeat^{L}(\cdots)$ 
 and
 $\rprog_{i}= \rprepeat(\cdots, \rprog_{i - 1}, \cdots)$ and
 there isn't any loop label (i.e., $LOOP'$) or $\rprepeat_i$ between $\rprog_{i}$ and $\rprog_{i - 1}$ for $i = n, \cdots, 1$.
% \\
% 2. 
% collect the loop chain: 
% $lpchain : \tpath \to \mathcal{P}(\mathcal{P}(\rprog)))$
% \\
% $lpchain(\tpath) = \rprog_n \to \rprog_{n-1} \to \cdots \to \tpath$
% % such that there is at least a $\rpchoose$ and isn't consecutive repeats $\rprepeat$ (i.e., at most one 
% % $\rprepeat$) between any $\rprog_{i - 1}$ and $\rprog_{i}$ for $i = n, \cdots, 1$.
% $\rprog_{i}= \rprepeat^{L_i}(\cdots, \rprog_{i - 1}, \cdots)$ and
%  there isn't any loop (i.e., $\rprepeat^{L}$) between $\rprog_{i}$ and $\rprog_{i - 1}$ for $i = n, \cdots, 1$.
% \\
% 2. Compute the local bound for every repeat chain as follows:
% \\
% $rpLB(L, \tpath) = \prod\limits_{\rprog_i \in rpchain(\tpath)}
% % \frac{chsInit(\rprog_i, \tpath) - chsFinal(\rprog_i, \tpath)}{varGD(\rprog_i, \tpath)}
% rLB(\rprog_i)$
% % \\
% where $\rprepeat^{L}$ is the closet loop containing $\tpath$, $\rprepeat^{L}(\cdots, \rprog, \cdots)$.
% \\
% 4. Compute the nested local bound for every loop chain as follows, for every of 
% $(\rprog_i, \tpath)$ such that $\rprog_i \in lpchain(\tpath)$,
% \\
% $lpLB(\rprog_i, \tpath) = 
% % \prod\limits_{\rprog_i \in lpchain(\tpath)}
% \frac{lpInit(\rprog_i, \tpath) - lpFinal(\rprog_i, \tpath)}{varGD(\rprog_i, \tpath)}$
% \\
\item  \emph{Local Repeat Chain Bound}.
\\
For every transition path $\tpath$ in its closet while loop $LOOP$, the \emph{Local Repeat Chain Bound} $rpLB(LOOP, \tpath)$ is computed as follows,
% $rpRB: \tpath \to \mathcal{A}_{in}$, $chsRB: (\rprog \times \tpath) \to \mathcal{A}_{in}$
% \\
% For each transition path $\tpath \in \rprog$,
% \\
% 1. First compute the path sensitive reachability choosing bound through their choose chain:
% \\
% $chsRB(\rprog_n, \tpath) = \prod\limits_{\rprog_i \in lpchain(\rprog_n, \tpath)}
% \frac{chsInit(\rprog_i, \tpath) - chsFinal(\rprog_i, \tpath)}{varGD(\rprog_i, \tpath)}$
  \\
  $rpLB(LOOP, \tpath) = \max \left\{ \prod\limits_{\rprog_i \in ch}  rLB(\rprog_i) 
  ~\middle\vert~ ch \in rp\mathcal{C}(LOOP_l, \tpath) \right\}
  $;
  \\
  $rpLB(LOOP, \tpath) = \bot
  $ if $LOOP$ isn't the closest while loop containing $\tpath$.
  %
  \end{enumerate}
% \\
% \\
% 2. Then compute the path sensitive reachability repeating bound for $\tpath$ as:
% \\
% $RB(\tpath) = \max\limits_{l \in lpchains(\tpath)} \{rpLB(n, \tpath) \prod\limits_{\rprog_i \in l} lpRB(\rprog_i, \tpath) \}$
% % For $chain \in rpchain(\tpath)$:
% where $lpchains(\tpath)$ is set of $lpchains(\tpath)$ containing all the loop chains of $\tpath$.
%
\item \emph{Nested Loop Bound}.
\\
For every transition path $\tpath \in \rprog$, 
its \emph{Nested Loop Bound} $lpLB(LOOP_i, \tpath) \in \mathcal{A}_{\lin}$ is the global path-sensitive bound within each while loop $LOOp_i$ it is nested in.
% {Collect Loop Chains for Nested Loop}
It is computed as follows,
\begin{enumerate}
  \item \emph{Loop Chain Set}.
  \\
  For each transition path in the refined program, $\tpath \in \rprog$, its \emph{Loop Chain Set} ($lp\mathcal{C}(\tpath) \in \mathcal{P}(\mathcal{P}(\rprog))$) is the set of 
  loop chains. Each loop chain $lpch(\tpath) \in \mathcal{P}(\rprog)$ is a list of nested loop label where 
  $\tpath$ is contained inside.
  \\
For each transition path in the refined program $\tpath \in \rprog$, 
its \emph{Loop Chain Set} is collected as follows: 
% repeat chain.
% then there is loop tree as follows for a program:
% $P = \rprepeat^{L_n}(\cdots, \rprepeat^{L_{n'}}(\cdots), \cdots, \rprepeat^{L_{n''}}(\cdots))$.
% \\
% For each transition path $\tpath$, we:
% from the head of the while loop.
\\
% 1. firstly collect the loop chain: 
$lp\mathcal{C}(\tpath) \in \mathcal{P}(\mathcal{P}(\rprog)) \triangleq \mathcal{P}(lpch(\tpath))$
\\
$lpch(\tpath) \in \mathcal{P}(\rprog) \triangleq 
LOOP_n \to LOOP_{n-1} \to \cdots \to \tpath$
\\
% such that there is at least a $\rpchoose$ and isn't consecutive repeats $\rprepeat$ (i.e., at most one 
% $\rprepeat$) between any $\rprog_{i - 1}$ and $\rprog_{i}$ for $i = n, \cdots, 1$.
such that 
$\rprog_{i}= LOOP_i : (\cdots, LOOP_{i - 1} : \rprog_{i-1}, \cdots)$ and
 there isn't any loop label (i.e., $LOOP'$) between $\rprog_{i}$ and $\rprog_{i - 1}$ for $i = n, \cdots, 1$.
\item  \emph{Nested Loop Bound}.
\\
Given a loop label $LOOP_i \in \rprog$ and a transition path $\tpath \in \rprog$ in a refined program $\rprog$,
the \emph{Nested Loop Bound} $lpLB(LOOP_i, \tpath)$ is a symbolic expression in $\mathcal{A}_{\lin}$.
This expression is the path-sensitive
reachability estimation for the number of $LOOP_i$'s execution iterations
%  will 
% bound the execution times of $LOOP_{t_0}$
% in each single execution of the $LOOP_{t_i}$ for every
such that, during these iterations, $\tpath$ will be executed. 
%  bound of $\tpath$ in the while loop with loop label $LOOP_i$.
\\
For a refined program $\rprog \in \mathcal{RP}$, 
% with its refined statement $\rprog \in \mathcal{RP}$,
% \\
% for 
each transition path in this refined program, $\tpath \in \rprog$ with its loop chain set $rp\mathcal{C}(\tpath)$,
\\
and every loop label $LOOP_i \in lpch(\tpath) $ and $lpch(\tpath)  \in rp\mathcal{C}(\tpath)$:
% \\
% $(LOOP_i, \tpath)$ such that $LOOP_i \in lpch(\tpath)$,
\\
$lpLB(LOOP_i, \tpath) = rpLB(LOOP_i, \tpath)$ if $rpLB(LOOP_i, \tpath) \neq \bot$.
\\
$lpLB(LOOP_i, \tpath) = 
% \prod\limits_{\rprog_i \in lpchain(\tpath)}
\frac{lpInit(LOOP_i, \tpath) - rFinal(\tpath)}{lpInit(LOOP_i, \tpath) - lpNext(LOOP_i, \tpath)}$
% \\
\begin{defn}
  \label{def:nested_loop_bound}
For each transition path in a refined program $\tpath \in \rprog$ and every loop $LOOP_i \in lpch(\tpath)$ where $lpch(\tpath)  \in rp\mathcal{C}(\tpath)$, 
the \emph{Nested Loop Bound} $lpLB(LOOP_i, \tpath)$ is computed as follows,
\\
$lpLB(LOOP_i, \tpath) = rpLB(LOOP_i, \tpath)$ if $rpLB(LOOP_i, \tpath) \neq \bot$.
\\
$lpLB(LOOP_i, \tpath) = 
% \prod\limits_{\rprog_i \in lpchain(\tpath)}
\frac{lpInit(LOOP_i, \tpath) - rFinal(\tpath)}{lpInit(LOOP_i, \tpath) - lpNext(LOOP_i, \tpath)}$
\end{defn}
The computations of the operations $lpInit(LOOP_i, \tpath)$ and $lpNext(LOOP_i, \tpath)$
are formally computed as follows,
\begin{itemize}
\item For a transition path $\tpath \in \paths(\absG(c))$ and a loop label $LOOP_i$ in this transition path's loop chain.
Let $\rprog_i$ be the refined program with this loop label $LOOP_i$, i.e., $LOOP_i: \rprog_i \in \rprog$, 
the abstract loop chain initial state $lpInit(LOOP_i, \tpath) \in \mathcal{P}(\absstate)$ is computed as follows,
\[
  lpInit(LOOP_i, \tpath) \triangleq 
  \bigwedge_{x \in VAR(c)}
  \left\{ 
  x = arg\max_{l_1}
  \left\{
      \begin{array}{l}
    (v, (l_1, dc, l_2)) \in \reset(x) 
        \\ 
    % \qquad
    \land \absinit(\rprog_i) \leq l_1 \leq \absinit(\tpath)
    \end{array}
    \right\}
  \right\}
  \]
\item
For a transition path $\tpath \in \paths(\absG(c))$ and a refined program $\rprog_i$ with the loop label $LOOP_i: \rprog_i$, its $rNext(LOOP_i, \tpath)$ is computed as follows,
%
\[
  \begin{array}{l}
  lpNext(LOOP_i, \tpath) \triangleq 
  \bigwedge\limits_{x \in VAR(\rprog)}
  \left\{ 
    \begin{array}{l}
  x =   \sum\limits_{(x, \absevent) \in \inc(x) }\left\{ 
    \varinvar(y) + c ~\middle\vert~ \absevent = (l, (y, c), \_) \land l \in \lvar(\rprog_i)\right\}
    \\ \qquad 
    - \sum\limits_{(x, \absevent) \in dec(x) }\left\{ 
      \varinvar(y) + c ~\middle\vert~ \absevent = (l, (y, c), \_) \land l \in \lvar(\rprog_i) \right\}
    \end{array}
  \right\}
  \end{array}
\]
$lpInit(LOOP_i, \tpath)$, and $lpNext(LOOP_i, \tpath)$ can be computed using the 
$INIT(c, i, \absinit(\tpath))$ and $NEXT(c, i, \absinit(\tpath))$ from paper \cite{GulwaniJK09}.
% \\
The $lpLB(LOOP_i, \tpath)$ can also be computed using the algorithm 
$\kw{BOUNDFINDERD(INIT(c, i, \absinit(\tpath)), NEXT(c, i, \absinit(\tpath)), V_{\ln})}$
from paper \cite{GulwaniJK09} with the equivalent result.
\end{itemize}
% $rpRB: \tpath \to \mathcal{A}_{in}$, $chsRB: (\rprog \times \tpath) \to \mathcal{A}_{in}$
% \\
% For each transition path $\tpath \in \rprog$,
% \\
% 1. First compute the path sensitive reachability choosing bound through their choose chain:
% \\
% $chsRB(\rprog_n, \tpath) = \prod\limits_{\rprog_i \in lpchain(\rprog_n, \tpath)}
% \frac{chsInit(\rprog_i, \tpath) - chsFinal(\rprog_i, \tpath)}{varGD(\rprog_i, \tpath)}$
% \\
\end{enumerate}
\item \emph{Global Loop Bound}.
\\
For every transition path $\tpath \in \rprog$, its \emph{Global Loop Bound} $tRB(\tpath) \in \mathcal{A}_{in}$ is 
% Compute 
its path sensitive reachability bound globally. It is computed as follows,
\begin{defn}
  \label{def:global_loop_bound}
$tRB(\tpath) = \max
\left \{ \prod\limits_{LOOP \in l} rpLB(LOOP, \tpath)
% lpRB(\rprog_i, \tpath) 
~ \middle\vert~ l \in lp\mathcal{C}(\tpath) \right\}$
\end{defn}
% For $chain \in rpchain(\tpath)$:
% where $lpchains(\tpath)$ is set of $lpchains(\tpath)$ containing all the loop chains of $\tpath$.
\end{enumerate}
\paragraph{Step 5: Path Sensitive Reachability Bound Computation for Every Location}
$psRB : \ldom \to \mathcal{A}_{in}$
\\
For every label in a refined program $l \in \lvar(\rprog)$,
 $psRB(l)$ is its path sensitive reachability bound.
 \\
 \begin{defn}
  \label{def:label_psrb}
Given a refined program $\rprog \in \mathcal{RP}$ with 
\emph{Global Loop Bound} $tRB(\tpath)$
computed for its every transition path $\tpath \in \rprog$  notated by $tRB(\tpath)$,
%  for each of its transition path $\tpath \in \rprog$ 
% with the \emph{Global Loop Bound}
% computed as above, notated by $tRB(\tpath)$.
the $psRB(l)$ for each label $l \in \rprog$ is computed as follows,
\\
\[ psRB(l) = \sum\limits_{\tpath \in \rprog \land 
l \in \tpath} tRB(\tpath)\]
 \end{defn}