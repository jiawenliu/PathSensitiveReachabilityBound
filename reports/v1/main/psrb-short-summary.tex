\paragraph{Step 1: Constraint Program}
\label{sec:abs_prog}
% \textbf{Step 1: Program Abstract Execution Control Flow Graph}
The Constraint program is an Abstract Transition Graph for program $c$,
% For a program $c$, this analysis first generates its abstract execution control flow graph notated as follows,
\[\absG(c) =(\absV(c), \absE(c)).\]
%
$\absV(c)$ is the set of program $c$'s control location, each edge in $\absE(c)$ is a transition
between two control locations if and only if there is a feasible control flow between the two locations.
Each edge is annotated by a difference constraint \cite{sinn2017complexity} or a boolean expression
from the set $\dcdom^{\top}$.
\paragraph{Step 2: Program Rephrase and Refinement}
\begin{enumerate}
  \item \textbf{ Program Rephrase through Path Collection on Abstract CFG}
  This step simply collects all the simple loop-free transition path
  % $\tpath \in \paths(\absG(c))$, (a simple path) 
  on this program's abstract control flow graph $\absG(c)$, notated as follows,
  \[ \tpath \in \paths(\absG(c)). \]
  Then it rewrites the program in the form of simple paths.
  % \\
  \item \textbf{ Program Refinement} This step adopts the program refinement from \cite{GulwaniJK09}.
  It refines the rephrased program above path-sensitively and generates the refined program 
  $\rprog \in \mathcal{RP}$.
  \end{enumerate}
  % \begin{enumerate}
  % \item \emph{Program Rephrase through Path Collection on Abstract CFG}.
  % \\
  % The transition path
  % $\tpath \in \paths(\absG(c))$, is a simple path on this program's abstract control flow graph $\absG(c)$.
  % \\
  % % For the part of the graph not in any SCC:
  % \\
  % $p \triangleq \tpath $ if $\tpath \in \paths(\absG(c))$ and $\tpath \not\in SCC(\absG(c))$;
  % \\
  % $p \triangleq \rpchoose\{p_1, p_2 \}$ if $p_1$ and $p_2$ has the same head and end;
  % \\
  % $p \triangleq p_1; p_2$ if head of $p_1$ is the same as head of $p_2$ and either $p_1$ or $p_2$ isn't a simple path. 
  % % \\
  % % For the part of the graph not in some SCCs:
  % % \\
  % % For every while loop with guard label $l$:
  % % \\
  % % $p \triangleq \tpath $ if $\tpath \in \paths(\absG(c))$
  % \\
  % $p \triangleq LOOP_l : \rprepeat(\rpchoose\{p_1, \cdots, p_m\})$ if head and end of $p'$ are both $l$;
  % % and end of $p'$ are both $l$;
  % % \\
  % %
  % \item \emph{Program Refinement}.
  % \\
  % Refined program:
  % $\rprog \in \mathcal{RP}$.
  % % For a Loop $L$,
  % % computes all the transition Paths : $\tpath \in \absG(c)$
  % % -> $\rprog \in \mathcal{RP}$
  % % in this loop and generate the refined statement.
  % \\
  % Given a rephrased program $p$, its refined program is computed as follows,
  % \\
  % $\rprog \triangleq \tpath $ if $p = \tpath$\\
  % $\rprog \triangleq \rpchoose\{\rprog_1, \rprog_2 \}$ where $p \triangleq \rpchoose\{p_1, p_2 \}$ and 
  % $\rprog_1$ and $\rprog_2$ are refined $p_1$ and $p_2$. 
  % \\
  % $\rprog \triangleq LOOP_l : \kw{REFINE(p_w)}$ if $p = \rprepeat(p_w)$ and $\kw{REFINE(p_w)}$ is the algorithm in \\
  % $\rprog \triangleq \kw{REFINE(p_1)}; \kw{REFINE(p_2)}$ if $p = p_1; p_2$ 
  % \cite{sinn2017complexity}.
  % \end{enumerate}
  % \\
  \paragraph{Step 3: Outside-In Algorithm}
\label{sec:outinalg}
\textbf{Outside-In} algorithm computes the bound for iteration numbers of every repeat pattern $\rprog$,
% (i.e., $\rprog$) 
by a deep first search strategy from the outside most $\rprepeat$ annotation
to the innermost one nested.
\\
This bound computation doesn't consider the influence of the outside $\rprog$
when it computes the inner nested $\rprog'$.
\\
Necessary computation operators:
%  Required by the \emph{Outside-In} Algorithm:
\\
The \emph{State}: 
$\absstate \in \mathcal{P}(\dcdom^{\top})$ : conjunctions of the edge annotations.
%  constraints.
\\
The states of a refined program ($\rprog$):
\\
\emph{Initial State} ($\rfinit(\rprog)$), 
\emph{Final State} ($\rffinal(\rprog)$), and \emph{Next State} ($\rfnext(\rprog)$)  $\in \absstate$.
\\
The \emph{Variable Grade Decedent}: $\varGD : \rprog \to \mathcal{A}_{in}$.
% The \emph{Initial State}: $\rfinit : \rprog \to \absstate $.
\\
Computing the $\outinB$ bound for a refined program $\rprog$. 
\\
$\outinB(\rpchoose\{\rprog_i\}) =  \max\{\outinB(\rprog_i)\}$
\\
$\outinB(\rprepeat\{\rprog\}) =  \frac{\rfinit(\rprog) - \rffinal(\rprog)}{\varGD(\rprog)}$
\\
$\outinB(\rpseq(\rprog_1, \rprog_2)) =  \outinB(\rprog_1)+ \outinB( \rprog_2)$
\\
$\outinB(\tpath) =  1$.
\\
The computations of the operations: $\rfinit(\rprog)$,
$\rffinal(\rprog)$ $\rfnext(\tpath)$ and $\varGD : \rprog \to \mathcal{A}_{in}$ is in Appendix.
Equivalent result can be computed for $\outinB(\rprepeat\{\rprog\})$ by alternative computation method from paper \cite{GulwaniJK09}.
  % \\
  % The \emph{Refined Abstract State}: 
  % $\absstate \in \mathcal{P}(\dcdom^{\top})$ : conjunctions of difference constraints.
  % \\
  % The \emph{Initial Refined Abstract State}: $rInit : \rprog \to \absstate $.
  % % % , computed as follows,
  % % % \[
  % % % rInit(\rprog) \triangleq 
  % % % \bigwedge_{x \in VAR(\rprog)}
  % % % \left\{ 
  % % % x = v ~\middle\vert~ 
  % % % v = arg\max_{l_1}\left\{ 
  % % % v~\middle\vert~ 
  % % % (v, \absevent) \in \reset(x) 
  % % % \land \absevent = (l_1, dc, l_2)
  % % % \land l_1 \leq \absinit(rprog)
  % % % \right\}
  % % % \right\}
  % % % \]
  % % \\
  % % The $rInit : \rprog \to \absstate $ can also be computed using $INIT(c, l_0, \absinit(\rprog))$ algorithm in \cite{GulwaniJK09}. 
  % % $INIT(c, l_0, \absinit(\rprog))$ computes the equivalent results.
  % \\
  % The \emph{Final Refined Abstract State} $rFinal : \rprog \to \absstate $.
  % % \[
  % % rFinal(\rprog) \triangleq 
  % % \bigwedge_{x \in VAR(\rprog)}
  % % \left\{ 
  % % \varinvar(x)
  % % \right\} \land \neg Invariant(\rprog)
  % % \]
  % \\
  % The \emph{Variable Grade Decedent}: $varGD : \rprog \to \mathcal{A}_{in}$, is the set of the variables' variation in one iteration;
  % \\
  % Given a refined program $\rprog$, its path-sensitive local reachability bound is computed as follows,
  % \\
  % $rLB(\rpchoose\{\rprog_i\}) = \max\{rLB(\rprog_i)\}$
  % \\
  % $rLB(\rprepeat\{\rprog\}) = \frac{rInit(\rprog) - rFinal(\rprog)}{varGD(\rprog)}$
  % \\
  % $rLB(\rprepeat\{\rprog\})$ can also be computed using the algorithm 
  % $\kw{BOUNDFINDERD(INITD(c, l_0, \absinit(\rprog)), NEXTD(c, l_0, \absinit(\rprog), V_{\lin} )}$ 
  % in paper~\cite{GulwaniJK09} with equivalent results.
  % \\
  % $rLB(\rpseq(\rprog_1, \rprog_2)) = rLB(\rprog_1)+ rLB( \rprog_2)$
  % \\
  % $rLB(\tpath) = 1$.
  % \\
  % \begin{itemize}
  % \item For a refined program $\rprog$, its initial refined abstract $rInit(\rprog) \in \mathcal{P}(\absstate)$ is computed as follows,
  % \[
  % rInit(\rprog) \triangleq 
  % \bigwedge_{x \in VAR(\rprog)}
  % \left\{ 
  % x = v ~\middle\vert~ 
  % v = arg\max_{l_1}\left\{ 
  % v~\middle\vert~ 
  % (v, \absevent) \in \reset(x) 
  % \land \absevent = (l_1, dc, l_2)
  % \land l_1 \leq \absinit(rprog)
  % \right\}
  % \right\}
  % \]
  % % \\
  % The $rInit : \rprog \to \absstate $ can also be computed using $INIT(c, l_0, \absinit(\rprog))$ algorithm in \cite{GulwaniJK09}. 
  % $INIT(c, l_0, \absinit(\rprog))$ computes the equivalent results.
  % %
  % \item The program $\rprog$'s final refined abstract state $rFinal(\rprog) \in \mathcal{P}(\absstate) $ is computed as follows, 
  % \[
  % rFinal(\rprog) \triangleq 
  % \bigwedge_{x \in VAR(\rprog)}
  % \left\{ 
  % \varinvar(x)
  % \right\} \land \neg Invariant(\rprog)
  % \]
  % % \\
  % \item The program $\rprog$'s Variable Gradient Decent set is computed as follows,
  % \\
  % $varGD(\rpchoose\{\rprog_i\}) = \max\{varGD(\rprog_i)\}$
  % \\
  % $varGD(\rprepeat\{\rprog\}) = rLB(\rprepeat\{\rprog\}) \times {varGD(\rprog)}$
  % \\
  % $varGD(\rpseq(\rprog_1, \rprog_2)) = varGD(\rprog_1)+ varGD( \rprog_2)$
  % \\
  % $varGD(\tpath) = rInit(\tpath) - rNext(\tpath)$
  % \\
  % \item For a transition path $\tpath \in \paths(\absG(c))$, its $rNext(\tpath)$ is computed as follows,
  % %
  % \[
  % \begin{array}{l}
  % rNext(\tpath) \triangleq 
  % \bigwedge\limits_{x \in VAR(\rprog)}
  % \left\{ 
  % \begin{array}{l}
  % x = \sum\limits_{(x, \absevent) \in \inc(x) }\left\{ 
  % \varinvar(y) + c ~\middle\vert~ \absevent = (l, (y, c), \_) \land l \in \lvar(\rprog)\right\}
  % \\ \qquad 
  % - \sum\limits_{(x, \absevent) \in dec(x) }\left\{ 
  % \varinvar(y) + c ~\middle\vert~ \absevent = (l, (y, c), \_) \land l \in \lvar(\rprog) \right\}
  % \end{array}
  % \right\}
  % \end{array}
  % \]
  % \end{itemize}
  %
  \paragraph{Step 4: Inside-Out Algorithm}
  \label{sec:inoutalg}
  Main Algorithm Steps:
  \begin{enumerate}
  \item Repeat Chain Bound:
  \\
  => Require : Repeat chain and $\outinB(\rprog)$ from~\ref{sec:outinalg}.
  \\
  => Compute : path sensitive bound on the iteration numbers for every \emph{simple transition path} inside its closet while loop.
  \\
  Steps:
  \\
  \textbf{Repeat Chain} $\rpch(L, \tpath) \in \mathcal{P}(\rprog)$.
  \\
  $L, \tpath$ 
  repeat chain w.r.t a transition path
  is a list of repeat pattern $\rprog$ with loop header annotation $L$,
  and the $\tpath$ is contained by these repeat patterns recursively.
  % inside the same while loop $L$ as the $\tpath$. It is computed as follows,
  % For a refined while loop program $\rprog_{l} = L_l : \rprog \in \mathcal{RP}$, 
  % with its refined statement $\rprog \in \mathcal{RP}$,
  \\
  \textbf{Repeat chain set}.
  $\rpchset(L, \tpath, \rprog) \in \mathcal{P}(\mathcal{P}(\rprog))$.
  \\
  For each transition path in the refined program $\tpath \in \rprog$, 
  its repeat chain set 
  $\rpchset(L, \tpath, \rprog) \in \mathcal{P}(\mathcal{P}(\rprog))$
   is a set of all the repeat chains for $L, \tpath \in \rprog$ in this program.
  \\
  \textbf{Repeat Chain Bound}. $\rpchB(L, \tpath, \rprog) \in \mathcal{A}_{in}$.
  \\
  For every transition path $\tpath$
  in its closet enclosed while loop $L$,
  % $rpRB: \tpath \to \mathcal{A}_{in}$, $chsRB: (\rprog \times \tpath) \to \mathcal{A}_{in}$
  % \\
  % For each transition path $\tpath \in \rprog$,
  % \\
  % 1. First compute the path sensitive reachability choosing bound through their choose chain:
  % \\
  % $chsRB(\rprog_n, \tpath) = \prod\limits_{\rprog_i \in lpchain(\rprog_n, \tpath)}
  % \frac{chsInit(\rprog_i, \tpath) - chsFinal(\rprog_i, \tpath)}{\varGD(\rprog_i, \tpath)}$
    \\
    $\rpchB(L, \tpath,  \rprog) = \max \left\{ \prod\limits_{\rprog_i \in ch}  \outinB(\rprog_i) 
    ~\middle\vert~ ch \in \rpchset(L_l, \tpath) \right\}
    $
    \\
    \item \emph{Related Loop Bound}:
  \\
  => Require: $\rpchB(L, \tpath,  \rprog)$ from~\ref{sec:outinalg} and Step 1 above.
  \\
  => Compute: For every \emph{simple transition path} the path-sensitive
  reachability bound for the \emph{parent loops}' iteration numbers
  such that, during these iterations, the \emph{simple transition path}'s closet enclosed loop will be executed. 
  \\
  %  will 
  \emph{Nested Loop Chain}.  $\lpch(\tpath, \rprog) \in \mathcal{P}(\rprog)$.
  \\
    Each loop chain $\lpch(\tpath, \rprog) \in \mathcal{P}(\rprog)$ 
    is a list of refined program $\rprog' \in \rprog$
    $\tpath$ is nested in, and every $\rprog'$ has different loop header annotation.
  \\
  \highlight{
  \emph{Related Loop Bound}. ($\lpchB(L_i, \tpath, \rprog) \in \mathcal{A}_{\lin}$)
  \\
  Given a loop label $L_i \in \rprog$ and a transition path $\tpath \in \rprog$ in a refined program $\rprog$,
  the \emph{Related Loop Bound} $\lpchB(L_i, \tpath, \rprog)$ 
  the path-sensitive
  reachability bound on the number of $L_i$'s iteration numbers,
  such that, during these iterations, $\tpath$ will be executed. 
  %
  \begin{defn}[Related Loop Bound ]
    \label{def:relatedloop_bound}
  For each transition path in a refined program $\tpath \in \rprog$
  and every loop $L_i \in \lpch(\tpath, \rprog)$,
  % where $\lpch(\tpath)  \in \rpchset(\tpath)$, 
  the \emph{Related Loop Bound} $\lpchB(L_i, \tpath, \rprog)$ is computed as follows,
  \\
  $\lpchB(L_i, \tpath, \rprog) = 
  \left\{
  \begin{array}{l}
    \rpchB(L_i, \tpath, \rprog)  \\
    \qquad \rpchB(L_i, \tpath) \neq \bot
    \\
    \frac{\lpinit(L_i, \tpath) - \rffinal(\tpath)}{\lpinit(L_i, \tpath) - \lpnext(L_i, \tpath)}
    \\ \qquad o.w.
  \end{array}
  \right\}$ 
  % if $\rpchB(L_i, \tpath) \neq \bot$.
  % \\
  % $\lpchB(L_i, \tpath, \rprog) = 
  % % \prod\limits_{\rprog_i \in lpchain(\tpath)}
  % \frac{\lpinit(L_i, \tpath) - \rffinal(\tpath)}{\lpinit(L_i, \tpath) - \lpnext(L_i, \tpath)}$ o.w.
  \end{defn}
  }
  The computations of the operations $\lpinit(L_i, \tpath)$ and $\lpnext(L_i, \tpath)$ are in Appendix.
  %
  \\
  \item \emph{Inside-Out Bound}: $\inoutB(\tpath, \rprog) \in \mathcal{A}_{in}$.
  \\
  => Require :
  $\outinB(\tpath)$ from~\ref{sec:outinalg}, $\rpchB(L, \tpath)$ from Step 1 and 
  $lp\mathcal{C}(\tpath)$
  Step 2 above.
  \\
  => Compute : 
  For every transition path $\tpath \in \rprog$, its \emph{Inside-Out Loop Bound}
   $\inoutB(\tpath, \rprog) \in \mathcal{A}_{in}$ is 
  % Compute 
  the path sensitive reachability bound on the iteration numbers of this path in this refined program $\rprog$. 
  %
  \begin{defn}[{Inside-Out Bound} ($\inoutB(\tpath, \rprog) \in \mathcal{A}_{in}$)]
    \label{def:outin_bound}
    Given a refined program $\rprog$, for every transition path $\tpath \in \rprog$, 
    its \emph{Inside-Out Bound}
    $\inoutB(\tpath, \rprog)$ is 
   % Compute 
   computed as follows,
  \[
    \inoutB(\tpath, \rprog) =
    \prod\limits_{L \in \lpch(\tpath, \rprog)} \rpchB(L, \tpath, \rprog)
    \]
  \end{defn}
\end{enumerate}
  \paragraph{Step 5: Path Sensitive Reachability Bound Computation}
  => Require : $\outinB(\tpath, \rprog)$ and $\inoutB(\tpath, \rprog)$ from~\ref{sec:outinalg} and \ref{sec:inoutalg}.
  \\
  => Compute : path sensitive reachability bound for every execution location (i.e., label).
  For every program control location $l \in \lvar(c)$, with $\rprog$ as its refined program,
  %  in a program $c$,
  %  refined program $l \in \lvar(\rprog)$,
   $\psRB(l, \rprog)$ is its path sensitive reachability bound on the executing times of this location $l$.
   \\
   \begin{defn}
    \label{def:label_psrb}
  Given a program $c$ with its refined program $\rprog \in \mathcal{RP}$
  %  with 
  % \emph{Global Loop Bound} $\inoutB(\tpath)$
  % computed for its every transition path $\tpath \in \rprog$  notated by $\inoutB(\tpath)$,
  %  for each of its transition path $\tpath \in \rprog$ 
  % with the \emph{Global Loop Bound}
  % computed as above, notated by $\inoutB(\tpath)$.
  the $\psRB(c, l)$ for every label $l \in \lvar(c)$ is computed as follows,
  \\
  \[ \psRB(c, l) = \sum\limits_{\tpath \in \rprog \land 
  l \in \tpath} \inoutB(\tpath)\]
   \end{defn}