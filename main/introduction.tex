The programs' execution complexity affects our daily life from many perspectives.
For example,
from the privacy and security perspective,
how much secret information is leaked by a program depends on the number of times a certain operation that leaks the data,
% either by direct or indirect information flow, 
is executed~\cite{Malacaria07};
the amount of perturbation in the output data values resulting
from a small perturbation or uncertainty in the input,
values depend on the number of times additive error propagation operators are applied; etc.
Estimating such quantitative properties requires us to know
how many times is a given control location inside the program that performs certain operations executed?
% \\
From the performance perspective, it is important to give a precise estimation
on the program's resource cost bound w.r.t. the program's inputs.
For example, in memory-constrained environments such as embedded systems,
it is important to bound the amount of memory required to run certain applications.
In real-time systems, it is important to bound the worst-case execution time of the program.
Applications running on low-power devices or low-bandwidth environments must use up little power or bandwidth respectively. 
With the advent of cloud computing, where users would be charged per program execution,
predicting resource usage characteristics would be a crucial component of accurate bid placement by cloud providers. 
One of the challenges in bounding this cost precisely is that resource consumption is location-sensitive.
In other words, different location has different resource cost as well as different execution times.
To give accurate estimation results on these execution properties,
% This brings me to one of 
the fundamental questions that need to be addressed 
% for computing such resource bounds:
is estimating the bound on the execution times
% How many times is 
a given control location inside the program that consumes these resources.
For these reasons, we focus on analyzing the bound on the execution times of a program's given control location in this paper.
This bound is referred to as the reachability-bound in the program analysis area,
which is firstly proposed by the paper~\cite{GulwaniZ10}.
In this paper, finding a symbolic worst-case bound on this quantitative reachability property
in terms of the inputs to that procedure
is referred to as the \emph{reachability-bound problem}.


Providing a good solution to this problem is challenging.
The paper~\cite{GulwaniZ10} that introduces this concept
gives a two-step solution by combining the abstract interpretation-based iterative technique
 and the non-iterative proof-rules-based technique.
 However, their solution
does not solve this problem in a path-sensitive manner.
It over-approximates the reachability-bounds on different paths inside a while loop.
% \\
 There are also many works in analyzing the program complexity~\cite{GustafssonEL05, HumenbergerJK18},
 or estimating the upper bound on a program's worst-case resource cost
~\cite{BrockschmidtEFFG16, AlbertAGP08, AliasDFG10, Flores-MontoyaH14}.
But their analysis
focus only on estimating 
the overall complexity 
by inferring the bounds on the loop iteration numbers,
 or the worst-case running time and resource cost of the program's entire execution.
 \\
 None of them computes the reachability-bound on a given program control location directly or path-sensitively.
 To leverage these limitations,
We introduce a path-sensitive reachability-bound analysis algorithm in this paper, which aims to solve 
the reachability-bounds problem efficiently and path-sensitively.
Our algorithm is more advanced by combining two lines of complexity analysis techniques.
 \begin{itemize}
 \item One line of loop bound analysis works based on the \emph{amortized complexity analysis} originated from Tarjan's influential paper~\cite{PotechinP17} combined with \emph{ranking function} estimation from~\cite{BradleyMS05} and developed in~\cite{ZulegerGSV11,SinnZV14,SinnZV17,LuCT21,AliasDFG10}.
 They do well in nested loops by alternating the loop bound computation with the ranking function estimation. This alternation lines up with the \emph{amortized complexity analysis} without requiring unrolling the paths in order to compose the nested loops.
 \\
But they have limitation, estimating the ranking function ignores the interleaving between multiple paths in the same loop.
It over-approximates the bound of each single path as well the loop bound under different path interleaving.
\item 
Another line of loop bound analysis through loop summarization and path refinement seek for precise loop path representation~\cite{ManoliosV06, BalakrishnanSIG09, SharmaDDA11, Flores-MontoyaH14, HumenbergerJK18, CyphertBKR19}, and compute loop bound over accurate loop summarization~\cite{GulwaniJK09, ZulegerGSV11}.
They do well in summarizing the loop path and computing the interleaving between paths, and then compute precise bound path single path considering the interleaving.
\\
Their limitation is the composition between nested loops. Recursively unrolling the nested paths and computing the interleaving between unrolled paths are heavy and non-terminating.
 \end{itemize}
Our combination improves in both limitations.

 The contributions of this work can be summarized as follows,
 \begin{itemize}
   \item the main contribution is the combination of the \emph{amortized program analysis} through ranking function estimation and the path refinement and loop summarization based on \emph{algebraic program analysis} in bound analysis algorithm.
    \item A path-sensitive reachability-bound computation algorithm.
    This algorithm can compute the evaluation times of each program point accurately and path-sensitively.
    \item A prototype implementation of this algorithm and an evaluation over 5 different benchmarks.
    The evaluation results show that we can compute tight bound on the evaluation times of each program point in a program. For program with multi-path loop, we compute different bounds for the points on different paths.
 \end{itemize}