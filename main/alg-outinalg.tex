\textbf{Computation Overview}
\[
  \begin{array}{rcl}
    \outinB(\tpath, \tpath) & \triangleq & 1 \\
    \outinB(\tpath', \tpath) & \triangleq & \highlight{0} \\
    \outinB(\rprog_1;\rprog_2, \tpath) & \triangleq & \outinB(\rprog_1, \tpath) + \outinB(\rprog_2, \tpath) \\
    \outinB(\rpchoose{\rprog_1, \cdots, \rprog_m }, \tpath) & \triangleq 
    & \max\left\{ \outinB(\rprog_1, \tpath), \cdots, \outinB(\rprog_m, \tpath) \right\} \\
    \outinB(\rprepeat(\rprog'), \tpath) & \triangleq 
    & \highlight{\frac{\rfinit(\rprog, \rprog') - \rffinal(\rprog, \rprog')}{\varGD(\rprog, \rprog')}}
     \times \outinB(\rprog', \tpath)
     \\
    \outinB(l: \rprog', \tpath) & \triangleq & \outinB(\rprog', \tpath) \\
    &  & \text{this case always equals to } 0 \\
  \end{array}\]
Three steps:
\begin{enumerate}
    \item It first collects three edge sets for each variable,
  in which the variable increases, decreases and reset respectively.
  \item
  Then, it assigns a variable to the edge on which this variable decreases as this edge's ranking function.
  \item
  In the last step, it computes the bound on the maximum value of the ranking function,
  the $\rfinit(\rprog, \rpattern)$,
  $\rffinal(\rprog, \rpattern)$, $\rfnext(\rprog, \rpattern)$ and $\varGD(\rprog, \rpattern) \in \mathcal{A}_{in}$
%   are formally presented as follows.
   and the bound on the execution
  times of the given execution path recursively.
%   \item In the last step, it combines the local bound of each transition path
%    but path-insensitively.
  \end{enumerate}

  The algorithm in this step is inspired from the Algorithm.2 in paper~\cite{SinnZV14},
  % which assigns a variable to each edge on which this variable decrease as its ranking function.
  the Algorithm.3 in paper~\cite{ZulegerGSV11},
  and the Definition.25 in Section 4 of paper~\cite{sinn2017complexity}.
  Algorithm.3 in paper~\cite{ZulegerGSV11} assigns a set of variables to each transition in which these variables decrease as the local bound
  and estimates the maximum value each variable in this set.
  Algorithm.2 in paper~\cite{SinnZV14} assigns a variable to each edge on which this variable decrease as its ranking function
  and then estimates the maximum value for the ranking function.
  The Definition.25 in paper~\cite{sinn2017complexity}
  assigns each transition with a variable that decreases in this transition, as the local bound and computes the bound similarly.

\subsection{Local Bound Computation}
  For every repeat pattern $\rpattern$ in a refined program $\rprog$, 
this algorithm
computes a bound, $\outinB(\rprog, \rpattern) \in \mathcal{A}_{in}$ on its iteration numbers locally and path-sensitively
named \textbf{Outside-In} bound.
%  of every repeat pattern 
\\
% $\outinB(\rprog, \rpattern)$
% computes the bound locally
% % is the \textbf{Outside-In} bound for a repeat pattern $\rpattern$ 
% in the sense that
% % for a repeat pattern 
% if 
For every $\rpattern'$ that is nested
in another repeat pattern $\rpattern$, i.e., $\rpattern = \rprepeat(\rpattern')$,
$\outinB(\rprog, \rpattern)$ only estimates
$\rpattern'$'s iteration number within $\rpattern$ by assuming that $\rprepeat(\rpattern')$ executes once.
% \\

\highlight{\textbf{Notations / Operators:}}
\\
The \emph{State}: 
$\absstate \in \mathcal{P}(\dcdom^{\top})$ is a conjunction of constraints.
%  constraints.
\\
% For a refined program ($\rprog$), there are the 
% \\
\emph{Initial State} ($\rfinit(\rprog, \rpattern) \in \absstate$), 
\\
\emph{Final State} ($\rffinal(\rprog, \rpattern) \in \absstate$),
\\
\emph{Next State} $\rfnext(\rprog, \rpattern) \in \absstate$,
% The \emph{Initial State}: $\rfinit : \rprog \to \absstate $.
\\
\emph{Variable Grade Decedent}: $\varGD(\rprog, \rpattern) \in \mathcal{A}_{in}$


\highlight{\textbf{Computation Steps:}}
  \textbf{Outside-In} Bound ($\outinB(\rprog, \rpattern) \in \mathcal{A}_{in}$)
  \\
For a repeat pattern $\rpattern$ in a refined program $\rprog$,
the \textbf{Outside-In} bound, $\outinB(\rprog, \rpattern) \in \mathcal{A}_{in}$ is computed recursively as follows,
% for every nested repeat patterns in $\rprog$,
% \\
% $\outinB(\rpchoose{\rprog_1, \cdots, \rprog_m}) =  \max\{\outinB(\rprog_1), \cdots, \outinB(\rprog_m)\}$
\\
$\outinB(\rprog, \rprepeat({\rpattern})) = 
\frac{\rfinit(\rprog, \rpattern) - \rffinal(\rprog, \rpattern)}{\varGD(\rprog, \rpattern)}$
\\
$\outinB(\rprog, \rpattern_1; \rpattern_2) =  \outinB(\rprog, \rpattern_1)+ \outinB(\rprog,  \rpattern_2)$
\\
$\outinB(\rprog, \tpath) =  1$.
\\
\highlight{
  \textbf{Improvement Discussion:}
\\
The computation of $\outinB(\rprepeat({\rpattern}))$ 
as below is more efficient and accurate than existing techniques.
% in the first case
% can be computed  as follows
\\
The method $T(c, l)$ in \cite{GulwaniJK09} can also compute a
bound on the iteration numbers of $\rprepeat({\rpattern})$ locally based on
% $\rprepeat(\rpattern)$ with loop header $l$
% has the 
the same semantics and syntax.
Specifically, $T(c, l)$ call the $\kw{BOUNDFINDERD}$ function as follows:
\\
$\kw{BOUNDFINDERD(INITD(c, l_0, \absinit(\rprepeat({\rpattern}))),
NEXTD(c, l_0, \absinit(\rprepeat({\rpattern})), V_{\lin} )}$
% based on
% % $\rprepeat(\rpattern)$ with loop header $l$
% % has the 
% the same semantics and syntax.
%  as the refined program in paper~\cite{GulwaniJK09},
% in paper~\cite{GulwaniJK09} 
% with equivalent results.
However, the $\kw{BOUNDFINDERD}$ function in \cite{GulwaniJK09} relies an arbitrary interface to
compute the bound on the iteration numbers.
The efficiency and accuracy of this computation fully depend on this arbitrary interface.
\\
In comparison to them, below I provide an efficient and accurate bound computation method using the
ranks computed in Definition~\ref{def:edge_pathinsensitivebound} over the abstract transition graph $\absG(c)$.
}
\\
% Formally defined as follows,
% \begin{defn}[Path-sensitive Local Reachability Bound]
%   \label{def:pathsensitive_lrb}
% \end{defn}




% \begin{enumerate}
%     \item It first collects three edge sets for each variable,
%   in which the variable increases, decreases and reset respectively.
%   \item
%   Then, it assigns a variable to the edge on which this variable decreases as this edge's ranking function.
%   \item
%   In the last step, it computes the bound on the maximum value of the variable,
%   the $\rfinit(\rprog, \rpattern)$,
%   $\rffinal(\rprog, \rpattern)$, $\rfnext(\rprog, \rpattern)$ and $\varGD(\rprog, \rpattern) \in \mathcal{A}_{in}$
% %   are formally presented as follows.
%    and the bound on the execution
%   times of the corresponding simple transition path recursively.
% %   \item In the last step, it combines the local bound of each transition path
% %    but path-insensitively.
%   \end{enumerate}

%   The algorithm in this step is inspired from the Algorithm.2 in paper~\cite{SinnZV14},
%   % which assigns a variable to each edge on which this variable decrease as its ranking function.
%   the Algorithm.3 in paper~\cite{ZulegerGSV11},
%   and the Definition.25 in Section 4 of paper~\cite{sinn2017complexity}.
%   Algorithm.3 in paper~\cite{ZulegerGSV11} assigns a set of variables to each transition in which these variables decrease as the local bound
%   and estimates the maximum value each variable in this set.
%   Algorithm.2 in paper~\cite{SinnZV14} assigns a variable to each edge on which this variable decrease as its ranking function
%   and then estimates the maximum value for the ranking function.
%   The Definition.25 in paper~\cite{sinn2017complexity}
%   assigns each transition with a variable that decreases in this transition, as the local bound and computes the bound similarly.
  %
  %
  \subsection{Collecting Variable Modifications}
  For each variable $x$ in a program $c$, this step computes three edge sets, $\inc(c, x)$, $\dec(c, x)$,
  and $\reset(c, x)$ for $x$.
  Every edge in a set corresponds to a transition in which $x$ is increased,
  %  $\inc(c, x)$,
  decreased
  % $\dec(c, x)$ and 
  or reset
  % $\reset(c, x)$, 
  respectively.
  \\
  $\inc: \cdom \to \mathcal{VAR} \to \mathcal{P}(\absevent) $
  is the set of the edges where the variable increase, 
  \\
  $\inc(c, x) = \{ \absevent | \absevent = (l, l', x' \leq x + v) \land \absevent \in \absflow(c)\}$
  \\
  $\dec: \mathcal{VAR} \to \mathcal{P}(\absevent) $
  is the set of abstract events where the variable decrease,
  \\
  $\dec(c, x) = \{\absevent| \absevent = (l, l', x' \leq x - v) \land \absevent \in \absflow(c)\}$
  \\
  $\reset: \cdom \to \mathcal{VAR} \to \mathcal{P}(\absevent) $
  is the set of the abstract events where the variable is reset,
  \\
  $\reset(c, x) = \{\absevent| \absevent = (l, l', x' \leq y - v) \land x \neq y \land \absevent \in \absflow(c)\}$
  \\
  $\resetchain: \cdom \to \mathcal{VAR} \to \mathcal{P}(\mathcal{P}(\absevent)) $
  is the set of the chain of abstract events where the variable is reset through the chain.
  \\
  In addition to
  collect the edge set that $x$ is reset on every edge in this set, i.e., compute the $\reset(c, x)$,
  we also compute a set, $\resetchain(c, x)$ contains sequences of edges for $x$
  based on the Definition.20 in \cite{sinn2017complexity}.
  In each sequence, $(e_0, \cdots, e_m) \in \resetchain(c, x)$
  a variable $x_i$ is reset by another variable $x_{i + 1}$ on edge $e_{i}$
  and $x_{i + 1}$ is reset on edge $e_{i + 1}$ recursively
  for every $i = 0, \cdots, m - 1$.
  $x$ is reset on the first edge $e_0$ of every sequence in $\resetchain(c, x)$.
  \highlight{Rephrase: Each edge $e_i$ in a sequence $(e_0, \cdots, e_m) \in \resetchain(c, x)$
  resets a variable $x_i$ by another variable $x_{i + 1}$ such that $x_{i + 1}$
  is reset on edge $e_{i + 1}$ recursively. The first edge $e_0$ of each sequence resets the variable $x$.}
  % 
  % Each chain in this set is  where a given variable is reset by the 
  % variables of the abstract events through the chain.
  %
  \\
  In the following steps, $c$ is omitted in $\inc(x)$,
  $\dec(x)$ and $\reset(x)$ for concise when the reference of a program $c$ is clear in the context.

  \subsection{Assigning The Ranking Function to An Edge}
  For each edge in the transition graph $\absG(c)$ of a program $c$,
  this step assigns the variable that decreases on this edge as the ranking function   of this edge.
  This step adopts the local bound computation method in Section 4 of \cite{sinn2017complexity} to assign the local bound to each edge,
  formally as follows.
  \begin{defn}[Ranking Function   Generatation]
    \label{def:ranking_gen}
  % Given a program $c$ with its abstract transition graph 
  % $\absG(c) = (\absV, \absE)$,
  For every edge $\absevent$ in the transition graph $\absG(c)$ of a program $c$,
  its \emph{ranking function/local bound}, $\locbound(\absevent, c)$
  is the variable that decreases on this edge, computed as follows,
  %
  \[ 
  \begin{array}{ll}
    \locbound(\absevent, c) \triangleq 1 
    & \absevent \notin SCC(\absG(c))
    \\
    \locbound(\absevent, c) \triangleq x
    & \absevent \in SCC(\absG(c)) \land \absevent \in \dec(x) \land  \absevent = (\_, \_ , x' \leq x - v) \\
    \locbound(\absevent, c) \triangleq x
    & \absevent \in SCC(\absG(c)) \land 
    \absevent  \notin \bigcup_{x \in \mathcal{VAR}} \dec(x)
    \land \absevent \notin SCC(\absG(c) \setminus \dec(x)).
  \end{array}
  \]
  $SCC(\absG(c))$ is the set of all the strong connected components of $\absG(c)$.
  \end{defn}
    The first case is straightforward. 
    For the label $l$ which is not in any while loop, 
    the labeled command with the label $l$ will be 
    evaluated at most once. 
    % we do not need to analyze the visiting times of every node in the graph from phase 1.
    The second and third cases are guaranteed by the \emph{Discussion on Soundness} in Section 4 in~\cite{sinn2017complexity}.
    The formal soundness proof is in Lemma~\ref{lem:local_bound_sound} in Appendix~\ref{apdx:pathinsensitive_rb_soundness}.
  %
  % \paragraph*{Invariant Inference and Closure Generation }
  % Then, computing the bound invariants for variables and the transition closures for abstract events:
  % \\ 
  % $ \varinvar: \mathcal{VAR} \cup \constdom \to EXPR(\constdom)$
  % \\
  % $\absclr: \absevent \to EXPR(\constdom)$
  % \\
  % $EXPR(\constdom)$ is symbolic expression 
  % over $\constdom$, which is a subset of arithmetic expressions over $\mathbb{N}$ with input variables and $ $.
  % We use $\mathcal{A}_{\lin}$ denotes the arithmetic expression 
  % over the symbolic variables, (i.e., $\mathbb{N}$ with input variables and $ $).
  % Then, the symbolic invariant for each variable 
  % as well as the symbolic transition closure for each transition is calculated as follows:
  % \[ 
  % \begin{array}{lll}
  %   \varinvar(x) & \triangleq c & c \in \constdom \\
  %   \varinvar(x) & \triangleq \incrs(v) + \max(\{\varinvar(a) + c | (t, a, c) \in \reset(x)\}) & c \notin \constdom
  % \end{array}
  % \]
  % %
  % \begin{defn}
  %   \label{def:edge_pathinsensitivebound_base}
  % \[ 
  % \begin{array}{lll}
  %   \absclr(\absevent) 
  %   & \triangleq x / v & \\ 
  %   & \locbound(\absevent) = (x, v) \in \constdom \times \mathbb{N} & \\
  %   \absclr(\absevent) 
  %   & \triangleq (\incrs(x) + 
  %   \sum\limits_{(\absevent', y, v') \in \reset(x)}
  %   \absclr(\absevent') \times \max(\varinvar(y) + v', 0) ) / v & \\
  %   & \locbound(\absevent) = (x, v) \land x \notin \constdom & 
  % \end{array}
  %   \]
  % \end{defn}
  % %
  % \paragraph*{Improved Variable Modification Tracking}
  % \\
  % $\incrs(v) \triangleq \sum\limits_{(\absevent, c) \in \inc(v)}\{\absclr(\absevent) \times v\}$
  %
  \subsection{Abstract States Computation}
  This step estimates the upper bound, $\varinvar(x, c) \in \mathcal{A}_{\lin}$
  on the maximum value for each ranking function   $x \in  \mathcal{VAR} \cup \constdom$.
  \\
  For a program $c$, the \emph{ranking function bound} of an,
  $\varinvar(\locbound(\absevent, c)) \in \mathcal{A}_{\lin}$ is 
  the bound on the maximum value of the ranking function  
  % $\locbound(\absevent)  \in \mathcal{VAR} \cup \constdom $,
  assigned to the edge $\absevent \in \absE(c)$, formally in Definition~\ref{def:ranking_bound} and~\ref{def:edge_pathinsensitivebound}.
  \\
  % $\varinvar(\locbound(\absevent)) \in \mathcal{A}_{\lin}$.
  % Then, computing the bound for vriables and the transition closures for abstract events:
  In order to estimate the maximum value of $\locbound(\absevent, c)$ assigned to edge $\absevent \in \absE(c)$,
  % for each (ranking function's) maximum value,
  the bound on the iteration times of each corresponding edge, $\absclr(\absevent, c)$ 
  is computed interactively in a path-insensitive manner.
  % , the 
  \\ 
  $ \varinvar: (\mathcal{VAR} \cup \constdom  \times \cdom) \to \mathcal{A}_{\lin}$
  \\
  $\absclr: (\absevent \times \cdom) \to \mathcal{A}_{\lin}$
  % \\
  % Then, the bound on the maximum value of each variable,
  % as well as the symbolic bound on the iteration times of each edge is calculated recursively as follows, path-insensitively.
  \begin{defn}[Ranking Function Estimation]
    \label{def:ranking_bound}
  For a program $c$ and an edge $\absevent \in \absE(c)$,
  the \emph{ranking function bound}, $\varinvar(\locbound(\absevent, c))$ for the ranking function $\locbound(\absevent, c)$
  of this edge
  % %  \in \mathcal{A}_{\lin}$ 
  % for $\locbound(\absevent) \in \mathcal{VAR} \cup \constdom$,
  % is the bound on the maximum value of the ranking function  
  % % $\locbound(\absevent)  \in \mathcal{VAR} \cup \constdom $,
  % assigned to the edge $\absevent \in \absG(c)$, 
  is computed as follows,
    \[ 
  \begin{array}{lll}
    \varinvar(x, c) & \triangleq x & x \in \constdom \\
    \varinvar(x, c) & \triangleq \incrs(x, c) + \max(\{\varinvar(y, c) + c ~\mid~ (l, x' \leq y + c, l') \in \reset(x)\}) & c \notin \constdom
  \end{array}
  \]
  %
  $\incrs(x, c) \triangleq \sum\limits_{\absevent \in \inc(v)}\{\absclr(\absevent, c) \times c ~\mid~ \absevent = (l, x' \leq x + c, l')\}$ where 
  $\absclr(\absevent, c) \in \mathcal{A}_{\lin}$  is computed as below,
\[ 
\begin{array}{lll}
  \absclr(\absevent, c) 
  & \triangleq \varinvar(\locbound(\absevent, c))  & \\
  & \quad \locbound(\absevent, c) \in \constdom & \\
  \absclr(\absevent, c) 
  & \triangleq \Big(
    \sum\limits_{y \in \{ y ~|~ 
    ch \in \resetchain(x), (l_1, x, y, v, l_2) \in ch \} } \incrs(y, c) & \\
    & \quad + 
  \sum\limits_{ch \in \resetchain(x, c)}
  \big( \min\left\{\absclr(\absevent', c) ~\mid~ \absevent' \in ch\right\} \times 
  \max\left\{\varinvar(in(ch), c) + \sum\limits_{(l_1, x, y, v, l_2) \in ch } v, 0 \right\}\big) \Big)  & \\
  &  \quad \locbound(\absevent, c) = x \land x \notin \constdom & ,
\end{array}
  \]
 where $in(ch)$ is the variable on the last edge on the reset chain $ch$.
\end{defn}
  %
The computations of $\outinB(\rprepeat({\rpattern}))$ and $\rfinit(\rprog, \rpattern)$,
$\rffinal(\rprog, \rpattern)$, $\rfnext(\rprog, \rpattern)$ and $\varGD(\rprog, \rpattern) \in \mathcal{A}_{in}$
are formally presented as follows.
  \begin{defn}[Abstract States Computation]
    \label{def:edge_pathinsensitivebound}
   The computations of $\outinB(\rprepeat({\rpattern}))$ and $\rfinit(\rprog, \rpattern)$,
   $\rffinal(\rprog, \rpattern)$, $\rfnext(\rprog, \rpattern)$ and $\varGD(\rprog, \rpattern) \in \mathcal{A}_{in}$
   are formally presented as follows.
   \begin{itemize}
    \item The \emph{Initial State}, $\rfinit(\rprog, \rpattern)$ for the repeat pattern $\rpattern$ of a refined program $\rprog$ is computed as follows,
   % \\
   %  For a refined program $\rprepeat$, its initial refined abstract $\rfinit(\rprepeat) \in \mathcal{P}(\absstate)$ is computed as follows,
   \[
     \rfinit(\rprog, \rpattern) \triangleq 
     \bigwedge_{x \in VAR(\rpattern)}
     % \left\{ 
     x
     % = v ~\middle\vert~ 
     % v 
     = arg\max_{l_1}\left\{
       \varinvar(y) + v ~\middle\vert~ 
       \begin{array}{l} 
         (l_1, x' \leq y + v, l_2) \in \reset(x) 
         \\
       % \land \absevent = (l_1, dc, l_2)
       \land l_1 \leq \absinit(\rpattern)
     \end{array}
     \right\}
     % \right\}
     \]
   % \\
   % The 
   $\rfinit(\rprog, \rpattern)$ can also be computed using the function $\kw{INIT(c, l_0, \absinit(\rpattern))}$ in \cite{GulwaniJK09}. 
   $\kw{INIT(c, l_0, \absinit(\rpattern))}$ computes the equivalent results.
   %
   \item  The {\emph{Final State}, $\rffinal(\rprog, \rpattern) \in \absstate$ for the repeat pattern $\rpattern$ of a refined program $\rprog$ is computed as follows, }
   % \\
   % The program $\rprepeat$'s final refined abstract state $\rffinal(\rprepeat) \in \mathcal{P}(\absstate) $ is computed as follows, 
   \[
     \rffinal(\rprog, \rpattern) \triangleq 
     \bigwedge_{x \in VAR(\rpattern)}
     % \left\{ 
     %   \varinvar(x)
     % \right\} \land 
     \neg \invariant(\rpattern)
     \]
     % \\
   \item The \emph{Next State}, $\rfnext(\rprog, \rpattern) \in \absstate$ for the repeat pattern $\rpattern$ of a refined program $\rprog$ is computed as follows,
   % \\
   %  For a transition path $\tpath \in \paths(\absG(c))$, its $\rfnext(\tpath)$ is computed as follows,
   %
   \[
     \begin{array}{l}
     \rfnext(\rprog, \tpath) \triangleq 
     \bigwedge\limits_{x \in VAR(\tpath)}
     \left\{ 
       \begin{array}{l}
     x =   \sum\limits_{(x, \absevent) \in \inc(x) }
     \left\{ 
       {\varinvar(y) \todo{==>} \reset(y)}+ v ~\middle\vert~ \absevent = (l, x' \leq y + v, \_) \land l \in \lvar(\tpath)\right\}
       \\ \qquad 
       - \sum\limits_{(x, \absevent) \in \dec(x) }\left\{ 
         \varinvar(y) + v ~\middle\vert~ \absevent = (l, x' \leq y + v, \_) \land l \in \lvar(\tpath) \right\}
       \end{array}
     \right\}
     \end{array}
   \]
   We only need to compute the $\rfnext(\rprog, \rpattern)$ for the repeat pattern which is a simple transition path,
   i.e., $\rfnext(\rprog, \tpath)$.
   \item  The \emph{Variable Gradient Decent}, $\varGD(\rprog, \rpattern)$ for the repeat pattern $\rpattern$ of a refined program $\rprog$ is computed as follows,
   % \\
   % $\varGD(\rpchoose{\rprog_1, \cdots, \rprog_m}) =  \max\left\{\varGD(\rprog_1), \cdots, \varGD(\rprog_m) \right\}$
   \\
   % \todo
   {$\varGD(\rprog, \rprepeat(\rpattern)) =  \outinB(\rprog, \rpattern) \times {\varGD(\rprog, \rpattern)}$}
   \\
   $\varGD(\rprog, \rpseq(\rpattern_1, \rpattern_2)) =  \varGD(\rprog, \rpattern_1)+ \varGD(\rprog, rpattern_2)$
   \\
   $\varGD(\rprog, \tpath) =  \rfinit(\tpath) - \rfnext(\rprog, \tpath)$   
    \end{itemize}
\end{defn}
    %
    

\subsection{Theorem Guarantee}
  \label{sec:outinalg-theorem}
  For a program $c$ and an edge $\absevent \in \absE(c)$,
  $\absclr(\absevent)$ is a sound upper bound
  %  for every label $l$ in this program 
  on the execution times of this transition,
  formally in Theorem~\ref{thm:pathinsensitive_rb_soundness}. 
  Proof of this theorem is in Appendix~\ref{apdx:pathinsensitive_rb_soundness}.

  \highlight{\textbf{Theorem Guarantee Discussion:}}
\\
\emph{Soundness} For every repeat pattern $\rpattern$, the $\outinB(\rpattern, \rprog)$
is a sound upper bound on its execution times locally.
\\
This bound is sound locally by assuming
that all the loops and repeat patterns where $\rpattern$ is nested execute only once.
This assumption comes from the computation of the $\varGD$ and the deep first search strategy.
\\
For every base case, i.e., a simple transition path, 
$\varGD(\rprog, \tpath) =  \rfinit(\tpath) - \rfnext(\rprog, \tpath)$
counts the variables' changes only once. In this way, it assumes all the outside patterns and loops execution only once.
In the other two cases,  $\varGD(\rprog, \rprepeat(\rpattern))$ and $\varGD(\rprog, \rpseq(\rpattern_1, \rpattern_2))$
has the same assumption when compute the variables' changes.
% \\
The soundness also relies on the operation $\frac{\rfinit(\rprog, \rpattern) - \rffinal(\rprog, \rpattern)}{\varGD(\rprog, \rpattern)}$,
which is solved by Definition~\ref{def:edge_pathinsensitivebound} and~\ref{def:ranking_bound}.
% a sound external solver. \todo{reference}
%
\begin{thm}[Soundness of Outside-In Bound]
  \label{thm:sound_outsidein}
  For every program $c$ with it refined program $\rprog$,
  if $\rpattern$ is a repeat pattern of this program, then the $\outinB(\rpattern, \rprog)$
is a sound local bound on the execution times of $\rpattern$ in program $c$.
  \[
    \begin{array}{l}
    \forall c \in \cdom, \rpattern, \rprog, \trace_0 \in \tdom_0(\rpattern), \trace_1, \trace \in \tdom \st 
    \rpattern \in \rprog \land
    \rprog = Alg.4(c)
    \\ \qquad
    \land
    \config{c, \trace_0} \to^* \config{\highlight{\ewhile (\rpattern, \cdots)}, \trace_0 \tracecat \trace_1} 
    \land 
    \config{\rpattern, \trace_0 \tracecat \trace_1} \to^* \config{{\eskip, \trace_0 \tracecat \trace_1 \tracecat \trace}}
    \\ \qquad
    \implies
    \todo{Change Counter}
    {\vcounter(\trace, L(\rpattern))} \leq \outinB(\rpattern, \rprog)(\trace_0 \tracecat \trace_1).
    \end{array}
    \]
\end{thm}
\highlight{Informal Discussion}
\\
Theorem~\ref{thm:sound_outsidein} guarantees that
the bound for every repeat pattern $\rpattern$, the $\outinB(\rpattern, \rprog)$
is a sound upper bound on its execution times, by assuming
that all the loops and repeat patterns where $\rpattern$ is nested execute only once.
\\
In paper \cite{sinn2017complexity} Definition~9, they informally discussed the local bound soundness.
$v$ is a local bound if it has the same decreasing time as the transition's execution time.
By assuming that certain program parts (those were e increases) are not executed,
then value of $v$ can limit the execution time of that transition.
\\
In our soundness, we assume all the code pieces not inside this repeat pattern are executed at most once (once if they show up in front of the program
zero time if not).
In this case, this bound limits the execution time of this repeat pattern.