%
A trace $\trace \in \tdom$ is a list of events, 
collecting the events generated during a specific program execution. 
\[
\begin{array}{llll}
\mbox{Trace} & \trace
& ::= & [] ~|~ \trace :: \event 
% ~|~ []^{\infty}
\end{array}
\]
A trace can be regarded as the program history, 
which records all the evaluations for assignment commands and guards in $\eif$ and $\ewhile$ command.
\\
{
A trace can be finite ($\trace \in \ftdom$) or infinite $\trace \in \inftdom$.
If a program doesn't terminate when executing under some initial trace,
it produces the infinite trace 
from $\inftdom$, which records a non-terminating computation.
So we denote by $\tdom$ the set of all traces, and $\tdom = \ftdom \cup \inftdom$.
The trace-based semantics with non-terminating execution is defined below following the maximal trace semantics in \cite{Cousot19}.}

We use list notation for traces, where $[]$ is the empty trace, the operator $\traceadd$ combines an event and a trace in a new event, 
and the operator $\tracecat$ concatenates two traces formally defined as follows. 
\begin{defn}[Trace Concatenation, $\tracecat: \tdom \to \tdom\to \tdom $]
  \label{def:trace_concate}
Given two traces $\trace_1 \in \tdom, \trace_2 \in \tdom$, the trace concatenation operator 
$\tracecat$ is defined as:
\[
\trace_1 \tracecat \trace_2 \triangleq
\left\{
\begin{array}{ll} 
  \trace_1 & \trace_2 = [] \lor \trace_1 \in \inftdom \\
  \trace_2 & \trace_1 = [] \lor \trace_2 \in \inftdom \\
  (\trace_1   \tracecat \trace_2'):: \event & \trace_1 \in \ftdom \land \trace_2 = \trace_2' :: \event
  % \trace_2 &  \trace_2 \in \inftdom \\
\end{array}
\right.
\]
\end{defn}

