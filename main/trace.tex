%
A trace $\trace \in \tdom$ is a list of events, 
collecting the events generated during a specific program execution. 
\[
\begin{array}{llll}
\mbox{Trace} & \trace
& ::= & [] ~|~ \trace :: \event 
% ~|~ []^{\infty}
\end{array}
\]
A trace can be regarded as the program history, 
which records all the evaluations for assignment commands and guards in $\eif$ and $\ewhile$ command.
\\
\highlight{
A trace can be finite ($\trace \in \ftdom$) or infinite $\trace \in \inftdom$.
If a program doesn't terminate when executing under some initial trace,
it produces the infinite trace 
from $\inftdom$, which records a non-terminating computation.
So we denote by $\tdom$ the set of all traces, and $\tdom = \ftdom \cup \inftdom$.
The trace-based semantics with non-terminating execution is defined below following the maximal trace semantics in \cite{Cousot19}.}

We use list notation for traces, where $[]$ is the empty trace, the operator $\traceadd$ combines an event and a trace in a new event, 
and the operator $\tracecat$ concatenates two traces formally defined as follows. 
\highlight{
\begin{defn}[Trace Concatenation, $\tracecat: \tdom \to \tdom\to \tdom $]
  \label{def:trace_concate}
Given two traces $\trace_1 \in \tdom, \trace_2 \in \tdom$, the trace concatenation operator 
$\tracecat$ is defined as:
\[
\trace_1 \tracecat \trace_2 \triangleq
\left\{
\begin{array}{ll} 
  \trace_1 & \trace_2 = [] \lor \trace_1 \in \inftdom \\
  ((\trace_1 :: \event)  \tracecat \trace_2') & \trace_1 \in \ftdom \land \trace_2 = [\event] \tracecat \trace_2'
  % \trace_2 &  \trace_2 \in \inftdom \\
\end{array}
\right.
\]
\end{defn}
}
\begin{defn}(An Event Belongs to A Trace)
  An event $\event \in \eventset$ belongs to a trace $\trace \in \tdom$, i.e., $\event \in \trace$ are defined as follows:
%
\begin{equation*}
  \event \in \trace  
  \triangleq \left\{
  \begin{array}{ll} 
    \etrue                  & \trace =  [\event] \tracecat \trace'
     \land \event = \event' \\
    \event \in \trace' & \trace =  [\event'] \tracecat \trace'
      \land \event \neq \event' \\ 
    \efalse                 & \trace = [] \lor \trace \in \inftdom
  \end{array}
  \right.
\end{equation*}
As usual, we denote by $\event \notin \trace$ that the event $\event$ doesn't belong to the trace $\trace$.
\end{defn}
%
In the rest of the paper, we denote by $\bot$ a value s.t. $\bot < n $ for all $n \in \mathbb{N}$.
\begin{defn}[Counter Notation]
  \label{def:counter}
The counting operator $\counter : \tdom \to \ldom \to (\mathbb{N} \cup \{\bot\})$
counts the appearance of a label in a trace.
\[
\begin{array}{llll}
\counter([(x, l, v)] \tracecat \trace', l ) \triangleq \counter(\trace', l) + 1 & \text{if}~ l = l
&
\counter([(b, l, v)] \tracecat \trace', l) \triangleq \counter(\trace', l) + 1 & \text{if}~ l = l
\\
\counter([(x, l', v)] \tracecat \trace', l) \triangleq \counter(\trace', l)   & \text{if}~ l' \neq l
&
\counter([(b, l', v)] \tracecat \trace', l) \triangleq \counter(\trace', l)   & \text{if}~ l' \neq l
\\
\counter({[]}, l) \triangleq 0 & 
&
\counter(\trace, l) \triangleq \bot & \text{if }~ \trace \in \inftdom
\end{array}
\]
{When the input trace is infinite, $\counter(\trace, l)$ returns $\bot$ for any program label $l$.}
\end{defn}
\begin{defn}[Counter Notation of Label List]
  \label{def:lcounter}
  The counting operator $\lcounter : \tdom \to \ldom \to (\mathbb{N} \cup \{\infty\})$
  counts the appearance of a list of labels $[l_1, \ldots, l_n]$ as follows.
\[
  \begin{array}{ll}
  \lcounter(\trace'' \tracecat \trace', [l_1, \ldots, l_n] ) 
  \triangleq \lcounter(\trace', [l_1, \ldots, l_n]) + 1  & \text{if}~ \pi_2(\trace''[i]) = l_i, \forall i = 1, \ldots, n
  \\ 
  \lcounter([(\_, l, \_)] \tracecat \trace', [l_1, \ldots, l_n] ) 
  \triangleq \lcounter(\trace', [l_1, \ldots, l_n]) & \text{if}~ l \neq l_1
  \\ 
  \lcounter(\trace, [l_1, \ldots, l_n] ) 
  \triangleq \bot & \text{if }~ \trace \in \inftdom
\end{array}
\]
{When the input trace is infinite, $\lcounter(\trace, L)$ returns $\bot$ for any list of labels as well.}
\end{defn}
%
We define the operator $\tracel : \tdom \to \mathcal{P}{(\ldom)}$ projects the label from every event in a trace as a list of program points,
defined as follows,
\[
\tracel([(\_, l, \_)] \tracecat \trace') \triangleq [l] \tracecat \tracel(\trace')
\qquad
\tracel([ ]) \triangleq []
\]