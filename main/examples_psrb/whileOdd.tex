\begin{example}
    [While Odds Algorithm]
    \label{ex:whileOdd}
    \[
      %
      \begin{array}{l}
          \kw{whileOdd}(k) \triangleq \\
          \clabel{ \assign{i}{k} }^{0} ; \\
              \ewhile ~ \clabel{i > 0}^{1} ~ \edo ~ \\
              \qquad \Big(
                \eif(\clabel{i \% 2 == 0 }^{2}, \\
                \qquad \qquad \clabel{\assign{i}{i - 1}}^{3},\\
                \qquad \qquad \clabel{\assign{i}{i - 3}}^{4});
                \Big)
          \end{array}
      \]
    % \end{example}
    { \small
    \begin{figure}
    \centering
    %
    \begin{subfigure}{.4\textwidth}
        \begin{centering}
        \begin{tikzpicture}[scale=\textwidth/15cm,samples=200]
    % Variables Initialization
    \draw[] (5, 1) circle (0pt) node{{ $x^1: {}^1_{1}$}};
    % Variables Inside the Loop
     \draw[] (0, 7) circle (0pt) node{\highlight{ $y^5: {}^{\frac{k}{2}}_{1}$}};
     \draw[] (0, 4) circle (0pt) node{\highlight{ \boldsymbol{$p^6: {}^{\frac{k}{2}}_{1}$}}};
     \draw[] (0, 1) circle (0pt) node{{ $\mathbf{x^7: {}^{k}_{1}}$}};
     % Counter Variables
     \draw[] (5, 7) circle (0pt) node {{$j^0: {}^{1}_{0}$}};
     \draw[] (5, 4) circle (0pt) node {{ $j^3: {}^{k}_{0}$}};
     %
    % Value Dependency Edges:
             % Value Dependency Edges:
             \draw[ thick, -latex,]  (0, 3.5) -- 
             node [] {\highlight{$k $}}(0, 1.5) ;
             \draw[ thick, -Straight Barb] (6.5, 4.5) arc (150:-150:1);
             \draw[](7, 4) node [] {\highlight{$k  $}};
             \draw[ thick, -latex] (5, 4.5)  -- 
             node [right] {\highlight{$k$}}(5, 6.5) ;
             % Value Dependency Edges on Initial Values:
             \draw[ thick, -latex,] (1.5, 1)  -- 
             node [above] {\highlight{$k$}}(4, 1) ;
             %
            \draw[ ultra thick, -latex, densely dotted,] (-0.6, 1.5)  to  [out=-220,in=220]  
            node [left] {\highlight{$k $}}(-0.5, 6.5);
            \draw[ ultra thick, -latex, densely dotted,]  (0.5, 6.5) to  [out=-30,in=30] 
            node [above] {\highlight{$k $}}(0.6, 1.6) ;
             % Control Dependency
            \draw[ thick,-latex] (1.5, 7)  -- 
            node [above] {\highlight{$k$}}(4, 6) ;
            \draw[ thick,-latex] (1.5, 4)  -- 
            node [above] {\highlight{$k$}}(4, 6) ;
             \draw[ thick,-latex] (1.5, 1)  -- 
             node [below] {\highlight{$k $}}(4, 6) ;
    
    \end{tikzpicture}
    \caption{}
    \end{centering}
    \end{subfigure}
    %
            \begin{subfigure}{.5\textwidth}
                \begin{centering}
                \begin{tikzpicture}[scale=\textwidth/11cm,samples=200]
            % Variables Initialization
             \draw[] (5, 1) circle (0pt) node{{ $x^1: {}^{w_1}_{1}$}};
            % Variables Inside the Loop
             \draw[] (0, 7) circle (0pt) node{{ $y^5: {}^{w_k/2}_{1}$}};
             \draw[] (0, 4) circle (0pt) node{{ $p^6: {}^{w_k/2}_{1}$}};
             \draw[] (0, 1) circle (0pt) node{{ $x^7: {}^{w_k}_{1}$}};
             % Counter Variables
             \draw[] (5, 7) circle (0pt) node {{$j^0: {}^{w_1}_{0}$}};
             \draw[] (5, 4) circle (0pt) node {{ $j^3: {}^{w_k}_{0}$}};
             %
             % Value Dependency Edges:
             \draw[ thick, -latex,]  (0, 3.5) -- 
             node [] {\highlight{$\trace_0 \to \env(\trace_0) k $}}(0, 1.5) ;
             \draw[ thick, -Straight Barb] (6.5, 4.5) arc (150:-150:1);
             \draw[](7, 4) node [] {\highlight{$\trace_0 \to \env(\trace_0) k  $}};
             \draw[ thick, -latex] (5, 4.5)  -- 
             node [right] {\highlight{$\trace_0 \to \env(\trace_0) k $}}(5, 6.5) ;
             % Value Dependency Edges on Initial Values:
             \draw[ thick, -latex,] (1.5, 1)  -- 
             node [above] {\highlight{$\trace_0 \to \env(\trace_0) k $}}(4, 1) ;
             %
            \draw[ ultra thick, -latex, densely dotted,] (-0.6, 1.5)  to  [out=-220,in=220]  
            node [left] {\highlight{$\trace_0 \to \env(\trace_0) k $}}(-0.5, 6.5);
            \draw[ ultra thick, -latex, densely dotted,]  (0.5, 6.5) to  [out=-30,in=30] 
            node [above] {\highlight{$\trace_0 \to \env(\trace_0) k $}}(0.6, 1.6) ;
             % Control Dependency
            \draw[ thick,-latex] (1.5, 7)  -- 
            node [above] {\highlight{$\trace_0 \to \env(\trace_0) k $}}(4, 6) ;
            \draw[ thick,-latex] (1.5, 4)  -- 
            node [above] {\highlight{$\trace_0 \to \env(\trace_0) k $}}(4, 6) ;
             \draw[ thick,-latex] (1.5, 1)  -- 
             node [below] {\highlight{$\trace_0 \to \env(\trace_0) k $}}(4, 6) ;
             \end{tikzpicture}
             \caption{}
                \end{centering}
                \end{subfigure}
    \caption{
    (a) The multiple rounds odd example 
    (b) The program-based dependency graph from $\THESYSTEM$
    (c) The execution-based dependency graph.}
        \label{fig:overappr_example}
    \end{figure}
    }
    %
    \end{example}    
    \begin{enumerate}
      \item Step 1: Abstract Transition Graph:
    
    \item Step 2: Path Insensitive Transition Bound Computation
    
    \item Step 3: Program Rephrase through Path Collection on Abstract CFG
    \\
    $\tpath_0 = (0 \to 1)$
    \\
    $\tpath_1 = (1 \to 2), (2 \to 3), (3 \to 1)$
    \\
    $\tpath_2 = (1 \to 2), (2 \to 4), (4 \to 1)$
    \\
    $\tpath_3 = (1 \to \lex)$
    \\
    Rephrased Program
    \[
    \tpath_0 ; LOOP1: \rprepeat(\rpchoose\{\tpath_1, \tpath_2 \}); \tpath_3
    \]
    \item Step 4: While Loop Refinement:
    \[
      \tpath_0 ; LOOP1: \rpchoose\{\rprepeat_3(\rprepeat_1(\tpath_1); \tpath_2) , \rprepeat_4(\rprepeat_2(\tpath_2); \tpath_1) \}; \tpath_3
      \]
    \item Step 5: Outside-In Algorithm
    \\
    $LB(\tpath_0) = 1$
    \\
    $LB(\tpath_3) = 1$
    \\
    $LB(\rprepeat_1(\tpath_1)) = 1 $
    \\
    $LB(\rprepeat_3(\rprepeat_1(\tpath_1); \tpath_2)) = \frac{n}{4} $
    \\
    $LB(\rprepeat_2(\tpath_2)) = 1 $
    \\
    $LB(\rprepeat_4(\rprepeat_2(\tpath_2); \tpath_1)) = \frac{n}{4} $
    % \\
    % $LB(LOOP1: \rpchoose(\rprepeat_2(\cdots), \rprepeat_1(\tpath_1))) 
    % = \max\{m, \frac{n}{m}\} $
    % \\
    \item Step 6: Inside-Out Algorithm
    \begin{itemize}
      \item \textbf{Repeat Chain Set}
      \\
      $rp\mathcal{C}(LOOP1, \tpath_1) = \{\rprepeat_4(\cdots, \tpath_1), \rprepeat_3(\rprepeat_1(\tpath_1); \tpath_2) \to \rprepeat_1(\tpath_1)\}$ \\
      $rp\mathcal{C}(LOOP1, \tpath_2) = \{\rprepeat_3(\cdots, \tpath_2), \rprepeat_4(\rprepeat_2(\tpath_2); \tpath_1) \to \rprepeat_2(\tpath_2)\}$ \\
      $rp\mathcal{C}(\_, \_) = \emptyset$ 
      % \\
      \item \textbf{{Local Repeat Chain Bound} for Every Transition Path $\tpath$ on its Repeat Chain}
      $rpLB(LOOP1, \tpath_1) = \frac{n}{4}$ \\
      $rpLB(LOOP1, \tpath_2) = \frac{n}{4}$ 
      %
      \item \textbf{Loop Chain}
      \\
      $lp\mathcal{C}(\tpath_1) = \{LOOP1\to \tpath_1\}$ \\
      $lp\mathcal{C}(\tpath_2) = \{LOOP1\to \tpath_2\}$ \\
      $lp\mathcal{C}(\tpath_0) = \{\tpath_0\}$ \\
      $lp\mathcal{C}(\tpath_3) = \{\tpath_3\}$ 
      \item \textbf{Nested Loop Bound for Every Transition Path $\tpath$ on its Loop Chain}
      \\
      $rpLB(LOOP1, \tpath_1) = \frac{n}{4}$ \\
      $rpLB(LOOP1, \tpath_2) = \frac{n}{4}$  \\
      $rpLB(\bot, \tpath_0) = 1$ \\
      $rpLB(\bot, \tpath_3) = 1$ 
      \item \textbf{Path Sensitive Reachability Bound For Every Transition Path $\tpath$ }
      \\
      $psRB(\tpath_1) = \frac{n}{4}$ \\
      $psRB(\tpath_2) = \frac{n}{4}$ \\
      $psRB(\tpath_0) = 1$ \\
      $psRB(\tpath_3) = 1$ 
    \end{itemize}
    \item Step 7: Path Sensitive Reachability Bound Computation for Every Location
    \\
    $psRB(\{0, 1\}) = 1$ \\
    $psRB(\{2, 3, 1 \}) = \frac{n}{4}$ \\
    $psRB(\{2, 4, 1\}) = \frac{n}{4}$ \\
    $psRB(\{\lex\}) = 1$ 
    \end{enumerate}

