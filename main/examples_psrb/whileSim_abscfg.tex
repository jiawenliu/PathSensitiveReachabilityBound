\begin{example}[The Abstract Control Flow Graph for The Simple While Loop Example Program]
  \label{ex:whileSim_abscfg}
    For the simple while loop example program, 
its abstract control flow graph is shown as in Figure~\ref{fig:whileSim_abscfg}(b).
For example, the edge $(0 \xrightarrow{a' \leq 0} 1)$ on the top, tells us the command 
$\clabel{\assign{a}{0}}^0$ is executed with next continuation point $1$,
where the 
command $\clabel{\assign{j}{k}}^1$ will be executed next.
The constraint $a' \leq 0$ is a difference constraint, generated by abstracting from the assignment command $\assign{a}{0}$.
It represents that the value of $a$ is less than or equals to $0$ after 
execution of $\clabel{\assign{a}{0}}^0$ and before executing $\clabel{\assign{j}{k}}^1$.
The difference constraint is an inequality relation, 
the left-hand side of the inequality $x'$ denotes the variable $x$
after executing the command at $l$
and the right-hand side describes the variable $x$ in the state before the execution. 
For example, the constraint $a' \leq a+x $ on edge $5 \xrightarrow{a' \leq a + x } 2$ describes the execution of
 the command at line $5$, 
$\clabel{\assign{a}{x + a}}^{5}$. 
The $a'$ on the left side of $a' \leq a+x$ represents the value of $a$ after the assignment,
while the right-hand side $a$ stores the value before the assignment. 
% $top$ means there is no assignment executed, for example, 
% I have 
The boolean constraint $j \leq 0 $ on the edge $2 \xrightarrow{j \leq 0} 6$, 
represents the negation of the testing guard $j > 0$
in the $\ewhile$ command with loop header at line $2$.
%
% The same way for the rest edges' constructions.
\begin{figure} 
  \centering
  \begin{subfigure}{.7\textwidth}
  \begin{centering}
  {\small
  $
  \kw{whileSim(k)} \triangleq
    \begin{array}{l}
        \clabel{ \assign{a}{0}}^{0} ;   
              \clabel{\assign{j}{k} }^{1} ;\\
              \ewhile ~ \clabel{j > 0}^{2} ~ \edo ~ 
              \Big(
               \clabel{\assign{x}{j} }^{3}  ;
               \clabel{\assign{j}{j-1}}^{4} ;
              \clabel{\assign{a}{x + a}}^{5}  \Big);\\
              \clabel{\assign{l}{k * a} }^{6}
          \end{array}
  $
  }
  \caption{}
  \end{centering}
  \end{subfigure}
    \begin{subfigure}{.45\textwidth}
    \begin{centering}
  \begin{tikzpicture}[scale=\textwidth/20cm,samples=200]
  \draw[] (-7, 10) circle (0pt) node{{ $0$}};
  \draw[] (0, 10) circle (0pt) node{{ $1$}};
  \draw[] (0, 7) circle (0pt) node{\textbf{$2$}};
  \draw[] (0, 4) circle (0pt) node{{ $3$}};
  \draw[] (0, 1) circle (0pt) node{{ $4$}};
  \draw[] (-7, 1) circle (0pt) node{{ $5$}};
  % Counter Variables
  \draw[] (6, 7) circle (0pt) node {\textbf{$6$}};
  \draw[] (6, 4) circle (0pt) node {{ $\lex$}};
  %
  % Control Flow Edges:
  \draw[ thick, -latex] (-6, 10)  -- node [above] {$a' \leq 0$}(-0.5, 10);
  \draw[ thick, -latex] (0, 9.5)  -- node [left] {$j' \leq k$} (0, 7.5) ;
  \draw[ thick, -latex] (0, 6.5)  -- node [right] {$j > 0$}  (0, 4.5);
  \draw[ thick, -latex] (0, 3.5)  -- node [right] {$x' \leq j$} (0, 1.5) ;
  \draw[ thick, -latex] (-0.5, 1)  -- node [above] {$j' \leq j - 1$} (-6, 1) ;
  \draw[ thick, -latex] (-6, 1.5)  -- node [left] {$a' \leq x + a$} (-0.5, 7)  ;
  \draw[ thick, -latex] (0.5, 7)  -- node [above] {$ j \leq 0 $}  (5.5, 7);
  \draw[ thick, -latex] (6, 6.5)  -- node [right] {$l' \leq k * a$} (6, 4.5) ;
  \end{tikzpicture}
  \caption{}
    \end{centering}
    \end{subfigure}
    \begin{subfigure}{.45\textwidth}
      \begin{centering}
    %   \todo{abstract-cfg for two round}
    \begin{tikzpicture}[scale=\textwidth/20cm,samples=200]
    \draw[] (-10, 10) circle (0pt) node{{ $0: 1$}};
    \draw[] (0, 10) circle (0pt) node{{ $1: 1$}};
    \draw[] (0, 7) circle (0pt) node{\textbf{$2: k$}};
    \draw[] (0, 4) circle (0pt) node{{ $3: k$}};
    \draw[] (0, 1) circle (0pt) node{{ $4: k$}};
    \draw[] (-10, 1) circle (0pt) node{{ $5: k$}};
    % Counter Variables
    \draw[] (6, 7) circle (0pt) node {\textbf{$6: 1$}};
    \draw[] (6, 4) circle (0pt) node {{ $\lex: 1$}};
    %
    % Control Flow Edges:
  \draw[ thick, -latex] (-8, 10)  -- node [above] {$a' \leq 0$}(-1.5, 10);
  \draw[ thick, -latex] (0, 9.5)  -- node [left] {$j' \leq k$} (0, 7.5) ;
  \draw[ thick, -latex] (0, 6.5)  -- node [right] {$j > 0 $}  (0, 4.5);
  \draw[ thick, -latex] (0, 3.5)  -- node [right] {$x' \leq j$} (0, 1.5) ;
  \draw[ thick, -latex] (-1.5, 1)  -- node [above] {$j' \leq j - 1$} (-8, 1) ;
  \draw[ thick, -latex] (-8, 1.5)  -- node [left] {$a' \leq x + a$} (-1.5, 7)  ;
  \draw[ thick, -latex] (1.5, 7)  -- node [above] {$j \leq 0 $}  (4.5, 7);
  \draw[ thick, -latex] (6, 6.5)  -- node [right] {$l' \leq k * a$} (6, 4.5) ;
    \end{tikzpicture}
    \caption{}
      \end{centering}
      \end{subfigure}
    \caption{(a) The Simple While Loop Example Program $\kw{whileSim(k)}$
    (b) The abstract control flow graph for $\kw{whileSim(k)}$  
    (c) The abstract control flow graph with the path insensitive reachability bound for $\kw{whileSim(k)}$.}
    \label{fig:whileSim_abscfg}
  \end{figure}
\end{example}