\begin{example}
    [While Single Algorithm]
    \label{ex:whileSigle}
    \[
      %
      \begin{array}{l}
          \kw{whileOdd}(k) \triangleq \\
          \clabel{ \assign{i}{k} }^{0} ; \\
              \ewhile ~ \clabel{i > 0}^{1} ~ \edo ~ \\
              \qquad \Big(
                \eif(\clabel{i = 2 }^{2}, \\
                \qquad \qquad \clabel{\assign{i}{i - 1}}^{3},\\
                \qquad \qquad \clabel{\assign{i}{i - 3}}^{4});
                \Big)
          \end{array}
      \]
    % \end{example}
    
    \begin{enumerate}
      \item Step 1: Abstract Transition Graph:
    
    \item Step 2: Path Insensitive Transition Bound Computation
    
    \item Step 3: Program Rephrase through Path Collection on Abstract CFG
    \\
    $\tpath_0 = (0, 1)$
    \\
    $\tpath_1 = (1 \to 2), (2 \to 3), (3 \to 1)$
    \\
    $\tpath_2 = (1 \to 2), (2 \to 4), (4 \to 1)$
    \\
    $\tpath_3 = (1 \to \lex)$
    \\
    Rephrased Program
    \[
    \tpath_0 ; LOOP1: \rprepeat(\rpchoose\{\tpath_1, \tpath_2 \}); \tpath_3
    \]
    \item Step 4: While Loop Refinement:
    \[
      \tpath_0 ; LOOP1: \rpchoose\{\rprepeat_3(\rprepeat_1(\tpath_1); \tpath_2) , \rprepeat_4(\rprepeat_2(\tpath_2); \tpath_1) \}; \tpath_3
      \]
    \item Step 5: Outside-In Algorithm
    \\
    $LB(\tpath_0) = 1$
    \\
    $LB(\tpath_3) = 1$
    \\
    $LB(\rprepeat_1(\tpath_1)) = 1 $
    \\
    $LB(\rprepeat_3(\rprepeat_1(\tpath_1); \tpath_2)) = \frac{n}{4} $
    \\
    $LB(\rprepeat_2(\tpath_2)) = 1 $
    \\
    $LB(\rprepeat_4(\rprepeat_2(\tpath_2); \tpath_1)) = \frac{n}{4} $
    % \\
    % $LB(LOOP1: \rpchoose(\rprepeat_2(\cdots), \rprepeat_1(\tpath_1))) 
    % = \max\{m, \frac{n}{m}\} $
    % \\
    \item Step 6: Inside-Out Algorithm
    \begin{itemize}
      \item \textbf{Repeat Chain}
      \\
      $rp\mathcal{C}(LOOP1, \tpath_1) = \{\rprepeat_4(\cdots, \tpath_1), \rprepeat_3(\rprepeat_1(\tpath_1); \tpath_2) \to \rprepeat_1(\tpath_1)\}$ \\
      $rp\mathcal{C}(LOOP1, \tpath_2) = \{\rprepeat_3(\cdots, \tpath_2), \rprepeat_4(\rprepeat_2(\tpath_2); \tpath_1) \to \rprepeat_2(\tpath_2)\}$ \\
      $rp\mathcal{C}(\_, \_) = \emptyset$ 
      % \\
      \item \textbf{Rp Local Bound for Every Transition Path $\tpath$ on its Repeat Chain}
      $rpLB(LOOP1, \tpath_1) = \frac{n}{4}$ \\
      $rpLB(LOOP1, \tpath_2) = \frac{n}{4}$ 
      %
      \item \textbf{Loop Chain}
      \\
      $lp\mathcal{C}(\tpath_1) = \{LOOP1\to \tpath_1\}$ \\
      $lp\mathcal{C}(\tpath_2) = \{LOOP1\to \tpath_2\}$ \\
      $lp\mathcal{C}(\tpath_0) = \{\tpath_0\}$ \\
      $lp\mathcal{C}(\tpath_3) = \{\tpath_3\}$ 
      \item \textbf{Nested Loop Bound for Every Transition Path $\tpath$ on its Loop Chain}
      \\
      $rpLB(LOOP1, \tpath_1) = \frac{n}{4}$ \\
      $rpLB(LOOP1, \tpath_2) = \frac{n}{4}$  \\
      $rpLB(\bot, \tpath_0) = 1$ \\
      $rpLB(\bot, \tpath_3) = 1$ 
      \item \textbf{Path Sensitive Reachability Bound For Every Transition Path $\tpath$ }
      \\
      $psRB(\tpath_1) = \frac{n}{4}$ \\
      $psRB(\tpath_2) = \frac{n}{4}$ \\
      $psRB(\tpath_0) = 1$ \\
      $psRB(\tpath_3) = 1$ 
    \end{itemize}
    \item Step 7: Path Sensitive Reachability Bound Computation for Every Location
    \\
    $psRB(\{0, 1\}) = 1$ \\
    $psRB(\{2, 3, 1 \}) = \frac{n}{4}$ \\
    $psRB(\{2, 4, 1\}) = \frac{n}{4}$ \\
    $psRB(\{\lex\}) = 1$ 
    \end{enumerate}

        \begin{figure}
     \centering
    \begin{subfigure}{.6\textwidth}
        \begin{centering}
        \begin{tikzpicture}[scale=\textwidth/15cm,samples=150]
    % Variables Initialization
    % \draw[] (-5, 1) circle (0pt) node{{ $z^1: {}^{w_1}_{1}$}};
    % \draw[] (-5, 7) circle (0pt) node{{$p^2: {}^{w_1}_{0}$}};
    \draw[] (-5, 4) circle (0pt) node{{ $z^1: {}^{w_1}_{1}$}};
    % Variables Inside the Loop
     \draw[] (0, 6) circle (0pt) node{{ $y^3: {}^{w_k}_{1}$}};
     \draw[] (0, 2) circle (0pt) node{{ $y^{5}: {}^{w_k}_{0}$}};
     % Counter Variables
     \draw[] (5, 6) circle (0pt) node {{$j^0: {}^{w_1}_{0}$}};
     \draw[] (5, 2) circle (0pt) node {{ $j^8: {}^{w_k}_{0}$}};
     %
     % Value Dependency Edges:
     \draw[ ultra thick, -Straight Barb, densely dotted,] (0.8, 7) arc (220:-100:1);
     % The Weight for this edge
     \draw[](1.2, 9.5) node 
     {\highlight{\footnotesize
            $\trace_0 \to 
            \left\{\begin{array}{ll}
               \env(\trace_0) k & \env(\trace_0) k  \leq 1 \\
           2 & \env(\trace_0) k \geq 2
            \end{array}\right\}
            $}};
     \draw[ thick, -latex] (-1, 6)  to  [out=-130,in=130]  
    % The Weight for this edge
    node [] {\highlight{$\trace_0 \to 1 $}} (-1, 2);
     % Value Dependency Edges on Initial Values:
     \draw[ ultra thick, -latex, densely dotted,] (-1.5, 6)  -- 
    % The Weight for this edge
    node [left] {\highlight{$\trace_0 \to \env(\trace_0) k $}} (-4, 4.7) ;
     %
     % Value Dependency For Control Variables:
     \draw[ thick, -Straight Barb] (6.5, 2.5) arc  (150:-150:1);
    % The Weight for this edge
    \draw[](8, 2) node [] {\highlight{$\trace_0 \to \env(\trace_0) k  $}};
     % Control Dependency
     \draw[ thick, -latex] (5, 2.5)  -- 
    % The Weight for this edge
    node [right] {\highlight{$\trace_0 \to \env(\trace_0) k $}} (5, 5.5);
     \draw[ thick,-latex] (1.5, 6)  -- (3.5, 6) ;
     \draw[ thick,-latex] (1.5, 1.8)  -- 
    % The Weight for this edge
    node [] {\highlight{$\trace_0 \to \env(\trace_0) k $}} (3.5, 6) ;
     \draw[ thick,-latex] (1.5, 6)  -- (3.5, 2) ;
     \draw[ thick,-latex] (1.5, 1.8)  -- (3.5, 2) ; 
    \end{tikzpicture}
     \caption{}
        \end{centering}
        \end{subfigure}
    % \end{wrapfigure}
    % \end{equation*}
    \vspace{-0.4cm}
     \caption{(a) The multi rounds single example
     (b) The execution-based dependency graph.}
    \label{fig:multipleRoundsSingle}
    \vspace{-0.5cm}
    \end{figure}
    \end{example}