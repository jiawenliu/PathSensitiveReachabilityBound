\textbf{Algorithm Overview} 
Given a program $c$ with $\rprog$ as its refined program,
\begin{enumerate}
  \item 
  \emph{Relative Loop Bound} $\lpchB(l, \tpath, \rprog)$,
  \[
    \lpchB(l, \tpath, \rprog) \triangleq
    \left\{
    \begin{array}{ll}
      \inoutB(\rprog, \tpath)
      & \qquad{ l:\rprog = \kw{enclose}(\tpath)}
      \\
      \frac{\lpinit(l, \tpath, \rprog) - \rffinal(\rprog, \tpath)}{\lpinit(l, \tpath, \rprog) - \lpnext(l, \tpath, \rprog)}
      & \qquad o.w.
    \end{array}
    \right\}
    \]
    \item  reachability bound:
    $\inoutB(\rprog, \tpath) \triangleq $

    \[
      \begin{array}{rcl}
        \inoutB(\tpath, \tpath) & \triangleq & 1 \\
        \inoutB(\tpath', \tpath) & \triangleq & \highlight{0} \\
        \inoutB(\rpchoose{\rprog_1, \cdots, \rprog_m }, \tpath) & \triangleq & \max(, \cdots, ) \\
        \inoutB(\rprepeat(\rprog'), \tpath) & \triangleq & \outinB(\rprepeat(\rprog'), \rprog) \times \inoutB(\rprog', \tpath)\\
        \inoutB(\rprepeat(\rprog'), \tpath) & \triangleq 
        & \highlight{\frac{\rfinit(\rprog, \rprog') - \rffinal(\rprog, \rprog')}{\varGD(\rprog, \rprog')}}
         \times \inoutB(\rprog', \tpath)
         \\
        \inoutB(l: \rprog', \tpath) & \triangleq & \lpchB(l, \tpath, \rprog) \times \inoutB(\rprog, \tpath) \\
        \inoutB(l: \rprog', \tpath) & \triangleq & 
        \highlight{
          \lpchB(l, \tpath, \rprog') \times 
        \max\limits_{l':\rprog'' \in \rprog' \land l = \kw{enclose}(l':\rprog'')}
       \{1, \inoutB(l':\rprog'', \tpath)\} }
        \\
        \end{array}
      \]
    
     For every label $l \in \lvar(c)$, $\psRB(c, l)$,
    \[ 
      \psRB(c, l) = \sum\limits_{\tpath \in \rprog \land 
    l \in \tpath} \inoutB(\tpath, \rprog)
    \]

\end{enumerate}


For every simple transition path $\tpath$ in a refined program $\rprog$, 
this algorithm
computes the \textbf{Inside-Out} bound, $\outinB(\rprog, \tpath) \in \mathcal{A}_{in}$
%  on its iteration numbers locally.
% computes the bound
on the execution numbers of $\tpath$ path-sensitively and globally
% for every simple transition path $\tpath$ in this program $\rprog$,
through the following steps.

\subsection{Path-sensitive Local Bound}
\highlight{\textbf{Computation Steps:}}
%
\emph{Repeat Chain Bound}.
  \\
  For every transition path $\tpath$ in a refined program $\rprog$,
  this step computes a path-sensitive
  bound on $\tpath$'s iteration times within its closest enclosing while loop, named \textbf{Repeat Chain} Bound.

  \begin{enumerate}
\item \emph{Repeat Chain} Computation $\rpch(\tpath, \rpattern) \in \mathcal{P}(\rprog)$.
\\
For a transition path $\tpath$ in the refined program $\rprog$ of a program $c$, 
this step computes a repeat chain for $\tpath$ in a repeat pattern $\rpattern$
within a while loop.
%  whose header is at program point $l$.
%  % 
% is a list of repeat patterns nested inside each other, formally defined as follows.
%  with loop header annotation $L$,
% and the $\tpath$ is contained by these repeat patterns recursively.
% % inside the same while loop $L$ as the $\tpath$. It is computed as follows,
% % For a refined while loop program $\rprog_{l} = L_l : \rprog \in \mathcal{RP}$, 
% % with its refined statement $\rprog \in \mathcal{RP}$,
% \\
% \emph{Repeat chain set}
% $\rpchset(l, \tpath, \rprog) \in \mathcal{P}(\mathcal{P}(\rprog))$.
% \\
% For each transition path in the refined program $\tpath \in \rprog$, 
% its repeat chain set 
% $\rpchset(l, \tpath, \rprog) \in \mathcal{P}(\mathcal{P}(\rprog))$
%  is a set of all the repeat chains for $l, \tpath \in \rprog$ in this program.
 %
%  \highlight{\textbf{Computation Steps:}}
%  \begin{defn}[Repeat Chain]
%   % ($\rpch(l, \tpath, \rprog) \in \mathcal{P}(\rprog)$)]
%   \label{def:repeatchain}
% For a refined program $\rprog$,
% a \emph{Repeat Chain},
% % $\rpch(l, \tpath, \rprog)$
% % i.e.,
% %  a repeat pattern with loop header annotation $l$.
% %  efined program contained inside the same while loop $L$ as the $\tpath$. 
% % It is computed as follows,
%   % $\rpch(L_l, \tpath) \in \mathcal{P}(\rprog)$\\
% % \[
%   % \rpch(l, \tpath, \rprog) \triangleq 
%   $\rpattern_m \to \rpattern_{n-1} \to \cdots \to \rpattern_{0} \to \tpath$
%   is a list of repeat patterns nested within each other, 
%   % and
%   $\rpattern_{i}= \rprepeat(\cdots, \rpattern_{i - 1}, \cdots)$ for every $i = m-1, \cdots, 1$.
% % \]
% It satisfies
% % $\rprog_{n} $ has the form 
% 1. $l : \rpattern_m$ is a sub-refined program in $\rprog$
% \\
% 2. $\rpattern_{i}= \rprepeat(\cdots, \rpattern_{i - 1}, \cdots)$ and
%  there isn't any nested loop header ($l'$) or $\rprepeat$ annotation between $\rpattern_{i}$ and $\rpattern_{i - 1}$
%  for every $i = m-1, \cdots, 1$.
% \end{defn}
%
  % \begin{defn}[Repeat Chain]
  %   % ($\rpch(l, \tpath, \rprog) \in \mathcal{P}(\rprog)$)]
  %   \label{def:repeatchain}
  % For a refined program $\rprog$,
  % a \emph{Repeat Chain},
  % % $\rpch(l, \tpath, \rprog)$
  % % i.e.,
  % %  a repeat pattern with loop header annotation $l$.
  % %  efined program contained inside the same while loop $L$ as the $\tpath$. 
  % % It is computed as follows,
  %   % $\rpch(L_l, \tpath) \in \mathcal{P}(\rprog)$\\
  % % \[
  %   % \rpch(l, \tpath, \rprog) \triangleq 
  %   $\rpattern_m \to \rpattern_{n-1} \to \cdots \to \rpattern_{0} \to \tpath$
  %   is a list of repeat patterns nested within each other, 
  %   % and
  %   $\rpattern_{i}= \rprepeat(\cdots, \rpattern_{i - 1}, \cdots)$ for every $i = m-1, \cdots, 1$.
  % % \]
  % % It satisfies
  % % % $\rprog_{n} $ has the form 
  % % 1. $l : \rpattern_m$ is a sub-refined program in $\rprog$
  % % \\
  % % 2. $\rpattern_{i}= \rprepeat(\cdots, \rpattern_{i - 1}, \cdots)$ and
  % %  there isn't any nested loop header ($l'$) or $\rprepeat$ annotation between $\rpattern_{i}$ and $\rpattern_{i - 1}$
  % %  for every $i = m-1, \cdots, 1$.
  % \end{defn}
  %
  \begin{defn}[Repeat Chain of A Simple Transition Path $\rpch(\tpath, \rpattern)$]
    \label{def:repeatchain-loop}
  For a refined program $\rprog$ and a simple transition path $\tpath$, we define
  % $\rpch(l, \tpath, \rprog)$ 
  $\rpch(\tpath, \rpattern) = \rpattern \to \rpattern_1 \to \cdots \to \rpattern_{m} \to \tpath$
  a repeat chain for $\tpath$ in the repeat pattern $\rpattern$
  % loop whose header is at program point $l$,
  if and only if,
  % is a list of repeat patterns nested in a repeat pattern with loop header annotation $l$.
  %  efined program contained inside the same while loop $L$ as the $\tpath$. 
  % It is computed as follows,
    % $\rpch(L_l, \tpath) \in \mathcal{P}(\rprog)$\\
  \begin{itemize}
    % \item $\rpch(l, \tpath, \rprog) = \rpattern_m \to \rpattern_{n-1} \to \cdots \to \rpattern_{0} \to \tpath$,
    % $\rprog_{n} $ has the form 
    % \item $l : \rpattern_m \in \rpchoose{l: \rpattern_1, \cdots, l: \rpattern_n}$ and $\rpchoose{l: \rpattern_1, \cdots, l: \rpattern_n}$ 
    % is the sub-refined program of $\rprog$ for the loop with header at $l$,
    %
    \item  $\rpattern_{i - 1} = \rprepeat(\cdots; \rpattern^{i}; \cdots)$ for every $i = 1, \cdots, m$ and $\rpattern_m = \rprepeat(\tpath)$,
    \item 
    there isn't any nested loop (i.e., no program point of loop header ($l'$)) or $\rprepeat$ annotation between $\rpattern_{i - 1}$ and $\rpattern_{i}$
    for every $i = 1, \cdots, m$.
  \end{itemize}
  \end{defn}
  %
\highlight{
  The repeat chain of a $\tpath$
  filters out the execution of other transition paths that are nested in the same loop as $\tpath$.
  In this way, the following repeat chain bound computations are path-sensitive
  because they only deal with $\tpath$'s execution over this chain.
  }
  %
  % $\rpch(l, \tpath, \rprog) \in \mathcal{P}(\rprog)$ computes 
%
% \begin{defn}[Repeat Chain Computation]
%   \label{def:repeatchain-loop}
% For a refined program $\rprog$ and a simple transition path $\tpath$,
% % $\rpch(l, \tpath, \rprog)$ 
% $\rpattern_m \to \rpattern_{n-1} \to \cdots \to \rpattern_{0} \to \tpath$
% ($\rpch(l, \tpath, \rprog) \in \mathcal{P}(\rprog)$ computes )
% is a repeat chain for the loop whose header is at program point $l$,
% if and only if,
% % is a list of repeat patterns nested in a repeat pattern with loop header annotation $l$.
% %  efined program contained inside the same while loop $L$ as the $\tpath$. 
% % It is computed as follows,
%   % $\rpch(L_l, \tpath) \in \mathcal{P}(\rprog)$\\
% \begin{itemize}
%   % \item $\rpch(l, \tpath, \rprog) = \rpattern_m \to \rpattern_{n-1} \to \cdots \to \rpattern_{0} \to \tpath$,
%   % $\rprog_{n} $ has the form 
%   \item $l : \rpattern_m \in \rpchoose{l: \rpattern_1, \cdots, l: \rpattern_n}$ and $\rpchoose{l: \rpattern_1, \cdots, l: \rpattern_n}$ 
%   is the sub-refined program of $\rprog$ for the loop with header at $l$,
%   %
%   \item $\rpattern_{i} = \rprepeat(\cdots; \rpattern_{i - 1}; \cdots)$ and
%  there isn't any nested loop (i.e., no program point of loop header ($l'$)) or $\rprepeat$ annotation between $\rpattern_{i}$ and $\rpattern_{i - 1}$
%  for every $i = 1, \cdots, m - 1$.
% \end{itemize}
% \end{defn}
%
\item 
\emph{Repeat Chain Set Computation}
$\rpchset(l, \tpath, \rprog) \in \mathcal{P}(\mathcal{P}(\rprog))$.
\\
For a transition path $\tpath$ in the refined program $\rprog$ of a program $c$, 
this step computes a set 
$\rpchset(l, \tpath, \rprog) \in \mathcal{P}(\mathcal{P}(\rprog))$
% computes a set containing
which contains
%  is a set of 
 all repeat chains of $\tpath$ inside the loop with header at $l$.
  % and $\tpath \in \rprog$.
 \\
Let $\rpchoose{l: \rpattern_1, \cdots, l: \rpattern_n}  \in \rprog$
be the subprogram in $\rprog$ corresponds to the  loop with header at $l$, for a simple transition path $\tpath$ in this loop,
the $\rpchset(l, \tpath, \rprog)$ is computed as follows,
 \[
    \rpchset(l, \tpath, \rprog) \triangleq \left\{\rpch(\tpath, \rpattern_m)
    ~\middle\vert~
    l : \rpattern_m \in \rpchoose{l: \rpattern_1, \cdots, l: \rpattern_n} 
    % \land \rpchoose{l: \rpattern_1, \cdots, l: \rpattern_n}  \in \rprog
    \right\}
 \]
%
% \begin{defn}[Repeat Chain Set ($\rpchset(l, \tpath, \rprog) \in \mathcal{P}(\mathcal{P}(\rprog))$)]
%   \label{def:repeatchainset}
  
%   For a refined program $\rprog$ and a simple transition path $\tpath$, the \emph{Repeat China Set} is the set of 
%   all repeat chains for $\rprog$ and $\tpath$.
%   \[
%     \rpchset(l, \tpath, \rprog) \triangleq \left\{\rpch(l, \tpath, \rprog) \right\}
%  \]
% \end{defn}
% \todo{Path-sensitivity Discussion}
%
%
 % \\
% 2. 
% collect the loop chain: 
% $lpchain : \tpath \to \mathcal{P}(\mathcal{P}(\rprog)))$
% \\
% $lpchain(\tpath) = \rprog_n \to \rprog_{n-1} \to \cdots \to \tpath$
% % such that there is at least a $\rpchoose$ and isn't consecutive repeats $\rprepeat$ (i.e., at most one 
% % $\rprepeat$) between any $\rprog_{i - 1}$ and $\rprog_{i}$ for $i = n, \cdots, 1$.
% $\rprog_{i}= \rprepeat^{l}(\cdots, \rprog_{i - 1}, \cdots)$ and
%  there isn't any loop (i.e., $\rprepeat^{L}$) between $\rprog_{i}$ and $\rprog_{i - 1}$ for $i = n, \cdots, 1$.
% \\
% 2. Compute the local bound for every repeat chain as follows:
% \\
% $\rpchB(l, \tpath, \rprog) = \prod\limits_{\rprog_i \in rpchain(\tpath)}
% % \frac{chsInit(\rprog_i, \tpath) - chsFinal(\rprog_i, \tpath)}{\varGD(\rprog_i, \tpath)}
% \outinB(\rprog_i)$
% % \\
% where $\rprepeat^{L}$ is the  closest loop containing $\tpath$, $\rprepeat^{L}(\cdots, \rprog, \cdots)$.
% \\
% 4. Compute the nested local bound for every loop chain as follows, for every of 
% $(\rprog_i, \tpath)$ such that $\rprog_i \in lpchain(\tpath)$,
% \\
% $\lpchB(\rprog_i, \tpath) = 
% % \prod\limits_{\rprog_i \in lpchain(\tpath)}
% \frac{\lpinit(\rprog_i, \tpath) - lpFinal(\rprog_i, \tpath)}{\varGD(\rprog_i, \tpath)}$
% \\
\item  \emph{Repeat Chain Bound Computation}.
\\
For every transition path $\tpath$
in its \highlight{closest enclosed} while loop $l$,
the \emph{Repeat Chain Bound} $\rpchB(l, \tpath, \rprog) \in \mathcal{A}_{in}$ is computed as follows,
% $rpRB: \tpath \to \mathcal{A}_{in}$, $chsRB: (\rprog \times \tpath) \to \mathcal{A}_{in}$
% \\
% For each transition path $\tpath \in \rprog$,
% \\
% 1. First compute the path sensitive reachability choosing bound through their choose chain:
% \\
% $chsRB(\rprog_n, \tpath) = \prod\limits_{\rprog_i \in lpchain(\rprog_n, \tpath)}
% \frac{chsInit(\rprog_i, \tpath) - chsFinal(\rprog_i, \tpath)}{\varGD(\rprog_i, \tpath)}$
  \\
  $\rpchB(l, \tpath, \rprog) = \max \left\{ \prod\limits_{\rprog_i \in ch}  \outinB(\rprog, \rprog_i) 
  ~\middle\vert~ ch \in \rpchset(l, \tpath, \rprog) \right\}
  $,
  \\
  $\rpchB(l, \tpath, \rprog) = \bot$ if $l$ isn't the closest while loop containing $\tpath$.
  %
  \end{enumerate}
  %
  \highlight{\textbf{Theorem Guarantee Discussion: The Soundness and Path-sensitivity}}
  \begin{itemize}
  \item 
  \emph{Soundness}
  \\
  \emph{Repeat-Chain Bound}  $\rpchB(l, \tpath, \rprog)$.
  For a simple transition path $\tpath$ with its closest enclosing while loop at $l$ in a program $\rprog$, $\rpchB(l, \tpath, \rprog)$
  is a sound bound on its execution times in side $l$.
  \\
  For a simple transition path $\tpath$ only enclosed by one repeat pattern $\rprepeat(\tpath)$, 
  we know $\rpchB(l, \tpath, \rprog) = \outinB(\rprepeat(\tpath), \rprog)$.
  Since $\outinB(\rprepeat(\tpath), \rprog)$ is a sound local bound on the iteration times
  of $\rprepeat(\tpath)$ by assuming all the outside loops executes only once.
  In this sense, $\outinB(\rprepeat(\tpath), \rprog)$ is also a sound bound on the iteration times globally.
  \\
  For a simple transition path $\tpath$ nested in multiple repeat patterns $\rpattern_1, \cdots, \rpattern_m$,
  we know $\rpchB(l, \tpath, \rprog) = \prod\limits_{i = 1, \cdots, m}\outinB(\rpattern_i, \rprog)$.
  By the same guarantee from $\outinB(\rpattern_i, \rprog)$, it is sound to multiply each of them.
  
  \item
  \highlight{\textbf{The Path-sensitivity Discussion Informally}}
  \\
  For every multiple-paths loop,
  this bound computes the bound for every simple transition path path-sensitively.
  In comparison to the traditional bound computation methods, they
  estimate the bound of different paths by taking maximum overall paths.
  \\
  The path-sensitivity is guaranteed by the following informal discussion.
  \\
  If a simple transition path isn't nested in any $\rprepeat$ annotation, then
  $\outinB(\tpath) = 1$ is sound and tight because it bounds one execution of the while body accurate.
  \\
  If a simple transition path is nested in some $\rprepeat$ annotations,
  $\rpch(l, \tpath, \rprog) \triangleq \rpattern_m \to \rpattern_{n-1} \to \cdots \to \rpattern_{0} \to \tpath$, we have the following guarantees.
  \\
  1. $\rprog_n$ is a unique repeat pattern of this loop,
  and isn't a sub-pattern of any other repeat patterns in this loop.
  \\
  2. for every $i = n, \cdots, 0$, $\outinB(\rpattern_{i - 1})$ bounds the execution time of $\rpattern_{i - 1}$ with the assumption that $\rprog_{i}$ executes only once.
  \\
  3. $\tpath$ only shows up once inside $\rprog_n$.
  \\
  By 1. 2. 3. guarantees, multiplication of $\outinB(\rpattern_{i - 1})$ for every $i = n, \cdots, 0$ is the accurate execution time of this
  simple transition path in the repeat pattern $\rprog_n$.
  \\
  In the case that loop has multiple repeat patterns, $\rpchoose{l: \rpattern_1, \cdots, l: \rpattern_n}$,
  \\
  We first find every repeat chain of this loop for this simple transition path.
  Then we compute the execution time of $\tpath$ on every repeat chain and take the maximum value.
%  
  \end{itemize}

\subsection{Relative Loop Bound}
\label{sec:}
% \\
% 2. Then compute the path sensitive reachability repeating bound for $\tpath$ as:
% \\
% $RB(\tpath) = \max\limits_{l \in lpchains(\tpath)} \{\rpchB(n, \tpath) \prod\limits_{\rprog_i \in l} lpRB(\rprog_i, \tpath) \}$
% % For $chain \in rpchain(\tpath)$:
% where $lpchains(\tpath)$ is set of $lpchains(\tpath)$ containing all the loop chains of $\tpath$.
%
\emph{Relative Loop Bound:} 
$\lpchB(l, \tpath, \rprog) \in \mathcal{A}_{\lin}$.
\\
For every simple transition path $\tpath$
and a loop whose header is at program point $l$ in a refined program $\rprog$,
 $\lpchB(l, \tpath, \rprog) \in \mathcal{A}_{\lin}$ computes a symbolic expression in $\mathcal{A}_{\lin}$
as the \emph{Relative Loop Bound} for the $\tpath$ and loop $l$.
\\
It bounds the iteration numbers of the loop $l$ w.r.t.
the inner loop $l'$ that is $\tpath$'s closest enclosing loop,
such that,
% the simple transition path $\tpath$.
% This $\tpath$'s closest enclosing loop has the loop header at $l'$ and $l'$ is nested inside the loop $l$.
% \\
% It estimates the iteration numbers of loop $l$ such that 
during these iterations of loop $l$, the nested loop $l'$ is executed.
\highlight{
% the for every program control location,
% how many times the innermost loop of this control location will be touched w.r.t. every
% outside loop it is nested in.
This is distinguished from the traditional methods, which only compute the bound on the iteration number
for the inner loop w.r.t one iteration of the outside loop where it is nested.
\emph{Relative Loop Bound} for $\tpath$ and $l$ bounds the number of the iterations for
the outside loop $l$,
% w.r.t. an inner loop.
%  of every control location.
such that during these iterations of the outside loop $l$, the inner loop $l'$ is entered. 
}
%  is
% the computed as follows.
\begin{enumerate}
  \item \emph{Loop Chain} Computation $\lpch(\tpath, \rprog) \in \mathcal{P}(\rprog)$ 
  \\
  For every simple transition path in a refined program $\tpath \in \rprog$,
  this step computes 
  % is 
  a list of sub-refined program $\rprog_1, \cdots, \rprog_n \in \rprog$
  as the loop chain $\lpch(\tpath, \rprog) \in \mathcal{P}(\rprog)$ of $\tpath$.
  Each of the $\rprog_i$ in this chain corresponds to a sub-refined program of a loop whose header is at program point $l_i$ and
 each loop $l_{i}$ is nested inside the other loop with header $l_{i+1}$.
 We only compute the longest loop chain for $\tpath$ by checking that
$\rprog_n$ isn't nested in any while loop.
%  It is formally defined as follows.
  % $\tpath$ is nested in, and every $\rprog'$ has different loop header annotation.
\begin{defn}[Loop Chain ($\lpch(\tpath, \rprog) \in \mathcal{P}(\rprog)$)]
  \label{def:loopchain}
For a refined program $\rprog$ and a simple transition path $\tpath$ in this program, we define
$\lpch(\tpath, \rprog) = 
\rprog_n \to \rprog_{n-1} \to \cdots \to \tpath$
the loop chain of
$\tpath$ in this program if and only if 
%
% \[ 
%   \lpch(\tpath, \rprog) 
%   \triangleq 
% \rprog_n \to \rprog_{n-1} \to \cdots \to \tpath
% \]
% such that there is at least a $\rpchoose$ and isn't consecutive repeats $\rprepeat$ (i.e., at most one 
% $\rprepeat$) between any $\rprog_{i - 1}$ and $\rprog_{i}$ for $i = n, \cdots, 1$.
% such that 
\begin{itemize}
\item $\rprog_{i}= l_i : (\cdots, l_{i - 1} : \rprog_{i-1}, \cdots)$ for $i = n, \cdots, 1$,
\item there isn't any nested loop between $\rprog_{i}$ and $\rprog_{i - 1}$ for $i = n, \cdots, 1$,
%  and 
 \item $\rprog_n$ isn't nested in any other while loop.
\end{itemize}
\end{defn}
%
\highlight{Theorem Guarantee:}
Every transition path in a refined program $\tpath \in \rprog$ has a unique longest loop chain.
%
%
%
\item  \highlight{\emph{\textbf{Relative Loop Bound:}}
$\lpchB(l, \tpath, \rprog) \in \mathcal{A}_{\lin}$}.
\\
For a loop label $l$ and a transition path $\tpath$ in a refined program $\rprog$,
 $\lpchB(l, \tpath, \rprog)$ computes a symbolic expression in $\mathcal{A}_{\lin}$ named \emph{Relative Loop Bound}
 for $\tpath$ and $l$.
\\
It bounds the number of $l$'s iteration numbers,
%  will 
% bound the execution times of $L_{t_0}$
% in each single execution of the $L_{t_i}$ for every
such that during these iterations, the closest loop $l'$ enclosing $\tpath$ will be executed.
It is formally computed as follows.
%  bound of $\tpath$ in the while loop with loop label $l$.
% \\
% For a refined program $\rprog$, 
% % with its refined statement $\rprog \in \mathcal{RP}$,
% % \\
% % for 
% each transition path in this refined program, $\tpath \in \rprog$ with its loop chain set $\rpchset(\tpath)$,
% \\
% and every loop label $l \in \lpch(\tpath) $ and $\lpch(\tpath)  \in \rpchset(\tpath)$:
% % \\
% % $(l, \tpath)$ such that $l \in \lpch(\tpath)$,
% \\
% $\lpchB(l, \tpath, \rprog) = \rpchB(l, \tpath, \rprog)$ if $\rpchB(l, \tpath, \rprog) \neq \bot$.
% \\
% $\lpchB(l, \tpath, \rprog) = 
% % \prod\limits_{\rprog_i \in lpchain(\tpath)}
% \frac{\lpinit(l, \tpath) - \rffinal(\tpath)}{\lpinit(l, \tpath) - \lpnext(l, \tpath)}$
% \\
\begin{defn}[Relative Loop Bound ($\lpchB(l, \tpath, \rprog) \in \mathcal{A}_{\lin}$)]
  \label{def:relatedloop_bound}
For every simple transition path in a refined program $\tpath \in \rprog$
and every loop $l \in \lpch(\tpath, \rprog)$,
% where $\lpch(\tpath)  \in \rpchset(\tpath)$, 
the \emph{Relative Loop Bound} $\lpchB(l, \tpath, \rprog)$ is computed as follows,
\[
  \lpchB(l, \tpath, \rprog) \triangleq
  \left\{
  \begin{array}{ll}
    \rpchB(l, \tpath, \rprog)  
    & \qquad \rpchB(l, \tpath, \rprog) \neq \bot
    \\
    \frac{\lpinit(l, \tpath, \rprog) - \rffinal(\rprog, \tpath)}{\lpinit(l, \tpath, \rprog) - \lpnext(l, \tpath, \rprog)}
    & \qquad o.w.
  \end{array}
  \right\}
  \]
\end{defn}
The computations of the operations $\lpinit(l, \tpath, \rprog)$ and $\lpnext(l, \tpath, \rprog)$
are formally computed as follows,
\begin{itemize}

\item For a transition path $\tpath \in \paths(\absG(c))$ and a loop label $l$ in this transition path's loop chain.
Let $\rpattern_l$ be the repeat pattern with label $l$, i.e., $l: \rpattern_l \in \rprog$, 
the abstract loop chain initial state $\lpinit(l, \tpath, \rprog) \in \mathcal{P}(\absstate)$ is computed as follows,
\[
  \lpinit(l, \tpath, \rprog) \triangleq 
  \bigwedge_{x \in VAR(\tpath)}
  % \left\{ 
  x = arg\max_{l_1}
  \left\{
      \begin{array}{l}
    (v, (l_1, dc, l_2)) \in \reset(x) 
        \\ 
    % \qquad
    \land \absinit(\rpattern_l) \leq l_1 \leq \absinit(\tpath)
    \land l: \rpattern_l \in \rprog
    \end{array}
    \right\}
  % \right\}
  \]
\item
For the transition path $\tpath \in \paths(\absG(c))$ and in a refined program $\rprog$,
%  with the loop label $l: \rprog_i$, its 
$\rfnext(l, \tpath, \rprog)$ is computed as follows,
%
\[
  \begin{array}{l}
  \lpnext(l, \tpath, \rprog) \triangleq 
  \bigwedge\limits_{x \in VAR(\tpath)}
  % \left\{ 
    x =   
    \begin{array}{l}
  \sum\limits_{(x, \absevent) \in \inc(x) }\left\{ 
    \varinvar(y) + v ~\middle\vert~ \absevent = (l, x' \leq y + v, \_) \land l \in \lvar(\rpattern_l)\right\}
    \\ \qquad 
    - \sum\limits_{(x, \absevent) \in dec(x) }\left\{ 
      \varinvar(y) + v 
      ~\middle\vert~ \absevent = (l, x' \leq y + v, \_) \land l \in \lvar(\rpattern_l) \right\}
      \\
      \qquad 
      \land l: \rpattern_l \in \rprog
    \end{array}
  % \right\}
  \end{array}
\]
%
\end{itemize}
% $rpRB: \tpath \to \mathcal{A}_{in}$, $chsRB: (\rprog \times \tpath) \to \mathcal{A}_{in}$
% \\
% For each transition path $\tpath \in \rprog$,
% \\
% 1. First compute the path sensitive reachability choosing bound through their choose chain:
% \\
% $chsRB(\rprog_n, \tpath) = \prod\limits_{\rprog_i \in lpchain(\rprog_n, \tpath)}
% \frac{chsInit(\rprog_i, \tpath) - chsFinal(\rprog_i, \tpath)}{\varGD(\rprog_i, \tpath)}$
% \\
\end{enumerate}
%

\highlight{\textbf{Theorem Guarantee Discussion: The Soundness and Improvements}}
\begin{itemize}
\item 
\emph{Soundness}
\\
\emph{Relative-Loop Bound} $\lpchB(l, \tpath, \rprog)$. In a refined program $\rprog$,
for its every loop at the program point $l$ and a transition path $\tpath$ inside this loop,
the \emph{Relative Loop Bound} $\lpchB(l, \tpath, \rprog)$ is a symbolic bound
on the number of $l$'s iteration numbers,
%  will 
% bound the execution times of $L_{t_0}$
% in each single execution of the $L_{t_i}$ for every
such that during these iterations, the closest loop $l'$ enclosing $\tpath$ will be executed.
\\
The soundness is guaranteed by the operation
 $\frac{\lpinit(\rprog, \tpath) - \rffinal(\rprog, \tpath)}{\lpinit(\rprog, \tpath) - \lpnext(\rprog, \tpath)}$.
 In this operation, $\lpnext(\rprog, \tpath)$ computes the variables states of the $\tpath$
after visited the program point $l$ the second time and before visiting any other program point.
% it is 
In the same time, the soundness also relies on an external solver. 
\item 
\highlight{
  \emph{Improvements}
  \\
  $\lpinit(l, \tpath)$, and $\lpnext(l, \tpath)$ can be computed using the 
  $\kw{INIT}(c, i, \absinit(\tpath))$ and $\kw{NEXT}(c, i, \absinit(\tpath))$ from paper \cite{GulwaniJK09}.
  % \\
Then, based on the same semantics and syntax,
the $\lpchB(l, \tpath)$ can be computed as
\\
$I(l, l') = \kw{BOUNDFINDERD(INIT(c, i, \absinit(\tpath)), NEXT(c, i, \absinit(\tpath)), V_{\ln})}$.
% from paper \cite{GulwaniJK09}.
\\
There are two improvements comparing to their method.
\\
1. $I(l, l')$ is the bound on iteration numbers of the inner loop $l'$ in each iteration of outside loop.
This is equivalent to the OutIn bound of $l'$ if $l$ is the closest enclosing loop of $l'$.
By multiplying $I(l, l')$ with $\outinB(l')$
 they assume $l'$ is executed in every $l$'s iteration.
 In this sense, they over-approximate the iteration numbers of the inner loop $l'$.
 \\
2. However, $\kw{BOUNDFINDERD}$ in paper~\cite{GulwaniJK09} is an arbitrary interface computing the bound on iteration numbers of $l$
separately.
The efficiency and accuracy of their algorithm fully depend on this arbitrary interface.
\\
In comparison to them, I also provide an efficient and accurate bound computation method using the
ranks computed in Definition~\ref{def:edge_pathinsensitivebound} over the abstract transition graph $\absG(c)$.
 }
\end{itemize}
%
\subsection{Path-sensitive Global Bound}

The \emph{Inside-Out Bound} ($\inoutB(\tpath, \rprog) \in \mathcal{A}_{in}$).
\\
For every simple transition path $\tpath \in \rprog$,
its \emph{Inside-Out Loop Bound}
 $\inoutB(\tpath, \rprog) \in \mathcal{A}_{in}$ is 
% Compute 
the path sensitive reachability-bound on $\tpath$'s execution numbers,
computed as follows.
\begin{defn}[{Inside-Out Loop Bound} ($\inoutB(\tpath, \rprog) \in \mathcal{A}_{in}$)]
  \label{def:outin_bound}
  Given a refined program $\rprog$, for every transition path $\tpath \in \rprog$, 
  its \emph{Inside-Out Loop Bound}
  $\inoutB(\tpath, \rprog)$ is 
 % Compute 
\[
  \inoutB(\tpath, \rprog) =
  \prod\limits_{l \in \lpch(\tpath, \rprog)} \rpchB(l, \tpath, \rprog).
% lpRB(\rprog_i, \tpath) 
% ~ \middle\vert~ l \in lp\mathcal{C}(\tpath) \right
\]
\end{defn}
% For $chain \in rpchain(\tpath)$:
% where $lpchains(\tpath)$ is set of $lpchains(\tpath)$ containing all the loop chains of $\tpath$.

\highlight{\textbf{Theorem Guarantee Discussion:}}
\emph{Soundness} of the \emph{Inside-Out Bound}  
For every simple transition path $\tpath$, the $\inoutB(\tpath, \rprog)$
is a sound upper bound on its execution times globally.
\\
In every base case, a simple transition path not nested in any loop, or just inside one loop.
Then $\inoutB(\tpath, \rprog) = \rpchB(l, \tpath, \rprog)$, which is the bound on its execution times without considering
outside loops' executions. Since there isn't any outside loop, this operation is sound.
In the same time, the soundness is guaranteed by the $\rpchB(l, \tpath, \rprog)$.
\\
For every simple transition path nested in multiple loops, if $l$ is the closest loop $\tpath$ comes from, then
$\inoutB(\tpath, \rprog) =
\prod\limits_{l' \in \lpch(\tpath, \rprog)} \rpchB(l', \tpath, \rprog) \times \rpchB(l, \tpath, \rprog)$.
Since $\lpchB(l', \tpath, \rprog)$ bounds the number of $l'$'s iteration numbers,
%  will 
% bound the execution times of $L_{t_0}$
% in each single execution of the $L_{t_i}$ for every
such that during these iterations, the $\tpath$ will be executed,
it is sound to multiply these bounds by $\rpchB(l, \tpath, \rprog)$.
In the same time, the soundness also relies on the soundness of $\rpchB(l, \tpath, \rprog)$ and $\rpchB(l', \tpath, \rprog)$. 
% Since 
% In this way, it assumes all the outside patterns and loop execution only once.
% In the other two cases,  $\varGD(\rprog, \rprepeat(\rpattern))$ and $\varGD(\rprog, \rpseq(\rpattern_1, \rpattern_2))$
% has the same assumption.
% \todo{reference}
% \end{itemize}
%
\begin{thm}[Soundness of Inside-Out Bound]
  \label{thm:sound_insideout}
  For every program $c$ with it refined program $\rprog$ and 
  every simple transition path $\tpath$ in this program,
   the $\inoutB(\tpath, \rprog)$
is a sound bound on the execution times of $\tpath$ in $c$.
  \[
    \begin{array}{l}
    \forall c \in \cdom, \tpath \in \absG(c), \trace_0 \in \tdom_0(c), \trace \in \tdom, \rprog \st 
    % \rpattern \in \rprog \land
    \rprog = Alg.4(c)
    % \\ \qquad
    \land
    \config{c, \trace_0} \to^* 
    % \config{\highlight{\tpath; \rprog', \trace_0 \tracecat \trace_1} 
    % \land 
    % \config{\rpattern, \trace_0 \tracecat \trace_1} \to^* 
    \config{{\eskip, \trace_0 \tracecat \trace}}
    \\ \qquad
    \implies
    \vcounter(\trace, L(\tpath)) \leq \inoutB(\tpath, \rprog)(\trace_0).
    \end{array}
    \]
\end{thm}
\todo{Formalize the Proof}
\\
Theorem~\ref{thm:sound_insideout} guarantees that
for every simple transition path $\tpath$ in a program $c$ with its refined program $\rprog$,
the $\inoutB(\tpath, \rprog)$
is a sound upper bound on its execution times in $c$.
% by assuming
% that all the loops and repeat patterns where $\rpattern$ is nested execute only once.
%
\subsection{Path-sensitive Reachability-Bound}
\label{sec:psrbcompute}
For every program $c$ and a program point $l \in \lvar(c)$, with $\rprog$ as its refined program,
%  in a program $c$,
its path-sensitive reachability-bound ($\psRB(l, \rprog)$) is a symbolic sound bound on the executing times of $l$.
%
 \begin{defn}
  \label{def:label_psrb}
Given a program $c$ with $\rprog$ as its refined program,
%  with 
% \emph{Global Loop Bound} $\inoutB(\tpath)$
% computed for its every transition path $\tpath \in \rprog$  notated by $\inoutB(\tpath)$,
%  for each of its transition path $\tpath \in \rprog$ 
% with the \emph{Global Loop Bound}
% computed as above, notated by $\inoutB(\tpath)$.
for every label $l \in \lvar(c)$, $\psRB(c, l)$ is computed as follows,
% \\
\[ 
  \psRB(c, l) = \sum\limits_{\tpath \in \rprog \land 
l \in \tpath} \inoutB(\tpath, \rprog)
\]
 \end{defn}
%  \textbf{Semantics of }
 \begin{thm}[Soundness of the Path-sensitive Reachability Bound Estimation]
  \label{thm:pathsensitive_rb_soundness}
For every program ${c}$ and every label $l$ in this program,
%  $c$,
% such that $(l, w) \in \exeRB(c)$, 
% and any initial trace $\trace_0 \in \tdom_0(c)$ with 
% $\config{{c}, \trace_0} \to^{*} \config{\eskip, \trace_0\tracecat \vtrace} $ 
% and $\config{\psRB(c, l), \trace_0} \earrow v$,
% % for some generated evaluation trace $\vtrace \in \tdom$,
% we have $ w(\trace_0) \leq v $.
$\config{\psRB(c, l)}$ is a \emph{Reachability-Bound} for $l$ in $c$.
%
\[
  \forall c \in \cdom, \trace_0 \in \tdom_0(c), \trace \in \tdom \st 
  \config{{c}, \trace} \to^{*} \config{\eskip, \trace_0 \tracecat \vtrace} 
  \implies \config{\psRB(c, l)}(\trace_0) \geq \vcounter(\vtrace, l) 
  \]
\end{thm}
%
% \begin{thm}[Soundness of the Path-sensitive Reachability Bound Estimation]
%   \label{thm:pathsensitive_rb_soundness}
% Given a program ${c}$, for every label $l$ of this program $c$ such that $(l, w) \in \exeRB(c)$, 
% and any initial trace $\trace_0 \in \tdom_0(c)$ with 
% % $\config{{c}, \trace_0} \to^{*} \config{\eskip, \trace_0\tracecat \vtrace} $ 
% and $\config{\psRB(c, l), \trace_0} \earrow v$,
% % for some generated evaluation trace $\vtrace \in \tdom$,
% we have $ w(\trace_0) \leq v $.
% %
% \[
%   \begin{array}{l}
%   \forall (l, w_{t}) \in \exeRB(c),
%   % (x^l, w_{p}) \in \progV, 
%   \trace_0 \in \tdom_0(c), 
%   v \in \mathbb{N} \st
%   \config{\psRB(c, l), \trace_0} \earrow v
%   \implies
%   % \right\} 
%   w_{t}(\trace_0) \leq v
%   \end{array}
% \]
% \end{thm}
%
Proof of this theorem is in Appendix~\ref{apdx:pathsensitive_rb_soundness}.
\paragraph{Path-sensitive Reachability Bound through An Example}
\begin{example}
  [\todo{While Single Algorithm}]
  \label{ex:whileSigle}
  $$
  \kw{whileSim(k)} \triangleq
    \begin{array}{l}
        \clabel{ \assign{a}{0}}^{0} ;   
              \clabel{\assign{j}{k} }^{1} ;\\
              \ewhile ~ \clabel{j > 0}^{2} ~ \edo ~ 
              \Big(
               \clabel{\assign{x}{j} }^{3}  ;
               \clabel{\assign{j}{j-1}}^{4} ;
              \clabel{\assign{a}{x + a}}^{5}  \Big);\\
              \clabel{\assign{l}{k * a} }^{6}
          \end{array}
  $$
\end{example}
  % 
  \begin{enumerate}
    \item  \textbf{The Constraint Program (Abstract Control Flow Graph)}
    is illustrated in Example~\ref{ex:whileSim_abscfg} in Figure~\ref{fig:whileSim_abscfg}(b)
  
    \item \textbf{Program Refinement}
    \\
    The loop free simple transition paths are computed as follows,
    \\
  $\tpath_0 =  (0 \to 1), (1 \to 2)$
  \\
  $\tpath_1 =  (2 \to 3), (3 \to 4), (4 \to 5), (5 \to 2)$
  \\
  $\tpath_2 = (2 \to 6 \to \lex)$
  \\
  For simplicity, I omitted the vertices sequence and the constraints for every edge in the transition path. 
  In the following examples in Section~\ref{sec:example}, they 
  are omitted as well. 
  \\
  In this example the complete transition paths are as follows,
  \\
  $\tpath_0 =  
    \left\{ \begin{array}{l}
    \text{Edges Sequence}: ((0, a \leq 0, 1), (1, j \leq k, 2))
    \\
    \text{Vertices Sequence}: (0, 1, 2)
    \end{array}\right\}
  $
  \\
  $\tpath_1 =  
  \left\{ \begin{array}{l}
    \text{Edges Sequence}: ((2, j > 0, 3), (3, x \leq j, 4), (4, j \leq j - 1, 5), (5, a \leq x + a, 2))
    \\
    \text{Vertices Sequence}: (2, 3, 4, 5, 2)
    \end{array}\right\}
  $
  \\
  $\tpath_2 = 
  \left\{ \begin{array}{l}
    \text{Edges Sequence}: ((2, j \geq 0, 6), (6, l \leq k*a, \lex))
    \\
    \text{Vertices Sequence}: (2, 6, \lex)
    \end{array}\right\}
    $
  \\
  \textbf{Refined Program}:
  \[
    \tpath_0 ; 1: \rprepeat(\tpath_1); \tpath_2
  \]
  \item \textbf{Outside-In Algorithm}: The \emph{OutIn} bound for the $\rprog$ and every nested repeat patterns.
  \\
  $\outinB(\tpath_0) = 1$
  \quad
  $\outinB(\tpath_2) = 1$
  \quad
  $\outinB(\rprepeat_1(\tpath_1)) = k $
  % \\
  \item \textbf{Inside-Out Algorithm}
  \begin{itemize}
    \item \textbf{Repeat Chain Set}
    \\
    $\rpchset(1, \tpath_1) = \{\rprepeat_1(\tpath_1)\}$ 
    % \\
    % $\rpchset(1, \tpath_2) = \{\rprepeat_2(\tpath_2), \rprepeat_3(\rprepeat_2(\tpath_2); \tpath_1) \to \rprepeat_2(\tpath_2)\}$ \\
    $\rpchset(\_, \_) = \emptyset$ 
    % \\
    \item \textbf{Repeat Chain Bound} for every simple transition path $\tpath$ through its \emph{Repeat Chain}s
    \\
    $\rpchB(1, \tpath_1) = k$
    %
    \item \textbf{Loop Chain}
    \\
    $\lpch(\tpath_1) = 1\to \tpath_1$ \quad
    $\lpch(\tpath_0) = \tpath_0$ \quad
    $\lpch(\tpath_3) = \tpath_3$ 
    \item \textbf{{Relative Loop Bound}} for every simple transition path $\tpath$ through its \emph{Loop Chain}
    \\
    $\rpchB(1, \tpath_1) =   k$
  \quad
    $\rpchB(\bot, \tpath_0) = 1$ \quad
    $\rpchB(\bot, \tpath_2) = 1$ 
    \item \textbf{Path-Sensitive Reachability-Bound} for every simple transition path $\tpath$
    \\
    $\inoutB(\tpath_1) = k$
\quad
    $\inoutB(\tpath_0) = 1$ \\
    $\inoutB(\tpath_2) = 1$ 
  \end{itemize}
  \item Step 7: Path Sensitive Reachability Bound Computation for Every Location
  \\
  $\psRB(\{0, 1, 2\}) = 1$ \\
  $\psRB(\{3, 4, 5, 2 \}) = k$
   \\
  $\psRB(\{6, \lex\}) = 1$
  \end{enumerate}
