Our static program analysis algorithm computes 
% an upper bound on the 
% execution-based reachability times 
a \emph{reachability-bound} for every program point $l$ in a program $c$ in a path-sensitive manner.
The algorithm is summarized into the following three steps,
\begin{enumerate}
\item In Section~\ref{sec:progabs}, we 
compute the \emph{Abstract Transition Graph}, $\absG(c)$ for a program $c$ through program abstraction.
This step first computes the abstract transition 
for every labeled command in $c$ and each abstract transition corresponds to an edge in $\absG(c)$.
\item The second step in Section~\ref{sec:refine}
computes the \emph{Refined Program}, $\rprog$ for a program $c$ based on 
its abstraction transition graph by the program refinement algorithm in~\cite{GulwaniJK09}.
This algorithm transforms the multiple-path loops
into multiple loops where
the interleaving of paths is explicit.

\item At the same time, we compute the \emph{Ranking Function} in Section~\ref{sec:rank}\footnote{\textbf{ranking function} is the name used in \cite{SinnZV14}
and \textbf{local bound} is the name used in \cite{ZulegerGSV11, SinnZV17}.
We refer to the two names as having the same meaning in this paper.} for every transition edge 
and estimate the upper bound invariant w.r.t, the input variables on every ranking function's maximum value.

\item The \emph{path-sensitive reachability-bound} algorithm computes the \emph{reachability-bound} for every program point based on the \emph{refined program} and the upper bound invariant of the \emph{ranking function} computed in previous stages.
It also requires computing several quantities to obtain the final result.
%  the \emph{path reachability-bound} and the \emph{loop reachability-bound}.
Section~\ref{sec:psrb} introduces these quantities that are needed to be computed in this algorithm and the following sections describe the computations of each quantity. 
% \begin{enumerate}
 \item Section~\ref{sec:pathlocalrb} gives the computation of the first quantity -- the local \emph{path reachability-bound}, $\outinB(\rprog, \tpath)$ for every \emph{simple transition path} $\tpath$ in Section~\ref{sec:pathlocalrb}. 
 This is the local bound on the numbers that  $\tpath$ is evaluated when executing the closet loop  $\rprog$ in which $\tpath$ is nested.
 This local bound considers only the execution of the innermost loop where $\tpath$ is located.
%   closest enclosing loop, i.e., $\kw{enclosed}(\tpath)$.
 \item Then in Section~\ref{sec:looprb}, we compute the second quantity -- \emph{loop reachability-bound}, $\lpchB(l: \rprog, \tpath)$ for every \emph{simple transition path} $\tpath$
 w.r.t its outer loop $l: \rprog$. 
 This is the bound on the iteration numbers of the outer loop $l$,
 such that during these iterations, the inner loop where $\tpath$ is located is reached.
 \item Then using the local \emph{path reachability-bound} and the  \emph{loop reachability-bound},
 we compute the \emph{path reachability-bound} globally.
 The \emph{path reachability-bound}, $\inoutB(\rprog, \tpath)$ is a global upper bound on the execution times of a \emph{simple transition path} $\tpath$ computed in Section~\ref{sec:pathrb}.
 
 \item With the global reachability-bound of every simple transition path, now we can sum up all the \emph{path reachability-bounds}, $\inoutB(\rprog, \tpath)$ over the $\tpath$ which contains the program point $l$, and compute the final result -- \emph{reachability-bound}, $\psRB(c, l)$ for every program point $l \in \lvar(c)$ as in Definition~\ref{def:point_psrb}.
 %% 
% \end{enumerate}
\end{enumerate}