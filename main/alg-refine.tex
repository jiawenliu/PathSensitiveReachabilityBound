There are three steps as follows.
\begin{enumerate}
  \item \textbf{Collecting} all \emph{simple transition paths}.
  Every \emph{simple transition paths}, $\tpath \in \paths(\absG(c))$ 
  contains only the edges of atomic assignment or guard transitions without interleaving other paths.
  Each of them corresponds to a path in the flatten program in Definition~4.1 in \cite{GulwaniJK09}.
  \item \textbf{Rewriting} the program $c$ by rearranging all \emph{simple transition paths} as the syntax in \cite{GulwaniJK09} and preserves the same semantics.
  \item \textbf{Refining} the program and computing the 
  refined program, $\rprog$ by Algorithm~1 in paper~\cite{GulwaniJK09}.
  This step invokes the algorithm REFINE from paper~\cite{GulwaniJK09} and compute the 
  refined program $\rprog$ for a program $c$ given the rewritten program as input.
\end{enumerate}

\subsection{The Simple Transition Path}
We first collect the loop headers $\loopl(c) \subseteq \lvar(c)$ from a program $c$, which is the set of all program points corresponding to the loop headers in program $c$.
\begin{defn}[Loop Headers ($\loopl : \cdom \to \mathcal{P}(\ldom)$)]
  \label{def:loopl}
  \[
  \loopl(c) \triangleq 
  \left\{
    \begin{array}{ll}
      \{\}  & {c} = \clabel{\assign x e}^{l} \\
      \loopl({c_1}) \cup \loopl({{c_2}})  & {c} = {c_1};{c_2} \\
      \loopl(c_t) \cup \loopl({{c_f}})   & {c} =\eif(\clabel{\bexpr}^{l}, c_t, c_f) \\
  \loopl(c_w) \cup \{l\}, &  {c}   = \ewhile \clabel{\bexpr}^{l} \edo (c_w)
  \end{array}
\right.
\]
  \end{defn}
% \begin{defn}[Loop Path]
%   \label{def:looppath}
% A simple transition path
% $\tpath \in \paths(\absG(c))$ for the program $c$, is a path on its abstract transition graph $\absG(c) = (\absV(c), \absE(c))$ with 
% \begin{itemize}
% \item a vertices sequence $(l_0, \ldots, l_n)$, where $l_i \in \absV(c)$ for every $i = 0, \ldots, n$ and
% %
% \item an edge sequence $(e_1, \ldots, e_n)$, where $e_i = (l_{i - 1}, dc_i, l_{i}) \in \absE(c)$ for every $i = 1, \ldots, n$,
% \end{itemize}
% %
% satisfying:
% \begin{itemize}
%   \item $l_i \neq l_j$ for every $i = 0, \ldots, n$ and $j = 0, \ldots, {n - 1}$,
%   \item $l_0$ is either the program point of a loop header or the program entrance ($l_0 = 0$),
%   i.e., $l_0 \in \loopl(c) \cup \{ 0 \}$
%   \item and $l_n$ is either the program point of a loop header or the program exit ($l_n = \lex$),
%   i.e., $l_0 \in \loopl(c) \cup \{ \lex \}$.
% \end{itemize}
% \end{defn}

\begin{defn}[Simple Tansition Path]
  \label{def:tpath}
A \emph{simple transition path}
$\tpath \in \paths(\absG(c))$ for the program $c$, is either a simple cyclic path, which has the same start- and end-point
or a simple path has either different while loop headers, the program entrance or exit as its start- and end-point
without visiting any loop header inside the path.
\\
Specifically, a path $l_0 \xrightarrow{dc_0} l_1 \xrightarrow{dc_1} \ldots l_n \in \paths(\absG(c))$ with the
vertices sequence $(l_0, \ldots, l_n)$, where $l_i \in \absV(c)$ for every $i = 0, \ldots, n$ and
%
the edge sequence $(e_1, \ldots, e_n)$, where $e_i = (l_{i - 1}, dc_i, l_{i}) \in \absE(c)$ for every $i = 1, \ldots, n$,
%
is a \emph{simple transition path} if and only if it satisfies,
\begin{itemize}
  \item $l_i \neq l_j$ for every $i = 0, \ldots, n$ and $j = 0, \ldots, {n - 1}$,
  \item $l_0$ is either the program point of a loop header or the program entrance ($l_0 = 0$),
  i.e., $l_0 \in \loopl(c) \cup \{ 0 \}$
  \item and $l_n$ is either the program point of a loop header or the program exit ($l_n = \lex$),
  i.e., $l_0 \in \loopl(c) \cup \{ \lex \}$,
  \item and $l_i \notin \loopl(c) \cup \{ 0, \lex \}$ for every $i = 1, \ldots, n-1$.
\end{itemize}
\end{defn}


\paragraph{Example.}
\todo{The walk through example}
$2 \to 3 \to 6 \to 2$ is a \emph{simple transition path} on $\absG(\kw{twoPathsWhile}(n, m))$ in Figure~\ref{fig:twoPathsWhile_abscfg}(b).
However, $2 \to 3 \to 6 \to 2 \to 3 \to 4 \to 5 \to 2$ is not a \emph{simple transition path} because it is not simple (the program points $2$ and $3$ are visited twice).
In Figure~\ref{fig:threeWhile-overview}(b), $1 \to 2 \to 3 \to 4 \to 5 \to 6$ is not a \emph{simple transition path} on $\absG(\kw{threeNestedWhile}(n, m, N))$ because it visits a loop header $3$ inside the path.

\subsection{Program Refinement}
\paragraph{Rewrite the Program.}
\begin{algorithm}
  \caption{Program Rewriting $\kw{Rewrite}$}
  \label{alg:alg-refine_rewrite}
  \begin{algorithmic}[1]
    \REQUIRE program $c$
    \STATE collects all $c$'s \emph{simple transition path}s from $\absG(c)$, $\tpath_1, \ldots, \tpath_n \in \paths(\absG(c))$.
    \STATE \textbf{init}: candidate set $W = \{c_1, \ldots, c_n\}$, where $c_i = \tpath_i$ and $i = 1, \ldots, n$
    \STATE \textbf{while} $W.size()> 1$:
    \STATE \quad create $c' = \rpchoose{c_1, \ldots, c_m}$ 
    s.t. $c_i \in W \land c_i[0] = c_j[0] \land c_i[-1] = c_j[-1], i, j = 1, \ldots, m$.
    \\ \quad $W.add(c')$ \qquad $W.remove(c_1, \ldots, c_m)$
    \STATE
    \quad create $c' = \rprepeat(c)$ s.t. $c_i \in W \land c[0] = c[-1] \land c[0] \in \loopl(c)$
    \\ \quad $W.add(c')$, \qquad $W.remove(c)$
    \STATE \quad create $c' = c_1; c_2$ s.t. $c_1, c_2 \in W \land c_1[-1] = c_2[0]$
    \\
    \quad $W.add(c')$ \qquad $W.remove(c_1, c_2)$
    \STATE \textbf{Endwhile}
    \\ $c^T = W[0]$
    \RETURN $c^T$.
\end{algorithmic}
\end{algorithm}
%
Algorithm~\ref{alg:alg-refine_rewrite} transformation the syntax of program following~\cite{GulwaniJK09} and preserving the semantics.
We first use simple depth first search strategy collects all the \emph{simple transition path}s satisfying the Definition~\ref{def:tpath}. It guarantees that every $\tpath$ is equivalent to a path $\rho$ in Definition~4.1 of \cite{GulwaniJK09}.
Then
in Line-2, we initialize each candidate $c_i$ with a \emph{simple transition path} $\tpath_i$. New candidates generated in Line-4, 5 and 6 correspond to the if,
while and sequence statement in paper~\cite{GulwaniJK09} respectively.
\begin{itemize}
  \item
  Line-4: for all the candidates $c_1, \ldots, c_m$ having the same starting and ending vertices, rewrite them into if statement as~\cite{GulwaniJK09}.
  \item
  Line-5: for every candidate $c'$, if it starts and ends with the same vertex, rewrite it into while loop statement as~\cite{GulwaniJK09}.
  \item
  Line-6: for every two candidates $c_1, c_2$, if $c_1$ ends with the same vertex as $c_2$'s starting label, rewrite them into sequence statement as~\cite{GulwaniJK09}.
\end{itemize}
% \todo{soundness of the program write}
\paragraph{Refined Program}
We implement the algorithm REFINE from paper~\cite{GulwaniJK09} and compute the 
refined program $\rprog$ for a rewritten program $c$.

\paragraph[example]{Example.}
\todo{The walk through example}