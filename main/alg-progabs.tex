This path-sensitive reachability-bound algorithm
is performed on basis of an \emph{Abstract Transition Graph} for the program $c$.
This step shows how to generate the abstract transition graph $\absG(c)$ of a
program $c$ by constructing its vertices and edges.

\subsection{Vertices Construction}
\label{sec:abs_prog-vertex}
Every 
vertex corresponds to a program execution point, 
and the vertices set is all the $c$'s program execution points with an extra exit point ${\lex}$, 
\[ 
 \absV(c) = \progl(c)\cup\{{\lex}\}
 \]

\subsection{Edges Construction}
\label{sec:abs_prog-edge}
Each edge $(l \xrightarrow{dc} l') \in \absE(c)$ is an abstract transition
between two program points $l, l'$. 
The edge is constructed by the standard control flow analysis satisfying
there is an edge from $l$ to $l'$ if and only if
the command with label $l'$ can execute right after the execution of the command with label $l$.
We use the same algorithm as in Section~\ref{sec:alg_abscfg-edge} to compute the constraint and edges.
Specifically we compute the \emph{constraint}, the \emph{initial and continuation state} for each command
and then generate the set of edges for the program, $\absflow(c)$ in which each edge is an
\emph{abstract event}.
\paragraph{Constraint Computation}
A constraint describes the abstract execution of the assignment command with the label $l$,
or abstract evaluation of the boolean expression in the guard with label $l$.

Again we first introduce some components of the constraint. Most of them are the same as the definitions in Section~\ref{sec:alg_abscfg-edge}. We present them in detail in our simplified program model.

\begin{defn}[Symbolic Constant, Symbolic Expression]
 \label{def:symbolic_expr}
The universe of the \emph{symbolic constant} is denoted by $\scvardom \subseteq \mathbb{N} \cup \vardom \cup \{\infty \}$, in which a symbolic constant can be a natural number, $\infty$, or a program's input variable.
 The \emph{symbolic expression} of a program $\scexpr(c) \subseteq \mathcal{A}$ is the set of all the arithmetic expressions over $\mathbb{N} \cup \inpvar(c) \cup \{\infty \}$ for the program $c$.
\end{defn}

\begin{defn}[Difference Constraints]
 The difference constraints $DC(\vardom \cup \scvardom)$ is the set of all the inequality of
form $x' \leq y + v$ or $x' \leq v$ where $x, y \in \vardom $ and $v \in \scvardom$.
An inequality $x' \leq y + v$ describes that the value of $x$ in the current state is
at most the value of $y$ in the previous state plus the symbolic constant $v$, and $x' \leq v$ describes that the value of $x$ in the current state is
at most $v$.
\end{defn}

\begin{defn}[Constraints]
A constraints $dc \in \dcdom^{\top} = DC(\vardom \cup \scvardom) \cup \booldom\cup \{\top\}$ can be either a
difference constraint $d \in DC(\vardom \cup \scvardom)$, a boolean expression $\bexpr \in \booldom$
or $\top$ denotes always true.
\end{defn}

When the constraint in a transition is a difference constrain, $l \xrightarrow{x' \leq y + v} l'$,
% Then $x'$ 
it denotes that
the value of variable $x$
after executing the command at $l$ is at most
% and the right-hand side describes 
the value of variable $y$ plus $v$ before the execution,
and $l \xrightarrow{x' \leq v} l'$ respectively denotes
% value of variable $x$
% after executing the command at $l$ is 
$x$ has at most
% and the right-hand side describes 
the value of the symbolic constant $v$.
% before the execution.
For example in Figure~\ref{fig:relatedNestedWhileOdd-overview}(b), constraint $i' \leq i - 1$ on the edge $7 \xrightarrow{i' \leq i - 1} 1$
describes the execution of
 the command at line $7$, 
$\clabel{\assign{i}{i - 1}}^{7}$. 

% For every expression in each of the label commands, it is computed in three steps via the program abstraction method adopted from Section~6 in~\cite{SinnZV17}. 

The boolean constraint, $\bexpr$ on an edge $l \xrightarrow{b} l'$ describes
that after evaluating the guard with label $l$,
$\bexpr$ holds and the command with label $l$ will execute right after.
In Figure~\ref{fig:relatedNestedWhileOdd-overview}, $i \leq 0 $ on the edge $1 \xrightarrow{i \leq 0} \lex$, 
represents the negation of the testing guard $\clabel{i > 0}^1$
in the $\ewhile$ command, and $i \leq 0$ must hold in order to perform this transition from program point $1$ to
the program exit. 
We also have $\top$, which is preserved for $\eskip$ command or the commands that don't interfere with any guard variable.

% \highlight{Computation Steps:}


\begin{defn}[Constraint Computation]
 \label{def:constraint_compute}
 For a program $c$, a boolean expression $\bexpr$ in the guard of a $\eif$ or $\ewhile$ command
 or an expression $\expr$ and a variable $x$
 in an assignment command $\assign{x}{\expr}$,
 the constraint $\absexpr(\bexpr, \_, c )$ or $\absexpr(x - v, x, c)$ is computed as follows.
 \begin{enumerate}
 \item Compute the symbolic constant set of program $c$, $\scvar(c)$ of program $c$ as follows.
 \[
 \scvar(c) = \mathbb{N} \cup \inpvar(c) \cup \{\infty \} \subseteq \scvardom
 \]
 \item Initialize 
 $\grdvar(c) = \{\}$ as the set of the variable used in the expression of every while or if guard in the program $c$.
 \item Then we compute $\absexpr(\bexpr, \_, c )$ and $\absexpr(\expr, x, c)$.
 \[
 \begin{array}{ll} 
 \absexpr(x - v, x, c) = x' \leq x - v & x \in \grdvar(c)\land v \in \scvar(c) \\
 \absexpr(y + v, x, c) = x' \leq y + v & x, y \in \grdvar(c)\land v \in \scvar(c)\\
 \absexpr(v, x, c) = x' \leq v & x \in \grdvar(c)\land v \in \scvar(c) \\
 \absexpr(y + v, x, c) = x' \leq y + v, & x \in \grdvar(c)\land v \in \scvar(c)\\
 \qquad \grdvar(c) = \grdvar(c) \cup \{y\} & \qquad \land y \notin \grdvar(c) \\
 \absexpr(\bexpr, \_, c) = \bexpr, \grdvar(c) = \grdvar(c) \cup \kw{FV}(\bexpr) & \bexpr \in \scvar(c) \\
 \absexpr(\expr, x, c) = \top & {o.w.} \\
 \end{array}
 \]
 \end{enumerate}
 $\absexpr(\expr, x, c)$ and $\grdvar(c)$ are iteratively updating until stabilized over every guard and assignment command in $c$. In the following part of the paper, we use $\absexpr(\expr, x, c)$
 to denote the stabilized computation result.
 \end{defn}
%
In case 4, if a variable $x$, belonging to the set 
 $\grdvar(c)$ is updated by a variable $y$, which isn't in this set, 
 we add $y$ into the set $\grdvar(c)$ and repeat 
 above procedure until $\grdvar(c)$ and $\absexpr(\expr, x, c)$ is stabilized. 
 \\
Specifically 
we handle a 
normalized boolean expression, $x > 0$
in guards of if and while loop headers, and 
the guard variable $x$ only increases, decreases, or is reset by 
simple arithmetic expression (mainly multiplication, division, minus, and plus and the extended operations max and min). 
The guard variable $x$ is generalized into the \emph{norm} when the guard 
in if or while has the form $\expr > 0$, where $\expr$ is a norm and $\grdvar$ is generalized into the set of norms.
Then in the last clause in Definition~\ref{def:constraint_compute}, i.e., $\absexpr(\expr, x, c)$,
we will check if $\expr$ is a norm or contains norms, and extend the $\grdvar(c)= \grdvar(c)\cup \{\expr\}$ if neither.
The constraint computation over the \emph{norm} follows the computation steps 1, 2, and 3 in Section 6.1 in paper~\cite{SinnZV17}. 
%
\paragraph{Initial and Continuation State Computation}
This step computes two sets for each command. 
The initial state is a set that contains the
program points before executing this command, which is computed by the standard initial state generation method from control flow analysis.
The continuation state is a set
that contains the constraint of this command and the program points after the execution of this command.
This set is enriched 
by the standard control flow analysis.

%
\begin{itemize}
 \item The \emph{initial state}, $\absinit(c) \in \ldom$
 for a command $c$.
 $\absinit(c)$ is a unique program label that corresponds to the command before executing this command. 
\\
Given a program $c$, its abstract initial state, $\absinit(c)$ is computed as follows,
%
\[
 \begin{array}{ll}
 \absinit(\clabel{\assign{x}{\expr}}{}^l) & = l \\
 \absinit(\clabel{\eskip}^{l}) & = l \\
 \absinit(\eif (\clabel{\bexpr}^l, c_t, c_f)) & = l \\
 \absinit(\ewhile \clabel{\bexpr}^l \edo (c_w)) & = l \\
 \absinit(c_1 ; c_2) & = \absinit(c_1) \\
 \end{array}
 \]
%
%
\item The \emph{continuation state}, $\absfinal(c)$ of program $c$, 
$\absfinal(c) \in \mathcal{P}(\ldom \times \dcdom^{\top})$
is a set of pairs, $(l, dc)$ with a
program point (i.e., a label), $l$ as the first component and a constraint, 
$dc$ as the second component.
The program point $l$ corresponds to the labeled command after the execution of $c$,
and the constraint $dc$ in this pair is computed by $\absexpr$ for the expression in $c$.
\\
Given a program $c$, its continuation state, $\absfinal(c)$ is computed as follows,
 \[
 \begin{array}{ll}
 \absfinal(\clabel{\assign{x}{\expr}}{}^l) & = \{(l, \absexpr(\expr, x, c))\} \\
 \absfinal(\clabel{\eskip}^{l}) 
 & = \{(l, \top)\} \\
 \absfinal(\eif (\clabel{\bexpr}^l, c_t, c_f)) & = \absfinal(c_t) \cup \absfinal(c_f) \\
 \absfinal(\ewhile \clabel{\bexpr}^l \edo (c_w)) & = \{(l, \absexpr(\bexpr, \top, c_w))\} \\
 \absfinal(c_1 ; c_2) & = \absfinal(c_2) \\
 \end{array}
 \]
 %
\end{itemize}
 \paragraph{Abstract Event Computation} Each abstract event is an edge between two vertices in the abstract transition graph.
 It is generated by computing the initial and continuation execution state interactively and recursively for a program $c$.
 
% \highlight{Notations and Formal Definitions:}
% \begin{itemize}
% \item \emph{Abstract Event}: 
% $\absevent \in $
% $\ldom \times \dcdom^{\top} \times \ldom$
% \item \emph{Abstract Trace}, $\absflow(c)$ of a program $c$: $\absflow(c) \in \mathcal{P}( \ldom \times \dcdom^{\top} \times \ldom )$
% \end{itemize}
 \begin{defn}[Abstract Event]
 \label{def:abs_event}
 Abstract Event: 
 $\absevent \in $
 $\ldom \times \dcdom^{\top} \times \ldom$
 is a 
 triple where the first and third components are labels,
 second component is a constraint from $\dcdom^{\top}$.
 \end{defn}
 In an abstract event $(l, dc, l')$ of a program $c$, 
 the first label $l \in \ldom$ is the \emph{initial state} $\absinit(c)$ of $c$, and 
 the second label $l' \in \ldom$ with the constraint $dc \in \dcdom^{\top}$ correspond to a \emph{continuation state} of $c$.
% We use the notation $\mathcal{P}(\absevent)$ to denote the power sets of all abstract events.

The set of the abstract events $\absflow(c)$ for a program $c$
is computed as follows in Definition~\ref{def:absevent_compute}.
 %
 \begin{defn}[Abstract Execution Trace Computation]
 \label{def:psRB-absevent_compute}
 The \emph{abstract trace}, $\absflow(c) \in \mathcal{P}( \ldom \times \dcdom^{\top} \times \ldom )$ of a program $c$ is computed as follows.

 We first append a $\eskip$ command with 
the label $\lex$, i.e., $\clabel{\eskip}^{{\lex}}$ at the end of the program $c$, and construct 
the program $c' = c;\clabel{\eskip}^{{\lex}}$.
Then, we compute the $\absflow(c) = \absflow'(c')$ for $c'$ as follows,
 %
 {
 \[
 \begin{array}{ll}
 \absflow'(\clabel{\assign{x}{\expr}}{}^l) & = \emptyset \\
 \absflow'(\clabel{\eskip}^{l}) & = \emptyset \\
 \absflow'(\eif (\clabel{\bexpr}^l, c_t, c_f)) & = \absflow'(c_t) \cup \absflow'(c_f)
 \\ & \quad 
 \cup \left \{(l, \absexpr(\bexpr, \top, c), \absinit(c_t) ) \right \} , 
 \\ & \quad 
 \cup \left \{ (l, \absexpr(\neg\bexpr, \top, c), \absinit(c_f)) \right \} \\
 \absflow'(\ewhile \clabel{\bexpr}^l \edo (c_w)) & = \absflow'(c_w) \cup \{(l, \absexpr(\bexpr, \top, c), \absinit(c_w)) \} 
 \\ & \quad 
 \cup \{(l', dc, l)| (l', dc) \in \absfinal(c_w) \} \\
 \absflow'(c_1 ; c_2) & = \absflow'(c_1) \cup \absflow'(c_2) 
 \\ & \quad 
 \cup \{ (l, dc, \absinit(c_2)) | (l, dc) \in \absfinal(c_1) \} \\
 \end{array}
 \]
 }
 \end{defn}
 Notice $\absflow'(\clabel{\assign{x}{\expr}}^{l})$ and $\absflow'(\clabel{\eskip}^{l})$ are both empty set. 
 For every event $\event$ with label $l$ in an execution trace $\trace$ of program $c$, 
 there is an abstract event in the program's abstract execution trace of the form $(l, \_, \_)$. 
 We also show the soundness of the \emph{abstract event computation} as Theorem~\ref{lem:abscfg_sound} below with proof in Appendix~\ref{apdx:abs_sound}.
\begin{lem}[Soundness of the Abstract Execution Trace]
\label{lem:psRB-abscfg_sound}
For any program $c$ with 
 % its abstract control flow graph $\absG(c) = (\absV(c), \absE(c))$,
 an arbitrary initial trace $\vtrace_0 \in \ftdom_0(c)$ and
 a finite execution trace $\trace \in \ftdom$ that is generated 
 when $c$ is executing under $\vtrace_0$,
 then for every event $\event \in \trace$ such that $\event = (\_, l, \_)$ for some label $l$ in program $c$,
 % an initial trace $\vtrace_0 \in \ftdom_0(c)$,
 there is an abstract event $\absevent = (l, \_, \_) \in \absflow(c)$ and $\absevent = (l, \_, \_)$.
%
\[
\begin{array}{l}
 \forall c, c' \in \cdom, \trace_0 \in \ftdom_0(c), \trace \in \ftdom , \event = (\_, l, \_) \in \eventset, l \in \progl(c) \st
 \\ \qquad 
 \big(
 \config{{c}, \trace_0} \to^{*} \config{c', \trace_0 \tracecat \trace} 
 \big)
 \land \event \in \trace 
 \implies \exists 
 \absevent \in \absflow(c) \land \absevent = (l, \_, \_)
\end{array}
\]
\end{lem}
If $l$ is the label of an assignment command in a program $c$,
then there is a unique abstract event in the program's abstract events set
$\absevent \in \absflow(c)$ of form $(l, \_, \_)$. 
The following Lemma~\ref{lem:abscfg_unique} states this uniqueness property and the proof is in Appendix~\ref{apdx:abs_sound}.
\begin{lem}[Uniqueness of the Abstract Events]
\label{lem:abscfg_unique}
For every program $c$ and
a finite execution trace $\trace \in \ftdom$ which is generated w.r.t.
an initial trace $\trace_0 \in \ftdom_0(c)$,
there is a unique abstract event $\absevent = (l, \_, \_) \in \absflow(c)$ 
for every assignment event $\event \in \eventset^{\asn}$ that has the same label $l$ in the
execution trace, specifically, $\event = (\_, l, \_) \in \eventset^{\asn}$ and $\event \in \trace$.
%
\[
 \begin{array}{l}
 \forall c \in \cdom, \trace_0 \in \ftdom_0(c), \trace \in \ftdom , \event = (\_, l, \_) \in \eventset^{\asn} \st
 \big(
 \config{{c}, \trace_0} \to^{*} \config{c', \trace_0 \tracecat \trace} 
 % \lor 
 % \config{{c}, \trace_0} \uparrow^{\infty} \trace_0 \tracecat \trace 
 \big)
 \land \event \in \trace 
 \\
 \qquad \implies \exists ! \absevent = (l, \_, \_) \in (\ldom\times \dcdom^{\top} \times \ldom) \st 
 \absevent \in \absflow(c)
 \end{array}
\]
\end{lem}
%
\paragraph{Edge Construction}
The edge for $c$'s abstract transition graph is constructed simply by computing the program's abstract events set, $\absflow(c)$ as follows,
 \[
 \absE(c) = \{(l_1, dc, l_2) | (l_1, dc, l_2) \in \absflow(c)\}
 \]
For each edge $(l, dc, l') \in \absE(c)$, $dc$ describes either an abstract execution of the assignment command with label $l$,
or the evaluation of the guard with label $l$.
%

Again in Figure~\ref{fig:relatedNestedWhileOdd-overview}(b),
the edge $(0 \xrightarrow{i' \leq n} 1)$ on the top tells us the command 
$\clabel{\assign{i}{n}}^0$ is executed with an abstract transition, $i' \leq n$ and a continuation point $1$. 
According to the continuation point $1$, the while loop with header at location $1$, $\ewhile \clabel{i > 0}^1 \edo \{\ldots\}$ will be executed right after.

\subsection{Abstract Transition Graph Construction} 
With the vertices $\absV(c)$ and edges $\absE(c)$ ready, we construct the abstract transition graph, formally in
Definition~\ref{def:abs_cfg}.
%
\begin{defn}[Abstract Transition Graph]
 \label{def:psRB-abs_cfg}
 The \emph{abstract transition graph} of a program $c$ is $\absG(c) \triangleq (\absV(c), \absE(c))$, where
 $\absV(c) \triangleq \progl(c)\cup\{\lex\}$
 and each edge $(l, dc, l') \in \absE(c)$ is an abstract transition
 $\absE(c) \triangleq \{(l_1, dc, l_2) | (l_1, dc, l_2) \in \absflow(c)\}$.
 % and
 % .
% between two program points $l, l'$ if and only if
% the command with label $l'$ can execute right after the execution of the command with label $l$.
% % if and only if there is a control flow between two program points.
% The constraint $dc \in \dcdom$ on each edge is generated from the command with label $l$, which describes the abstract execution of this command.
\end{defn}

\begin{defn}[Path]
 \label{def:abs_cfgpath} 
 A path on $\absG(c)$ is a sequence, $ l_0 \xrightarrow{dc_0} l_1 \xrightarrow{dc_1} \ldots $ with
 \begin{itemize}
 \item the vertices sequence $(l_0, \ldots)$, where $l_i \in \absV(c)$ for every $i = 0, 1, \ldots$ and
 %
 \item the edge sequence $(e_0, \ldots)$, where $e_i = (l_{i}, dc_i, l_{i + 1}) \in \absE(c)$ for every $i = 0, 1, \ldots$.
 \end{itemize}
 A path is cyclic, if it has the same start- and end-point. A path is simple, if it does not visit a location twice except for the start- and end-location. We use $\paths(\absG(c))$ to denote the set of all the paths on $\absG(c)$,
 and $\pathl(p)$ to represent the list of program points corresponding to the vertices sequence of a path $p \in \paths(\absG(c))$.
%  where $\pathl : \paths(\absG(c)) \to \mathcal{P}{(\ldom)}$.
 \end{defn}
 For example in Figure~\ref{fig:relatedNestedWhileOdd-overview}(b), $1 \to 2 \to 3 \to 4 \to 5 \to 4$ is a \emph{path}, but it is not simple (the program points $4$ is visited twice). The path $4 \to 5 \to 4$ is both cyclic and simple.

\paragraph{Example.}
In Figure~\ref{fig:relatedNestedWhileOdd-overview}(b),
the edge $(0 \xrightarrow{i' \leq n} 1)$ on the top tells us the command 
$\clabel{\assign{i}{n}}^0$ is executed with a continuation point $1$, and the
command $\ewhile \clabel{i > 0}^1 \edo \{\ldots\}$ will be executed next.
The annotation $i' \leq n$ is a difference constraint 
computed by $\absexpr$ over
the expression $n$ in the assignment command $\assign{i}{n}$.
It represents that the value of $i$ is less than or equal to the value of $n$ after the
execution of $\clabel{\assign{i}{n}}^0$ and before evaluating the guard $\clabel{i > 0}^1$.
Another example constraint $i' \leq i - 1$ on the edge $7 \xrightarrow{i' \leq i - 1} 1$
describes the execution of
 the command at line $7$, 
$\clabel{\assign{i}{i - 1}}^{7}$. 
The $i'$ on the left side of $i' \leq i - 1$ represents the value of $i$ after the assignment operation,
and the right-hand side $i$ stores the value before the assignment.
The boolean constraint $i \leq 0 $ on the edge $1 \xrightarrow{i \leq 0} \lex$, 
represents the negation of the testing guard $\clabel{i > 0}^1$
in the $\ewhile$ command.
$1 \xrightarrow{i \leq 0} \lex$ denotes that $i \leq 0$ must hold to perform this transition from program point $1$ to
the program exit. 