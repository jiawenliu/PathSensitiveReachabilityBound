\subsection{Loop Bound}
$BD(\rprepeat({\rprog}))$  is path-insensitive because we only compute the loop bound
for the single loop path in a refined program $\rprog$.
But our alternative computation method in Section~\ref{sec:looprb} has a better performance in the following sense.
The computation of $BD(\rprepeat({\rprog}))$ 
as below is more efficient and accurate than existing techniques.
\\
The method $T(c, l)$ in \cite{GulwaniJK09} can also compute a
bound on the iteration numbers of $\rprepeat({\rprog})$ locally based on
the same semantics and syntax.
Specifically, $T(c, l)$ call the $\kw{BOUNDFINDERD}$ function as follows:
\\
$\kw{BOUNDFINDERD(INITD(c, l_0, \absinit(\rprepeat({\rprog}))),
NEXTD(c, l_0, \absinit(\rprepeat({\rprog})), V_{\lin} )}$.
However, the $\kw{BOUNDFINDERD}$ function in \cite{GulwaniJK09} relies an arbitrary interface to
compute the bound on the iteration numbers.
The efficiency and accuracy of this computation fully depend on this arbitrary interface.
\\
In comparison to them, the alternative loop bound computation method in 
Definition~\ref{def:loopbound} based on the abstract transition graph $\absG(c)$ is more accurate.

\subsection{Path Local Reachability-Bound}
\highlight{
  The computation
  filters out the execution of other transition paths that are nested in the same loop as $\tpath$.
  In this way, the following repeat chain bound computations are path-sensitive
  because they only deal with $\tpath$'s execution over this chain.
  }
  \highlight{\textbf{The Path-sensitivity Discussion Informally}}
  \\
  For every multiple-paths loop,
  this bound computes the bound for every simple transition path path-sensitively.
  In comparison to the traditional bound computation methods, they
  estimate the bound of different paths by taking maximum overall paths.
  \\
  The path-sensitivity is guaranteed by the following informal discussion.
  \\
  If a simple transition path isn't nested in any $\rprepeat$ annotation, then
  $\outinB(\tpath) = 1$ is sound and tight because it bounds one execution of the while body accurate.
  \\
  If a simple transition path is nested in some $\rprepeat$ annotations,
  $\rpch(l: \rprog, \tpath) \triangleq \rprog_m \to \rprog_{n-1} \to \ldots \to \rprog_{0} \to \tpath$, we have the following guarantees.
  \\
  1. $\rprog_n$ is a unique transition path of this loop,
  and isn't a sub-pattern of any other transition paths in this loop.
  \\
  2. for every $i = n, \ldots, 0$, $\outinB(\rprog_{i - 1})$ bounds the execution time of $\rprog_{i - 1}$ with the assumption that $\rprog_{i}$ executes only once.
  \\
  3. $\tpath$ only shows up once inside $\rprog_n$.
  \\
  By 1. 2. 3. guarantees, multiplication of $\outinB(\rprog_{i - 1})$ for every $i = n, \ldots, 0$ is the accurate execution time of this
  simple transition path in the transition path $\rprog_n$.
  \\
  In the case that loop has multiple transition paths, $\rpchoose{l: \rprog_1, \ldots, l: \rprog_n}$,
  \\
  We first find every repeat chain of this loop for this simple transition path.
  Then we compute the execution time of $\tpath$ on every repeat chain and take the maximum value.

\subsection{Loop Reachability-Bound}
\highlight{
  \emph{Improvements}
  \\
  $\lpchB(l: \rprog, \tpath) \in \mathcal{A}_{\lin}$ bounds the iteration numbers of the loop $l$ w.r.t.
the inner loop $l'$ that is $\tpath$'s closest enclosing loop,
such that,
during these iterations of loop $l$, the nested loop $l'$ is executed.
\highlight{
This is distinguished from the traditional methods, which only compute the bound on the iteration number
for the inner loop w.r.t one iteration of the outside loop where it is nested.
\emph{Relative Loop Bound} for $\tpath$ and $l$ bounds the number of the iterations for
the outside loop $l$,
such that during these iterations of the outside loop $l$, the inner loop $l'$ is entered. 
}
$\lpinit(l, \tpath)$, and $\lpnext(l, \tpath)$ can be computed using the 
  $\kw{INIT}(c, i, \absinit(\tpath))$ and $\kw{NEXT}(c, i, \absinit(\tpath))$ from paper \cite{GulwaniJK09}.
  % \\
Then, based on the same semantics and syntax,
the $\lpchB(l, \tpath)$ can be computed as
\\
$I(l, l') = \kw{BOUNDFINDERD(INIT(c, i, \absinit(\tpath)), NEXT(c, i, \absinit(\tpath)), V_{\ln})}$.
\\
There are two improvements comparing to their method.
\\
1. $I(l, l')$ is the bound on iteration numbers of the inner loop $l'$ in each iteration of outside loop.
This is equivalent to the OutIn bound of $l'$ if $l$ is the closest enclosing loop of $l'$.
By multiplying $I(l, l')$ with $\outinB(l')$
 they assume $l'$ is executed in every $l$'s iteration.
 In this sense, they over-approximate the iteration numbers of the inner loop $l'$.
 \\
2. However, $\kw{BOUNDFINDERD}$ in paper~\cite{GulwaniJK09} is an arbitrary interface computing the bound on iteration numbers of $l$
separately.
The efficiency and accuracy of their algorithm fully depend on this arbitrary interface.
\\
In comparison to them, the computation method in Definition~\ref{def:looprb} based the
ranking functions and abstract states over the abstract transition graph $\absG(c)$ is more accurate.
 }

\subsection{Path Global Reachability-Bound}
\subsubsection{Path-sensitivity}
