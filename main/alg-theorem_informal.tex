\subsection{Loop Bound}
\subsubsection{Path-sensitivity}
  For every repeat pattern $\rpattern$ in a refined program $\rprog$, 
this algorithm
computes a bound, $\outinB(\rprog, \rpattern) \in \mathcal{A}_{in}$ on its iteration numbers locally and path-sensitively
named \textbf{Outside-In} bound.
%  of every repeat pattern 
\\
% $\outinB(\rprog, \rpattern)$
% computes the bound locally
% % is the \textbf{Outside-In} bound for a repeat pattern $\rpattern$ 
% in the sense that
% % for a repeat pattern 
% if 
For every $\rpattern'$ that is nested
in another repeat pattern $\rpattern$, i.e., $\rpattern = \rprepeat(\rpattern')$,
$\outinB(\rprog, \rpattern)$ only estimates
$\rpattern'$'s iteration number within $\rpattern$ by assuming that $\rprepeat(\rpattern')$ executes once.
% \\

\highlight{\textbf{Notations / Operators:}}
\\
The \emph{State}: 
$\absstate \in \mathcal{P}(\dcdom^{\top})$ is a conjunction of constraints.
\\
\emph{Initial State} ($\rfinit(\rprog, \rpattern) \in \absstate$), 
\\
\emph{Final State} ($\rffinal(\rprog, \rpattern) \in \absstate$),
\\
\emph{Next State} $\rfnext(\rprog, \rpattern) \in \absstate$,
% The \emph{Initial State}: $\rfinit : \rprog \to \absstate $.
\\
\emph{Variable Grade Decedent}: $\varGD(\rprog, \rpattern) \in \mathcal{A}_{in}$


\highlight{\textbf{Computation Steps:}}
  \textbf{Outside-In} Bound ($\outinB(\rprog, \rpattern) \in \mathcal{A}_{in}$)
  \\
For a repeat pattern $\rpattern$ in a refined program $\rprog$,
the \textbf{Outside-In} bound, $\outinB(\rprog, \rpattern) \in \mathcal{A}_{in}$ is computed recursively as follows,
% for every nested repeat patterns in $\rprog$,
% \\
% $\outinB(\rpchoose{\rprog_1, \cdots, \rprog_m}) =  \max\{\outinB(\rprog_1), \cdots, \outinB(\rprog_m)\}$
\\
$\outinB(\rprog, \rprepeat({\rpattern})) = 
\frac{\rfinit(\rprog, \rpattern) - \rffinal(\rprog, \rpattern)}{\varGD(\rprog, \rpattern)}$
\\
$\outinB(\rprog, \rpattern_1; \rpattern_2) =  \outinB(\rprog, \rpattern_1)+ \outinB(\rprog,  \rpattern_2)$
\\
$\outinB(\rprog, \tpath) =  1$.
\\
\highlight{
  \textbf{Improvement Discussion:}
\\
The computation of $\outinB(\rprepeat({\rpattern}))$ 
as below is more efficient and accurate than existing techniques.
% in the first case
% can be computed  as follows
\\
The method $T(c, l)$ in \cite{GulwaniJK09} can also compute a
bound on the iteration numbers of $\rprepeat({\rpattern})$ locally based on
% $\rprepeat(\rpattern)$ with loop header $l$
% has the 
the same semantics and syntax.
Specifically, $T(c, l)$ call the $\kw{BOUNDFINDERD}$ function as follows:
\\
$\kw{BOUNDFINDERD(INITD(c, l_0, \absinit(\rprepeat({\rpattern}))),
NEXTD(c, l_0, \absinit(\rprepeat({\rpattern})), V_{\lin} )}$
% based on
% % $\rprepeat(\rpattern)$ with loop header $l$
% % has the 
% the same semantics and syntax.
%  as the refined program in paper~\cite{GulwaniJK09},
% in paper~\cite{GulwaniJK09} 
% with equivalent results.
However, the $\kw{BOUNDFINDERD}$ function in \cite{GulwaniJK09} relies an arbitrary interface to
compute the bound on the iteration numbers.
The efficiency and accuracy of this computation fully depend on this arbitrary interface.
\\
In comparison to them, the alternative loop bound computation method in 
Definition~\ref{def:loopbound} based on the abstract transition graph $\absG(c)$ is more accurate.
}

\subsubsection{Soundness}
\subsection{Theorem Guarantee}
  \label{sec:outinalg-theorem}
  For a program $c$ and an edge $\absevent \in \absE(c)$,
  $\absclr(\absevent)$ is a sound upper bound
  %  for every label $l$ in this program 
  on the execution times of this transition,
  formally in Theorem~\ref{thm:pathinsensitive_rb_soundness}. 
  Proof of this theorem is in Appendix~\ref{apdx:pathinsensitive_rb_soundness}.

  \highlight{\textbf{Theorem Guarantee Discussion:}}
\\
\emph{Soundness} For every repeat pattern $\rpattern$, the $\outinB(\rpattern, \rprog)$
is a sound upper bound on its execution times locally.
\\
This bound is sound locally by assuming
that all the loops and repeat patterns where $\rpattern$ is nested execute only once.
This assumption comes from the computation of the $\varGD$ and the deep first search strategy.
\\
For every base case, i.e., a simple transition path, 
$\varGD(\rprog, \tpath) =  \rfinit(\tpath) - \rfnext(\rprog, \tpath)$
counts the variables' changes only once. In this way, it assumes all the outside patterns and loops execution only once.
In the other two cases,  $\varGD(\rprog, \rprepeat(\rpattern))$ and $\varGD(\rprog, \rpseq(\rpattern_1, \rpattern_2))$
has the same assumption when compute the variables' changes.
% \\
The soundness also relies on the operation $\frac{\rfinit(\rprog, \rpattern) - \rffinal(\rprog, \rpattern)}{\varGD(\rprog, \rpattern)}$,
which is solved by Definition~\ref{def:edge_pathinsensitivebound} and~\ref{def:ranking_bound}.
% a sound external solver. \todo{reference}
%
\begin{thm}[Soundness of Outside-In Bound]
  \label{thm:sound_outsidein}
  For every program $c$ with it refined program $\rprog$,
  if $\rpattern$ is a repeat pattern of this program, then the $\outinB(\rpattern, \rprog)$
is a sound local bound on the execution times of $\rpattern$ in program $c$.
  \[
    \begin{array}{l}
    \forall c \in \cdom, \rpattern, \rprog, \trace_0 \in \tdom_0(\rpattern), \trace_1, \trace \in \tdom \st 
    \rpattern \in \rprog \land
    \rprog = Alg.4(c)
    \\ \qquad
    \land
    \config{c, \trace_0} \to^* \config{\highlight{\ewhile (\rpattern, \cdots)}, \trace_0 \tracecat \trace_1} 
    \land 
    \config{\rpattern, \trace_0 \tracecat \trace_1} \to^* \config{{\eskip, \trace_0 \tracecat \trace_1 \tracecat \trace}}
    \\ \qquad
    \implies
    \todo{Change Counter}
    {\vcounter(\trace, L(\rpattern))} \leq \outinB(\rpattern, \rprog)(\trace_0 \tracecat \trace_1).
    \end{array}
    \]
\end{thm}
\highlight{Informal Discussion}
\\
Theorem~\ref{thm:sound_outsidein} guarantees that
the bound for every repeat pattern $\rpattern$, the $\outinB(\rpattern, \rprog)$
is a sound upper bound on its execution times, by assuming
that all the loops and repeat patterns where $\rpattern$ is nested execute only once.
\\
In paper \cite{sinn2017complexity} Definition~9, they informally discussed the local bound soundness.
$v$ is a local bound if it has the same decreasing time as the transition's execution time.
By assuming that certain program parts (those were e increases) are not executed,
then value of $v$ can limit the execution time of that transition.
\\
In our soundness, we assume all the code pieces not inside this repeat pattern are executed at most once (once if they show up in front of the program
zero time if not).
In this case, this bound limits the execution time of this repeat pattern.

\subsection{Path Local Reachability Bound}
\subsubsection{Path-sensitivity}
\highlight{
  The computation
  filters out the execution of other transition paths that are nested in the same loop as $\tpath$.
  In this way, the following repeat chain bound computations are path-sensitive
  because they only deal with $\tpath$'s execution over this chain.
  }
  \highlight{\textbf{The Path-sensitivity Discussion Informally}}
  \\
  For every multiple-paths loop,
  this bound computes the bound for every simple transition path path-sensitively.
  In comparison to the traditional bound computation methods, they
  estimate the bound of different paths by taking maximum overall paths.
  \\
  The path-sensitivity is guaranteed by the following informal discussion.
  \\
  If a simple transition path isn't nested in any $\rprepeat$ annotation, then
  $\outinB(\tpath) = 1$ is sound and tight because it bounds one execution of the while body accurate.
  \\
  If a simple transition path is nested in some $\rprepeat$ annotations,
  $\rpch(l, \tpath, \rprog) \triangleq \rpattern_m \to \rpattern_{n-1} \to \cdots \to \rpattern_{0} \to \tpath$, we have the following guarantees.
  \\
  1. $\rprog_n$ is a unique repeat pattern of this loop,
  and isn't a sub-pattern of any other repeat patterns in this loop.
  \\
  2. for every $i = n, \cdots, 0$, $\outinB(\rpattern_{i - 1})$ bounds the execution time of $\rpattern_{i - 1}$ with the assumption that $\rprog_{i}$ executes only once.
  \\
  3. $\tpath$ only shows up once inside $\rprog_n$.
  \\
  By 1. 2. 3. guarantees, multiplication of $\outinB(\rpattern_{i - 1})$ for every $i = n, \cdots, 0$ is the accurate execution time of this
  simple transition path in the repeat pattern $\rprog_n$.
  \\
  In the case that loop has multiple repeat patterns, $\rpchoose{l: \rpattern_1, \cdots, l: \rpattern_n}$,
  \\
  We first find every repeat chain of this loop for this simple transition path.
  Then we compute the execution time of $\tpath$ on every repeat chain and take the maximum value.

\subsubsection{Soundness}
\emph{Soundness}
    \\
    \emph{Path Local Reachability Bound}  $\outinB(\rprog, \tpath)$.
    For a simple transition path $\tpath$ with its closest enclosing while loop at $l$ in a program $\rprog$, 
    $\outinB(\rprog, \tpath)$
    is a sound bound on its execution times in side $l$.
    \\
    For a simple transition path $\tpath$ only enclosed by one repeat pattern $\rprepeat(\tpath)$, 
    we know $\outinB(\rprog, \tpath) = \outinB(\rprepeat(\tpath), \rprog)$.
    Since $\outinB(\rprepeat(\tpath), \rprog)$ is a sound local bound on the iteration times
    of $\rprepeat(\tpath)$ by assuming all the outside loops executes only once.
    In this sense, $\outinB(\rprepeat(\tpath), \rprog)$ is also a sound bound on the iteration times globally.
    \\
    For a simple transition path $\tpath$ nested in multiple repeat patterns $\rpattern_1, \cdots, \rpattern_m$,
    we know $\outinB(\rprog, \tpath) = \prod\limits_{i = 1, \cdots, m}\outinB(\rpattern_i, \rprog)$.
    By the same guarantee from $\outinB(\rpattern_i, \rprog)$, it is sound to multiply each of them.
    
  %  


\subsection{Loop Reachability Bound}
\subsubsection{Path-sensitivity}
$\lpchB(l, \tpath, \rprog) \in \mathcal{A}_{\lin}$.
\\
For every simple transition path $\tpath$
and a loop whose header is at program point $l$ in a refined program $\rprog$,
 $\lpchB(l, \tpath, \rprog) \in \mathcal{A}_{\lin}$ computes a symbolic expression in $\mathcal{A}_{\lin}$
as the \emph{Relative Loop Bound} for the $\tpath$ and loop $l$.
\\
It bounds the iteration numbers of the loop $l$ w.r.t.
the inner loop $l'$ that is $\tpath$'s closest enclosing loop,
such that,
% the simple transition path $\tpath$.
% This $\tpath$'s closest enclosing loop has the loop header at $l'$ and $l'$ is nested inside the loop $l$.
% \\
% It estimates the iteration numbers of loop $l$ such that 
during these iterations of loop $l$, the nested loop $l'$ is executed.
\highlight{
% the for every program control location,
% how many times the innermost loop of this control location will be touched w.r.t. every
% outside loop it is nested in.
This is distinguished from the traditional methods, which only compute the bound on the iteration number
for the inner loop w.r.t one iteration of the outside loop where it is nested.
\emph{Relative Loop Bound} for $\tpath$ and $l$ bounds the number of the iterations for
the outside loop $l$,
% w.r.t. an inner loop.
%  of every control location.
such that during these iterations of the outside loop $l$, the inner loop $l'$ is entered. 
}

\highlight{
  \emph{Improvements}
  \\
  $\lpinit(l, \tpath)$, and $\lpnext(l, \tpath)$ can be computed using the 
  $\kw{INIT}(c, i, \absinit(\tpath))$ and $\kw{NEXT}(c, i, \absinit(\tpath))$ from paper \cite{GulwaniJK09}.
  % \\
Then, based on the same semantics and syntax,
the $\lpchB(l, \tpath)$ can be computed as
\\
$I(l, l') = \kw{BOUNDFINDERD(INIT(c, i, \absinit(\tpath)), NEXT(c, i, \absinit(\tpath)), V_{\ln})}$.
% from paper \cite{GulwaniJK09}.
\\
There are two improvements comparing to their method.
\\
1. $I(l, l')$ is the bound on iteration numbers of the inner loop $l'$ in each iteration of outside loop.
This is equivalent to the OutIn bound of $l'$ if $l$ is the closest enclosing loop of $l'$.
By multiplying $I(l, l')$ with $\outinB(l')$
 they assume $l'$ is executed in every $l$'s iteration.
 In this sense, they over-approximate the iteration numbers of the inner loop $l'$.
 \\
2. However, $\kw{BOUNDFINDERD}$ in paper~\cite{GulwaniJK09} is an arbitrary interface computing the bound on iteration numbers of $l$
separately.
The efficiency and accuracy of their algorithm fully depend on this arbitrary interface.
\\
In comparison to them, the computation method in Definition~\ref{def:looprb} based the
ranking functions and abstract states over the abstract transition graph $\absG(c)$ is more accurate.
 }

\subsubsection{Soundness}

\emph{Soundness}
\\
\emph{Relative-Loop Bound} $\lpchB(l, \tpath, \rprog)$. In a refined program $\rprog$,
for its every loop at the program point $l$ and a transition path $\tpath$ inside this loop,
the \emph{Relative Loop Bound} $\lpchB(l, \tpath, \rprog)$ is a symbolic bound
on the number of $l$'s iteration numbers,
%  will 
% bound the execution times of $L_{t_0}$
% in each single execution of the $L_{t_i}$ for every
such that during these iterations, the closest loop $l'$ enclosing $\tpath$ will be executed.
\\
The soundness is guaranteed by the operation
 $\frac{\lpinit(\rprog, \tpath) - \rffinal(\rprog, \tpath)}{\lpinit(\rprog, \tpath) - \lpnext(\rprog, \tpath)}$.
 In this operation, $\lpnext(\rprog, \tpath)$ computes the variables states of the $\tpath$
after visited the program point $l$ the second time and before visiting any other program point.
% it is 
In the same time, the soundness also relies on an external solver. 



\subsection{Path Reachability Bound}
\subsubsection{Path-sensitivity}
\subsubsection{Soundness}


\emph{Soundness} of the \emph{Path Reachability Bound}  
For every simple transition path $\tpath$, the $\inoutB(\tpath, \rprog)$
is a sound upper bound on its execution times globally.
\\
In every base case, a simple transition path not nested in any loop, or just inside one loop.
Then $\inoutB(\tpath, \rprog) = \outinB(\rprog, \tpath)$, which is the bound on its execution times without considering
outside loops' executions. Since there isn't any outside loop, this operation is sound.
In the same time, the soundness is guaranteed by the $\outinB(\rprog, \tpath)$.
\\
For every simple transition path nested in multiple loops, if $l$ is the closest loop $\tpath$ comes from, then
$\inoutB(\tpath, \rprog) =
\prod\limits_{l' \in \lpch(\tpath, \rprog)} \rpchB(l', \tpath, \rprog) \times \outinB(\rprog, \tpath)$.
Since $\lpchB(l', \tpath, \rprog)$ bounds the number of $l'$'s iteration numbers,
%  will 
% bound the execution times of $L_{t_0}$
% in each single execution of the $L_{t_i}$ for every
such that during these iterations, the $\tpath$ will be executed,
it is sound to multiply these bounds by $\outinB(\rprog, \tpath)$.
In the same time, the soundness also relies on the soundness of $\outinB(\rprog, \tpath)$ and $\rpchB(l', \tpath, \rprog)$. 
% Since 
% In this way, it assumes all the outside patterns and loop execution only once.
% In the other two cases,  $\varGD(\rprog, \rprepeat(\rpattern))$ and $\varGD(\rprog, \rpseq(\rpattern_1, \rpattern_2))$
% has the same assumption.
% \todo{reference}
% \end{itemize}
%
\begin{thm}[Soundness of Path Reachability Bound]
  \label{thm:sound_path-rb}
  For every program $c$ with it refined program $\rprog$ and 
  every simple transition path $\tpath$ in this program,
   the $\inoutB(\tpath, \rprog)$
is a sound bound on the execution times of $\tpath$ in $c$.
  \[
    \begin{array}{l}
    \forall c \in \cdom, \tpath \in \absG(c), \trace_0 \in \tdom_0(c), \trace \in \tdom, \rprog \st 
    % \rpattern \in \rprog \land
    \rprog = Alg.4(c)
    % \\ \qquad
    \land
    \config{c, \trace_0} \to^* 
    % \config{\highlight{\tpath; \rprog', \trace_0 \tracecat \trace_1} 
    % \land 
    % \config{\rpattern, \trace_0 \tracecat \trace_1} \to^* 
    \config{{\eskip, \trace_0 \tracecat \trace}}
    \\ \qquad
    \implies
    \vcounter(\trace, L(\tpath)) \leq \inoutB(\tpath, \rprog)(\trace_0).
    \end{array}
    \]
\end{thm}
\todo{Formalize the Proof}
\\
Theorem~\ref{thm:sound_path-rb} guarantees that
for every simple transition path $\tpath$ in a program $c$ with its refined program $\rprog$,
the $\inoutB(\tpath, \rprog)$
is a sound upper bound on its execution times in $c$.
