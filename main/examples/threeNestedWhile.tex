\begin{example}[Nested Loop with Related Iterators]
  \label{ex:threeNestedWhile}
  %
  This example program is the same as the motivating program in Figure~\ref{fig:threeWhile-overview}(a) that has three nested loops, and they contain related iterators.
  The iterators of the inner loop is reset by the outer loop. 
  At the meantime, the inner loop modifies the counters of the outer loop.
  We illustrate step-by-step here how we compute the expected reachability-bound of each program point.
  %  Example
  %
  { \small
\begin{figure}
\centering
\begin{subfigure}{.4\textwidth}
  \begin{centering}
  {\small
  $
  \begin{array}{l}
      \kw{relatedNestedWhile}(n, m, N) \triangleq \\
      \clabel{ \assign{i}{0} }^{0} ; \\
          \ewhile ~ \clabel{i < n}^{1} ~ \edo ~ \\
          \qquad \Big(
           \clabel{\assign{j}{m}}^{2} ;\\
           \qquad \ewhile ~ \clabel{j > 0}^{3} ~ \edo ~ \\
           \qquad \qquad \Big(
            \clabel{\assign{j}{j-1}}^{4};
            \clabel{\assign{w}{i}}^{5};\\
            \qquad \qquad \ewhile ~ \clabel{w < N}^{6} ~ \edo ~
            \\ \qquad \qquad 
            \Big(
              \clabel{\assign{w}{w + 1}}^{7}
                \Big); \\
                \qquad \qquad \clabel{\assign{i}{w}}^{8}
                \Big); \\
                \qquad \clabel{\assign{i}{i+1}}^{9}
            \Big)
      \end{array}
  $
  }
  \caption{}
  \end{centering}
  \end{subfigure}
\begin{subfigure}{.55\textwidth}
  \begin{centering}
%   \todo{abstract-cfg for two round}
\begin{tikzpicture}[scale=\textwidth/18cm,samples=200]
\draw[] (-7, 10) circle (0pt) node{{ $0$}};
\draw[] (0, 10) circle (0pt) node{{ $1$}};
\draw[] (6, 10) circle (0pt) node {{$\lex$}};
\draw[] (0, 7) circle (0pt) node{{$2$}};
\draw[] (0, 4) circle (0pt) node{{ $3$}};
\draw[] (-7, 4) circle (0pt) node{{ $9$}};
\draw[] (0, 1) circle (0pt) node{{ $4$}};
\draw[] (3, 1) circle (0pt) node{{ $5$}};
\draw[] (6, 1) circle (0pt) node{{ $6$}};
\draw[] (6, -3) circle (0pt) node{{ $7$}};
\draw[] (5, 6) circle (0pt) node{{ $8$}};
% Counter Variables
%
% Control Flow Edges:
\draw[ thick, -latex] (-6, 10)  -- node [above] {$i' \leq 0$}(-0.5, 10);
\draw[ thick, -latex] (0, 9.5)  -- node [left] {$i < n$} (0, 7.5) ;
\draw[ thick, -latex] (0, 6.5)  -- node [left] {$j' \leq m$} (0, 4.5) ;
\draw[ thick, -latex] (0, 3.5)  -- node [left] {$j > 0$} (0, 1.5) ;
\draw[ thick, -latex] (-0.5, 4)  -- node [below] {$j \leq 0$} (-6.5, 4) ;
\draw[ thick, -latex] (-6.5, 4.5)  to  [out=90,in=180]  node [left] {$i' \leq i + 1$ }(-0.5, 9.5);
\draw[ thick, -latex] (0.5, 10)  -- node [above] {$i \geq n$}  (5, 10);
\draw[ thick, -latex] (0.5, 1)  -- node [below] {$j' \leq j - 1$}  (2.5, 1);
\draw[ thick, -latex] (3.5, 1)  -- node [above] {$w' \leq i$}  (5.5, 1);
\draw[ thick, -latex] (6, 0.5)  -- node [left] {$w < N$}  (6, -2.5);
\draw[ thick, -latex] (6, 1.5)  to [out=90,in=-60] node [right] {$w \geq N$}  (5, 5.5);
\draw[ thick, -latex] (6, -2.5)  to  [out=0,in=0] node [right] {$w' \leq w + 1$}  (6, 0.5);
\draw[ thick, -latex] (5, 5.5)  to  node [above] {$i' \leq w$ }(0.5, 4);
\end{tikzpicture}
\caption{}
  \end{centering}
  \end{subfigure}
\caption{
(a) The example of nested loop with related iterators.
  (b) The abstract transition graph.}
    \label{fig:threeNestedWhile}
\end{figure}
}
\end{example}

\begin{enumerate}
  \item  \textbf{The Abstract Transition Graph}: Figure~\ref{fig:threeNestedWhile}(b).

  \item \textbf{Program Refinement}
  \\
  {Simple Transition Paths:}
  %  are computed as follows,
  \\
$
      \begin{array}{llll}
          \tpath_0 = (0 \to 1)
          &
          \tpath_1 = (1 \to 2 \to 3)
          &           
          \tpath_2 = (3 \to 4 \to 5 \to 6)
          &
          \tpath_3 = (6 \to 7 \to 6)
          \\
          \tpath_6 = (1 \to \lex)
          &
          \tpath_4 = (6 \to 8 \to 3)
          &
          \tpath_5 = (3 \to 9 \to 1)
      \end{array}
$
  \\
  Refined Program:
\\
$
  \rprog = \tpath_0 ; 
1: \rprepeat(\tpath_1; 3: \rprepeat(\tpath_2; 6: \rprepeat(\tpath_3); \tpath_4); \tpath_5); \tpath_6
$
\\
Define sub loop programs as follows,

$\rprog_1 = \rprepeat(\tpath_1; 3: \rprepeat(\tpath_2; 6: \rprepeat(\tpath_3); \tpath_4); \tpath_5)$
\\
$\rprog_3 = \rprepeat(\tpath_2; 6: \rprepeat(\tpath_3); \tpath_4)$
\\
$\rprog_6 = \rprepeat(\tpath_3)$
  \item {Path Local Reachability-bound}:
\\
$\outinB(1: \rprog_1, \tpath_1) = n - N$ \quad
$\outinB(1: \rprog_1, \tpath_5) = n - N$ \\
$\outinB(3: \rprog_3, \tpath_2) = m$ \quad
$\outinB(3: \rprog_3, \tpath_4) = m$ \\
$\outinB(6: \rprog_6, \tpath_3) = N$ \quad
%
\\
Loop Bounds:
\\
$BD(\tpath_0) = 1$
\quad
$BD(\tpath_6) = 1$
\quad
$BD( \rprog_6) = N $
\\
$BD(\rprog_3) = m $
\quad
$BD(\rprog_1) = n - N $
%
\item Loop Reachability-bound:
\\
% $\lpchB(1: \rprog_1, \tpath_1) = n - N$ \quad
% $\lpchB(1: \rprog_1, \tpath_5) = n - N$ \\
$\lpchB(1: \rprog_1, \tpath_2) = n - N$ \quad 
$\lpchB(1: \rprog_1, \tpath_4) = n - N$ \\
$\lpchB(1: \rprog_1, \tpath_3) = 1$ \quad
% $\lpchB(3: \rprog_3, \tpath_4) = m$ \quad
% $\lpchB(3: \rprog_3, \tpath_2) = m$ \\
$\lpchB(3: \rprog_3, \tpath_3) = 1$ \quad 
% $\lpchB(6: \rprog_6, \tpath_3) = N$
%
%
\item Path Global Reachability-bound:
\\
$\inoutB(\rprog, \tpath_1) = n - N$ \quad
$\inoutB(\rprog, \tpath_5) = n - N$ \quad
\\
$\inoutB(\rprog, \tpath_2) = (n - N) \times m$ \quad
$\inoutB(\rprog, \tpath_4) = (n - N) \times m$ 
\\
$\inoutB(\rprog, \tpath_3) = N$
\\
$\inoutB(\rprog, \tpath_6) = 1$ \quad
$\inoutB(\rprog, \tpath_0) = 1$ 
%
\item The Reachability-bound:
\\
$\psRB(0) = \psRB(\lex) = 1$ \quad
$\psRB(1) = n - N + 1$ \quad
$\psRB(2) = \psRB(9) = n - N$ \\
$\psRB(3) = n - N + (n - N) \times m$ \\
$\psRB(4) = \psRB(5) = \psRB(8) = (n - N) \times m$ \\
$\psRB(6) = N + (n - N) \times m$ \quad
$\psRB(7) = N$ \quad
\end{enumerate}