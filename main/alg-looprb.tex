\textbf{Computation Overview}
\\
%  Loop Reachability Bound
% \[
%   \lpchB(l: \rprog, \tpath) \triangleq
%     \frac{\lpinit(l: \rprog, \tpath) - \rffinal(\tpath)}{\lpnext(l: \rprog, \tpath)}
% \]
%  Loop Bound (the alternative computation method):
%   \[
%     BD(\rprepeat(\rprog')) \triangleq
%     \frac{\rfinit(\rprog') - \rffinal(\rprog')}{\varGD(\rprog')}
% \]
  To solve the \emph{loop-reachability bound}, $\lpchB(l: \rprog, \tpath)$ and loop bound, $BD(\rprog)$:
  \\
  1. Compute the ranking function for every transition and estimate its maximum value, as in Section~\ref{sec:rank}.
  \\
  2. Compute the abstract states,
  $\rfinit(\rprog)$, $\rffinal(\rprog)$, $\rfnext(\rprog)$, $\varGD(\rprog, \rprog')$, 
  $\lpinit(l: \rprog, \tpath)$, and $\lpnext(l: \rprog, \tpath)$ using ranking function,
  as in Definition~\ref{def:alg-absstate}.
  \\
  3. Solve the equation using abstract states above and SMT solver.
  % \\
\begin{defn}[Abstract States Computation]
    \label{def:alg-absstate}
   The computations of 
   $\rfinit(\rprog)$, $\rffinal(\rprog)$, $\rfnext(\rprog)$, $\varGD(\rprog, \rprog')$, 
   $\lpinit(l: \rprog, \tpath)$, and $\lpnext(l: \rprog, \tpath) \in \mathcal{A}_{in}$
   are formally presented as follows.
   \begin{itemize}
    \item The \emph{Initial State}, 
    $\rfinit(\rprog)$ is a set of equations $x = e$, where $e \in \mathcal{A}_{in}$ is a
    symbolic expression describing the initial value of $x$ before executing $\rprog$.
    Each $x$ is the ranking function of a simple transition path $\tpath$ in this program. 
    It contains the initial value for every ranking function of the simple transition path $\tpath$ in this program.
   \[
     \rfinit(\rprog) \triangleq 
     \bigcup_{\tpath \in \rprog, x = \locbound(\tpath)}
     \left \{ 
     x = \arg\max_{l_1}\left\{
       \varinvar(y) + v ~\middle\vert~ 
       \begin{array}{l} 
         (l_1, x' \leq y + v, l_2) \in \reset(x) 
         \\
       \land l_1 \leq \absinit(\rprog)
     \end{array}
     \right\}
     \right\}
     \]
   $\rfinit(\tpath)$ can also be computed using the function $\kw{INIT(c, l_0, \absinit(\rprog))}$ in \cite{GulwaniJK09}. 
   $\kw{INIT(c, l_0, \absinit(\rprog))}$ computes the equivalent results.
   %
   \item  The \emph{Final State}, $\rffinal(\rprog)$ a conjunction of boolean expressions.
   It is the post-condition
   after the execution of $\rprog$.
   % \\
   \[
     \rffinal(\rprog) \triangleq 
     \bigcup_{\tpath \in \rprog, x = \locbound(\tpath)}
     \left \{ 
     x = \arg\min_{l_1}\left\{
       v_{post} ~\middle\vert~ 
       \begin{array}{l} 
        l_1 \geq \absfinal(\rprog)[2]
       \land \bigwedge_{b \in \kw{Guard}(\rprog)} \neg b
     \end{array}
     \right\}
     \right\}
     \]
    $\kw{Guard}(\rprog)$ is the set of all the unique boolean expressions (i.e., the boolean constraints) on this program.
   \item The \emph{Next State}, $\rfnext(\tpath) \in \mathcal{A}_{in}$ 
   is a
   symbolic expression describing how much $\tpath$'s ranking function ($\locbound(\tpath)$) is changed after the first execution of $\rprog$ and before the second execution.
   \[
     \begin{array}{l}
     \rfnext(\tpath) \triangleq 
       \begin{array}{l}
    \sum\limits_{\absevent \in \inc(x) }
     \left\{ v ~\middle\vert~ \absevent = (l, x' \leq x + v, \_) \land l \in \tpath\right\}
     \\ \qquad 
     + \arg\max\limits_{l' }
        \left\{ \varinvar(y) + v ~\middle\vert~ (l, x' \leq y + v, l') \in \reset(x) \land l \in \tpath\right\}
        \\ \qquad 
       - \sum\limits_{ \absevent \in \dec(x) }\left\{ 
         v ~\middle\vert~ \absevent = (l, x' \leq x - v, \_) \land l \in \tpath 
         \right\}
       \end{array}
     \end{array}
     , x = \locbound(\tpath)
   \]
   Indeed we only compute the $\rfnext(\tpath)$ because that the recursion is exhausted into the base case, i.e. $\tpath$ when computing $\varGD(\rprog)$ as below.
   \item  The \emph{Variable Gradient Decent}, 
   $\varGD(\rprog)$
   is a set of equations $x = e$ for the same variables in the $\rfinit(\rprog)$.
   It computes the set of next states for every ranking function of the simple transition path $\tpath$ in $\rprog$,
   and recursively computes the loop bound, $BD$ in the meantime.
   \\
   {$\varGD(\rprepeat(\rprog)) =    BD(\rprog)  \times
  %  \highlight{\frac{\rfinit(\rprepeat(\rprog)) - \rffinal(\rprepeat(\rprog))}{\varGD(\rprepeat(\rprog))}}  \times 
 {\varGD(\rprog)}$}
   \\
   $\varGD(\rprog_1;\rprog_2) =  \varGD(\rprog_1) \cup \varGD(\rprog_2)$
   \\
   $\varGD(\tpath) =  \left\{x = \rfnext(\tpath) ~|~ x = \locbound(\tpath) \right\} $   
%
   \item 
%    For a transition path $\tpath \in \paths(\absG(c))$ and a loop label $l$ in this transition path's loop chain.
% Let $\tpath_l$ be the repeat pattern with label $l$, i.e., $l: \tpath_l \in \rprog$, 
% the abstract loop chain initial state 
The loop initial state 
$\lpinit(l: \rprog, \tpath) \in \inpexpr$ is symbolic expression as well. 
It describes the abstract initial value of $\tpath$'s ranking function before
any visit of $\tpath$ and during the first execution of $l: \rprog$.
\[
  \lpinit(l: \rprog, \tpath) \triangleq 
  \arg\max_{l_1}\left\{
       \varinvar(y) + v ~\middle\vert~ 
       \begin{array}{l} 
         (l_1, x' \leq y + v, l_2) \in \reset(x) 
         \\
         \land \absinit(\rprog) \leq l_1 \leq \absinit(\tpath)
       \end{array}
     \right\}
    , x = \locbound(\tpath)
  \]
\item
The loop next state 
$\lpnext(l: \rprog, \tpath) \in \inpexpr$ 
describes how much $\tpath$'s ranking function
is modified before
the second visit of $\tpath$ but during the second execution of $l: \rprog$.
\footnote{$l' \in \rprog$: the $\in$ notation is abused to denote
the program point $l'$ is a vertex on a path in the program $\rprog$.}
%
\[
  \begin{array}{l}
  \lpnext(l: \rprog, \tpath) \triangleq 
    \begin{array}{l}
  \sum\limits_{\absevent \in \inc(x) }
  \left\{ 
      v ~\middle\vert~ \absevent = (l', x' \leq x + v, \_) \land  l' \in \rprog 
      \land l' \notin \tpath \right\}
      \\ \qquad 
      + \arg\max\limits_{l_2 }
         \left\{ \varinvar(y) + v ~\middle\vert~ 
         (l_1, x' \leq y + v, l_2) \in \reset(x) \land l_1 \in \rprog \land l_1 \notin \tpath\right\}
     \\ \qquad 
      - \sum\limits_{ \absevent \in \dec(x) }\left\{ 
      v 
      ~\middle\vert~ \absevent = (l', x' \leq x + v, \_) \land l' \in \rprog \land l' \notin \tpath \right\}
      \\ \qquad 
      + \outinB(\tpath) \times \rfnext(\tpath)
    \end{array}
  \end{array}
  , x = \locbound(\tpath)
  \]
    \end{itemize}
\end{defn}

Formally
\begin{defn}[Loop Bound (Alternative Computation Method)]
\label{def:loopbound}
\[
  \begin{array}{rcl}
    BD(\tpath) & \triangleq & 1 \\
    BD(\rprog_1;\rprog_2) & \triangleq & BD(\rprog_1) + BD(\rprog_2) \\
    BD(\rpchoose{\rprog_1, \cdots, \rprog_m }) & \triangleq 
    & \max\left\{ BD(\rprog_1), \cdots, BD(\rprog_m) \right\} \\
    BD(\rprepeat(\rprog')) & \triangleq 
    &
    \max\limits_{\tpath \in \rprog, x = \locbound(\tpath)}
    \Big\{ \highlight{\frac{a - b}{a'}} ~\vert~
    x = a \in \rfinit(\rprog')
    \\ & & \qquad \qquad
    \land x = b \in \rffinal(\rprog')
    \land x = a' \in \varGD(\rprog')
     \Big\} 
     \\
    BD(l: \rprog') & \triangleq & BD(\rprog')
  \end{array}
  \]
\end{defn}

\begin{defn}[Loop Reachability Bound]
  \label{def:looprb}
  Loop Reachability Bound
  \[
    \lpchB(l: \rprog, \tpath) \triangleq
      \frac{\lpinit(l: \rprog, \tpath) - \rffinal(\tpath)}{\lpnext(l: \rprog, \tpath)}
  \]
\end{defn}