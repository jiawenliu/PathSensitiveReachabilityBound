\textbf{Computation Overview}
To solve the \emph{loop-reachability bound}, $\lpchB(l: \rprog, \tpath)$
  \\
  1. Compute the loop abstract states,
  $\lpinit(l: \rprog, \tpath, c)$, and $\lpnext(l: \rprog, \tpath, c)$ using ranking function,
  as in Definition~\ref{def:alg-absstate}.
  \\
  2. Solve the equation using abstract states above and SMT solver.
\begin{defn}[Loop Abstract States Computation]
\label{def:alg-loopabsstate}
Given a program $c$, with $l:\rprog$ as a sub-program (a loop) in $c$'s refined program, 
where $l: \rprog \subseteq REFINE(\kw{Rewrite(c)})$, 
we compute four two abstract states for the ranking functions of this loop, 
   $\lpinit(l: \rprog, \tpath, c)$, and $\lpnext(l: \rprog, \tpath, c) \in \scexpr(c)$
   are formally presented as follows.
   \begin{itemize}%
   \item 
The loop initial state 
$\lpinit(l: \rprog, \tpath, c) \in \scexpr(c)$ is symbolic expression as well. 
It describes the abstract initial value of $\tpath$'s ranking function before
any visit of $\tpath$ and during the first execution of $l: \rprog$.
\[
  \lpinit(l: \rprog, \tpath, c) \triangleq 
  \arg\max_{l_1}\left\{
       \varinvar(y, c) + v ~\middle\vert~ 
       \begin{array}{l} 
         (l_1, x' \leq y + v, l_2) \in \reset(x) 
         \\
         \land \absinit(\rprog) \leq l_1 \leq \absinit(\tpath)
         \land
         x \in \left\{ \locbound(\absevent, c) | \absevent \in \tpath \right\}
       \end{array}
     \right\}
  \]
\item
The loop next state 
$\lpnext(l: \rprog, \tpath, c) \in \scexpr(c)$ 
describes how much $\tpath$'s ranking function
is modified before
the second visit of $\tpath$ but during the second execution of $l: \rprog$.
\footnote{$l' \in \rprog$: the $\in$ notation is abused to denote
the program point $l'$ is a vertex on a path in the program $\rprog$.}
%
\[
  \begin{array}{l}
  \lpnext(l: \rprog, \tpath, c) \triangleq 
  \\
  \max\limits_{x \in \left\{ \locbound(\absevent, c) | \absevent \in \tpath \right\}}
  \left\{
    \begin{array}{l}
  \sum\limits_{\absevent \in \inc(x, c) }
  \left\{ 
      v ~\middle\vert~ \absevent = (l', x' \leq x + v, \_) \land  l' \in \rprog 
      \land l' \notin \tpath \right\}
      \\ \qquad 
      + \arg\max\limits_{l_2 }
         \left\{ \varinvar(y, c) + v ~\middle\vert~ 
         (l_1, x' \leq y + v, l_2) \in \reset(x, c) \land l_1 \in \rprog \land l_1 \notin \tpath\right\}
     \\ \qquad 
      - \sum\limits_{ \absevent \in \dec(x, c) }\left\{ 
      v 
      ~\middle\vert~ \absevent = (l', x' \leq x + v, \_) \land l' \in \rprog \land l' \notin \tpath \right\}
      \\ \qquad 
      + BD(\kw{enclosed}(\tpath)) \times \rfnext(\tpath)
    \end{array}
    \right\}
  \end{array}
  \]
    \end{itemize}
\end{defn}
%
\begin{defn}[Loop Reachability-Bound]
  \label{def:looprb}
  Given a refined program $\rprog$ and a simple transition path $\tpath$ in this program, 
  let $l: \rprog_l$ be a loop program in $\rprog$,
  then $l: \rprog_l$'s \emph{loop reachability-bound} w.r.t. $\tpath$
  is computed recursively as follows. 
  \[
    \lpchB(l: \rprog, \tpath, c) \triangleq
    \frac{\lpinit(\rprog, \tpath, c) - \rffinal(\tpath, c)}{\lpnext(\rprog, \tpath, c)}
  \]
\end{defn}
%
We show that $\lpchB(l: \rprog, \tpath)$
is a sound upper bound on iteration numbers of the outside loop $l$,
such that,
during these iterations, the nested loop $l' = \kw{enclosed(\tpath)}$ is executed, i.e., reached in Lemma~\ref{lem:looprb-sound} with proof in \highlight{Appendix~\ref{apdx:looprb-sound}}.
\begin{lem}[Soundness of the Loop Reachability-Bound]
  \label{lem:looprb-sound}
  For any loop $l: \rprog$ and a simple transition path $\tpath$ in a refined program, if $l_t: \rprog_t$ is the closest loop enclosing the $\tpath$, then the entering times of $l_t: \rprog_t$ when executing the $l: \rprog$ under initial trace $\trace_0 \in \ftdom_0(c)$, is bounded by $\econfig{\lpchB(l: \rprog, \tpath, c)}(\trace_0)$.
  \[
    \begin{array}{l}
    \forall c, c_l \in \cdom, \tpath \in \absG(c), \trace_0 \in \ftdom_0(c_l), \trace \in \tdom, l, l' \in \ldom, \rprog \st 
    \rprog = REFINE(c)
    \\ \qquad
    \land l_t: \rprog_t = \kw{enclosed}(\rprog, \tpath)
    \land 
    \rprog = \algrewrite(c_l)
    \land
    \Big(
    \config{c_l, \trace_0} \to^* \config{\clabel{\eskip}^{l'}, \trace_0 \tracecat \trace}
    \lor \config{c_l, \trace_0} \uparrow^{\infty} \trace_0 \tracecat \trace 
    \Big)
    \\ \qquad
    \implies
    \econfig{\lpchB(l: \rprog, \tpath, c)}(\trace_0) 
    \\ \qquad \geq 
      \sum\Big\{
      \lcounter(\trace, \tracel(\trace_{l \to l_t})) ~\vert~ \trace_{l \to l_t} \in 
      \big\{\trace_{l \to l_t} ~\vert~ 
      % \\ \qquad \qquad \qquad \qquad
      \trace_{l \to l_t}, \trace' \in \ftdom, \trace'' \in \tdom
      \land \trace = \trace_{l \to l_t} \tracecat \trace''
      \\ \qquad \qquad \qquad \qquad
      \land \trace_{l \to l_t} = [(\_, l, \etrue)] \tracecat \trace' \tracecat [(\_, l_t, \etrue)]
      \land \counter(\trace', l) = \counter(\trace', l_t) = 0 
      \big\}
      \Big\}.
\end{array}
  \]
% \[
  %   \begin{array}{l}
  %   \forall c, c_l, c_l' \in \cdom, \tpath \in \absG(c), \trace_0 \in \ftdom_0(c_l), \trace \in \tdom, l, l' \in \ldom, \rprog \st 
  %   \rprog = REFINE(c)
  %   \\ \qquad
  %   \land l_t: \rprog_t = \kw{enclosed}(\rprog, \tpath)
  %   \land \rprog = \algrewrite(c_l)
  %   \land
  %   \Big(
  %   \config{c_l, \trace_0} \to^* \config{\clabel{\eskip}^{l'}, \trace_0 \tracecat \trace}
  %   \lor \config{c_l, \trace_0} \uparrow^{\infty} \trace_0 \tracecat \trace 
  %   \Big)
  %   \\ \qquad
  %   \implies
  %   \econfig{\lpchB(l: \rprog, \tpath)}(\trace_0) 
  %   \\ \qquad \geq 
  %   \# \left\{\trace_{l \to l_t} ~\vert~ \trace_{l \to l_t} \in \trace, L' \in \mathcal{LIST}(\ldom) \land 
  %   \tracel(\trace_{l \to l_t}) = [l] ++ L' ++ [l_t]
  %   \land l, l_t \notin L'
  %   \right\}.
  %   \end{array}
  % \]
\end{lem}
%
The $\lpchB(l:\rprog, \tpath)$ 
can also be computed by
$I(l, l') = \kw{BOUNDFINDERD(INIT(c, i, \absinit(\tpath)), NEXT(c, i, \absinit(\tpath)), V_{\ln})}$ in paper\cite{GulwaniJK09}.
However, $I(l, l')$ over-approximates this value, because it computes the bound on iteration numbers of $l'$ in one iteration of $l$.
