%
%
\subsection{Labeled Language}
% \mg{It is ok to list all the operations in the appendix but for the main paper it is better to save space.}
\[
\begin{array}{llll}
\mbox{Arithmetic Operators} 
& \oplus_a & ::= & + ~|~ - ~|~ \times 
%
~|~ \div ~|~ \max ~|~ \min\\  
% ~|~ \div \\  
\mbox{Boolean Operators} 
& \oplus_b & ::= & \lor ~|~ \land
\\
%
\mbox{Relational Operators} 
& \sim & ::= & < ~|~ \leq ~|~ == 
\\  
%
\mbox{Arithmetic Expression} 
& \aexpr & ::= & 
n ~|~ {x} ~|~ \aexpr \oplus_a \aexpr  
 ~|~ \elog \aexpr  ~|~ \esign \aexpr
\\
%
\mbox{Boolean Expression} & \bexpr & ::= & 
%
\etrue ~|~ \efalse  ~|~ \neg \bexpr
 ~|~ \bexpr \oplus_b \bexpr
%
~|~ \aexpr \sim \aexpr 
\\
%
\mbox{Expression} & \expr & ::= & v ~|~ \aexpr ~|~ \bexpr ~|~ [\expr, \dots, \expr]
\\  
%
\mbox{Value} 
& v & ::= & { n ~|~ \etrue ~|~ \efalse ~|~ [] ~|~ [v, \dots, v]}  
\\ 
&&&
\highlight
{
~|~ (r, x_1, \ldots, x_n) := c
}
\\
%
\mbox{Query Expression} 
& {\qexpr} & ::= 
& { \qval ~|~ \aexpr ~|~ \qexpr \oplus_a \qexpr ~|~ \chi[\aexpr]} 
\\
%
\mbox{Query Value} & \qval & ::= 
& {n ~|~ \chi[n] ~|~ \qval \oplus_a  \qval ~|~ n \oplus_a  \chi[n]
    ~|~ \chi[n] \oplus_a  n}
\\
% \\%
\mbox{Label} 
& l & ::= & (n \in \mathbb{N} \cup \{\lin, \lex\}) ~|~ (l, n)
\\ 
%
\mbox{Labeled Command} 
& {c} & ::= &  
\clabel{\assign{x}{\expr}}^l 
% ~|~ \clabel{\assign{x}{\query(\qexpr)}}^l
~|~  \clabel{\eskip}^l
~|~ \ewhile \clabel{\bexpr}^{l} \edo {c}
~|~ \eif(\clabel{\bexpr}^{l} , {c}, {c}) 
\\ 
&&&
\highlight
{
~|~ \clabel{\efun}^l: x(r, x_1, \ldots, x_n) := c
~|~ \clabel{\assign{x}{\ecall(x, e_1, \ldots, e_n)}}^l
}
~|~ {c};{c}  
\\ 
% \\
\mbox{Event} 
& \event & ::= & 
% ~|~ ({x}, l, v, \qval)
({x}, l, v) ~ \mbox{Assignment Event} 
% \\
% &&& 
~|~(\bexpr, l, v) ~ \mbox{Testing Event}
\\
% &&& \text{\mg{I think it would be better to use quadruples for events, where the}}\\
% &&& \text{\mg{first element is either a variable or a boolean expression and }}\\
% &&& \text{\mg{the last is either a query value or some default value $\bullet$}}\\
%
% \mbox{Trace} & \trace
% & ::= & \cdot | \trace \cdot \event | \trace \tracecat \trace 
% \\
%
% \mbox{Trace} & \trace
% & ::= & [] ~|~ \event:: \trace ~|~ \trace \tracecat \trace  \\
\mbox{Trace} & \trace
& ::= & [] ~|~ \trace :: \event\\
% &&& \text{\mg{I don't understand why you need both :: and ++ as constructors.}}\\
% &&& \text{\jl{Because append is to the left but we are adding element to the left in the OS}}\\
% &&& \text{\jl{I was too sticky to the convention, it is a good idea to append to the left and just use $::$}}
% %
% \mbox{Event Signature} & \sig
% & ::= & (x, l, n) | (x, l, n, \query) | (b, l, n)
% \\
% %
\end{array}
\]
% \todo{change trace notation into list, and update corresponding operator nations}
% \\
% \wqside{"$\cdot$" has two meanings? empty, delimit. Trace is list of event?}
We use following notations to represent the set of corresponding terms:
\[
\begin{array}{lll}
\mathcal{VAR} & : & \mbox{Set of Variables}  
\\ 
%
\mathcal{VAL} & : & \mbox{Set of Values} 
% \\ 
% %
% \mathcal{QVAL} & : & \mbox{Set of Query Values} 
\\ 
%
\cdom & : & \mbox{Set of Commands} 
\\ 
%
\eventset  & : & \mbox{Set of Events}  
\\
%
\eventset^{\asn}  & : & \mbox{Set of Assignment Events}  
\\
%
\eventset^{\test}  & : & \mbox{Set of Testing Events}  
\\
%
\ldom  & : & \mbox{Set of Labels}  
\\
%%
% \mathcal{VAL}  & : & \mbox{Set of Labeled Variables}  
% \\
%%
% \dbdom  & : & \mbox{{Set of Databases}} 
% \\
%
{\mathcal{T}} & : & \mbox{Set of Traces}
\\
%
\mathcal{T}_0(c) & : & \mbox{Set of Initial Traces, where all the input variables of the program $c$ are initialized.
}
\end{array}
\]
%
% The labeled variables and assigned variables are set of variables annotated by a label. 
% We use  
% %$\mathcal{LVAR} = \mathcal{VAR} \times \mathcal{L} $ 
% $\mathcal{LV}$ represents the universe of all the labeled variables and 
% $\avar_c \in \mathcal{P}(\mathcal{VAR} \times \mathbb{N}) \subset \mathcal{LV}$ and 
$\lvar_c \in \mathcal{P}(\ldom)$,
represents the set of labels
% assigned variables and labeles variables 
for a labeled command $c$,
% defined in Definition~\ref{def:lvar} and 
\ref{def:avar}.
%
\\
$FV: \expr \to \mathcal{P}(\mathcal{VAR})$, computes the set of free variables in an expression. To be precise,
$FV(\aexpr)$, $FV(\bexpr)$ and $FV(\qexpr)$ represent the set of free variables in arithmetic
expression $\aexpr$, boolean expression $\bexpr$ and query expression $\qexpr$ respectively.
Labeled variables in $c$ is the set of assigned variables and all the free variables
showing up in $c$ with a default label $in$. 
The free variables
showing up in $c$, which aren't defined before be used, are actually the input variables of this program.
%
% \begin{defn}[Assigned Variables (
% % $\avar_{c} \subseteq \mathcal{VAR} \times \mathbb{N}$ or 
% $\avar : \cdom \to \mathcal{P}(\mathcal{VAR} \times \mathbb{N})$)]
% % labelled Variables 
% % (
% % % $\lvar_{c} \subseteq \mathcal{VAR} \times \mathbb{N}$ or 
% % $\lvar : \cdom \to \mathcal{P}(\mathcal{VAR} \times \mathcal{L})$
% \label{def:avar}
% {\footnotesize
% $$ \avar_{c} \triangleq
%   \left\{
%   \begin{array}{ll}
%       \{{x}^l\}                   
%       & {c} = [{\assign x e}]^{l} 
%       \\
%       \{{x}^l\}                   
%       & {c} = [{\assign x \query(\qexpr)}]^{l} 
%       \\
%       \avar_{{c_1}} \cup \avar_{{c_2}}  
%       & {c} = {c_1};{c_2}
%       \\
%       \avar_{{c}} \cup \avar_{{c_2}} 
%       & {c} =\eif([\bexpr]^{l}, c_1, c_2) 
%       \\
%       \avar_{{c}'}
%       & {c}   = \ewhile ([\bexpr]^{l}, {c}')
% \end{array}
% \right.
% $$
% }
% \end{defn}
% %

\begin{defn}[Program Labels
(
% $\lvar_{c} \subseteq \mathcal{VAR} \times \mathbb{N}$ or 
$\lvar : \cdom \to \mathcal{P}(\mathcal{LV})$]
\label{def:lvar}
{\footnotesize
$$
  \lvar_{c} \triangleq
  \left\{
  \begin{array}{ll}
      \{l\}                  
      & {c} = [{\assign x e}]^{l} 
      \\
      \{l\}                
      & {c} = [{\assign x \query(\qexpr)}]^{l} 
      \\
      \lvar_{{c_1}} \cup \lvar_{{c_2}}  \cup \{l\} 
      & {c} = {c_1};{c_2}
      \\
      \lvar_{{c}} \cup \lvar_{{c_2}} \cup  \cup \{l\} 
      & {c} =\eif([\bexpr]^{l}, c_1, c_2) 
      \\
      \lvar_{{c}'} \cup \{l\} 
      & {c}   = \ewhile ([\bexpr]^{l}, {c}')
\end{array}
\right.
$$
}
\end{defn}
%
\highlight{
  \begin{defn}[Label Increase]
    \label{def:label_inc}  
    Label Increase $ + : {\ldom \to \mathbb{N} \to \ldom}$, increase a label $l$ by a natural number $n$:
\[
    n + n' \triangleq n'' ~ n, n' \in \mathbb{N} \land \config{[], n + n'} \aarrow n''
   \qquad (l, n) + n' \triangleq (l + n', n'') ~ n, n' \in \mathbb{N} \land \config{[], n + n'} \aarrow n''
   \]
\end{defn}
The case of $(l, n) + n'$ will never happen during evaluation.
By Operational semantics, the only place the label increase is in rule \textbf{fun-def},
$c' = (c)^{+n}$, where $c$ is the function body.
By the rule \textbf{fun-call}, and the label augment in Definition~\ref{def:comlabel_aug}, the function body $c$ will never be augmented.
%
\begin{defn}[Command Label Increase] 
  \label{def:comlabel_inc}
Command Label Increase $ {(\cdot)}{}^{+n} : {\cdom \to \cdom}$, increase the label in command by $n$.
\[
\begin{array}{ll}
  (\clabel{\assign{x}{\expr}}^l){}^{+n} & \triangleq \clabel{\assign{x}{\expr}}^{l + n}\\
(\clabel{\assign{x}{\query(\qexpr)}}^l)^{+n} & \triangleq \clabel{\assign{x}{\query(\qexpr)}}^{l + n}\\
(\clabel{\eskip}^l)^{+n} & \triangleq \clabel{\eskip}^{l + n}\\
(\ewhile \clabel{\bexpr}^{l} \edo {c'})^{+n} & \triangleq \ewhile \clabel{\bexpr}^{l+n} \edo {(c')^{+n}}\\
(\eif(\clabel{\bexpr}^{l} , {c_1}, {c_2}))^{+n}  & \triangleq \eif(\clabel{\bexpr}^{l+n} , {(c_1)^{+n}}, {(c_2)^{+n}})\\
% (\clabel{\efun}^l: x(r^l, x_1, \ldots, x_n) := c)^{+n} & \triangleq \clabel{\efun}^{l + n}: x(r^l, x_1, \ldots, x_n) := (c)^{+n} \\
(\clabel{\efun}^l: x(r^l, x_1, \ldots, x_n) := c)^{+n} & \triangleq \clabel{\efun}^{l + n}: x(r^l, x_1, \ldots, x_n) := c \\
(\clabel{\assign{x}{\ecall(x, e_1, \ldots, e_n)}}^l)^{+n} & \triangleq \clabel{\assign{x}{\ecall(x, e_1, \ldots, e_n)}}^{l + n}\\
({c_1};{c_2})^{+n} &  \triangleq {(c_1)}^{+n};{(c_2)}^{+n}
\end{array}
\]
\end{defn}
%
\begin{defn}[Command Label Augment] 
  \label{def:comlabel_aug}
  Command Label Augment $ \clabel{\cdot}^{l} : {\cdom \to \cdom}$, augment the label in command with a label $l$ 
in order to record the calling site.
\[
\begin{array}{ll}
  \clabel{\clabel{\assign{x}{\expr}}^{l'}}{}^{l} & \triangleq \clabel{\assign{x}{\expr}}^{(l, l')}\\
  \clabel{\clabel{\assign{x}{\query(\qexpr)}}^{l'}}^{l} & \triangleq \clabel{\assign{x}{\query(\qexpr)}}^{(l, l')}\\
  \clabel{\clabel{\eskip}^{l'}}^{l} & \triangleq \clabel{\eskip}^{(l, l')}\\
  \clabel{\ewhile \clabel{\bexpr}^{l'} \edo {c'}}^{l} & \triangleq \ewhile \clabel{\bexpr}^{(l, l')} \edo {(c')^{l}}\\
  \clabel{\eif(\clabel{\bexpr}^{l'} , {c_1}, {c_2})}^{l}  & \triangleq \eif(\clabel{\bexpr}^{(l, l')} , {(c_1)^{l}}, {(c_2)^{l}})\\
  \clabel{\clabel{\efun}^{l'}: x(r^l, x_1, \ldots, x_n) := c}^{l} & \triangleq \clabel{\efun}^{(l, l')}: x(r^l, x_1, \ldots, x_n) := c \\
  \clabel{\clabel{\assign{x}{\ecall(x, e_1, \ldots, e_n)}}^{l'}}^{l} & \triangleq \clabel{\assign{x}{\ecall(x, e_1, \ldots, e_n)}}^{(l, l')}\\
  \clabel{{c_1};{c_2}}^{l} &  \triangleq \clabel{c_1}^{l};\clabel{c_2}^{l}
\end{array}
\]
\end{defn}
}
% Each command is now labeled with a label $l$, either a natural number standing for the line of code where the command appears, or a symbol of $in$ or $ex$ used for annotating the input variables, and the exist point of the program. Notice that we associate the label $l$ to the conditional predicate $\bexpr$ in the if statement, and to the while guard counter $\bexpr$ in the $\ewhile$ statement.
% We abuse the same notation $c$ for labeled command in the rest of the paper.
% \\
% \todo{notation}
%
%
%
% % $$
% % }
% % \end{defn}
% %
% It is easy to see that a program $c$'s query variables is a subset of 
% its labeled variables, $\qvar(c) \subseteq \lvar(c)$.
%
% \mg{In this definition as well as in others, I have the impression that you assume that the labelled variables are unique in the program. For example, it would not make sense to assign a query to the same labelled variable over and over. If this is the case, we need to make this very explicit in the paper.}
% \jl{TODO}
%
Every labeled variable in a program is unique, formally as follows with proof in Appendix~\ref{apdx:lemma_sec123}.
\begin{lem}[Uniqueness of the Labeled Variables]
  \label{lem:lvar_unique}
  For every program $c \in \cdom$ and every two labeled variables such that
  $x^i, y^j \in \lvar(c)$, then $x^i \neq y^j$.
  \[
    \forall c \in \cdom, x^i, y^j \in \mathcal{L} \st x^i, y^j \in \lvar(c)\implies x^i \neq y^j.
    \]
\end{lem}
%
%
%
% \subsection{Trace-based Operational Semantics for Language \mg{What is ``Language''?}}
\subsection{{Trace-based Operational Semantics for {\tt Query While} Language}}

\subsubsection{Event}
Event projection operators $\pi_i$ projects the $i$th element from an event: 
\\
$\pi_i : 
\eventset \to \mathcal{VAR}\cup \mbox{Boolean Expression}  \cup \mathbb{N} \cup \mathcal{VAL} \cup \mathcal{QVAL} $ 
% \wqside{use b for Boolean expression?}
\\
% $\pi_{(i,j)} (\event) \triangleq (\pi_i(\event), \pi_j(\event)) $
% %
% \\
% Event Signature : $\pi_{\sig} : \eventset \to (\mathcal{VAR}\cup \mbox{Boolean Expression}) \times\mathbb{N}\times \mathbb{N}$
% \[
% \begin{array}{lll}
% \pi_{\sig} (x, l, n, v) \triangleq (x, l, n)
% &
% \pi_{\sig} (x, l, n, \qval, v) \triangleq (x, l, n, \query)
% &
% \pi_{\sig} (b, l, n, v)  \triangleq (b, l, n)
% \end{array}
% \]
%
%
Free Variables: $FV: \expr \to \mathcal{P}(\mathcal{VAR})$, 
the set of free variables in an expression.
\\
$FV(\qexpr)$ is the set of free variables in the query expression $\qexpr$.

\subsubsection{Trace}
%
A trace $\trace \in \tdom$ is a list of events, 
collecting the events generated during a specific program execution. 
\[
\begin{array}{llll}
\mbox{Trace} & \trace
& ::= & [] ~|~ \trace :: \event 
% ~|~ []^{\infty}
\end{array}
\]
A trace can be regarded as the program history, 
which records all the evaluations for assignment commands and guards in $\eif$ and $\ewhile$ command.
\\
\highlight{
A trace can be finite ($\trace \in \ftdom$) or infinite $\trace \in \inftdom$.
If a program doesn't terminate when executing under some initial trace,
it produces the infinite trace 
from $\inftdom$, which records a non-terminating computation.
So we denote by $\tdom$ the set of all traces, and $\tdom = \ftdom \cup \inftdom$.
The trace-based semantics with non-terminating execution is defined below following the maximal trace semantics in \cite{Cousot19}.}

We use list notation for traces, where $[]$ is the empty trace, the operator $\traceadd$ combines an event and a trace in a new event, 
and the operator $\tracecat$ concatenates two traces formally defined as follows. 

\begin{defn}[Trace Concatenation, $\tracecat: \tdom \to \tdom\to \tdom $]
  \label{def:trace_concate}
Given two traces $\trace_1 \in \tdom, \trace_2 \in \tdom$, the trace concatenation operator 
$\tracecat$ is defined as:
\[
\trace_1 \tracecat \trace_2 \triangleq
\left\{
\begin{array}{ll} 
  \trace_1 & \trace_2 = [] \lor \trace_1 \in \inftdom \\
  \trace_2 & \trace_1 = [] \lor \trace_2 \in \inftdom \\
  (\trace_1   \tracecat \trace_2'):: \event & \trace_1 \in \ftdom \land \trace_2 = \trace_2' :: \event
  % \trace_2 &  \trace_2 \in \inftdom \\
\end{array}
\right.
\]
\end{defn}

\begin{defn}(An Event Belongs to A Trace)
  An event $\event \in \eventset$ belongs to a trace $\trace \in \tdom$, i.e., $\event \in \trace$ are defined as follows:
%
\begin{equation*}
  \event \in \trace  
  \triangleq \left\{
  \begin{array}{ll} 
    \etrue                  & \trace =  [\event] \tracecat \trace'
     \land \event = \event' \\
    \event \in \trace' & \trace =  [\event'] \tracecat \trace'
      \land \event \neq \event' \\ 
    \efalse                 & \trace = [] \lor \trace \in \inftdom
  \end{array}
  \right.
\end{equation*}
As usual, we denote by $\event \notin \trace$ that the event $\event$ doesn't belong to the trace $\trace$.
\end{defn}
%
In the rest of the paper, we denote by $\bot$ a value s.t. $\bot < n $ for all $n \in \mathbb{N}$.
\begin{defn}[Counter Notation for Program Point]
  \label{def:counter}
The counting operator $\counter : \tdom \to \ldom \to (\mathbb{N} \cup \{\bot, \infty\})$
counts the appearance of a label in a trace.
\[
\begin{array}{llll}
\counter([(x, l, v)] \tracecat \trace', l ) \triangleq \counter(\trace', l) + 1 & \text{if}~ l = l
&
\counter([(b, l, v)] \tracecat \trace', l) \triangleq \counter(\trace', l) + 1 & \text{if}~ l = l
\\
\counter([(x, l', v)] \tracecat \trace', l) \triangleq \counter(\trace', l)   & \text{if}~ l' \neq l
&
\counter([(b, l', v)] \tracecat \trace', l) \triangleq \counter(\trace', l)   & \text{if}~ l' \neq l
\\
\counter({[]}, l) \triangleq 0 & 
&
\counter(\trace, l) \triangleq \bot & \text{if }~ \trace \in \inftdom
\end{array}
\]
{When the input trace is infinite, $\counter(\trace, l)$ returns $\bot$ for any program label $l$.}
\end{defn}
\begin{defn}[Counter Notation for List of Program Point]
  \label{def:lcounter}
  The counting operator $\lcounter : \tdom \to \mathcal{P}(\ldom) \to (\mathbb{N} \cup \{\infty\})$
  counts the appearance of a list of labels $[l_1, \ldots, l_n]$ as follows.
\[
  \begin{array}{ll}
  \lcounter(\trace'' \tracecat \trace', [l_1, \ldots, l_n] ) 
  \triangleq \lcounter(\trace', [l_1, \ldots, l_n]) + 1  & \text{if}~ \pi_2(\trace''[i]) = l_i, \forall i = 1, \ldots, n
  \\ 
  \lcounter([(\_, l, \_)] \tracecat \trace', [l_1, \ldots, l_n] ) 
  \triangleq \lcounter(\trace', [l_1, \ldots, l_n]) & \text{if}~ l \neq l_1
  \\ 
  \lcounter(\trace, [l_1, \ldots, l_n] ) 
  \triangleq \bot & \text{if }~ \trace \in \inftdom
\end{array}
\]
{When the input trace is infinite, $\lcounter(\trace, L)$ returns $\bot$ for any list of labels as well.}
\end{defn}
%
We define the operator $\tracel : \tdom \to \mathcal{P}{(\ldom)}$ projects the label from every event in a trace as a list of program points,
defined as follows,
\[
\tracel([(\_, l, \_)] \tracecat \trace') \triangleq [l] \tracecat \tracel(\trace')
\qquad
\tracel([ ]) \triangleq []
\]
\subsubsection{Environment} $ \env : {\mathcal{T}}  \to \mathcal{VAR} \to \mathcal{VAL} \cup \{\bot\}$
% \mgside{The following definition is missing one case, also it is better to say that $y\neq x$.}
% \[
% \begin{array}{lll}
% \env(\trace  \tracecat [(x, l, v, \cdot)]) x \triangleq v
% &
% \env(\trace \tracecat [(y, l, v, \cdot)]) x \triangleq \env(\trace) x, y \neq x
% &
% \env(\trace \tracecat [(b, l, v, \cdot)]) x \triangleq \env(\trace) x
% \\
% \env(\trace \tracecat [(x, l, v, \qval)]) x \triangleq v
% &
% \env(\trace \tracecat [(y, l, v, \qval)]) x \triangleq \env(\trace) x, y \neq x
% &
% \env({[]} ) x \triangleq \bot
% \end{array}
% \]
\[
\begin{array}{lll}
\env(\trace  \traceadd (x, l, v, \bullet)) x \triangleq v
&
\env(\trace \traceadd (y, l, v, \bullet)) x \triangleq \env(\trace) x, y \neq x
&
\env(\trace \traceadd (b, l, v, \bullet)) x \triangleq \env(\trace) x
\\
\env(\trace \traceadd (x, l, v, \qval)) x \triangleq v
&
\env(\trace \traceadd (y, l, v, \qval)) x \triangleq \env(\trace) x, y \neq x
&
\env({[]} ) x \triangleq \bot
\end{array}
\]
\subsubsection{Operational Semantics Rules}
{
\begin{mathpar}
\boxed{ \config{\trace,\aexpr} \aarrow v \, : \, \mbox{Trace  $\times$ Arithmetic Expr $\Rightarrow$ Arithmetic Value} }
\\
% \text{\mg{Missing. Without these rules it is difficult to understand why we need a trace to evaluate expressions.}}
% \\
\inferrule{ 
  \empty
}{
 \config{\trace,  n} 
 \aarrow n
}
\and
\inferrule{ 
  \env(\trace) x = v
}{
 \config{\trace,  x} 
 \aarrow v
}
\and
\inferrule{ 
  \config{\trace, \aexpr_1} \aarrow v_1
  \and 
  \config{\trace, \aexpr_2} \aarrow v_2
  \and 
   v_1 \oplus_a v_2 = v
}{
 \config{\trace,  \aexpr_1 \oplus_a \aexpr_2} 
 \aarrow v
}
\and
\inferrule{ 
  \config{\trace, \aexpr} \aarrow v'
  \and 
  \elog v' = v
}{
 \config{\trace,  \elog \aexpr} 
 \aarrow v
}
\and
\inferrule{ 
  \config{\trace, \aexpr} \aarrow v'
  \and 
  \esign v' = v
}{
 \config{\trace,  \esign \aexpr} 
 \aarrow v
}
\\
\boxed{ \config{\trace, \bexpr} \barrow v \, : \, \mbox{Trace $\times$ Boolean Expr $\Rightarrow$ Boolean Value} }
\\% \\
\inferrule{ 
  \empty
}{
 \config{\trace,  \efalse} 
 \barrow \efalse
}
\and 
\inferrule{ 
  \empty
}{
 \config{\trace,  \etrue} 
 \barrow \etrue
}
\and 
\inferrule{ 
  \config{\trace, \bexpr} \barrow v'
  \and 
  \neg v' = v
}{
 \config{\trace,  \neg \bexpr} 
 \barrow v
}
\and 
\inferrule{ 
  \config{\trace, \bexpr_1} \barrow v_1
  \and 
  \config{\trace, \bexpr_2} \barrow v_2
  \and 
   v_1 \oplus_b v_2 = v
}{
 \config{\trace,  \bexpr_1 \oplus_b \bexpr_2} 
 \barrow v
}
\and 
\inferrule{ 
  \config{\trace, \aexpr_1} \aarrow v_1
  \and 
  \config{\trace, \aexpr_2} \aarrow v_2
  \and 
   v_1 \sim v_2 = v
}{
 \config{\trace,  \aexpr_1 \sim \aexpr_2} 
 \barrow v
}
\\
\boxed{ \config{\trace, \expr} \earrow v \, : \, \mbox{Trace $\times$ Expression $\Rightarrow$ Value} }
\\
\inferrule{ 
  \config{\trace, \aexpr} \aarrow v
}{
 \config{\trace,  \aexpr} 
 \earrow v
}
\and
\inferrule{ 
  \config{\trace, \bexpr} \barrow v
}{
 \config{\trace,  \bexpr} 
 \earrow v
}
\and
\inferrule{ 
  \config{\trace, \expr_1} \earrow v_1
  \cdots
  \config{\trace, \expr_n} \earrow v_n
}{
 \config{\trace,  [\expr_1, \cdots, \expr_n]} 
 \earrow [v_1, \cdots, v_n]
}
\and
\inferrule{ 
  \empty
}{
 \config{\trace,  v} 
 \earrow v
}
% \\
% \boxed{ \config{\trace, \qexpr} \qarrow \qval \, : \, \mbox{Trace  $\times$ Query Expr $\Rightarrow$ Query Value} }
% \\
% \inferrule{ 
%   \config{\trace, \aexpr} \aarrow n
% }{
%  \config{\trace,  \aexpr} 
%  \qarrow n
% }
% \and
% \inferrule{ 
%   \config{\trace, \qexpr_1} \qarrow \qval_1
%   \and
%   \config{\trace, \qexpr_2} \qarrow \qval_2
% }{
%  \config{\trace,  \qexpr_1 \oplus_a \qexpr_2} 
%  \qarrow \qval_1 \oplus_a \qval_2
% }
% \and
% \inferrule{ 
%   \config{\trace, \aexpr} \aarrow n
% }{
%  \config{\trace, \chi[\aexpr]} \qarrow \chi[n]
% }
% \and
% \inferrule{ 
%   \empty
% }{
%  \config{\trace,  \qval} 
%  \qarrow \qval
% }
 \end{mathpar}
%
The trace based operational semantics rules are defined in Figure \ref{fig:os}.
%
\begin{figure}
%   \text{\mg{Several skip are missing labels. Do we need fresh labels or we reuse l?}}
%   \\
%   \text{\jl{Both are good for OS, but generate fresh label will need extra arguments in soundness proof, so rescuing l is better}}
%   \\
% \text{\mg{Also, why we use ++, cannot we just define lists as adding elements on the right?}}  \\
% \text{\jl{I was too sticky to the convention, it is a good idea to append to the left and just use $::$ as construtor}}  \\
% \text{\mg{It is also unclear why we store the boolean expression in if and while, besides the boolean value.}}\\
% \text{\jl{When proving the soundness of dependency between trace-based and program-based,}}\\
% \text{\jl{The variable used in the boolean expression is useful in proving the inversion Lemmas.}}
{
\begin{mathpar}
\boxed{
\mbox{Command $\times$ Trace}
\xrightarrow{}
\mbox{Command $\times$ Trace}
}
\and
\boxed{\config{{c, \trace}}
\xrightarrow{} 
\config{{c',  \trace'}}
}
\\
\inferrule
{
\empty
}
{
\config{\clabel{\eskip}^l,  \trace } 
\xrightarrow{} 
\config{\clabel{\eskip}^l, \trace}
}
~\textbf{skip}
%
\and
%
\inferrule
{
\event = ({x}, l, v, \bullet)
}
{
\config{[\assign{{x}}{\aexpr}]^{l},  \trace } 
\xrightarrow{} 
\config{\clabel{\eskip}^l, \trace \traceadd \event}
}
~\textbf{assn}
%
% \and
% %
% {
% \inferrule
% {
%  \config{\trace, \qexpr }\qarrow \qval
%  \and 
% \query(\qval) = v
% \and 
% \event = ({x}, l, v, \qval)
% }
% {
% \config{{[\assign{x}{\query(\qexpr)}]^l, \trace}}
% \xrightarrow{} 
% \config{{\clabel{\eskip}^l,  \trace \traceadd \event} }
% }
% ~\textbf{query}
% }
%
\and
%
\inferrule
{
  \config{\trace, b} \barrow \etrue
 \and 
 \event = (b, l, \etrue, \bullet)
}
{
\config{{\ewhile [b]^{l} \edo c, \trace}}
\xrightarrow{} 
\config{{
c; \ewhile [b]^{l} \edo c,
\trace \traceadd \event}}
}
~\textbf{while-t}
%
%
\and
%
\inferrule
{
  \config{\trace, b} \barrow \efalse
 \and 
 \event = (b, l, \efalse, \bullet)
}
{
\config{{\ewhile [b]^{l}, \edo c, \trace}}
\xrightarrow{} 
\config{{
  \clabel{\eskip}^l,
\trace \traceadd \event}}
}
~\textbf{while-f}
%
%
\and
%
%
\inferrule
{
\config{{c_1, \trace}}
\xrightarrow{}
\config{{c_1',  \trace'}}
}
{
\config{{c_1; c_2, \trace}} 
\xrightarrow{} 
\config{{c_1'; c_2, \trace'}}
}
~\textbf{seq1}
%
\and
%
\inferrule
{
  \config{{c_2, \trace}}
  \xrightarrow{}
  \config{{c_2',  \trace'}}
}
{
\config{{\clabel{\eskip}^l; c_2, \trace}} \xrightarrow{} \config{{ c_2', \trace'}}
}
~\textbf{seq2}
%
\and
%
%
\inferrule
{
  \config{\trace, b} \barrow \etrue
 \and 
 \event = (b, l, \etrue, \bullet)
}
{
 \config{{
\eif([b]^{l}, c_1, c_2), 
\trace}}
\xrightarrow{} 
\config{{c_1, \trace \traceadd \event}}
}
~\textbf{if-t}
%
\and
%
\inferrule
{
 \config{\trace, b} \barrow \efalse
 \and 
 \event = (b, l, \efalse, \bullet)
}
{
\config{{\eif([b]^{l}, c_1, c_2), \trace}}
\xrightarrow{} 
\config{{c_2, \trace \traceadd \event}}
}
~\textbf{if-f}
% %
\and
%
\highlight{
\inferrule
{
 c' = (c)^{+n}
 \and 
 \event = (f, l, (r, x_1, \ldots, x_n) := c', \bullet)
}
{
\config{{
  [\efun]^l: f(r, x_1, \ldots, x_n) := c, \trace}}
\xrightarrow{} 
\config{{\clabel{\eskip}^l, \trace \traceadd \event}}
}
~\textbf{fun-def}
%
}
\\
\highlight{
%
\inferrule
{
  \config{ \trace, f} \earrow (r, x_1, \ldots, x_n) := c
\and 
\config{{
  \clabel{\assign{x_1}{e_1}}^{(l, 1)}; \ldots;
  \clabel{\assign{x_n}{e_n}}^{(l, n)}, \trace}} 
  \xrightarrow{}^* 
  \config{{\clabel{\eskip}^{(l, n)}, \trace_1}}
  \\ 
  \config{{\clabel{c}^{(l)}, \trace_1}}
  \xrightarrow{}^* 
  \config{{\clabel{\eskip}^{l}, \trace'}}
  \and
  \config{\trace', r } \earrow v
  \and
 \event = (x, l, v, \bullet)
}
{
\config{{
  \clabel{\assign{x}{\ecall(f, e_1, \ldots, e_n)}}^l, \trace}}
\xrightarrow{} 
\config{{\clabel{\eskip}^l, \trace' :: \event}}
}
~\textbf{fun-call}
}
%
\end{mathpar}
}
% \end{subfigure}
    \caption{Trace-based Operational Semantics for Language.}
    \label{fig:os}
\end{figure}
%
\\

%
%
% \subsection{Event and Trace}
% \label{subsec:event_trace}

%
\clearpage
