\subsection{Labeled Language}
\[
\begin{array}{llll}
\mbox{Arithmetic Operators} 
& \oplus_a & ::= & + ~|~ - ~|~ \times 
%
~|~ \div ~|~ \emax ~|~ \emin
\\  
\mbox{Arithmetic Expression} 
& \aexpr & ::= & 
n ~|~ {x} ~|~ \aexpr \oplus_a \aexpr  
 ~|~ \elog \aexpr  ~|~ \esign \aexpr
\\
\mbox{Boolean Expression} & \bexpr & ::= & 
%
\etrue ~|~ \efalse  ~|~ \neg \bexpr
 ~|~ \bexpr \land \bexpr
%
~|~ \bexpr \lor \bexpr
~|~ \aexpr \leq \aexpr 
~|~ \aexpr < \aexpr 
~|~ \aexpr = \aexpr 
\\
\mbox{Expression} & \expr & ::= & v ~|~ \aexpr ~|~ \bexpr ~|~ [\expr, \dots, \expr]
\\  
%
\mbox{Value} 
& v & ::= & { n ~|~ \etrue ~|~ \efalse ~|~ [] ~|~ [v, \dots, v]} \\
%
% \\%
\mbox{Label} 
& l & \in & (\mathbb{N} \cup \{\lin, \lex\}) 
% ~|~ (l, n)
\\ 
%
\mbox{Labeled Command} 
& {c} & ::= &  
\clabel{\assign{x}{\expr}}^l 
~|~  \clabel{\eskip}^l
~|~ \ewhile \clabel{\bexpr}^{l} \edo {c}
~|~ \eif(\clabel{\bexpr}^{l} , {c}, {c}) 
~|~ {c};{c}  
\\ 
\mbox{Event} 
& \event & ::= & 
({x}, l, v) ~ \mbox{Assignment Event} 
% \\
% &&& 
~|~(\bexpr, l, v) ~ \mbox{Testing Event}
\\
\mbox{Trace} & \trace
& ::= & [] ~|~ \trace :: \event
\\
\end{array}
\]
We use following notations to represent the sets of corresponding terms:
\[
\begin{array}{lll}
\mathcal{VAR} & : & \mbox{Set of Variables}  
\\ 
%
\cdom & : & \mbox{Set of Commands} 
\\ 
%
\eventset  & : & \mbox{Set of Events}  
\\
%
\eventset^{\asn}  & : & \mbox{Set of Assignment Events}  
\\
%
\eventset^{\test}  & : & \mbox{Set of Testing Events}  
\\
%
\ldom  & : & \mbox{Set of Labels}  
\\
%
\highlight{\tdom} & : & \mbox{\highlight{Set of All Finite Execution Traces}}
\\
\highlight{\tdom^{\infty}} & : & \mbox{\highlight{Set of All Finite Or Infinite  Execution Traces}}
\\
%
\tdom_0(c) & : & \mbox{Set of Initial Traces, which is finite and all the input variables of the program $c$ are initialized.
}
\end{array}
\]
$FV: \expr \to \mathcal{P}(\mathcal{VAR})$, computes the set of free variables in an expression. To be precise,
$FV(\aexpr)$ and $FV(\bexpr)$
represent the sets of free variables in arithmetic
expression $\aexpr$ and boolean expression $\bexpr$
respectively.
%
\subsection{{Trace-based Operational Semantics}}
\label{sec:operational_semantics}
\paragraph*{Event}
An event is a triple.
An \emph{assignment event} corresponds to the evaluation of an assignment command $\clabel{\assign{x}{\expr}}^l$.
Its first element is the variable name $x$,
or a boolean expression (from the guard of if or while command), 
following by 
 the label, $l$ associated to this command and the value assigned to the variable.
 \\
 A \emph{testing event} corresponds to the evaluation of the boolean expression $b$ in the guard of a $\eif(\clabel{b}^l, c_1, c_2)$ command or $\ewhile \clabel{b}^l \edo c$.
 Its first element is the boolean expression $b$, following by 
 the label $l$ associated to the guard and the evaluated boolean result.
%
\[
\begin{array}{llll}
  \mbox{Event} 
  & \event & ::= & 
  ({x}, l, v) ~ \mbox{Assignment Event} 
  ~|~(\bexpr, l, v) ~ \mbox{Testing Event}
\end{array}
\]
Event projection operators $\pi_i$ projects the $i$th element from an event: 
\\
$\pi_i : 
\eventset \to \mathcal{VAR}\cup \mbox{Boolean Expression}  \cup \mathbb{N} $
% \subsubsection{Trace}
\paragraph*{Trace}
%
A trace $\trace \in \tdom $ is a list of events, 
collecting the events generated along the program execution. 
\[
\begin{array}{llll}
\mbox{Trace} & \trace
& ::= & [] ~|~ \trace :: \event
\end{array}
\]
A trace can be regarded as the program history, 
which records all the evaluations for assignment commands and guards in $\eif$ and $\ewhile$ command.
\\
\highlight{If a program doesn't terminate when executing under some initial trace, it produces an infinite trace $\trace \in \tdom^{\infty}$.
$\tdom^{\infty}$ is the set of all finite or infinite traces.}
\\
$\tracecat: \mathcal{T} \to \mathcal{T} \to \mathcal{T}$ is the trace concatenation operator, which combines two traces,
and $\vcounter : \mathcal{T} \to \mathbb{N} \to \mathbb{N}$ is the counting operator, 
which counts the occurrence of a labeled variable in the trace. When the input trace is infinite, it returns $\infty$.
$\event \in \trace $ or $\event \notin \trace $ denotes that $\event$ belongs to $trace$ or not.
All the definition details are in the appendix.
%
The counter operator is abused when the input is a sequence of labels $L = (l_1, \cdots, l_n)$ by counting the occurrence
of this sequence in trace. Specifically,
$\vcounter(\trace :: (\_, l_1, \_) :: \cdots :: (\_, l_n, \_), L ) \triangleq \vcounter(\trace, L) + 1$
and $\vcounter(\trace :: (\_, l, \_), L ) 
\triangleq \vcounter(\trace, L) ~ l \neq l_n$, etc.
The operator $\tlabel : \tdom \to \mathcal{P}{(\ldom)}$ gives the set of labels in every event belonging to a trace.
$\tlabel{(\trace  :: (\_, l, \_)])} \triangleq \{l\} \cup \tlabel{(\trace )}$ and $\tlabel({[ ]}) \triangleq \{\}$.
%
\paragraph{Environment} $ \env : {\tdom}  \to \mathcal{VAR} \to( \mathbb{N} \cup \{\bot\})$
\[
\begin{array}{llll}
\env(\trace  \traceadd (x, l, v)) x \triangleq v
&
\env(\trace \traceadd (y, l, v)) x \triangleq \env(\trace) x, y \neq x
&
\env(\trace \traceadd (b, l, v)) x \triangleq \env(\trace) x
&
\env({[]} ) x \triangleq \bot
\end{array}
\]
\paragraph{Arithmetic Expression Semantics}
\begin{mathpar}
  \boxed{ \econfig{\aexpr}(\trace) = v \, : \, \mbox{Trace  $\times$ Arithmetic Expr $\Rightarrow$ Arithmetic Value} }
  \\
  \inferrule{ 
    \empty
  }{
   \econfig{n} (\trace)
   = n
  }
  \and
  \inferrule{ 
    \env(\trace) x = v
  }{
   \econfig{x} 
   = v
  }
  \and
  \inferrule{ 
    \econfig{\aexpr_1}(\trace) = v_1
    \and 
    \econfig{\aexpr_2}(\trace) = v_2
    \and 
     v_1 \oplus_a v_2 = v
  }{
   \econfig{\aexpr_1 \oplus_a \aexpr_2} 
   = v
  }
  \and
  \inferrule{ 
    \econfig{\aexpr}(\trace) = v'
    \and 
    \elog v' = v
  }{
   \econfig{\elog \aexpr}(\trace) 
   = v
  }
  \and
  \inferrule{ 
    \econfig{\aexpr}(\trace) = v'
    \and 
    \esign v' = v
  }{
   \econfig{\esign \aexpr} 
   = v
  }
   \end{mathpar}
\paragraph{Operational Semantics Rules}
The operational semantics rules for expression evaluation and command execution is presented below.

\highlight{
  Given an initial trace $\trace_0 \in \tdom_0(c)$ w.r.t. a program $c$,
we use $\to^*$ to represent the multiple-step execution. $\config{c, \trace_0} \to^{*} \config{\eskip, \trace_0 \tracecat \trace}$
represents that the program's execution terminates and produces a finite execution trace $\trace \in \tdom$.
When the program execution doesn't terminate under $\trace_0$, 
we use $\config{c, \trace_0} \to^{\infty} \config{\cdot, \trace_0 \tracecat \trace}$ to
represent the non-terminated execution w.r.t. the infinite trace $\trace_0 \in \tdominf$.
\\
If we observe the operational semantics rules, we can find that no rule will shrink the trace. 
So we have the Lemma~\ref{lem:tracenondec} with proof in Appendix~\ref{apdx:lemma_sec123}, 
specifically the trace has the property that its length never decreases during the program execution.
\begin{lem}
[Trace Non-Decreasing]
\label{lem:tracenondec}
For any program $c \in \cdom$ and traces $\trace_0 \in \tdom_0(c), \trace \in \tdominf$, if 
$\config{c, \trace_0} \rightarrow^{*} \config{\eskip, \trace} $ or 
$\config{c, \trace_0} \rightarrow^{\infty} \config{\cdot, \trace}$,
then there exists a trace $\trace' \in \tdominf$ with $\trace_0 \tracecat \trace' = \trace$
%
$$
\forall \trace_0 \in \tdom_0(c), \trace \in \tdominf, c \st
\Big(
  \config{c, \trace_0} \rightarrow^{*} \config{\eskip, \trace} 
  \lor 
  \config{c, \trace_0} \rightarrow^{\infty} \config{\cdot, \trace}
\Big)
\implies \exists \trace' \in \tdominf \st \trace_0 \tracecat \trace' = \trace
$$
\end{lem}
}

{
\begin{mathpar}
\boxed{ \config{\trace, \bexpr} \barrow v \, : \, \mbox{Trace $\times$ Boolean Expr $\Rightarrow$ Boolean Value} }
\\% \\
\inferrule{ 
  \empty
}{
 \config{\trace,  \efalse} 
 \barrow \efalse
}
\and 
\inferrule{ 
  \empty
}{
 \config{\trace,  \etrue} 
 \barrow \etrue
}
\and 
\inferrule{ 
  \config{\trace, \bexpr} \barrow v'
  \and 
  \neg v' = v
}{
 \config{\trace,  \neg \bexpr} 
 \barrow v
}
\and 
\inferrule{ 
  \config{\trace, \bexpr_1} \barrow v_1
  \and 
  \config{\trace, \bexpr_2} \barrow v_2
  \and 
   v_1 \land v_2 = v
}{
 \config{\trace,  \bexpr_1 \land \bexpr_2} 
 \barrow v
}
\and 
\inferrule{ 
  \config{\trace, \bexpr_1} \barrow v_1
  \and 
  \config{\trace, \bexpr_2} \barrow v_2
  \and 
   v_1 \lor v_2 = v
}{
 \config{\trace,  \bexpr_1 \lor \bexpr_2} 
 \barrow v
}
\and 
\inferrule{ 
  \config{\trace, \aexpr_1} \aarrow v_1
  \and 
  \config{\trace, \aexpr_2} \aarrow v_2
  \and 
   v_1 \leq v_2 = v
}{
 \config{\trace,  \aexpr_1 \leq \aexpr_2} 
 \barrow v
}
\and 
\inferrule{ 
  \config{\trace, \aexpr_1} \aarrow v_1
  \and 
  \config{\trace, \aexpr_2} \aarrow v_2
  \and 
   v_1 < v_2 = v
}{
 \config{\trace,  \aexpr_1 < \aexpr_2} 
 \barrow v
}
\and 
\inferrule{ 
  \config{\trace, \aexpr_1} \aarrow v_1
  \and 
  \config{\trace, \aexpr_2} \aarrow v_2
  \and 
   v_1 = v_2 = v
}{
 \config{\trace,  \aexpr_1 = \aexpr_2} 
 \barrow v
}
\\
\boxed{ \config{\trace, \expr} \earrow v \, : \, \mbox{Trace $\times$ Expression $\Rightarrow$ Value} }
\\
\inferrule{ 
  \econfig{\aexpr}(\trace) = v
}{
 \config{\trace,  \aexpr} 
 \earrow v
}
\and
\inferrule{ 
  \config{\trace, \bexpr} \barrow v
}{
 \config{\trace,  \bexpr} 
 \earrow v
}
\and
\inferrule{ 
  \config{\trace, \expr_1} \earrow v_1
  \cdots
  \config{\trace, \expr_n} \earrow v_n
}{
 \config{\trace,  [\expr_1, \cdots, \expr_n]} 
 \earrow [v_1, \cdots, v_n]
}
\and
\inferrule{ 
  \empty
}{
 \config{\trace,  v} 
 \earrow v
}
 \end{mathpar}
%
The trace based operational semantics rules are defined as follows,
\begin{mathpar}
\boxed{
\mbox{Command $\times$ Trace}
\xrightarrow{}
\mbox{Command $\times$ Trace}
}
\and
\boxed{\config{{c, \trace}}
\xrightarrow{} 
\config{{c',  \trace'}}
}
\\
\inferrule
{
\empty
}
{
\config{\clabel{\eskip}^l,  \trace } 
\xrightarrow{} 
\config{\clabel{\eskip}^l, \trace}
}
~\textbf{skip}
%
\and
%
\inferrule
{
  \config{\expr, \trace} \earrow v
  \and
\event = ({x}, l, v)
}
{
\config{[\assign{{x}}{\expr}]^{l},  \trace } 
\xrightarrow{} 
\config{\clabel{\eskip}^l, \trace \traceadd \event}
}
~\textbf{assn}
\and
%
\inferrule
{
  \config{\trace, b} \earrow \etrue
 \and 
 \event = (b, l, \etrue)
}
{
\config{{\ewhile [b]^{l} \edo c, \trace}}
\xrightarrow{} 
\config{{
c; \ewhile [b]^{l} \edo c,
\trace \traceadd \event}}
}
~\textbf{while-t}
%
%
\and
%
\inferrule
{
  \config{\trace, b} \earrow \efalse
 \and 
 \event = (b, l, \efalse)
}
{
\config{{\ewhile [b]^{l}, \edo c, \trace}}
\xrightarrow{} 
\config{{
  \clabel{\eskip}^l,
\trace \traceadd \event}}
}
~\textbf{while-f}
%
%
\and
%
%
\inferrule
{
\config{{c_1, \trace}}
\xrightarrow{}
\config{{c_1',  \trace'}}
}
{
\config{{c_1; c_2, \trace}} 
\xrightarrow{} 
\config{{c_1'; c_2, \trace'}}
}
~\textbf{seq1}
%
\and
%
\inferrule
{
  \config{{c_2, \trace}}
  \xrightarrow{}
  \config{{c_2',  \trace'}}
}
{
\config{{\clabel{\eskip}^l; c_2, \trace}} \xrightarrow{} \config{{ c_2', \trace'}}
}
~\textbf{seq2}
%
\and
%
%
\inferrule
{
  \config{\trace, b} \earrow \etrue
 \and 
 \event = (b, l, \etrue)
}
{
\config{{
\eif([b]^{l}, c_1, c_2), 
\trace}}
\xrightarrow{} 
\config{{c_1, \trace \traceadd \event}}
}
~\textbf{if-t}
%
\and
%
\inferrule
{
 \config{\trace, b} \earrow \efalse
 \and 
 \event = (b, l, \efalse)
}
{
\config{{\eif([b]^{l}, c_1, c_2), \trace}}
\xrightarrow{} 
\config{{c_2, \trace \traceadd \event}}
}
~\textbf{if-f}
%
\end{mathpar}
}