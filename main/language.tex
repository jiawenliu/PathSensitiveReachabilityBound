\subsection{Labeled Language}
\[
\begin{array}{llll}
\mbox{Arithmetic Operators} 
& \oplus_a & ::= & + ~|~ - ~|~ \times 
%
~|~ \div ~|~ \emax ~|~ \emin
\\  
\mbox{Arithmetic Expression} 
& \aexpr & ::= & 
n ~|~ {x} ~|~ \aexpr \oplus_a \aexpr  
 ~|~ \elog \aexpr  ~|~ \esign \aexpr
\\
\mbox{Boolean Expression} & \bexpr & ::= & 
%
\etrue ~|~ \efalse  ~|~ \neg \bexpr
 ~|~ \bexpr \land \bexpr
%
~|~ \bexpr \lor \bexpr
~|~ \aexpr \leq \aexpr 
~|~ \aexpr < \aexpr 
~|~ \aexpr = \aexpr 
\\
\mbox{Expression} & \expr & ::= & v ~|~ \aexpr ~|~ \bexpr ~|~ [\expr, \dots, \expr]
\\  
%
\mbox{Value} 
& v & ::= & { n \in \mathbb{N}^{\infty} ~|~ \etrue ~|~ \efalse ~|~ [] ~|~ [v, \dots, v]} \\
%
% \\%
\mbox{Label} 
& l & \in & (\mathbb{N} \cup \{\lin, \lex\}) 
\\ 
%
\mbox{Labeled Command} 
& {c} & ::= &  
\clabel{\assign{x}{\expr}}^l 
~|~  \clabel{\eskip}^l
~|~ \ewhile \clabel{\bexpr}^{l} \edo ({c})
~|~ \eif(\clabel{\bexpr}^{l} , {c}, {c}) 
~|~ {c};{c}  
\\ 
\mbox{Event} 
& \event & ::= & 
({x}, l, v) ~ \mbox{Assignment Event} 
% \\
% &&& 
~|~(\bexpr, l, v) ~ \mbox{Testing Event}
\\
\mbox{Trace} & \trace
& ::= & [] ~|~ \trace :: \event
\\
\end{array}
\]
We denote by $\infty$ a value s.t. $n < \infty $ for all $n \in \mathbb{N}$.
We use following notations to represent the sets of corresponding terms:
\[
\begin{array}{lll}
\vardom & : & \mbox{Set of Variables}  
\\ 
%
\booldom & : & \mbox{Set of Boolean Expressions}  
\\ 
%
\cdom & : & \mbox{Set of Commands} 
\\ 
%
\eventset  & : & \mbox{Set of Events}  
\\
%
\eventset^{\asn}  & : & \mbox{Set of Assignment Events}  
\\
%
\eventset^{\test}  & : & \mbox{Set of Testing Events}  
\\
%
\ldom  & : & \mbox{Set of Labels}  
\\
%
\highlight{\ftdom} & : & \mbox{\highlight{Set of All Finite Execution Traces}}
\\
\highlight{\inftdom} & : & \mbox{\highlight{Set of Infinite  Execution Traces}}
\\
\highlight{\tdom} & : & \mbox{\highlight{Set of All Finite Or Infinite  Execution Traces}}
\\ 
%
\inpvar(c) & : & \mbox{Set of Program $c$'s Input Variables}  
\\
%
\ftdom_0(c) & : & \mbox{Set of Program $c$'s Initial Traces.}
\\ & & \mbox{Each initial trace $\trace_0 \in \ftdom_0(c)$ is finite and every input variable of the program $c$ has an initial value in $\trace_0$.}
\end{array}
\]
%
\subsection{{Trace-based Operational Semantics}}
\label{sec:operational_semantics}
\paragraph{Event}
An event is a triple.
Its first element is the variable name $x$,
or a boolean expression (from the guard of if or while command), 
following by 
 the label, $l$ associated to this command and the value assigned to the variable.

 We have two kinds of events: \emph{assignment events} and \emph{testing events},
 and we use $\eventset^{\asn}$ and $\eventset^{\test}$ to denote the set of all assignment events and testing events, respectively.

 An \emph{assignment event} tracks the execution of an assignment and consists of the assigned variable, the label of the command that generates it, the value assigned to the variable.

 A \emph{testing event} tracks the execution of if and while commands, specifically the evaluation of the boolean expression $b$ in the guard of a $\eif(\clabel{b}^l, c_1, c_2)$ command or $\ewhile \clabel{b}^l \edo c$.
 It consists of the boolean expression $b$ in the guard of the command, the label of the guard, the result of evaluating the guard.
%
\[
\begin{array}{llll}
  \mbox{Event} 
  & \event & ::= & 
  ({x}, l, v) ~ \mbox{Assignment Event} 
  ~|~(\bexpr, l, v) ~ \mbox{Testing Event}
\end{array}
\]
Event projection operators $\pi_i$ projects the $i^{th}$ element from an event: 
% \\
$\pi_i : 
\eventset \to \vardom \cup \booldom \cup \ldom $

\paragraph{Trace.}
%
A trace $\trace \in \tdom$ is a list of events, 
collecting the events generated during a specific program execution. 
\[
\begin{array}{llll}
\mbox{Trace} & \trace
& ::= & [] ~|~ \trace :: \event 
% ~|~ []^{\infty}
\end{array}
\]
A trace can be regarded as the program history, 
which records all the evaluations for assignment commands and guards in $\eif$ and $\ewhile$ command.
\\
\highlight{
A trace can be finite ($\trace \in \ftdom$) or infinite $\trace \in \inftdom$.
If a program doesn't terminate when executing under some initial trace,
it produces the infinite trace 
from $\inftdom$, which records a non-terminating computation.
So we denote by $\tdom$ the set of all traces, and $\tdom = \ftdom \cup \inftdom$.
The trace-based semantics with non-terminating execution is defined below following the maximal trace semantics in \cite{Cousot19}.}

We use list notation for traces, where $[]$ is the empty trace, the operator $\traceadd$ combines an event and a trace in a new event, 
and the operator $\tracecat$ concatenates two traces formally defined as follows. 

\begin{defn}[Trace Concatenation, $\tracecat: \tdom \to \tdom\to \tdom $]
  \label{def:trace_concate}
Given two traces $\trace_1 \in \tdom, \trace_2 \in \tdom$, the trace concatenation operator 
$\tracecat$ is defined as:
\[
\trace_1 \tracecat \trace_2 \triangleq
\left\{
\begin{array}{ll} 
  \trace_1 & \trace_2 = [] \lor \trace_1 \in \inftdom \\
  \trace_2 & \trace_1 = [] \lor \trace_2 \in \inftdom \\
  (\trace_1   \tracecat \trace_2'):: \event & \trace_1 \in \ftdom \land \trace_2 = \trace_2' :: \event
  % \trace_2 &  \trace_2 \in \inftdom \\
\end{array}
\right.
\]
\end{defn}

\begin{defn}(An Event Belongs to A Trace)
  An event $\event \in \eventset$ belongs to a trace $\trace \in \tdom$, i.e., $\event \in \trace$ are defined as follows:
%
\begin{equation*}
  \event \in \trace  
  \triangleq \left\{
  \begin{array}{ll} 
    \etrue                  & \trace =  [\event] \tracecat \trace'
     \land \event = \event' \\
    \event \in \trace' & \trace =  [\event'] \tracecat \trace'
      \land \event \neq \event' \\ 
    \efalse                 & \trace = [] \lor \trace \in \inftdom
  \end{array}
  \right.
\end{equation*}
As usual, we denote by $\event \notin \trace$ that the event $\event$ doesn't belong to the trace $\trace$.
\end{defn}
%
In the rest of the paper, we denote by $\bot$ a value s.t. $\bot < n $ for all $n \in \mathbb{N}$.
\begin{defn}[Counter Notation for Program Point]
  \label{def:counter}
The counting operator $\counter : \tdom \to \ldom \to (\mathbb{N} \cup \{\bot, \infty\})$
counts the appearance of a label in a trace.
\[
\begin{array}{llll}
\counter([(x, l, v)] \tracecat \trace', l ) \triangleq \counter(\trace', l) + 1 & \text{if}~ l = l
&
\counter([(b, l, v)] \tracecat \trace', l) \triangleq \counter(\trace', l) + 1 & \text{if}~ l = l
\\
\counter([(x, l', v)] \tracecat \trace', l) \triangleq \counter(\trace', l)   & \text{if}~ l' \neq l
&
\counter([(b, l', v)] \tracecat \trace', l) \triangleq \counter(\trace', l)   & \text{if}~ l' \neq l
\\
\counter({[]}, l) \triangleq 0 & 
&
\counter(\trace, l) \triangleq \bot & \text{if }~ \trace \in \inftdom
\end{array}
\]
{When the input trace is infinite, $\counter(\trace, l)$ returns $\bot$ for any program label $l$.}
\end{defn}
\begin{defn}[Counter Notation for List of Program Point]
  \label{def:lcounter}
  The counting operator $\lcounter : \tdom \to \mathcal{P}(\ldom) \to (\mathbb{N} \cup \{\infty\})$
  counts the appearance of a list of labels $[l_1, \ldots, l_n]$ as follows.
\[
  \begin{array}{ll}
  \lcounter(\trace'' \tracecat \trace', [l_1, \ldots, l_n] ) 
  \triangleq \lcounter(\trace', [l_1, \ldots, l_n]) + 1  & \text{if}~ \pi_2(\trace''[i]) = l_i, \forall i = 1, \ldots, n
  \\ 
  \lcounter([(\_, l, \_)] \tracecat \trace', [l_1, \ldots, l_n] ) 
  \triangleq \lcounter(\trace', [l_1, \ldots, l_n]) & \text{if}~ l \neq l_1
  \\ 
  \lcounter(\trace, [l_1, \ldots, l_n] ) 
  \triangleq \bot & \text{if }~ \trace \in \inftdom
\end{array}
\]
{When the input trace is infinite, $\lcounter(\trace, L)$ returns $\bot$ for any list of labels as well.}
\end{defn}
%
We define the operator $\tracel : \tdom \to \mathcal{P}{(\ldom)}$ projects the label from every event in a trace as a list of program points,
defined as follows,
\[
\tracel([(\_, l, \_)] \tracecat \trace') \triangleq [l] \tracecat \tracel(\trace')
\qquad
\tracel([ ]) \triangleq []
\]
%
\paragraph{Environment.} $\env : {\ftdom}  \to \vardom \to(\mathbb{N} \cup \{\bot\})$
\[
\begin{array}{llll}
\env(\trace  \traceadd (x, l, v)) x \triangleq v
&
\env(\trace \traceadd (y, l, v)) x \triangleq \env(\trace) x, y \neq x
&
\env(\trace \traceadd (b, l, v)) x \triangleq \env(\trace) x
&
\env({[]} ) x \triangleq \bot
\end{array}
\]
%
\begin{lem}[Initial Traces]
  \label{lem:initial_trace}
  \[
    \forall c \in \cdom, \trace \in \ftdom \st \trace \in \ftdom_0(c) \iff 
    \forall x \in \inpvar(c) \st \env(\trace_0) x \neq \bot
    \]
\end{lem}
%
\paragraph{Configuration.}
%
\paragraph{Expression Semantics}
The evaluation notation for arithmetic expression is $\econfig{} : \mathcal{A} \to \tdom \to \mathcal{V}$.
The $\econfig{\aexpr}(\trace)$ evaluates an arithmetic expression $\aexpr$ under trace $\trace$ following the arithmetic expression evaluation rules in Figure~\ref{fig:aexpr-eval}.
\begin{figure}
\begin{mathpar}
  \boxed{ \econfig{} \, : \, \mbox{Arithmetic Expression $\to$ Trace $\to$ Arithmetic Value}}
  \\
  \inferrule{ 
    \empty
  }{
   \econfig{n} (\trace)
   = n
  }
  \and
  \inferrule{ 
    \env(\trace) x = v
  }{
   \econfig{x} 
   = v
  }
  \and
  \inferrule{ 
    \econfig{\aexpr_1}(\trace) = v_1
    \and 
    \econfig{\aexpr_2}(\trace) = v_2
    \and 
     v_1 \oplus_a v_2 = v
  }{
   \econfig{\aexpr_1 \oplus_a \aexpr_2} 
   = v
  }
  \and
  \inferrule{ 
    \econfig{\aexpr}(\trace) = v'
    \and 
    \elog v' = v
  }{
   \econfig{\elog \aexpr}(\trace) 
   = v
  }
  \and
  \inferrule{ 
    \econfig{\aexpr}(\trace) = v'
    \and 
    \esign v' = v
  }{
   \econfig{\esign \aexpr} 
   = v
  }
   \end{mathpar}
   \caption{Evaluation Rules of Arithmetic Expression}
   \label{fig:aexpr-eval}
   \end{figure}

 The evaluation rules for boolean expression and standard expression are in Figure~\ref{fig:bexpr-eval} and Figure~\ref{fig:expr-eval}.
 \begin{figure}
  \begin{mathpar}
  \boxed{ \barrow \, : \, \mbox{ Boolean Expression $\times$ Trace $\rightarrow$ Boolean Value} }
  \\
  \inferrule{ 
    \empty
  }{
   \config{\efalse, \trace} 
   \barrow \efalse
  }
  \and 
  \inferrule{ 
    \empty
  }{
   \config{\etrue, \trace} 
   \barrow \etrue
  }
  \and 
  \inferrule{ 
    \config{\bexpr, \trace} \barrow v'
    \and 
    \neg v' = v
  }{
   \config{\neg \bexpr, \trace} 
   \barrow v
  }
  \and 
  \inferrule{ 
    \config{\bexpr, \trace_1} \barrow v_1
    \and 
    \config{\bexpr, \trace_2} \barrow v_2
    \and 
     v_1 \land v_2 = v
  }{
   \config{\bexpr, \trace_1 \land \bexpr_2} 
   \barrow v
  }
  \and 
  \inferrule{ 
    \config{\bexpr, \trace_1} \barrow v_1
    \and 
    \config{\bexpr, \trace_2} \barrow v_2
    \and 
     v_1 \lor v_2 = v
  }{
   \config{\bexpr, \trace_1 \lor \bexpr_2} 
   \barrow v
  }
  \end{mathpar}
  \caption{Evaluation Rules of Boolean Expression}
  \label{fig:bexpr-eval}
  \end{figure}
  
  \begin{figure}
    \begin{mathpar}
  \boxed{ \earrow \, : \, \mbox{Expression $\times$ Trace $\rightarrow$ Value} }
  \\
  \inferrule{ 
    \econfig{\aexpr}(\trace) = v
  }{
   \config{\aexpr, \trace} 
   \earrow v
  }
  \and
  \inferrule{ 
    \config{\bexpr, \trace} \barrow v
  }{
   \config{\bexpr, \trace} 
   \earrow v
  }
  \and
  \inferrule{ 
    \config{\expr_1, \trace} \earrow v_1
    \cdots
    \config{\expr_n, \trace} \earrow v_n
  }{
   \config{ [\expr_1, \cdots, \expr_n], \trace} 
   \earrow [v_1, \cdots, v_n]
  }
  \and
  \inferrule{ 
    \empty
  }{
   \config{v, \trace} 
   \earrow v
  }
   \end{mathpar}
   \caption{Evaluation Rules of Standard Expression}
   \label{fig:expr-eval}
   \end{figure}

\paragraph{Operational Semantics Rules}
%
The trace based operational semantics rules are defined as in Figure~\ref{fig:command-os}.
\begin{figure}
  \begin{mathpar}
\boxed{
\mbox{Command $\times$ Trace}
\xrightarrow{}
\mbox{Command $\times$ Trace}
}
\and
\boxed{\config{{c, \trace}}
\xrightarrow{} 
\config{{c',  \trace'}}
}
%
\\
%
\inferrule
{
\config{\expr, \trace} \earrow v
  \and
\event = ({x}, l, v)
}
{
\config{\clabel{\assign{{x}}{\expr}}^{l},  \trace } 
\xrightarrow{} 
\config{\clabel{\eskip}^l, \trace \traceadd \event}
}
~\rname{assn}
\and
%
\inferrule
{
  \config{\bexpr, \trace} \earrow \etrue
 \and 
 \event = (\bexpr, l, \etrue)
}
{
\config{{\ewhile \clabel{\bexpr}^{l} \edo (c), \trace}}
\xrightarrow{} 
\config{{
c; \ewhile \clabel{\bexpr}^{l} \edo (c),
\trace \traceadd \event}}
}
~\rname{while-t}
%
%
\and
%
\inferrule
{
  \config{\bexpr, \trace} \earrow \efalse
 \and 
 \event = (\bexpr, l, \efalse)
}
{
\config{{\ewhile \clabel{\bexpr}^{l} \edo (c), \trace}}
\xrightarrow{} 
\config{{
  \clabel{\eskip}^l,
\trace \traceadd \event}}
}
~\rname{while-f}
%
%
\and
%
%
\inferrule
{
\config{{c_1, \trace}}
\xrightarrow{}
\config{{c_1',  \trace'}}
}
{
\config{{c_1; c_2, \trace}} 
\xrightarrow{} 
\config{{c_1'; c_2, \trace'}}
}
~\rname{seq1}
%
\and
%
\inferrule
{
  \config{{c_2, \trace}}
  \xrightarrow{}
  \config{{c_2',  \trace'}}
}
{
\config{{\clabel{\eskip}^l; c_2, \trace}} \xrightarrow{} \config{{ c_2', \trace'}}
}
~\rname{seq2}
%
\and
%
%
\inferrule
{
  \config{\bexpr, \trace} \earrow \etrue
 \and 
 \event = (\bexpr, l, \etrue)
}
{
\config{{
\eif(\clabel{\bexpr}^{l}, c_1, c_2), 
\trace}}
\xrightarrow{} 
\config{{c_1, \trace \traceadd \event}}
}
~\rname{if-t}
%
\and
%
\inferrule
{
 \config{\bexpr, \trace} \earrow \efalse
 \and 
 \event = (\bexpr, l, \efalse)
}
{
\config{{\eif(\clabel{\bexpr}^{l}, c_1, c_2), \trace}}
\xrightarrow{} 
\config{{c_2, \trace \traceadd \event}}
}
~\rname{if-f}
%
\end{mathpar}
\caption{Operational Semantics Rules}
\label{fig:command-os}
\end{figure}


Given an initial trace $\trace_0 \in \ftdom_0(c)$ of the program $c$,
we use $\to^*$ for the reflexive and transitive closure of $\to$. 
If $\config{c, \trace_0} \rightarrow^{*} \config{\clabel{\eskip}^l, \trace_0 \tracecat \trace}$,
then the program's execution terminates and produces a finite execution trace $\trace \in \ftdom$.
\\
\begin{defn}[Non-terminating and Infinite Trace]
  \label{def:non-terminating}
  Given a program $c$ and an initial trace $\trace \in \ftdom_0(c)$,
  when $c$ executes with $\trace$,  we define the execution of $c$ under $\trace$ is non-terminating and produces an infinite trace $\trace' \in \inftdom$, as 
  $\config{c, \trace_0} \uparrow^{\infty} \trace' \in \lim(\uparrow)$
  where the limit is defined as follows.
  \[
    \begin{array}{l}
      \lim(\uparrow) 
      % \in \left( (\cdom \times \ftdom) \times (\cdom \times \inftdom) \right) 
      \triangleq 
    \\ \quad
    \Big\{
      (\config{c, \trace}, \trace') ~\vert~ 
      c\in \cdom, \trace \in \ftdom_0(c),
      \trace' \in \inftdom 
      \land \exists \trace_0 \in \ftdom, c_0 \in \cdom \st 
      \config{c, \trace} \to \config{c_0, \trace_0}
      \\ \qquad \qquad \qquad 
      \land \forall i \in \mathbb{N}, \exists \trace_i, \trace_{i + 1} \in \ftdom, \trace'' \in \inftdom, c_i, c_{i + 1} \in \cdom \st 
      \config{c_i, \trace_i} \to \config{c_{i + 1}, \trace_{i + 1}} 
      \land  \trace' = \trace_{i + 1} \tracecat \trace''
    \Big\}
    \end{array}
  \]
\end{defn}
%
% \begin{defn}[Non-terminating and Infinite Trace (alternative way)]
%   \label{def:non-terminating-2}
%   Given a program $c$ and an initial trace $\trace_0 \in \ftdom_0(c)$,
%   when $c$ executes with $\trace_0$,  we define $c$ is non-terminating under $\trace_0$, $\config{c, \trace_0} \uparrow^{\infty}$ if and only if there exists a function
%   $f : \mathbb{N} \to \cdom \times \tdom$ such that $f(0) = \config{c, \trace_0}$ and
%   for every $i \in \mathbb{N}$ there exist  $\trace_i, \trace_{i + 1}\in \tdom$, $c_i, c_{i + 1} \in \cdom$ such that  $f(i) = \config{c_i, \trace_i}$, $f(i + 1) =  \config{c_{i + 1}, \trace_{i + 1}}$ and
%   $\config{c_i, \trace_i} \to \config{c_{i + 1}, \trace_{i + 1}}$. 
%   \[
%     \begin{array}{l}
%     \forall \trace_0 \in \ftdom_0(c), c \in \cdom \st
%     \config{c, \trace_0} \uparrow^{\infty}
%     \\
%     \iff \exists f : \mathbb{N} \to \cdom \times \tdom \st 
%     f(0) = \config{c, \trace_0}
%     \\ \qquad \land
%     \forall i \in \mathbb{N}, \exists \trace_i, \trace_{i + 1} \in \tdom, c_i, c_{i + 1} \in \cdom\st 
%     \\ \qquad \quad
%     f(i) = \config{c_i, \trace_i}$, $f(i + 1) =  \config{c_{i + 1}, \trace_{i + 1}} \land \config{c_i, \trace_i} \to \config{c_{i + 1}, \trace_{i + 1}}
%     \end{array}
%   \]
%   Given a program $c$ and an initial trace $\trace_0 \in \ftdom_0(c)$, if $\config{c, \trace_0} \uparrow^{\infty}$, 
%   let $f$ be the function such that for every $i \in \mathbb{N}$,  $\trace_i, \trace_{i + 1}\in \tdom$, $c_i, c_{i + 1} \in \cdom$ where $\config{c_i, \trace_i} \to \config{c_{i + 1}, \trace_{i + 1}}$, we have $f(i) = \config{c_i, \trace_i}$, $f(i + 1) =  \config{c_{i + 1}, \trace_{i + 1}}$. 
%   Let $\pi_2 : (\cdom \times \tdom) \to \tdom$ be the projector which projects the trace from a configuration,
%   then we define $\config{c, \trace_0} \uparrow^{\infty} \trace'$ produces an infinite trace $\trace' = \pi_2(\lim\limits_{i \to \infty}(f(i))) \in \inftdom$.
%   \[ \trace' = \lim( \pi_2 \circ (f(i))). \]
% \end{defn}
% %
% \begin{defn}[Non-terminating and Infinite Trace (third way)]
%   \label{def:infinite-trace}
%   Given a program $c$ and an initial trace $\trace_0 \in \ftdom_0(c)$,
%   when $c$ executes with $\trace_0$,  we define $c$ is non-terminating under $\trace_0$, denoted as $\config{c, \trace_0} \uparrow^{\infty} \trace'$ and produce an infinite trace $\trace' \in \inftdom$ 
%   if and only if there exists a function
%   $f : \mathbb{N} \to \cdom \times \tdom$ such that $f(0) = \config{c, \trace_0}$ and
%   for every $i \in \mathbb{N}$ there exist  $\trace_i, \trace_{i + 1}\in \tdom$, $c_i, c_{i + 1} \in \cdom$ such that  $f(i) = \config{c_i, \trace_i}$, $f(i + 1) =  \config{c_{i + 1}, \trace_{i + 1}}$ and
%   $\config{c_i, \trace_i} \to \config{c_{i + 1}, \trace_{i + 1}}$. 
%   \[
%     \begin{array}{l}
%     \forall \trace_0 \in \ftdom_0(c), \trace' \in \inftdom, c \in \cdom \st
%     \config{c, \trace_0} \uparrow^{\infty} \trace'
%     \\
%     \iff \exists f : \mathbb{N} \to \cdom \times \tdom \st 
%     f(0) = \config{c, \trace_0}
%     \\ \qquad \land
%     \forall i \in \mathbb{N}, \exists \trace_i, \trace_{i + 1} \in \tdom, c_i, c_{i + 1} \in \cdom\st 
%     \\ \qquad \quad
%     f(i) = \config{c_i, \trace_i}$, $f(i + 1) =  \config{c_{i + 1}, \trace_{i + 1}} \land \config{c_i, \trace_i} \to \config{c_{i + 1}, \trace_{i + 1}}.
%     \end{array}
%   \]
%   Let $\pi_2 : (\cdom \times \tdom) \to \tdom$ be the projector which projects the trace from a configuration,
%   then the infinite trace $\trace'$ produced by $\config{c, \trace_0} \uparrow^{\infty} \trace'$ is
%   \[ \trace' = \pi_2(\lim\limits_{i \to \infty}(f(i))) \in \inftdom. \]
% \end{defn}
%
This follows the maximal trace semantics in \cite{cousot2019abstract} Section 2.5 Equation (12).
\\
If we observe the operational semantics rules, we can find that no rule will shrink the trace. 
So we have the Lemma~\ref{lem:tracenondec} with proof in Appendix~\ref{apdx:lem_language}, 
specifically the trace has the property that its length never decreases during the program execution.
\begin{lem}
  [Trace Non-Decreasing]
  \label{lem:tracenondec}
  For any program $c \in \cdom$ and initial trace $\trace_0 \in \ftdom_0(c)$,
  if there exists $\trace \in \tdom$ and $c' \in \cdom $ such that $\config{c, \trace_0} \rightarrow^{*} \config{c', \trace} $ or 
  $\config{c, \trace_0} \uparrow^{\infty} \trace$  
  then there exists a trace $\trace' \in \tdom$ such that $\trace_0 \tracecat \trace' = \trace$ formally as follows.
  %
  \[
    \begin{array}{l}
    \forall \trace_0 \in \ftdom_0(c), \trace \in \tdom, c, c' \in \cdom \st
    \Big( \config{c, \trace_0} \rightarrow^{*} \config{c', \trace} 
    \lor  \config{c, \trace_0} \uparrow^{\infty} \trace \Big)
    \\ \quad
    \implies \exists \trace' \in \tdom \st \trace_0 \tracecat \trace' = \trace 
    \end{array}
    \]
  \end{lem}
  % \begin{lem}
  %   [Trace Non-Decreasing (based on the alternative non-termination definition)]
  %   \label{lem:tracenondec2}
  %   For any program $c \in \cdom$ and initial trace $\trace_0 \in \ftdom_0(c)$,
  %   \begin{itemize}
  %     \item if there exists $\trace \in \tdom$ and $c' \in \cdom $ such that $\config{c, \trace_0} \rightarrow^{*} \config{c', \trace} $
  %     then there exists a trace $\trace' \in \ftdom$ such that $\trace_0 \tracecat \trace' = \trace$;
  %     \item if $\config{c, \trace_0} \uparrow^{\infty} $ and produces an infinite trace $\trace \in \inftdom$ as defined in Definition~\ref{def:non-terminating}
  %     then there exists an infinite trace $\trace' \in \inftdom$ such that $\trace_0 \tracecat \trace' = \trace$ formally as follows.  
  %   \end{itemize}
  %   %
  %   \[
  %     \begin{array}{l}
  %     \forall \trace_0 \in \ftdom_0(c), \trace \in \tdom, c, c' \in \cdom \st
  %     \config{c, \trace_0} \rightarrow^{*} \config{c', \trace} 
  %     \implies \exists \trace' \in \ftdom \st \trace_0 \tracecat \trace' = \trace 
  %     \\
  %     \todomath{\land \config{c, \trace_0} \uparrow^{\infty} \implies \exists \trace' \in \inftdom \st \trace_0 \tracecat \trace' \in \inftdom}
  %     \end{array}
  %     \]
  %   \end{lem}