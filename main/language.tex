\[
\begin{array}{llll}
\mbox{Arithmetic Operators} 
& \oplus_a & ::= & + ~|~ - ~|~ \times 
%
~|~ \div
\\  
\mbox{Linear Arithmetic} 
& \aexpr & ::= & 
v \in \mathbb{Z} ~|~ x \in \vardom ~|~ v \times \aexpr ~|~ \aexpr + \aexpr ~|~ \aexpr - \aexpr
  % ~|~ \emax (a, a) ~|~ \emin(a, a)
\\
\mbox{Boolean Expression} & \bexpr & ::= & 
%
\etrue ~|~ \efalse  ~|~ \neg \bexpr
 ~|~ \bexpr \land \bexpr
%
~|~ \bexpr \lor \bexpr 
\\
& & &
~|~ \aexpr \leq \aexpr 
~|~ \aexpr < \aexpr 
~|~ \aexpr = \aexpr 
\\
\mbox{Expression} & \expr & ::= & v ~|~ \aexpr ~|~ \bexpr ~|~ [\expr, \dots, \expr]
\\  
%
\mbox{Value} 
& v & ::= & { n \in \mathbb{Z} ~|~ \etrue ~|~ \efalse ~|~ [] ~|~ [v, \dots, v]} \\
%
% \\%
\mbox{Label} 
& l & \in & (\mathbb{N} \cup \{\lin, \lex\}) 
\\ 
%
\mbox{Labeled Command} 
& {c} & ::= &  
\clabel{\assign{x}{\expr}}^l 
~|~  \clabel{\eskip}^l 
\\
& & &
~|~ \ewhile \clabel{\bexpr}^{l} \edo ({c})
~|~ \eif(\clabel{\bexpr}^{l} , {c}, {c}) 
~|~ {c};{c}  
\\ 
\mbox{Event} 
& \event & ::= & 
({x}, l, v) ~ \mbox{Assignment Event} 
% \\
% &&& 
~|~(\bexpr, l, v) ~ \mbox{Testing Event}
\\
\mbox{Trace} & \trace
& ::= & [] ~|~ \trace :: \event
\\
\end{array}
\]
We use the following notations to represent the sets of corresponding terms:
\[
\begin{array}{lll}
\vardom & : & \mbox{Set of Variables}  
\\ 
%
\booldom & : & \mbox{Set of Boolean Expressions}  
\\ 
%
\cdom & : & \mbox{Set of All Labeled Commands} 
\\ 
%
\eventset  & : & \mbox{Set of Events}  
\\
%
\eventset^{\asn}  & : & \mbox{Set of Assignment Events}  
\\
%
\eventset^{\test}  & : & \mbox{Set of Testing Events}  
\\
%
\ldom  & : & \mbox{Set of Labels}  
\\
%
\highlight{\ftdom} & : & \mbox{\highlight{Set of All Finite Execution Traces}}
\\
\highlight{\inftdom} & : & \mbox{\highlight{Set of Infinite  Execution Traces}}
\\
\highlight{\tdom} & : & \mbox{\highlight{Set of All Finite Or Infinite  Execution Traces}}
\\ 
%
\inpvar(c) & : & \mbox{Set of Program $c$'s Input Variables}  
\\
%
\ftdom_0(c) & : & \mbox{Set of Program $c$'s Well-formed Initial Traces.}
\\ & & \mbox{Each well formed initial trace $\trace_0 \in \ftdom_0(c)$ is finite}
\\ & & \mbox{and every input variableof the program $c$ has an initial value in $\trace_0$.}
\end{array}
\]
%
\paragraph{Expression}
Expressions include
standard arithmetic expression, $\aexpr \in \mathcal{A}$ (with the value $n \in \mathbb{Z}$ and the variable $x$ from a finite universe $\vardom$) and boolean expression, $\bexpr \in \booldom$.
We denote by $\infty$ a value s.t. $n < \infty $ for all $n \in \mathbb{Z}$.

An arithmetic expression can be a constant $n$ denoting integer, a variable $x$ from some countable set $\vardom$ or an expression over variables $x_1, x_2, \ldots \in \vardom$
and integers $n_0, n_1, n_2, \ldots \in \mathbb{Z}$ that is composed using arithmetic operators,
multiplication($\times $), mod($\%$), addition($+$), subtraction($-$) in the form of linear polynomial such as $v_0 + n_1 x_1 + n_2 x_2 + \ldots$ or $n_0 + x_1 \% n_1 + n_2 x_2 \ldots$ where operators $\times$ and $\%$ can only compose an arithmetic expression with an integer.

%
A boolean expression can be either {\tt true} or {\tt false}, basic boolean connectives including logical negation,  $\neg \bexpr$, logical and, $\bexpr \land \bexpr$ and logic or
$\bexpr \lor \bexpr$ and basic comparison between arithmetic expressions, 
including $\aexpr \leq \aexpr$, $\aexpr < \aexpr$ and
$\aexpr = \aexpr$.
Additionally, we also have list in expression.

\paragraph{Labeled Command}
The commands are the typical ones from while languages. Each command is annotated with a label $l \in (\mathbb{N} \cup \{\lin, \lex\})$. The program is a labeled command $c$ having the following syntax. 
\[
{c} ::= 
\clabel{\assign{x}{\expr}}^l 
~|~  \clabel{\eskip}^l
~|~ \ewhile \clabel{\bexpr}^{l} \edo ({c})
~|~ \eif(\clabel{\bexpr}^{l} , {c}, {c}) 
~|~ {c};{c} 
\]
We use label to record
the location of each command, so that we can uniquely identify each program point.
We denote by $\cdom$ the universe of all labeled commands and $\ldom$ for all labels.

\paragraph{Free Variables and Input Variables}
  We denote by $\kw{FV}$ the operator $\kw{FV}: \expr \to \mathcal{P}(\mathcal{V})$, which computes the set of free variables in an expression. For example,
  $\kw{FV}(\aexpr)$ and $\kw{FV}(\bexpr)$ represent the set of free variables in arithmetic
  expression $\aexpr$ and boolean expression $\bexpr$ respectively.
  The union of all the free variables
  showing up in $c$ that are never defined before using, are actually the input variables of this program.
  % We use $\mathcal{V}_{\lin}$ to denote the universe of all input variables and $\kw{V_{\lin}}(c)$ the set of input variables in a given program $c$.
  We denote by $\inpvardom$ the universe of all input variables and $\inpvar(c)$ is the set of input variables in a program $c$, which is all the free variables
  in program expressions that are used before it is defined.
  
\paragraph{Event}
An event is a triple.
Its first element is the variable name $x$
or a boolean expression (from the guard of if or while command), 
following by 
 the label, $l$ associated to this command and the value assigned to the variable.

 We have two kinds of events: \emph{assignment events} and \emph{testing events},
 and we use $\eventset^{\asn}$ and $\eventset^{\test}$ to denote the set of all assignment events and testing events, respectively.

 An \emph{assignment event} tracks the execution of an assignment and consists of the assigned variable, the label of the command that generates it, the value assigned to the variable.

 A \emph{testing event} tracks the execution of if and while commands, specifically the evaluation of the boolean expression $b$ in the guard of a $\eif(\clabel{b}^l, c_1, c_2)$ command or $\ewhile \clabel{b}^l \edo (c)$.
 It consists of the boolean expression $b$ in the guard of the command, the label of the guard, the result of evaluating the guard.
%
\[
\begin{array}{llll}
  \mbox{Event} 
  & \event & ::= & 
  ({x}, l, v) ~ \mbox{Assignment Event} 
  ~|~(\bexpr, l, v) ~ \mbox{Testing Event}
\end{array}
\]
Event projection operators $\pi_i$ projects the $i^{th}$ element from an event: 
$\pi_i : \eventset \to \vardom \cup \booldom \cup \ldom $

\paragraph{Trace.}
%
A trace $\trace \in \tdom$ is a list of events, 
collecting the events generated during a specific program execution. 
\[
\begin{array}{llll}
\mbox{Trace} & \trace
& ::= & [] ~|~ \trace :: \event 
% ~|~ []^{\infty}
\end{array}
\]
A trace can be regarded as the program history, 
which records all the evaluations for assignment commands and guards in $\eif$ and $\ewhile$ command.
\\
{
A trace can be finite ($\trace \in \ftdom$) or infinite $\trace \in \inftdom$.
If a program doesn't terminate when executing under some initial trace,
it produces the infinite trace 
from $\inftdom$, which records a non-terminating computation.
So we denote by $\tdom$ the set of all traces, and $\tdom = \ftdom \cup \inftdom$.
The trace-based semantics with non-terminating execution is defined below following the maximal trace semantics in \cite{Cousot19}.}

We use list notation for traces, where $[]$ is the empty trace, the operator $\traceadd$ combines an event and a trace in a new event, 
and the operator $\tracecat$ concatenates two traces formally defined as follows. 
\begin{defn}[Trace Concatenation, $\tracecat: \tdom \to \tdom\to \tdom $]
  \label{def:trace_concate}
Given two traces $\trace_1 \in \tdom, \trace_2 \in \tdom$, the trace concatenation operator 
$\tracecat$ is defined as:
\[
\trace_1 \tracecat \trace_2 \triangleq
\left\{
\begin{array}{ll} 
  \trace_1 & \trace_2 = [] \lor \trace_1 \in \inftdom \\
  \trace_2 & \trace_1 = [] \lor \trace_2 \in \inftdom \\
  (\trace_1   \tracecat \trace_2'):: \event & \trace_1 \in \ftdom \land \trace_2 = \trace_2' :: \event
  % \trace_2 &  \trace_2 \in \inftdom \\
\end{array}
\right.
\]
\end{defn}
We use the notation $\trace[j]$ for the $j^{th}$ event in the trace $\trace$ counting from the head of the list, with $\trace[1]$ denoting the first event.
We use an operator $\tracel: \tdom \to \mathcal{P}{(\ldom)}$ which projects the labels of every event in a trace as a list of program points. This is defined as follows,
\[
\begin{array}{rcll}
\tracel(\trace) &\triangleq & [] &\qquad  \mbox{if}~~ \trace = [] \lor \trace \in \inftdom
\\ 
\tracel([(\_, l, \_)] \tracecat \trace') &\triangleq & [l] \tracecat \tracel(\trace') &\qquad  \mbox{otherwise}.
\end{array}
\]
%
\paragraph{Environment.} $\env : {\tdom}  \to \vardom \to \mathbb{Z}$.  
We don't use separate data structure as the environment to record each variable's value or state. Instead, we use the operator $\env (\trace) x$ to fetch the latest value assigned to $x$ in the trace $\trace$. 
$\env$ returns $\bot$ if the variable does not show up in the trace or the trace is an infinite trace.
\[
\begin{array}{lll}
\env(\trace  \traceadd (x, l, v)) x \triangleq v
&
\env(\trace \traceadd (y, l, v)) x \triangleq \env(\trace) x, y \neq x
&
\env(\trace \traceadd (b, l, v)) x \triangleq \env(\trace) x
\\
\env({[]} ) x \triangleq \bot
&
\env(\trace ) x \triangleq \bot ~ \text{if}~ \trace \in \inftdom
\end{array}
\]
%
\begin{defn}[Well-formed Initial Traces]
  \label{def:initial_trace}
  Given a program $c$, $\trace$ is a well-formed initial trace of $c$ if and only if all the input variables of $c$ has an initial value in this trace.
  \[
    \forall c \in \cdom, \trace \in \ftdom \st \trace \in \ftdom_0(c) \iff 
    \forall x \in \inpvar(c) \st \env(\trace_0) x \neq \bot
    \]
\end{defn}
%
% In the rest part of the paper, we implicitly consider $\trace$ as a well-formed initial trace of $c$ 
% when $\trace \in \ftdom_0(c)$, and
Again, $\ftdom_0(c)$ denotes the set of all well-formed initial traces for program $c$.
%
\paragraph{Expression Semantics}
The evaluation notation for arithmetic expression is $\aconfig{} : \mathcal{A} \to \tdom \to \mathcal{V}$.
The $\aconfig{\aexpr}(\trace)$ evaluates an arithmetic expression $\aexpr$ under trace $\trace$ following the arithmetic expression evaluation rules in Figure~\ref{fig:aexpr-eval}.
We also define for $\infty$ the following mathematic computation with arbitrary $\trace \in \tdom$.
 \[
  \begin{array}{llll}
  \aconfig{\aexpr + \infty} (\trace) = \infty
 &
 \aconfig{\aexpr - \infty} (\trace) = 0
 &
 \aconfig{\infty - \aexpr} (\trace) = \infty
  &
 \aconfig{\aexpr \div \infty} (\trace) = 0
 \\
 \aconfig{\emin(\aexpr, \infty)} (\trace) = \aexpr
 &
 \aconfig{\emax(\aexpr, \infty)} (\trace) = \infty
  \end{array}
 \]
% and
\begin{figure}
\begin{mathpar}
  \boxed{ \aconfig{} \, : \, \mbox{Arithmetic Expression $\to$ Trace $\to$ Value}}
  \\
  \inferrule{ 
    \empty
  }{
   \aconfig{n} (\trace)
   = n
  }
  \and
  \inferrule{ 
    \env(\trace) x = v
  }{
   \aconfig{x} (\trace)
   = v
  }
  \and
  \inferrule{ 
    \aconfig{\aexpr_1}(\trace) = v_1
    \and 
    \aconfig{\aexpr_2}(\trace) = v_2
    \and 
     v_1 \oplus_a v_2 = v
  }{
   \aconfig{\aexpr_1 \oplus_a \aexpr_2} (\trace)
   = v
  }
  % \and
  % \inferrule{ 
  %   \aconfig{\aexpr_1}(\trace) = v_1
  %   \and 
  %   \aconfig{\aexpr_2}(\trace) = v_2
  %   \and 
  %    v_1 < v_2
  % }{
  %  \aconfig{ \emax(\aexpr_1, \aexpr_2) } (\trace)
  %  = v_2
  % }
  % \and
  % \inferrule{ 
  %   \aconfig{\aexpr_1}(\trace) = v_1
  %   \and 
  %   \aconfig{\aexpr_2}(\trace) = v_2
  %   \and 
  %    v_1 \geq v_2
  % }{
  %  \aconfig{ \emax(\aexpr_1, \aexpr_2) } (\trace)
  %  = v_1
  % }
  % \and
  % \inferrule{ 
  %   \aconfig{\aexpr_1}(\trace) = v_1
  %   \and 
  %   \aconfig{\aexpr_2}(\trace) = v_2
  %   \and 
  %    v_1 < v_2
  % }{
  %  \aconfig{ \emin(\aexpr_1, \aexpr_2) } (\trace)
  %  = v_1
  % }
  % \and
  % \inferrule{ 
  %   \aconfig{\aexpr_1}(\trace) = v_1
  %   \and 
  %   \aconfig{\aexpr_2}(\trace) = v_2
  %   \and 
  %    v_1 \geq v_2
  % }{
  %  \aconfig{ \emin(\aexpr_1, \aexpr_2) } (\trace)
  %  = v_2
  % }
   \end{mathpar}
   \caption{Evaluation rules of arithmetic expression}
   \label{fig:aexpr-eval}
   \end{figure}

 The evaluation rules for boolean expression, $\bconfig{} : \booldom \to \tdom \to \mathcal{V}$ are in Figure~\ref{fig:bexpr-eval}.
 Figure~\ref{fig:expr-eval} shows the
 standard expression evaluation, $\econfig{} : (\mathcal{A} \cup \booldom) \to \tdom \to \mathcal{V}$, which includes the evaluation rules of both the boolean and arithmetic expressions.
%  respectively.
 \begin{figure}
  \begin{mathpar}
  \boxed{ \bconfig{} \, : \, \mbox{ Boolean Expression $\times$ Trace $\rightarrow$ Boolean Value} }
  \\
  \inferrule{ 
    \empty
  }{
   \bconfig{\efalse}(\trace) 
   = \efalse
  }
  \and 
  \inferrule{ 
    \empty
  }{
   \bconfig{\etrue}(\trace) 
   = \etrue
  }
  \and 
  \inferrule{ 
    \bconfig{\bexpr}(\trace)  = v'
    \and 
    \neg v' = v
  }{
   \bconfig{\neg \bexpr}(\trace) 
   = v
  }
  \and 
  \inferrule{ 
    \bconfig{\bexpr_1}(\trace)  = v_1
    \and 
    \bconfig{\bexpr_2}(\trace)  = v_2
    \and 
     v_1 \land v_2 = v
  }{
   \bconfig{\bexpr \land \bexpr_2}(\trace) 
   = v
  }
  \and 
  \inferrule{ 
    \bconfig{\bexpr_1}(\trace)  = v_1
    \and 
    \bconfig{\bexpr_2}(\trace)  = v_2
    \and 
     v_1 \lor v_2 = v
  }{
   \bconfig{\bexpr_1 \lor \bexpr_2}(\trace)  
   = v
  }
  \end{mathpar}
  \caption{Evaluation rules of boolean expression}
  \label{fig:bexpr-eval}
  \end{figure}
  
  \begin{figure}
    \begin{mathpar}
  \boxed{ \econfig{} \, : \, \mbox{Expression $\times$ Trace $\rightarrow$ Value} }
  \\
  \inferrule{ 
    \aconfig{\aexpr}(\trace) = v
  }{
   \econfig{\aexpr}(\trace) 
   = v
  }
  \and
  \inferrule{ 
    \bconfig{\bexpr}(\trace)  = v
  }{
   \bconfig{\bexpr}(\trace)  
   = v
  }
  \and
  \inferrule{ 
    \econfig{\expr_1}(\trace)  = v_1
    \and 
    \ldots
    \and 
    \econfig{\expr_n}(\trace)  = v_n
  }{
   \econfig{ [\expr_1, \ldots, \expr_n]}(\trace) 
   = [v_1, \ldots, v_n]
  }
  \and
  \inferrule{ 
    \empty
  }{
   \econfig{v}(\trace) 
   = v
  }
   \end{mathpar}
   \caption{Evaluation rules of standard expression}
   \label{fig:expr-eval}
   \end{figure}

\paragraph{Operational Semantics Rules}
%
The trace based operational semantics is described in terms of a small step evaluation relation $\config{c, \trace} \to \config{c', \trace'}$ describing how a configuration program-trace evaluates to another configuration
program-state.
The rules are presented in Figure~\ref{fig:command-os}.
\begin{figure}
  \begin{mathpar}
\boxed{
\mbox{Command $\times$ Trace}
\xrightarrow{}
\mbox{Command $\times$ Trace}
}
\and
\boxed{\config{{c, \trace}}
\xrightarrow{} 
\config{{c',  \trace'}}
}
%
\\
%
\inferrule
{
\econfig{\expr}(\trace)  = v
  \and
\event = ({x}, l, v)
}
{
\config{\clabel{\assign{{x}}{\expr}}^{l},  \trace } 
\xrightarrow{} 
\config{\clabel{\eskip}^l, \trace \traceadd \event}
}
~\rname{assn}
\and
%
\inferrule
{
  \bconfig{\bexpr}(\trace)  = \etrue
 \and 
 \event = (\bexpr, l, \etrue)
}
{
\config{{\ewhile \clabel{\bexpr}^{l} \edo (c), \trace}}
\xrightarrow{} 
\config{{
c; \ewhile \clabel{\bexpr}^{l} \edo (c),
\trace \traceadd \event}}
}
~\rname{while-t}
%
%
\and
%
\inferrule
{
  \bconfig{\bexpr}(\trace)  = \efalse
 \and 
 \event = (\bexpr, l, \efalse)
}
{
\config{{\ewhile \clabel{\bexpr}^{l} \edo (c), \trace}}
\xrightarrow{} 
\config{{
  \clabel{\eskip}^l,
\trace \traceadd \event}}
}
~\rname{while-f}
\and
%
%
\inferrule
{
\config{{c_1, \trace}}
\xrightarrow{}
\config{{c_1',  \trace'}}
}
{
\config{{c_1; c_2, \trace}} 
\xrightarrow{} 
\config{{c_1'; c_2, \trace'}}
}
~\rname{seq1}
%
\and
%
\inferrule
{
  \config{{c_2, \trace}}
  \xrightarrow{}
  \config{{c_2',  \trace'}}
}
{
\config{{\clabel{\eskip}^l; c_2, \trace}} \xrightarrow{} \config{{ c_2', \trace'}}
}
~\rname{seq2}
%
\and
%
%
\inferrule
{
  \bconfig{\bexpr}(\trace)  = \etrue
 \and 
 \event = (\bexpr, l, \etrue)
}
{
\config{{
\eif(\clabel{\bexpr}^{l}, c_1, c_2), 
\trace}}
\xrightarrow{} 
\config{{c_1, \trace \traceadd \event}}
}
~\rname{if-t}
%
\and
%
\inferrule
{
 \bconfig{\bexpr}(\trace)  = \efalse
 \and 
 \event = (\bexpr, l, \efalse)
}
{
\config{{\eif(\clabel{\bexpr}^{l}, c_1, c_2), \trace}}
\xrightarrow{} 
\config{{c_2, \trace \traceadd \event}}
}
~\rname{if-f}
%
\end{mathpar}
\caption{Operational semantics rules}
\label{fig:command-os}
\end{figure}


Given an initial trace $\trace_0 \in \ftdom_0(c)$ of the program $c$,
we use $\to^*$ for the reflexive and transitive closure of $\to$. 
If $\config{c, \trace_0} \rightarrow^{*} \config{\clabel{\eskip}^l, \trace_0 \tracecat \trace}$,
then the program's execution terminates and produces a finite execution trace $\trace \in \ftdom$.
\\
\begin{defn}[Non-terminating and Infinite Trace]
  \label{def:non-terminating}
  Given a program $c$ and an initial trace $\trace \in \ftdom_0(c)$,
  when $c$ executes with $\trace$,  we define the execution of $c$ under $\trace$ is non-terminating and produces an infinite trace $\trace' \in \inftdom$, as 
  $\config{c, \trace} \uparrow^{\infty} \trace' \in \lim(\uparrow)$
  where the limit is defined as follows.
  \[
    \begin{array}{l}
      \lim(\uparrow) 
      % \in \left( (\cdom \times \ftdom) \times (\cdom \times \inftdom) \right) 
      \triangleq 
    \\ \quad
    \Big\{
      (\config{c, \trace}, \trace') ~\vert~ 
      c\in \cdom, \trace \in \ftdom_0(c),
      \trace' \in \inftdom 
      \land \exists \trace_0 \in \ftdom, c_0 \in \cdom \st 
      \config{c, \trace} \to \config{c_0, \trace_0}
      \\ \qquad \qquad \qquad 
      \land \forall i \in \mathbb{N}, \exists \trace_i, \trace_{i + 1} \in \ftdom, \trace'' \in \inftdom, c_i, c_{i + 1} \in \cdom \st 
      \\ \qquad \qquad \qquad 
      \qquad \qquad \qquad 
      \config{c_i, \trace_i} \to \config{c_{i + 1}, \trace_{i + 1}} 
      \land  \trace' = \trace_{i + 1} \tracecat \trace''
    \Big\}
    \end{array}
  \]
\end{defn}
%
% \begin{defn}[Non-terminating and Infinite Trace (alternative way)]
%   \label{def:non-terminating-2}
%   Given a program $c$ and an initial trace $\trace_0 \in \ftdom_0(c)$,
%   when $c$ executes with $\trace_0$,  we define $c$ is non-terminating under $\trace_0$, $\config{c, \trace_0} \uparrow^{\infty}$ if and only if there exists a function
%   $f : \mathbb{N} \to \cdom \times \tdom$ such that $f(0) = \config{c, \trace_0}$ and
%   for every $i \in \mathbb{N}$ there exist  $\trace_i, \trace_{i + 1}\in \tdom$, $c_i, c_{i + 1} \in \cdom$ such that  $f(i) = \config{c_i, \trace_i}$, $f(i + 1) =  \config{c_{i + 1}, \trace_{i + 1}}$ and
%   $\config{c_i, \trace_i} \to \config{c_{i + 1}, \trace_{i + 1}}$. 
%   \[
%     \begin{array}{l}
%     \forall \trace_0 \in \ftdom_0(c), c \in \cdom \st
%     \config{c, \trace_0} \uparrow^{\infty}
%     \\
%     \iff \exists f : \mathbb{N} \to \cdom \times \tdom \st 
%     f(0) = \config{c, \trace_0}
%     \\ \qquad \land
%     \forall i \in \mathbb{N}, \exists \trace_i, \trace_{i + 1} \in \tdom, c_i, c_{i + 1} \in \cdom\st 
%     \\ \qquad \quad
%     f(i) = \config{c_i, \trace_i}$, $f(i + 1) =  \config{c_{i + 1}, \trace_{i + 1}} \land \config{c_i, \trace_i} \to \config{c_{i + 1}, \trace_{i + 1}}
%     \end{array}
%   \]
%   Given a program $c$ and an initial trace $\trace_0 \in \ftdom_0(c)$, if $\config{c, \trace_0} \uparrow^{\infty}$, 
%   let $f$ be the function such that for every $i \in \mathbb{N}$,  $\trace_i, \trace_{i + 1}\in \tdom$, $c_i, c_{i + 1} \in \cdom$ where $\config{c_i, \trace_i} \to \config{c_{i + 1}, \trace_{i + 1}}$, we have $f(i) = \config{c_i, \trace_i}$, $f(i + 1) =  \config{c_{i + 1}, \trace_{i + 1}}$. 
%   Let $\pi_2 : (\cdom \times \tdom) \to \tdom$ be the projector which projects the trace from a configuration,
%   then we define $\config{c, \trace_0} \uparrow^{\infty} \trace'$ produces an infinite trace $\trace' = \pi_2(\lim\limits_{i \to \infty}(f(i))) \in \inftdom$.
%   \[ \trace' = \lim( \pi_2 \circ (f(i))). \]
% \end{defn}
% %
% \begin{defn}[Non-terminating and Infinite Trace (third way)]
%   \label{def:infinite-trace}
%   Given a program $c$ and an initial trace $\trace_0 \in \ftdom_0(c)$,
%   when $c$ executes with $\trace_0$,  we define $c$ is non-terminating under $\trace_0$, denoted as $\config{c, \trace_0} \uparrow^{\infty} \trace'$ and produce an infinite trace $\trace' \in \inftdom$ 
%   if and only if there exists a function
%   $f : \mathbb{N} \to \cdom \times \tdom$ such that $f(0) = \config{c, \trace_0}$ and
%   for every $i \in \mathbb{N}$ there exist  $\trace_i, \trace_{i + 1}\in \tdom$, $c_i, c_{i + 1} \in \cdom$ such that  $f(i) = \config{c_i, \trace_i}$, $f(i + 1) =  \config{c_{i + 1}, \trace_{i + 1}}$ and
%   $\config{c_i, \trace_i} \to \config{c_{i + 1}, \trace_{i + 1}}$. 
%   \[
%     \begin{array}{l}
%     \forall \trace_0 \in \ftdom_0(c), \trace' \in \inftdom, c \in \cdom \st
%     \config{c, \trace_0} \uparrow^{\infty} \trace'
%     \\
%     \iff \exists f : \mathbb{N} \to \cdom \times \tdom \st 
%     f(0) = \config{c, \trace_0}
%     \\ \qquad \land
%     \forall i \in \mathbb{N}, \exists \trace_i, \trace_{i + 1} \in \tdom, c_i, c_{i + 1} \in \cdom\st 
%     \\ \qquad \quad
%     f(i) = \config{c_i, \trace_i}$, $f(i + 1) =  \config{c_{i + 1}, \trace_{i + 1}} \land \config{c_i, \trace_i} \to \config{c_{i + 1}, \trace_{i + 1}}.
%     \end{array}
%   \]
%   Let $\pi_2 : (\cdom \times \tdom) \to \tdom$ be the projector which projects the trace from a configuration,
%   then the infinite trace $\trace'$ produced by $\config{c, \trace_0} \uparrow^{\infty} \trace'$ is
%   \[ \trace' = \pi_2(\lim\limits_{i \to \infty}(f(i))) \in \inftdom. \]
% \end{defn}
%
This definition has the similar intuition as the maximal trace semantics in Equation (12) from Section 2.5 of paper~\cite{Cousot19a}, and the semantics of the program model in~\cite{SinnZV17}.
\\
% If we observe the operational semantics rules, we can find that no rule will shrink the trace. 
% So we 
We also have the trace non-decreasing property in this language model as below.
It is similar to the Lemma~\ref{lem:tracenondec} with proof in Appendix~\ref{apdx:lem_language}. 
\begin{lem}
  [Trace Non-Decreasing]
  \label{lem:psRB-tracenondec}
  For any program $c \in \cdom$ and initial trace $\trace_0 \in \ftdom_0(c)$,
  if there exists $\trace \in \tdom$ and $c' \in \cdom $ such that $\config{c, \trace_0} \rightarrow^{*} \config{c', \trace} $ or 
  $\config{c, \trace_0} \uparrow^{\infty} \trace$  
  then there exists a trace $\trace' \in \tdom$ such that $\trace_0 \tracecat \trace' = \trace$ formally as follows.
  %
  \[
    \begin{array}{l}
    \forall \trace_0 \in \ftdom_0(c), \trace \in \tdom, c, c' \in \cdom \st
    \Big( \config{c, \trace_0} \rightarrow^{*} \config{c', \trace} 
    \lor  \config{c, \trace_0} \uparrow^{\infty} \trace \Big)
    \\ \quad
    \implies \exists \trace' \in \tdom \st \trace_0 \tracecat \trace' = \trace 
    \end{array}
    \]
  \end{lem}
  Specifically the trace has the property that its length never decreases during the program execution.
