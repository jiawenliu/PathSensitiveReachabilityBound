\begin{lem}[Uniqueness of the Program Labels]
  For every program $c \in \cdom$ and every two labels such that
  $i, j \in \lvar(c)$, then $i \neq j$.
  \[
    \forall c \in \cdom, i, j \in \ldom \st i, j \in \lvar(c)\implies i \neq j.
    \]
\end{lem}
  \begin{proof}
    The proof is trivially by induction on program $c$.
  \end{proof}
  \begin{lem}
    [Trace Non-Decreasing]
    For every program $c \in \cdom$ and traces $\trace, \trace' \in \tdom$, if 
    $\config{c, \trace} \rightarrow^{*} \config{\eskip, \trace'}$,
    then there exists a trace $\trace'' \in \tdom$ with $\trace \tracecat \trace'' = \trace'$
    %
    $$
    \forall \trace, \trace' \in \tdom, c \st
    \config{c, \trace} \rightarrow^{*} \config{\eskip, \trace'} 
    \implies \exists \trace'' \in \tdom \st \trace \tracecat \trace'' = \trace'
    $$
    \end{lem}
    \begin{proof}
      Taking arbitrary trace $\trace \in \tdom$, by induction on program $c$, we have the following cases:
      \caseL{$c = [\assign{x}{\expr}]^{l}$}
      By the evaluation rule $\rname{assn}$, we have
      $
      {
      \config{[\assign{{x}}{\aexpr}]^{l},  \trace } 
      \xrightarrow{} 
      \config{\eskip, \trace :: ({x}, l, v, \bullet)}
      }$, for some $v \in \mathbb{N}$.
      \\
      Picking $\trace' = \trace ::({x}, l, v, \bullet)$ and $\trace'' =  [({x}, l, v, \bullet) ]$,
      it is obvious that $\trace \tracecat \trace'' = \trace'$.
      % \\
      % There are 2 cases, where $l' = l$ and $l' \neq l$.
      % \\
      % In case of $l' \neq l$, we know $\event \not\eventin \trace$, then this Lemma is vacuously true.
      %   \\
      %   In case of $l' = l$, by the abstract Execution Trace computation, we know 
      %   $\absflow(c) = \absflow'([x := \expr]^{l}; \clabel{\eskip}^{l_e}) = \{(l, \absexpr(\expr), l_e)\}$  
      %   \\
      % Then we have $\absevent = (l, \absexpr(\expr), l_e) $ and $\absevent \in \absflow(c)$.
      \\
      This case is proved.
      \caseL{$\ewhile [b]^{l_w} \edo c$}
      By the first rule applied to $c$, there are two cases:
      \subcaseL{$\textbf{while-t}$}
      If the first rule applied to is $\rname{while-t}$, we have
      \\
      $\config{{\ewhile [b]^{l_w} \edo c_w, \trace}}
        \xrightarrow{} 
        \config{{
        c_w; \ewhile [b]^{l_w} \edo c_w,
        \trace :: (b, l_w, \etrue, \bullet)}}~ (1)
      $.
      \\
      Let $\trace_w' \in \tdom$ be the trace satisfying following execution:
      \\
      $
      \config{{
      c_w,
      \trace :: (b, l_w, \etrue, \bullet)}}
      \xrightarrow{*} 
      \config{{
      \eskip, \trace_w'}}
    $
    \\
    By induction hypothesis on sub program $c_w$ with starting trace 
    $\trace :: (b, l_w, \etrue, \bullet)$ and ending trace $\trace_w'$, 
    we know there exist
    $\trace_w \in \tdom$ such that $\trace_w' = \trace :: (b, l_w, \etrue, \bullet) \tracecat \trace_w$.
    \\
    Then we have the following execution continued from $(1)$:
    \\
    $
    \config{{\ewhile [b]^{l_w} \edo c_w, \trace}}
        \xrightarrow{} 
        \config{{
        c_w; \ewhile [b]^{l_w} \edo c_w,
        \trace :: (b, l_w, \etrue, \bullet)}}
        \xrightarrow{*} 
        \config{\ewhile [b]^{l_w} \edo c_w, \trace :: (b, l_w, \etrue, \bullet) \tracecat \trace_w}
        ~(2)
    $
    By repeating the execution (1) and (2) until the program is evaluated into $\eskip$,
    with trace $\trace_w^{i'} $ for $i = 1, \cdots, n n \geq 1$ in each iteration, we know 
    in the $i-th$ iteration,
     there exists  $\trace_w^i \in \tdom$ such that  
    $\trace_w^{i'} = \trace_w^{(i-1)'} :: (b, l_w, \etrue, \bullet) ++ \trace_w^{i'}$
    \\
    Then we have the following execution:
    \\
    $
    \config{{\ewhile [b]^{l_w} \edo c_w, \trace}}
        \xrightarrow{} 
        \config{{
        c_w; \ewhile [b]^{l_w} \edo c_w,
        \trace :: (b, l_w, \etrue, \bullet)}}
        \xrightarrow{*} 
        \config{\ewhile [b]^{l_w} \edo c_w, \trace_w^{n'}}
        \xrightarrow{}^\rname{{while-f}}
        \config{\eskip, \trace_w^{n'}:: (b, l_w, \efalse, \bullet)}
    $ and $\trace_w^{n'} = \trace :: (b, l_w, \etrue, \bullet) \tracecat \trace_w^{1} :: \cdots :: (b, l_w, \etrue, \bullet) \tracecat \trace_w^{n} $.
    \\
    Picking $\trace' = \trace_w^{n'} :: (b, l_w, \efalse, \bullet)$ and $\trace'' = [(b, l_w, \etrue, \bullet)] \tracecat \trace_w^{1} :: \cdots :: (b, l_w, \etrue, \bullet) \tracecat \trace_w^{n}$,
    we have 
    $\trace ++ \trace'' = \trace'$.
    \\
    This case is proved.
      \subcaseL{$\textbf{while-f}$}
      If the first rule applied to $c$ is $\rname{while-f}$, we have
      \\
      $
      {
        \config{{\ewhile [b]^{l_w} \edo c_w, \trace}}
        \xrightarrow{}^\rname{while-f}
        \config{{
        \eskip,
        \trace :: (b, l_w, \efalse, \bullet)}}
      }$,
      In this case, picking $\trace' = \trace ::(b, l_w, \efalse, \bullet)$ and $\trace'' =  [(b, l_w, \efalse, \bullet) ]$,
      it is obvious that $\trace \tracecat \trace'' = \trace'$.
      \\
      This case is proved.
      \caseL{$\eif([b]^l, c_t, c_f)$}
      This case is proved in the same way as \textbf{case: $c = \ewhile [b]^{l} \edo c$}.
      \caseL{$c = c_{s1};c_{s2}$}
     By the induction hypothesis on $c_{s1}$ and $c_{s2}$ separately,
     we have this case proved.
    \end{proof}
    %