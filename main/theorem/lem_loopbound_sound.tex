
The three loop bound computation methods all computes the sound upper bounds, $BD(\rprepeat(\rprog'))$ for every loop $\rprepeat(\rprog')$ in a refined program $\rprog$:
\begin{enumerate}
    \item The Equation~\ref{eq:absBD} in Section~\ref{sec:rank} is path-insensitively and sound.
    Its soundness relies on the soundness of the $\absclr(\absevent, c)$ in Definition~\ref{def:ranking_bound} in Section~\ref{sec:rank}.
    For a program $c$ and an edge $\absevent \in \absE(c)$, $\absclr(\absevent, c)$ is a sound upper bound on the execution times of this transition, formally in Theorem~\ref{thm:pathinsensitive_rb_soundness} with proof in Appendix~\ref{apdx:pathinsensitive_rb_soundness}    
    \item The second computation fully relies on the soundness of $BOUND(\rprepeat(\rprog'))$ from paper\cite{GulwaniJK09}, we do not repeat their soundness in this paper.
    \item   The soundness of the alternative computation method we provided in Definition~\ref{def:loopbound} is presented below.
  \end{enumerate}

For every transition path $\rprog$, the $BD(\rprog)$
is a sound upper bound on its execution times.
\\
This bound is sound locally by assuming
that all the loops and transition paths where $\rprog$ is nested execute only once.
This assumption comes from the computation of the $\varGD$ and the depth first search strategy.
\\
%
\begin{lemma}[Soundness of Loop Bound]
    % \label{lem:loopbound_sound}
    For every program $c$ with it refined program $\rprog$,
    for every loop $\rprepeat(\rprog')$ inside $\rprog$, 
    $BD(\rprepeat(\rprog'))$ is a sound upper bound on the execution times of this loop when executing the program $c$.
  \end{lemma}
\begin{proof}
Lemma~\ref{lem:loopbound_sound} guarantees that
the bound for every transition path $\rprog$, the $BD(\rprog)$
is a sound upper bound on its execution times, by assuming
that all the loops and transition paths where $\rprog$ is nested execute only once.
\caseL{$\rprog' = \tpath$ is a simple transition path}
For every base case, i.e., a simple transition path, 
$\varGD(\rprog) =  \rfinit(\tpath) - \rfnext(\rprog)$
counts the variables' changes only once. In this way, it assumes all the outside patterns and loops execution only once.
In paper \cite{sinn2017complexity} Definition~9, they informally discussed the local bound soundness.
$v$ is a local bound if it has the same decreasing time as the transition's execution time.
By assuming that certain program parts (those were e increases) are not executed,
then value of $v$ can limit the execution time of that transition.
\\
In our soundness, we assume all the code pieces not inside this transition path are executed at most once (once if they show up in front of the program
zero time if not).
In this case, this bound limits the execution time of this transition path.
\\
The soundness also relies on the operation $\frac{\rfinit(\rprog) - \rffinal(\rprog)}{\varGD(\rprog)}$,
which can be solved by external SMT solver,
or solved by Definition~\ref{def:ranking_bound} in Section~\ref{sec:rank}.
\caseL{$\rprog' = \rpchoose{\rprog_1, \ldots}$}
Sound by taking the maximum.
\caseL{$\rprog' = \rprog_1; \rprog_2$}
Sound by taking the sum.
\end{proof}