\subsection{Proof of Lemma~\ref{lem:pathlocalrb-sound}}
\label{apdx:pathlocalrb-sound}
\begin{lemma}[Soundness of the Path Local Reachability-Bound]
  \label{lem:pathlocalrb-sound}
  For any program $c$ with its refined program $\rprog$ and a simple transition path $\tpath$ in $\rprog$,
  if $l: \rprog_l = \kw{enclosed}(\rprog, \tpath)$ is the closest loop where $\tpath$ is nested in this program,
  then the execution times of $\tpath$ when executing the $\rprog_l$ under initial trace $\trace_l \in \ftdom_0(c_l)$ is bounded by $\econfig{\outinB(\rprog_l, \tpath)}(\trace_0)$ with any initial trace $\trace_0 \in \ftdom_0(c)$.
  \[
    \begin{array}{l}
    \forall c, c_l \in \cdom, \tpath \in \absG(c), 
    \trace_l \in \ftdom_0(c_l), \trace_0 \in \ftdom_0(c), \trace \in \tdom, l, l' \in \ldom, \rprog \st 
    \\ \qquad
    \rprog = REFINE(\algrewrite(c))
    \land
    l: \rprog_l = \kw{enclosed}(\rprog, \tpath)
    \land 
    \rprog_l = \algrewrite(c_l)
    \\ \qquad
    \land
    \Big(
    \config{c_l, \trace_l} \rightarrow^* \config{\clabel{\eskip}^{l'}, \trace_l \tracecat \trace}
    \lor \config{c_l, \trace_l} \uparrow^{\infty} \trace_l \tracecat \trace 
    \Big)
    \\ \qquad
    \implies
    \econfig{\outinB(\rprog_l, \tpath)}(\trace_0) \geq \lcounter(\trace, \pathl(\tpath)).
    \end{array}
  \]  
\end{lemma}
\begin{proof}
\emph{Path Local Reachability-Bound} Soundness.
\\
Taking an arbitrary program $c$, let $\rprog$ be its refined program and $\tpath$ be an arbitrary transition path in $\rprog$.
\\
Let $l: \rprog_l = \kw{enclosed}(\rprog, \tpath)$ be the closest loop program where $\tpath$ is nested and $c_l$ be the while language program of $\rprog_l$ such that $\kw{Rewrite(c_l)} = \rprog_l$. 
\\
Let $\trace_l \in \ftdom_0(c_l)$ be an arbitrary initial trace of the closest loop $c_l$, and $\trace \in \tdom $ be an execution trace such that
\[
  \config{c_l, \trace_l} \to^* \config{\clabel{\eskip}^{l'}, \trace_l \tracecat \trace} \lor \config{c_l, \trace_l} \uparrow^{\infty} \trace_l \tracecat \trace 
\]
\textbf{$\bullet$ The execution terminates and {$\config{{c_l}, \trace_l} \to^{*} \config{\clabel{\eskip}^{l}, \trace_l \tracecat \trace}$}.} 
\\
 In this case we know $\trace \in \ftdom$ and is sufficient to show,
\[
  \econfig{\outinB(\rprog_l, \tpath)}(\trace_0) \geq \lcounter(\trace, \pathl(\tpath)).
\]
By induction on the loop program $\rprog_l$,
we have the following cases:
\caseL{$\rprog_l = \tpath$}
In this case, we have $\outinB(\tpath, \tpath) = 1$ by Definition~\ref{def:pathlocalrb}. 
\\
Since $\tpath$ doesn't contain any loop, we know it will be executed at most once,  i.e., $\lcounter(\trace, \pathl(\tpath)) \leq 1$.
\\
Then we have the following
\[\econfig{\outinB(\rprog_l, \tpath)}(\trace_0) = 1 = \lcounter(\trace, \pathl(\tpath)) \]
This case is proved.
\caseL{$\rprog_l = \tpath' \land \tpath' \neq \tpath$}
In this case, we have $\outinB(\tpath', \tpath) = \highlight{0} $ by Definition~\ref{def:pathlocalrb}.
\\
Since $\tpath' \neq \tpath$, we know $\tpath$ isn't executed when executing $c_l$, i.e., $\lcounter(\trace, \pathl(\tpath)) = 0$.
\\
Then we have
\[
  \econfig{\outinB(\rprog_l, \tpath)}(\trace_0) = 0 = \lcounter(\trace, \pathl(\tpath)),
  \]
and this case is proved.
\caseL{$\rprog_l = \rprog_1;\rprog_2$}
In this case, we have $\outinB(\rprog_1;\rprog_2, \tpath) = \outinB(\rprog_1, \tpath) + \outinB(\rprog_2, \tpath) $ by Definition~\ref{alg:alg-refine_rewrite}, and
it is sufficient to show
\[
  \econfig{\outinB(\rprog_1, \tpath) + \outinB(\rprog_2, \tpath)}(\trace_0) \geq \lcounter(\trace, \pathl(\tpath)) 
\]
Let $c_1, c_2 \in \cdom$ be the while language program corresponding to $\rprog_1$ and $\rprog_2$ such that $\algrewrite(c_1) = \rprog_1$ and $\algrewrite(c_2) = \rprog_2$.
According to the Algorithm~\ref{alg:alg-refine_rewrite}, we have $c_l = c_1; c_2$.
\\
According to the operational semantics, let $\trace_1, \trace_2 \in \ftdom$ be two execution traces such that 
\[
  \config{c_1; c_2, \trace_l} \to^* \config{c_2, \trace_l \tracecat \trace_1} \to^* \config{\clabel{\eskip}^{l'}, \trace_l \tracecat \trace_1 \tracecat \trace_2}
\]
By the trace non-decreasing property in Lemma~\ref{lem:tracenondec}, we have $\trace = \trace_1 \tracecat \trace_2$ and 
\[
  \lcounter(\trace, \pathl(\tpath))  = \lcounter(\trace_1,  \pathl(\tpath))  + \lcounter(\trace_2,  \pathl(\tpath)) 
\]
By induction hypothesis on $\trace_l$, $c_1$, $\trace_1$ and $c_2$, we have
\[
  \econfig{\outinB(\rprog_1, \tpath)}(\trace_0) \geq \lcounter(\trace_1,  \pathl(\tpath)) 
\]
 and 
  \[
    \econfig{\outinB(\rprog_2, \tpath)}(\trace_0) \geq \lcounter(\trace_2,  \pathl(\tpath)) 
  \]
Then we have this case proved as
\[
  \begin{array}{l}
  \econfig{\outinB(\rprog_1, \tpath) + \outinB(\rprog_2, \tpath)}(\trace_0)
  \\
  \geq \econfig{\outinB(\rprog_1, \tpath)}(\trace_0) + \econfig{\outinB(\rprog_2, \tpath)}(\trace_0)
  \\
  \geq \lcounter(\trace_1,  \pathl(\tpath))  + \lcounter(\trace_2,  \pathl(\tpath)) 
  \\
  = \lcounter(\trace, \pathl(\tpath))
  \end{array}
  \] 
\caseL{$\rprog_l = \rpchoose{\rprog_1, \ldots, \rprog_m }$}
We have $\outinB(\rpchoose{\rprog_1, \ldots, \rprog_m }, \tpath) = \max\left\{ \outinB(\rprog_1, \tpath), \ldots, \outinB(\rprog_m, \tpath) \right\}$ by Definition~\ref{alg:alg-refine_rewrite}, and
it is sufficient to show
\[
  \econfig{\max\left\{ \outinB(\rprog_1, \tpath), \ldots, \outinB(\rprog_m, \tpath) \right\}}(\trace_0) \geq \lcounter(\trace, \pathl(\tpath)) 
\]
Let $c_1, \ldots, c_m \in \cdom$ be the while language program corresponding to $\rprog_1, \ldots, \rprog_m$ such that $\algrewrite(c_1) = \rprog_1$, $\ldots$ and  $\algrewrite(c_m) = \rprog_m$.
\\
Let $\trace_1, \ldots, \trace_m \in \ftdom$ be the execution traces such that 
\[
  \config{c_i, \trace_r} \to^* 
  \config{\clabel{\eskip}^{l}, \trace_r \tracecat \trace_i}
\]
Then by induction hypothesis on each program $\rprog_i$, we know for each $\rprog_i$,
\[
  \econfig{\outinB(\rprog_i, \tpath)}(\trace_0)  \geq \lcounter(\trace_i, \pathl(\tpath)) 
\]
Since execution of each $\rprog_i$ corresponds to one possible execution of $c_r$ as shown in proof of Lemma~\ref{lem:loopbound_sound} in Appendix~\ref{apdx:loopbound-sound}, we know
\[
  \max\left\{\lcounter(\trace_0, \pathl(\tpath)), \ldots, \lcounter(\trace_m, \pathl(\tpath)) \right\} 
  \geq \lcounter(\trace, \pathl(\tpath)).
\]
Then we have
\[
  \begin{array}{l}
  \econfig{\outinB(\rpchoose{\rprog_1, \ldots, \rprog_m }, \tpath)}(\trace_0)
  \\
  = \econfig{\max\left\{ \outinB(\rprog_1, \tpath), \ldots, \outinB(\rprog_m, \tpath) \right\}}(\trace_0)
  \\
  = \max\left\{\econfig{\outinB(\rprog_1, \tpath)}(\trace_0), \ldots, \econfig{ \outinB(\rprog_m, \tpath)}(\trace_0) \right\}
  \\
  \geq \max\left\{\lcounter(\trace_0, \pathl(\tpath)), \ldots, \lcounter(\trace_m, \pathl(\tpath)) \right\} 
  \\
  \geq \lcounter(\trace, \pathl(\tpath)),
  \end{array}
\] 
and this case is proved.
\caseL{$\rprog_l = \rprepeat(\rprog')$ }
We have $\outinB(\rprepeat(\rprog'), \tpath) = BD(\rprepeat(\rprog'), c) \times \outinB(\rprog', \tpath)$
by Definition~\ref{alg:alg-refine_rewrite}, and
it is sufficient to show
\[
  \econfig{ BD(\rprepeat(\rprog'), c) \times \outinB(\rprog', \tpath)} (\trace_0) \geq \lcounter(\trace, \pathl(\tpath)) 
\]
Let $c' \in \cdom$ be the while language program corresponding to $\rprog'$ such that $\algrewrite(c') = \rprog'$.
According to the Algorithm~\ref{alg:alg-refine_rewrite}, we have $c_l = \ewhile \clabel{\bexpr}^{l} \edo c'$.
\\
According to the operational semantics, let $\trace' \in \ftdom$ be the execution trace of executing $c'$ one time with a program $c'' \in \cdom$ such that 
\[
  \config{\ewhile \clabel{\bexpr}^{l} \edo c', \trace_l} \to \config{c', \trace_l \tracecat (\bexpr, \etrue, l)} \to^* \config{\clabel{\eskip}^{l}; \ewhile \clabel{\bexpr}^{l} \edo c', \trace_l \tracecat (\bexpr, \etrue, l) \tracecat \trace'}
  \to^* \config{\clabel{\eskip}^{l}, \trace_l \tracecat \trace}
\]
By the trace non-decreasing property in Lemma~\ref{lem:tracenondec}, we know $\trace$ contains multiple execution traces of $c'$.
\\
By Lemma~\ref{lem:loopbound_sound}, we know $BD(\rprepeat(\rprog'), c)$ bounds the iterations of $\rprepeat(\rprog')$. Then we have
\[
  \econfig{BD(\rprepeat(\rprog'), c)}(\trace_0) \times \lcounter(\trace', \pathl(\tpath))  
  \geq \lcounter(\trace, \pathl(\tpath)) 
\]
By induction hypothesis on $\rprog'$ and $c'$, we also have
\[
  \econfig{\outinB(\rprog', \tpath)}(\trace_0) \geq \lcounter(\trace',  \pathl(\tpath)) 
\]
Then we have this case proved by
\[
  \begin{array}{l}
  \econfig{BD(\rprepeat(\rprog'), c) \times \outinB(\rprog', \tpath)}(\trace_0) 
  \\
  =
  \econfig{BD(\rprepeat(\rprog'), c)}(\trace_0) \times \econfig{\outinB(\rprog', \tpath)}(\trace_0) 
  \\
  \geq \econfig{BD(\rprepeat(\rprog'), c)}(\trace_0) \times \lcounter(\trace',  \pathl(\tpath))  
  \\
  \geq \lcounter(\trace, \pathl(\tpath)) 
  \end{array}
  \]
  \caseL{$\rprog = l : \rprog_l$}
  In this case, we have
  \[
    \outinB(l: \rprog_l, c) = \outinB(\rprog_l, c)
  \]
  Let $c_l \in \cdom$ be the while language program corresponding to $\rprog_l$ such that $\algrewrite(c_l) = \rprog_l$.
  \\
  According to the Algorithm~\ref{alg:alg-refine_rewrite} and the program refinement algorithm REFINE, we know
  $c_l = c_l$.
  \\
  Then we have this case proved trivially by induction hypothesis on $\rprog_l$ and $c_l$ with the same initial and execution trace $\trace_l$.

  \caseL{$\rprog_l = l': \rprog'$ and $l':\rprog' \neq \kw{enclosed}(\tpath)$}
  In this case, we have $\outinB(l': \rprog', \tpath) = \highlight{0} $ by Definition~\ref{def:pathlocalrb}.
  \\
  Since $l': \rprog' \neq \kw{enclosed}(\tpath)$, we know $\tpath$ isn't executed when executing $c_l$, i.e., $\lcounter(\trace, \pathl(\tpath)) = 0$.
  \\
  Then we have
  \[
    \econfig{\outinB(l': \rprog', \tpath)}(\trace_0) = 0 = \lcounter(\trace, \pathl(\tpath)),
    \]
  and this case is proved.
\\
\textbf{$\bullet$ The execution is non-terminating and {$\config{{c_l}, \trace} \uparrow^{\infty} \trace_l \tracecat \trace$}.} 
\\
 In this case we know $\trace \in \inftdom$ and is sufficient to show,
\[
  \econfig{\outinB(\rprog_l, \tpath)}(\trace_0) \geq \lcounter(\trace, \pathl(\tpath)).
\]
By induction on the loop program $\rprog_l$,
we have the following cases:
\caseL{$\rprog_l = \tpath$ }
In this case, we have $\outinB(\tpath, \tpath) = 1$ by Definition~\ref{def:pathlocalrb}. 
\\
Since $\tpath$ doesn't contain any loop, we know it will be executed at most once and terminates.
\\
Then this case is vacuously true.
\caseL{$\rprog_l = \tpath' \land \tpath' \neq \tpath$}
This case is vacuously true as well for the same reason as above.
\caseL{$\rprog_l = \rprog_1;\rprog_2$}
In this case, we have $\outinB(\rprog_1;\rprog_2, \tpath) = \outinB(\rprog_1, \tpath) + \outinB(\rprog_2, \tpath) $ by Definition~\ref{alg:alg-refine_rewrite}, and
it is sufficient to show
\[
  \econfig{\outinB(\rprog_1, \tpath) + \outinB(\rprog_2, \tpath)}(\trace_0) \geq \lcounter(\trace, \pathl(\tpath)) 
\]
Let $c_1, c_2 \in \cdom$ be the while language program corresponding to $\rprog_1$ and $\rprog_2$ such that $\algrewrite(c_1) = \rprog_1$ and $\algrewrite(c_2) = \rprog_2$.
According to the Algorithm~\ref{alg:alg-refine_rewrite}, we have $c_l = c_1; c_2$.
\\
According to the operational semantics, we have two possible executions as follows,
\subcaseL{The execution of $c_1$ terminates}
Then let $\trace_1 \in \ftdom, \trace_2 \in \inftdom$ be two execution traces, we have
\[
  \config{c_1; c_2, \trace_l} \to^* \config{c_2, \trace_l \tracecat \trace_1} \uparrow^{\infty} \trace_l \tracecat \trace_1 \tracecat \trace_2
\]
By the trace non-decreasing property in Lemma~\ref{lem:tracenondec}, we have $\trace = \trace_1 \tracecat \trace_2$ and,
\[
  \lcounter(\trace, \pathl(\tpath))  = 
  \lcounter(\trace_1,  \pathl(\tpath))  + \lcounter(\trace_2,  \pathl(\tpath)) = \lcounter(\trace_1,  \pathl(\tpath)) + \bot
\]
By induction hypothesis on $\trace_l$, $c_1$, $\trace_1$ and $c_2$, we know
\[
  \econfig{\outinB(\rprog_1, \tpath)}(\trace_0) \geq \lcounter(\trace_1,  \pathl(\tpath)) 
  \land 
  \econfig{\outinB(\rprog_2, \tpath)}(\trace_0)  \geq \lcounter(\trace_2,  \pathl(\tpath))  
\]
Then we have this case proved as
\[
  \begin{array}{l}
  \econfig{\outinB(\rprog_1, \tpath) + \outinB(\rprog_2, \tpath)}(\trace_0)
  \\
  \geq \econfig{\outinB(\rprog_1, \tpath)}(\trace_0) + \econfig{\outinB(\rprog_2, \tpath)}(\trace_0)
  \\
  \geq \lcounter(\trace_1,  \pathl(\tpath))  + \bot 
  \\
  = \lcounter(\trace, \pathl(\tpath))
  \end{array}
  \] 
  \subcaseL{The execution of $c_1$ is non-terminating}
  In this case, let $\trace_1, \trace_2 \in \inftdom$ be two execution traces, we have
  \[
    \config{c_1; c_2, \trace_l} \uparrow^{\infty} \trace_l \tracecat \trace_1
  \]
  By the trace concatenation in Definition~\ref{def:trace_concate} and the non-decreasing property in Lemma~\ref{lem:tracenondec}, we know $\trace = \trace_1$ and
  \[
    \lcounter(\trace, \pathl(\tpath))  = \lcounter(\trace_1, \pathl(\tpath))  = \bot
  \]
  By induction hypothesis on $\trace_l$, $c_1$, $\trace_1$, we have
  in the first case
  \[
    \econfig{\outinB(\rprog_1, \tpath)}(\trace_0) \geq \lcounter(\trace,  \pathl(\tpath)) 
  \]
  Then we have this case proved as
  \[
    \begin{array}{l}
    \econfig{\outinB(\rprog_1, \tpath) + \outinB(\rprog_2, \tpath)}(\trace_0)
    \\
    \geq \econfig{\outinB(\rprog_1, \tpath)}(\trace_0) + \econfig{\outinB(\rprog_2, \tpath)}(\trace_0)
    \\
    \geq \econfig{\outinB(\rprog_1, \tpath)}(\trace_0)
    \\
    \geq \lcounter(\trace_1,  \pathl(\tpath))
    \\
    = \lcounter(\trace, \pathl(\tpath))
    = \bot
    \end{array}
    \] 
   
\caseL{$\rprog_l = \rpchoose{\rprog_1, \ldots, \rprog_m }$}
We have $\outinB(\rpchoose{\rprog_1, \ldots, \rprog_m }, \tpath) = \max\left\{ \outinB(\rprog_1, \tpath), \ldots, \outinB(\rprog_m, \tpath) \right\}$ by Definition~\ref{alg:alg-refine_rewrite}.
Then it is sufficient to show
\[
  \econfig{\max\left\{ \outinB(\rprog_1, \tpath), \ldots, \outinB(\rprog_m, \tpath) \right\}}(\trace_0) \geq \lcounter(\trace, \pathl(\tpath)).
\]
Let $c_1, \ldots, c_m \in \cdom$ be the while language program corresponding to $\rprog_1, \ldots, \rprog_m$ such that $\algrewrite(c_1) = \rprog_1$, $\ldots$ and  $\algrewrite(c_m) = \rprog_m$.
\\
Since there exists at least a program $c_j$ whose execution under the initial trace $c_r$ doesn't terminate, by induction hypothesis on $c_j$, we know 
\[
  \econfig{\inoutB(\rprog_j, \tpath)}(\trace_0) = \infty.
\]
Then by induction hypothesis on all the other programs and taking the maximum value overall $\outinB(\rprog_i, \tpath)$, we get
\[
  \econfig{\max\left\{ \outinB(\rprog_1, \tpath), \ldots, \outinB(\rprog_m, \tpath) \right\}}(\trace_0) = \infty,
\]
and this case is proved.
\caseL{$\rprog_l = \rprepeat(\rprog')$ }
We have $\outinB(\rprepeat(\rprog'), \tpath) = BD(\rprepeat(\rprog'), c) \times \outinB(\rprog', \tpath)$
by Definition~\ref{alg:alg-refine_rewrite}, and
it is sufficient to show
\[
  \econfig{ BD(\rprepeat(\rprog'), c) \times \outinB(\rprog', \tpath)} (\trace_0) \geq \lcounter(\trace, \pathl(\tpath)) 
\]
Since execution of $\rprepeat(\rprog')$ non-terminates, and it is a loop program,
by Lemma~\ref{lem:loopbound_sound} second sub0case in the non-terminating case, we know
\[
  BD(\rprepeat(\rprog'), c) = \infty.
\]
Then we have this case proved by
\[
  \begin{array}{l}
  \econfig{BD(\rprepeat(\rprog'), c) \times \outinB(\rprog', \tpath)}(\trace_0) 
  \\
  =
  \econfig{ \infty }(\trace_0) 
  \\
  \geq \lcounter(\trace, \pathl(\tpath)) 
  \end{array}
  \]
  \caseL{$\rprog = l : \rprog_l$}
  In this case, we have
  \[
    \outinB(l: \rprog_l, c) = \outinB(\rprog_l, c)
  \]
  Let $c_l \in \cdom$ be the while language program corresponding to $\rprog_l$ such that $\algrewrite(c_l) = \rprog_l$.
  \\
  According to the Algorithm~\ref{alg:alg-refine_rewrite} and the program refinement algorithm REFINE, we know
  $c_l = c_l$.
  \\
  Then we have this case proved trivially by induction hypothesis on $\rprog_l$ and $c_l$ with the same initial and execution trace $\trace_l$.
\caseL{$\rprog_l = l': \rprog'$ and $l':\rprog' \neq \kw{enclosed}(\tpath)$}
$\outinB(l': \rprog', \tpath) = \outinB(\rprog', \tpath)$ 
Since $l': \rprog' \neq \kw{enclosed}(\tpath)$, we know $\tpath$ isn't executed when executing $c_l$, i.e., $\lcounter(\trace, \pathl(\tpath)) = 0$.
\\
Then we have
\[
  \econfig{\outinB(l': \rprog', \tpath)}(\trace_0) = 0 = \lcounter(\trace, \pathl(\tpath)),
  \]
and this case is proved.
\end{proof}

\highlight{
  Informal Discussion: For a simple transition path $\tpath$ only enclosed by one transition path $\rprepeat(\tpath)$, 
  we know $\outinB(\rprog, \tpath) = \outinB(\rprepeat(\tpath), \rprog)$.
  Since $\outinB(\rprepeat(\tpath), \rprog)$ is a sound local bound on the iteration times
  of $\rprepeat(\tpath)$ by assuming all the outside loops executes only once.
  In this sense, $\outinB(\rprepeat(\tpath), \rprog)$ is also a sound bound on the iteration times globally.
  \\
  For a simple transition path $\tpath$ nested in multiple transition paths $\rprog_1, \ldots, \rprog_m$,
  we know $\outinB(\rprog, \tpath) = \prod\limits_{i = 1, \ldots, m}\outinB(\rprog_i, \rprog)$.
  By the same guarantee from $\outinB(\rprog_i, \rprog)$, it is sound to multiply each of them.
}


\subsection{Proof of Lemma~\ref{lem:looprb-sound}}
\label{apdx:looprb-sound}
\begin{lemma}[Soundness of the Loop Reachability-Bound]
    For any loop $l: \rprog$ and a simple transition path $\tpath$ in a refined program, if $l_t: \rprog_t$ is the closest loop enclosing the $\tpath$, then the entering times of $l_t: \rprog_t$ when executing the $l: \rprog$, is bounded by $\lpchB(l: \rprog, \tpath)$.
    \[
      \begin{array}{l}
      \forall c, c_l \in \cdom, \tpath \in \absG(c), \trace_0 \in \tdom_0(c_l), \trace \in \inftdom, \rprog \st 
      \rprog = REFINE(c)
      \land
      l_t: \rprog_t = \kw{enclosed}(\rprog, \tpath)
      \\ \qquad
      \land 
      \rprog_l = \algrewrite(c_l)
      \land
      \config{c_l, \trace_0} \to^* 
      \config{\eskip, \trace_0 \tracecat \trace}
      \\ \qquad
      \implies
      \config{\lpchB(l: \rprog_l, \tpath)}(\trace_0) \geq \counter(\vtrace, l').
      \end{array}
    \]  
  \end{lemma}
  \begin{proof}
    \emph{Loop Reachability-Bound} Soundness.
    \\
Taking an arbitrary program $c$, let $\rprog$ be its refined program and $\tpath$ be an arbitrary transition path in this program.
\\
Let $l_t: \rprog_t = \kw{enclosed}(\rprog, \tpath)$ be the closest loop program where $\tpath$ is nested
and $l: \rprog_l \in \rprog$ be any loop in this program.
\\
Let $\trace_0 \in \tdom_0(c_l)$ be an arbitrary initial trace and $\trace \in \inftdom(c_l)$ be an execution trace such that $\config{c_l, \trace_0} \to^* \config{\eskip, \trace_0 \tracecat \trace}$.
\\
To show that $\lpchB(l: \rprog_l, \tpath)$ bounds the iteration numbers of the loop $l$ w.r.t.
the inner loop $l_t: \rprog_t$ 
such that during the iterations of loop $l$, the nested loop $l_t: \rprog_t$ is executed,
it is sufficient to show
\[
  \config{\lpchB(l: \rprog_l, \tpath)}(\trace_0) \geq \counter(\vtrace, l')
\]
By Definition~\ref{def:looprb}, we have
\[
    \lpchB(l: \rprog_l, \tpath) \triangleq
    \frac{\lpinit(\rprog, \tpath) - \rffinal(\rprog, \tpath)}{\lpinit(\rprog, \tpath) - \lpnext(\rprog, \tpath)}.
\]
The soundness is guaranteed by the operation
 $\frac{\lpinit(\rprog, \tpath) - \rffinal(\rprog, \tpath)}{\lpnext(\rprog, \tpath)}$ and the three states,
 $\lpinit(\rprog, \tpath)$, $\rffinal(\rprog, \tpath)$, $\lpnext(\rprog, \tpath)$ computed in Definition~\ref{def:alg-loopabsstate}.
\\
Let $x = \locbound(\tpath)$ be the ranking function of this path and $c_t \in \cdom$ such that $\rprog_t = REFINE(c_t)$, we prove the soundness of the three states respectively as follows.
\begin{itemize}
  \item \emph{Loop Initial State} Soundness.
  By Definition~\ref{def:alg-loopabsstate}, we have $\lpinit(\rprog, \tpath)$ 
  \[
    \lpinit(l: \rprog, \tpath) \triangleq 
    \arg\max_{l_1}\left\{
      \varinvar(y) + v ~\middle\vert~ 
      \begin{array}{l} 
        (l_1, x' \leq y + v, l_2) \in \reset(x) 
        \\
        \land \absinit(\rprog) \leq l_1 \leq \absinit(\tpath)
      \end{array}
    \right\}
  \]
Let $\trace' \in \tdom$ be the execution trace before first execution of $l_t: \rprog_t$, then we have
\[
  \config{c_l, \trace_0} \to^* \config{c_t;c', \trace_0 \tracecat \trace'} \to^* \config{\eskip, \trace_0 \tracecat \trace}
\]
  By the soundness of the rank estimation in Lemma~\ref{lem:local_bound_sound} and Theorem~\ref{thm:pathinsensitive_rb_soundness}, we know 
  for each variable $y$ such that $(l_1, x' \leq y + v, l_2) \in \reset(x) $,
  $\varinvar(y)$ is the sound upper bound on its maximum value during the execution. Then we have
  \[
    \config{\varinvar(y)}(\trace_0) \geq \max \left\{ v ~|~  \env(\trace) y = v \right\}  \geq \env(\trace') y 
  \]
%
  In this sense, we have 
  $
  \arg\max_{l_1}\left\{
    \varinvar(y) + v ~\middle\vert~ 
    \begin{array}{l} 
      (l_1, x' \leq y + v, l_2) \in \reset(x) 
      \\
      \land \absinit(\rprog) \leq l_1 \leq \absinit(\tpath)
    \end{array}
  \right\}$
  is a sound initial value estimation of the value of $x$ before first execution of $l_t: \rprog_t$, i.e.,
  \[
    \config{\arg\max_{l_1}\left\{
      \varinvar(y) + v ~\middle\vert~ 
      \begin{array}{l} 
        (l_1, x' \leq y + v, l_2) \in \reset(x) 
        \\
        \land \absinit(\rprog) \leq l_1 \leq \absinit(\tpath)
      \end{array}
    \right\}}(\trace_0) 
    \geq \max \left\{ v ~|~  \env(\trace) x = v \right\}  \geq \env(\trace') x
    \]
  %
  \item  $\lpnext(\rprog, \tpath)$ computes the variables states of the $\tpath$
  after visited the program point $l$ the second time and before visiting any other program point.
  \\
  By Definition~\ref{def:alg-loopabsstate}, we have $\lpnext(l:\rprog_, \tpath)$ as
  \[
    \begin{array}{l}
      \lpnext(l: \rprog, \tpath) \triangleq 
      \begin{array}{l}
        \sum\limits_{\absevent \in \inc(x) }
        \left\{ 
        v ~\middle\vert~ \absevent = (l', x' \leq x + v, \_) \land  l' \in \rprog 
        \land l' \notin \tpath \right\}
        \\ \qquad 
        + \arg\max\limits_{l_2 }
            \left\{ \varinvar(y) + v ~\middle\vert~ 
            (l_1, x' \leq y + v, l_2) \in \reset(x) \land l_1 \in \rprog \land l_1 \notin \tpath\right\}
        \\ \qquad 
        - \sum\limits_{ \absevent \in \dec(x) }\left\{ 
        v 
        ~\middle\vert~ \absevent = (l', x' \leq x + v, \_) \land l' \in \rprog \land l' \notin \tpath \right\}
        \\ \qquad 
        + \outinB(\tpath) \times \rfnext(\tpath)
      \end{array}
    \end{array}
  \]
  Let $\trace' \in \tdom$ be the execution trace before the execution of $l_t: \rprog_t$ in the second iteration of $l:\rprog_l$, then we have
  \[
    \config{c_l, \trace_0} \to^* \config{c_t;c', \trace_0 \tracecat \trace'} \to^* \config{\eskip, \trace_0 \tracecat \trace}
  \]
    By the soundness of the rank estimation in Lemma~\ref{lem:local_bound_sound} and Theorem~\ref{thm:pathinsensitive_rb_soundness}, we know 
    for each variable $y$ such that $(l_1, x' \leq y + v, l_2) \in \reset(x) $,
    $\varinvar(y)$ is the sound upper bound on its maximum value during the execution. Then we have
    \[
      \config{\varinvar(y)}(\trace_0) \geq \max \left\{ v ~|~  \env(\trace) y = v \right\}  \geq \env(\trace') y 
    \]
    Same for each abstract event such that $((l', x' \leq x + v, \_) \in \inc(x) $ and $(l', x' \leq x + v, \_) \in \dec(x)$,
    we know 
    $ \begin{array}{l}
      \sum\limits_{\absevent \in \inc(x) }
      \left\{ 
      v ~\middle\vert~ \absevent = (l', x' \leq x + v, \_) \land  l' \in \rprog 
      \land l' \notin \tpath \right\}
      \\ \qquad 
      + \arg\max\limits_{l_2 }
          \left\{ \varinvar(y) + v ~\middle\vert~ 
          (l_1, x' \leq y + v, l_2) \in \reset(x) \land l_1 \in \rprog \land l_1 \notin \tpath\right\}
      \\ \qquad 
      - \sum\limits_{ \absevent \in \dec(x) }\left\{ 
      v 
      ~\middle\vert~ \absevent = (l', x' \leq x + v, \_) \land l' \in \rprog \land l' \notin \tpath \right\}
    \end{array}
    $ bounds the value modification of $x$ in one iteration of $l: \rprog_l$.
  %
  \\
  By Lemma~\ref{lem:pathlocalrb-sound} and Lemma~\ref{lem:loopbound_sound}, we also know $BD(\rprog_t)$ is sound upper bound of $l_t$'s iterations and $\rfnext(\tpath)$ is the upper bound of $x$'s value modification in one iteration of $l_t: \rprog_t$.
  \\
  In this sense, we know
  $\begin{array}{l}
      \sum\limits_{\absevent \in \inc(x) }
      \left\{ 
      v ~\middle\vert~ \absevent = (l', x' \leq x + v, \_) \land  l' \in \rprog 
      \land l' \notin \tpath \right\}
      \\ \qquad 
      + \arg\max\limits_{l_2 }
          \left\{ \varinvar(y) + v ~\middle\vert~ 
          (l_1, x' \leq y + v, l_2) \in \reset(x) \land l_1 \in \rprog \land l_1 \notin \tpath\right\}
      \\ \qquad 
      - \sum\limits_{ \absevent \in \dec(x) }\left\{ 
      v 
      ~\middle\vert~ \absevent = (l', x' \leq x + v, \_) \land l' \in \rprog \land l' \notin \tpath \right\}
      \\ \qquad 
      + BD(l_t: \rprog_t) \times \rfnext(\tpath)
    \end{array}
    $
    is a sound upper bound on the value of $x$ modified from first iteration of $l: \rprog$ until the second iteration and before the execution of $l_t: \rprog_t$, i.e.,
    \[
      \begin{array}{l}
        \config{
        \lpinit(l: \rprog, \tpath)}(\trace_0)
        -
      \config{
          \begin{array}{l}
            \sum\limits_{\absevent \in \inc(x) }
            \left\{ 
            v ~\middle\vert~ \absevent = (l', x' \leq x + v, \_) \land  l' \in \rprog 
            \land l' \notin \tpath \right\}
            \\ \qquad 
            + \arg\max\limits_{l_2 }
                \left\{ \varinvar(y) + v ~\middle\vert~ 
                (l_1, x' \leq y + v, l_2) \in \reset(x) \land l_1 \in \rprog \land l_1 \notin \tpath\right\}
            \\ \qquad 
            - \sum\limits_{ \absevent \in \dec(x) }\left\{ 
            v 
            ~\middle\vert~ \absevent = (l', x' \leq x + v, \_) \land l' \in \rprog \land l' \notin \tpath \right\}
            \\ \qquad 
            + BD(l_t: \rprog_t) \times \rfnext(\tpath)
          \end{array}
      }(\trace_0) \\
      \leq \max \left\{ v ~|~  \env(\trace) x = v \right\}  \leq \env(\trace') x
    \end{array}
    \]
  %
  \item $\rffinal(\tpath)$ computes the value of $\locbound(\tpath)$ after the iteration of $l_t:\rprog_t$ finished.
  \\
  Let $\trace' \in \tdom$ be the execution trace after the execution of $l_t: \rprog_t$ in the first iteration of $l:\rprog_l$ and the $c'$ be the program after the execution of $c_t$ in the first iteration of $l:\rprog_l$,  then we have
  \[
    \config{c_l, \trace_0} \to^* \config{c', \trace_0 \tracecat \trace'} \to^* \config{\eskip, \trace_0 \tracecat \trace}
  \]
  By Lemma~\ref{lem:loopbound_sound}, we know $\rffinal(\tpath)$ computes the lower bound on the value of $\locbound(\tpath)$ after the iteration of $l_t:\rprog_t$ finished.
  \\
  Then we have 
  $\rffinal(\tpath) \leq \env(\trace') x$ and $\lpinit(\rprog, \tpath) - \rffinal(\rprog, \tpath) \geq \env(\trace') x$.
\end{itemize}
Then there are two cases as follows.
\caseL{$l_t: \rprog_t \notin \rprog_l$}
In this case, we know loop $l_t: \rprog_t$ will never be entered, and this case is proved.

\caseL{$l_t: \rprog_t \in \rprog_l$}
In this case, based on the soundness of the three states above, we know 
$\frac{\lpinit(\rprog, \tpath) - \rffinal(\rprog, \tpath)}{\lpinit(\rprog, \tpath) - \lpnext(\rprog, \tpath)}$ 
soundly computes the maximum number of $l: \rprog_l$'s iteration times such that during these iterations, the nested loop $l_t: \rprog_t$ is executed.
\end{proof}

\subsection{Proof of Lemma~\ref{lem:pathrb-sound}}
\label{apdx:pathrb-sound}
\begin{lemma}[Soundness of Path Reachability Bound]
  % \label{lem:pathrb-sound}
  For any program with its refined program $\rprog$ and a simple transition path $\tpath$ in this program,
  the execution times of $\tpath$ when executing the $\rprog$ is bounded by $\inoutB(\rprog, \tpath)$.
  \[
    \begin{array}{l}
    \forall c \in \cdom, \tpath \in \absG(c), \trace_0 \in \tdom_0(c), \trace \in \inftdom, \rprog \st 
    \rprog = REFINE(c)
    \land
    \config{c, \trace_0} \to^* 
    \config{{\eskip, \trace_0 \tracecat \trace}}
    \\ \qquad
    \implies
    \config{\inoutB(\rprog, \tpath)}(\trace_0) \geq \counter(\vtrace, L(\tpath)).
    \end{array}
    \]
\end{lemma}
%
\begin{proof}
\emph{Soundness} of the \emph{Path Reachability Bound}.
  \\
  Taking an arbitrary program $c$, let $\rprog$ be its refined program and $\tpath$ be an arbitrary transition path in this program.
  \\
Taking an arbitrary initial trace $\trace_0 \in \tdom_0(c)$  and an execution trace $\trace \in \tdom$
 such that $\config{{c}, \trace} \to^{*} \config{\eskip, \trace_0 \tracecat \vtrace}$,
 it is sufficient to show,
 %
 \[
 \config{\inoutB(\rprog, \tpath)}(\trace_0) \geq \counter(\vtrace, L(\tpath)) .
 \]
By induction on program $\rprog$, we have the same cases as in the proof of \emph{Path Local Reachability-Bound} Soundness in Appendix~\ref{apdx:pathlocalrb-sound}.
The proof of the first four cases are almost the same as the proof of Lemma~\ref{lem:pathlocalrb-sound}, I don't unfold and repeat all the detail in this proof.
\caseL{$\rprog = \tpath$}
By Definition~\ref{def:pathrb}, we have $\inoutB(\tpath, \tpath) = 1$, which is sound.
\caseL{$\rprog = \tpath'$}
By Definition~\ref{def:pathrb}, we have $\inoutB(\tpath', \tpath) = \highlight{0}$.
\\
Since $\tpath' \neq \tpath$, we know $\tpath$ isn't executed when executing $\tpath'$, which is sound.
\caseL{$\rprog = \rprog_1;\rprog_2$}
By Definition~\ref{def:pathrb}, we have $\inoutB(\rprog_1;\rprog_2, \tpath) = \inoutB(\rprog_1, \tpath) + \inoutB(\rprog_2, \tpath) $.
\\
Then we have this case proved by induction hypothesis on $\inoutB(\rprog_1, \tpath)$ and $\inoutB(\rprog_2, \tpath)$ in the same way as in Appendix~\ref{apdx:pathlocalrb-sound}.
\caseL{$\rprog = \rpchoose{\rprog_1, \ldots, \rprog_m }$}
By Definition~\ref{def:pathrb}, we have $\inoutB(\rpchoose{\rprog_1, \ldots, \rprog_m }, \tpath) = 
\max\left\{ \inoutB(\rprog_1, \tpath), \ldots, \inoutB(\rprog_m, \tpath) \right\}$ 
By induction hypothesis on $\rprog_1, \ldots, \rprog_m$, we have this case proved in the same way.

\caseL{$\rprog = l': \rprog'$ }
By Definition~\ref{def:pathrb}, we have 
\[  
  \begin{array}{rcl}
  \inoutB(l': \rprog', \tpath) & \triangleq & 
  \highlight{\outinB(\rprog', \tpath), \qquad \text{if } l': \rprog' = \kw{enclosed}(\tpath)}
  \\ &  \triangleq & 
  0, \qquad \qquad \qquad \quad ~~ \text{if }  \kw{enclosed}(\tpath) \notin \kw{enclosing}(\rprog')
  \\ &  \triangleq & 
  \lpchB(l':\rprog', \tpath )
  \times \max\limits_{l': \rprog' = \kw{enclosed}(l'':\rprog'')}
  \left\{ \inoutB(l'':\rprog'', \tpath) \right\}, o.w. 
  \end{array}
\]
%
Then, we have the following three cases:
\subcaseL{$l': \rprog' = \kw{enclosed}(\tpath)$ }
In this case, we have $\inoutB(l': \rprog', \tpath) = {\outinB(\rprog, \tpath)}$.
\\
Since $l': \rprog' = \kw{enclosed}(\tpath)$, we know $l': \rprog'$ is the closet loop where $\tpath$ is nested.
\\
By Lemma~\ref{lem:pathlocalrb-sound}, we know $\outinB(\rprog', \tpath)$ is the
sound upper bound on $\tpath$'s reaching times when executing $l': \rprog'$. 
\\
So we have this case proved by
\[
  \config{\inoutB(\rprog, \tpath)}(\trace_0) = \config{\outinB(\rprog, \tpath)}(\trace_0) \geq \counter(\trace, L(\tpath)).
\]
%
\subcaseL{$\kw{enclosed}(\tpath) \notin \kw{enclosing}(\rprog')$}
In this case, we have $\inoutB(l: \rprog_l, \tpath) = 0$.
\\
Since $\tpath$ doesn't belong to this loop, $0$ is sound because $\tpath$ will not be executed during the execution of $\rprog_l$.
%
\subcaseL{ $o.w.$ }
In this case, we have 
\[
  \inoutB(l': \rprog', \tpath) = \lpchB(l':\rprog', \tpath )
  \times \max\limits_{l': \rprog' = \kw{enclosed}(l'':\rprog'')}
  \{\inoutB(l'':\rprog'', \tpath)\}.
\]
Then we need to show:
\[
  \config{\lpchB(l':\rprog', \tpath )
  \times \max\limits_{l': \rprog' = \kw{enclosed}(l'':\rprog'')}
  \{\inoutB(l'':\rprog'', \tpath)\}}(\trace_0) 
  \geq \counter(\trace, L(\tpath)).
  \]
%
Since $\rprog = l': \rprog'$, according to Algorithm~\ref{alg:alg-refine_rewrite}, we know $c$ has the form
$\ewhile \clabel{\bexpr}^{l'} \edo c'$ for some boolean expression $\bexpr$.
\\
For each $l'':\rprog''$ such that $l': \rprog' = \kw{enclosed}(l'':\rprog'')$, 
let $c''\in \cdom$ be the while program corresponding to $\rprog''$ such that $\algrewrite(c'') = \rprog''$,
$c_1, c_2 \in \cdom$ such that $c' = c_1; c''; c_2$
$\trace'' \in \inftdom$ be the execution trace of $c''$ and $\trace_1, \trace_2 \in \inftdom$ be the execution traces before and after executing $c''$, then we have
\[
  \begin{array}{l}
  \config{c, \trace_0} \to \config{c';c, \trace_0 \tracecat [(\bexpr, l', \etrue)]} \\
  \to^* \config{c'';c_2;c, \trace_0 \tracecat [(\bexpr, l', \etrue)] \tracecat \trace_1} \\
  \to^* \config{c_2;c, \trace_0 \tracecat [(\bexpr, l', \etrue)] \tracecat \trace_1 \tracecat \trace''} \\
  \to^* \config{\eskip, \trace_0 \tracecat [(\bexpr, l', \etrue)] \tracecat \trace_1 \tracecat \trace'' \tracecat \trace_2}
  \end{array}
\]
By the trace non-decreasing property in Lemma~\ref{lem:tracenondec},
we know $\trace = [(\bexpr, l', \etrue)] \tracecat \trace_1 \tracecat \trace'' \tracecat \trace_2$.
\\
By induction hypothesis on $\inoutB(l'':\rprog'', \tpath)$ and $c''$, we have
\[
  \config{\inoutB(l'':\rprog'', \tpath)}(\trace_0 \tracecat [(\bexpr, l', \etrue)] \tracecat \trace_1) 
  \geq \counter(\trace'', L(\tpath)).
\]
Then for all $l'':\rprog''$ such that $l': \rprog' = \kw{enclosed}(l'':\rprog'')$, we have 
\[
  \begin{array}{l}
  \config{\max\limits_{l': \rprog' = \kw{enclosed}(l'':\rprog'')} \{\inoutB(l'':\rprog'', \tpath)\}}
  (\trace_0 \tracecat [(\bexpr, l', \etrue)] \tracecat \trace_1)
  \geq \counter(\trace'', L(\tpath))
  \end{array}
\]
Since $\trace''$ is the execution trace for $c''$ in one iteration of $c'$,
we know $\trace_2$ contains multiple execution traces of  $c''$ in all iterations of $c'$.
\\
There are two ways to prove the soundness.
\\
By Lemma~\ref{apdx:loopbound-sound}, we know $BD(l':\rprog') \geq$ the iteration numbers of $c'$. Then we have
\[
  \begin{array}{l}
  \config{ BD(l':\rprog')}(\trace_0) \times \counter(\trace'', L(\tpath)) \\
  \geq \counter(\trace'' \tracecat \trace_2, L(\tpath)) \\
  = \counter([(\bexpr, l', \etrue)] \tracecat \trace_1 \tracecat \trace'' \tracecat \trace_2)\\
  = \counter(\trace, L(\tpath))
  \end{array}
\]
Then we have this case proved as the following.
\[
  \begin{array}{l}
  \config{ BD(l':\rprog') \times \max\limits_{l': \rprog' = \kw{enclosed}(l'':\rprog'')} \{\inoutB(l'':\rprog'', \tpath)\}}(\trace_0)\\
  \geq \config{ BD(l':\rprog')}(\trace_0) \times 
  \config{\max\limits_{l': \rprog' = \kw{enclosed}(l'':\rprog'')} \{\inoutB(l'':\rprog'', \tpath)\}}
  (\trace_0 \tracecat [(\bexpr, l', \etrue)] \tracecat \trace_1)\\
  \geq \config{ BD(l':\rprog')}(\trace_0) \times \counter(\trace'', L(\tpath)) \\
  \geq \counter(\trace, L(\tpath))
  \end{array}
\]
However, this isn't precise enough since $\tpath$ might not be executed in every iteration of $c'$ because it is located in nested loops. 
\\
In this sense, we prove the soundness of the tighter bound based on $\lpchB(l:\rprog', \tpath)$ and Lemma~\ref{lem:looprb-sound}.
\\
By Lemma~\ref{lem:looprb-sound}, we know $\lpchB(l:\rprog', \tpath)$ is a sound upper bound on the maximum number of $l': \rprog'$'s iteration times such that during these iterations, the $\tpath$ is executed.
\\
In this sense, $\lpchB(l:\rprog', \tpath) \times \inoutB(\rprog, \tpath)$ is a sound upper bound on $\tpath$'s execution times during the execution of $l': \rprog'$.
\\
Then we have
\[
  \begin{array}{l}
  \config{ \lpchB(l:\rprog', \tpath)}(\trace_0) \times \counter(\trace'', L(\tpath)) \\
  \geq \counter(\trace'' \tracecat \trace_2, L(\tpath))\\
  = \counter([(\bexpr, l', \etrue)] \tracecat \trace_1 \tracecat \trace'' \tracecat \trace_2)\\
  = \counter(\trace, L(\tpath))
  \end{array}
\]
Then we have this case proved as the following.
\[
  \begin{array}{l}
  \inoutB(l':\rprog') \\
  = \config{ \lpchB(l:\rprog', \tpath) \times \max\limits_{l': \rprog' = \kw{enclosed}(l'':\rprog'')} \{\inoutB(l'':\rprog'', \tpath)\}}(\trace_0) \\
  \geq \config{ \lpchB(l:\rprog', \tpath) }(\trace_0) \times
  \config{\max\limits_{l': \rprog' = \kw{enclosed}(l'':\rprog'')} \{\inoutB(l'':\rprog'', \tpath)\}} 
  (\trace_0 \tracecat [(\bexpr, l', \etrue)] \tracecat \trace_1) \\
  \geq \config{ \lpchB(l:\rprog', \tpath)}(\trace_0) \times \counter(\trace'', L(\tpath)) \\
  \geq \counter(\trace, L(\tpath))
  \end{array}
\]
%
\caseL{$\rprog = \rprepeat(\rprog')$}
By Definition~\ref{def:pathrb}, we have
\[
  \inoutB(\rprepeat(\rprog'), \tpath) = \inoutB(\rprepeat(\rprog'), \rprog) \times \inoutB(\rprog', \tpath).
\]
We will never meet this case by the program rewriting in Algorithm~\ref{alg:alg-refine_rewrite}.
\\
Because $\rprepeat(\rprog')$ only shows up in a loop, we always match the case $l' : \rprepeat(\rprog')$ in the previous case before match this case.
%
\end{proof}





\subsection{Proof of Soundness of Path-Sensitive Reachability Bounds Estimation, Theorem~\ref{thm:pathsensitive_rb_soundness}}
\label{apdx:psrb-sound}
\begin{thm}[Soundness of the Path Sensitive Reachability Bound Estimation]
  Given a program ${c}$, for every label $l$ of this program $c$ such that $(l, w) \in \exeRB(c)$, 
  and any initial trace $\trace_0 \in \mathcal{T}_0(c)$ with 
  and $\config{ppsRB(l), \trace_0} \earrow v$,
  we have $ w(\trace_0) \leq v $.
  %
  \[
    \begin{array}{l}
    \forall (l, w_{t}) \in \exeRB(c),
    % (x^l, w_{p}) \in \progV, 
    \trace_0 \in \mathcal{T}_0(c), 
    v \in \mathbb{N} \st
    \config{psRB(l), \trace_0} \earrow v
    \implies
    % \right\} 
    w_{t}(\trace_0) \leq v
    \end{array}
  \]
  \end{thm}
Proof Summary:
\begin{enumerate}
\item Step 1 - 2: rely on soundness of path insensitive reachability bound by Theorem~\ref{thm:pathinsensitive_rb_soundness}
\\
\item Step 3: Program Rephrase : soundness of Step - 1
By the soundness of the abstract execution trace in Lemma~\ref{lem:abscfg_sound}
and the uniqueness of the abstract execution trace in Lemma~\ref{lem:absevent_unique},
the rephrased program $p$ is equivalent to the program $c$.
\\
\item Step 4: While loop Refinement : soundness of Paper \cite{GulwaniJK09}
\\
\item Step 5: Outside-In 
$rLB(\rprog)$ is local bound of execution time of $\rprog$ without considering outside loops.
\\
\item Step 6: Inside-Out
\begin{enumerate}
\item: $rpLB(LOOP_t, \tpath)$, the local bound of execution time of 
transition path in the closest $LOOP$.
by repeat chain 
\item $RB(\tpath)$, the bound of execution time for every transition path in the program
\\
By collecting loop chains.
\\
For every $LOOP_i$ on loop chain of $\tpath$, $lpRB(LOOP_i, LOOP_t)$ bound 
% the execution times of $LOOP_t$ inside the $LOOP_i$
% bound for 
the number of $LOOP_i$'s execution iterations
        %  will 
        % bound the execution times of $LOOP_{t_0}$
        % in each single execution of the $LOOP_{t_i}$ for every
        such that, during these iterations, $\tpath$ will be executed. 
\\
Then, $lRB(LOOP_t) = \prod\limits_{LOOP_i} lpRB(LOOP_i, LOOP_t)$ is the bound of execution time for the $LOOP_t$ in program, globally.
\\
$RB(\tpath) = lRB(LOOP_t) \times rpLB(LOOP_t, \tpath) $ is the bound of execution time for transition path $\tpath$ 
in program globally.
\end{enumerate}
\item Reachability Bound for every location.
$RB(l) = \sum_{l \in \tpath}RB(\tpath)$ is the bound of execution times for location $l$.

\end{enumerate}
  \begin{proof}
        Taking arbitrary program $c$, a pair  $(l, w) \in \exeRB(c)$, 
        and an initial trace $\trace_0 \in \mathcal{T}_0(c)$,  
        we know $w_t$ is the execution-based 
        reachability bound 
        for label $l \in \lvar(c)$, 
        % with a natural number
        % $v \in \mathbb{N}$ satisfying
        % $\config{psRB(l), \trace_0} \earrow v$, 
        it is sufficient to show,
        % \\
        \[
            \forall v \in \mathbb{N} \st \config{psRB(l), \trace_0} \earrow v
            \implies w_{t}(\trace_0) \leq v.\]
        By the Definition~\ref{def:exe_rb}, 
        let $\trace \in \mathcal{T}$ be an arbitrary execution trace 
        satisfying 
        $\config{{c}, \trace_0} \to^{*} \config{\eskip, \trace_0 \tracecat \vtrace} $,
        it is sufficient to show 
        \[
            \counter(\vtrace, l) \leq v.
        \]
        Let $\rprog$ be the rephrased program for $c$,
        by computation of $psRB(l)$ in Definition~\ref{def:label_psrb}, it is sufficient to show 
        \[
          \forall v \in \mathbb{N} \st 
          \config{\sum\limits_{\tpath \in \rprog \land 
          l \in \tpath} tRB(\tpath), \trace_0
          } \earrow v \implies  \counter(\vtrace, l) \leq v\]
          %
          By the soundness of the abstract execution trace in Lemma~\ref{lem:abscfg_sound}, 
          the uniqueness of the abstract execution trace in Lemma~\ref{lem:absevent_unique},
          we have an abstract event $\absevent = (l, \_, \_) \in \absE(c)$.
          \\
          Then we know there exists $\tpath \in \paths(\absG(c))$ such that 
          $l \in \tpath$.
          \\
          There are two cases as follows,
        $\tpath \in SCC(\absG(c))$ and $\tpath \notin SCC(\absG(c))$.
        %   \\
          \caseL{$\tpath \not\in SCC(\absG(c))$}
          In case of  $\tpath \not\in SCC(\absG(c))$, there is only one $\tpath$ contains $l$ and $tRB(\tpath) = 1$.
          \\
          Then we have $\config{\sum\limits_{\tpath} 1, \trace_0
          } \earrow 1$ and $ \counter(\vtrace, l) \leq 1$.
          \\
          This case is proved.
          \caseL{$\tpath \in SCC(\absG(c))$}
          In this case, let $TP$ be the set of all transition paths containing 
          label $l$ in program $\rprog$, then it is sufficient to show 
          \[
          \forall v \in \mathbb{N} \st 
          \config{\sum\limits_{\tpath \in TP} tRB(\tpath), \trace_0
          } \earrow v \implies  \counter(\vtrace, l) \leq v
          \]
          For each transition path $\tpath \in TP$, let $\trace_t$ be the execution trace 
          containing all the executions of $\tpath$
          under initial trace $\trace_0$, then we know 
          \[
            \counter(\vtrace, l) \leq \sum_{\tpath \in TP} \counter(\vtrace_t, l) 
          \]
          Then it is sufficient to show 
          \[
          \forall v \in \mathbb{N} \st 
          \config{\sum\limits_{\tpath \in TP} tRB(\tpath), \trace_0
          } \earrow v \implies \sum_{\tpath \in TP} \counter(\vtrace_t, l) \leq v
          \]
          Taking arbitrary transition path $\tpath \in TP$, it is sufficient to show 
          \[
            \forall v \in \mathbb{N} \st 
            \config{tRB(\tpath), \trace_0
            } \earrow v \implies \counter(\vtrace_t, l) \leq v
            \]
                    %
          By the \emph{Global Loop Bound} computation and the uniqueness of the 
          nested loop chain from Lemma~\ref{lem:lpch_unique}, 
          we have the only one loop chain $lpch(\tpath)$ for $\tpath$.
          \\
          Then it is sufficient to show 
          \[
            \forall v \in \mathbb{N} \st 
          \config{\prod_{LOOP_{t_i} \in lpch(\tpath)} lpRB(LOOP_{t_i}, \tpath), \trace_0
          } \earrow v \implies  \counter(\vtrace_t, l) \leq v
        \]
        For each transition path $\tpath \in TP$, 
        let $\{LOOP_{t_n} \to \cdots \to LOOP_{t_0} \to \tpath\}$
        be its loop chain $lpch(\tpath)$. 
        Let $\rprog_{t_i}$ be the refined program for every while loop 
        with label $LOOP_{t_i} \in lpch(\tpath)$ such that,
        \[
          \rprog_{t_{i}} = \cdots, LOOP_{t_{i - 1}}:\rprog_{t_{i - 1}}, i = n, \cdots, {1}.
        \] 
        Let $\trace_{t_i}, \trace_{t_i}' \in \mathcal{T}$ for $i = n, \cdots, t$ and $\trace' \in \mathcal{T}$ be the execution traces satisfying
        %
        \[
          \begin{array}{l}
          \config{c, \trace_0} \to^{*} \config{\rprog_{t_n};c', \trace_0 \tracecat \trace_{t_n}'}
        \to^{*} \config{\rprog_{t_{n - 1}};c', \trace_0 \tracecat \trace_{t_n}' \tracecat \trace_{t_n}}
        \to^{*} 
        \\ \qquad 
        \cdots \to^{*} \config{c', \trace_0 \tracecat \trace_{t_n}' \tracecat \cdots \tracecat
        \trace_{t_0}'} \to^{*} \config{\eskip, \trace_0 \tracecat \trace_{t_n}' \tracecat \cdots 
        \tracecat \trace_{t_0}' \tracecat \trace'}.
          \end{array}
        \]
        % It is sufficient to show 
        % \[
        %     \forall v \in \mathbb{N} \st
        %     \config{
        % %   \prod_{LOOP_i \in lpch(\tpath)}
        % lpRB(LOOP_i, \tpath), \trace_0
        %   } \earrow v \implies  \counter(\vtrace, l) \leq v
        % \]        
        By the label consistency and computation of the 
        \emph{Nest Loop Chain}, we know
        \[
          \counter(\vtrace_t, l) \leq \counter(\trace_{t_n}' \tracecat \cdots 
          \tracecat \trace_{t_0}', l)
          \] 
          % $\trace_t = \trace_i' \tracecat \trace_i$, then
        Then, it is sufficient to show 
        \[
          \forall v \in \mathbb{N} \st 
        \config{\prod_{LOOP_{t_i} \in lpch(\tpath)} lpRB(LOOP_{t_i}, \tpath), \trace_0
        } \earrow v 
        \implies  \counter(\trace_{t_n}' \tracecat \cdots 
        \tracecat \trace_{t_0}', l) \leq v
      \]
%
Let $\trace_{t_i} \in \mathcal{T}$ for $i = n, \cdots, t$ 
% and $\trace' \in \mathcal{T}$ 
be the execution traces corresponds to the single execution of $\rprog_i$ under initial trace $\trace_0$.
\\
Since the evaluation results in different iterations doesn't change the program label,
we know $\counter(\trace_{t_i}, l) = \counter(\trace_{t_i}', l)$ for two different iterations of $\rprog_i$.
\\
Then we know:
\[
  \counter(\vtrace_t, l) \leq 
  \counter(\trace_{t_n}, l_{t_{n-1}}) \times \cdots 
  \counter(\trace_{t_1}, l_{t_{0}}) \times \counter(\trace_{t_0}, l)
  = \prod\limits_{LOOP_{t_i} \in lpch(\tpath)} \counter(\trace_{t_i}, l_{t_{i-1}})
  \]
%
Then, it is sufficient to show 
\[
  \forall v \in \mathbb{N} \st 
\config{\prod_{LOOP_{t_i} \in lpch(\tpath)} lpRB(LOOP_{t_i}, \tpath), \trace_0
} \earrow v 
\implies  
\prod\limits_{LOOP_{t_i} \in lpch(\tpath)} \counter(\trace_{t_i}, l_{t_{i-1}})
\]
%
Let $r_n, \cdots, r_1$ be the number of iterations for each $LOOP_{t_n}, \cdots, LOOP_{t_0} \in lpch(\tpath)$, 
such that
during the execution,
the $\tpath$ is executed in the $r_i$ iterations of loop $LOOP_{t_i}$.
\\
Then we know
\[
  \counter(\vtrace_t, l) \leq 
  r_n \times \cdots 
  r_1 \times \counter(\trace_{t_0}, l)
  = 
  % \counter(\trace_{t_0}, l) \times 
  \prod\limits_{LOOP_{t_i} \in lpch(\tpath)} r_i
  \]
%  for each $LOOP_{t_n}, \cdots, LOOP_{t_0} \in lpch(\tpath)$ where 
%
Then taking arbitrary while loop $LOOP_i$ from $lpch(\tpath)$, it is sufficient to show
% By the computation of \emph{Nested Loop Bound} in Definition~\ref{def:nested_loop_bound}, we know 
% $lpRB(LOOP_{t_i}, \tpath)$ is the
% bound for the number of $LOOP_{t_i}$'s execution iterations
% %  will 
% % bound the execution times of $LOOP_{t_0}$
% % in each single execution of the $LOOP_{t_i}$ for every
% such that, during these iterations, $LOOP_{t_0}$ will be executed. 
% $LOOP_{t_i} \in lpch(\tpath)$.
% 
        \[
          \forall v \in \mathbb{N} \st 
          \config{lpRB(LOOP_i, \tpath), \trace_0
          } \earrow v_i 
          \implies  
          r_i \leq v_i
        \]   
  By the computation of \emph{Nested Loop Bound} in Definition~\ref{def:nested_loop_bound},
        there are two cases:
  \\
  Informal Proof Summary: 
  \\
  By the computation of \emph{Nested Loop Bound} in Definition~\ref{def:nested_loop_bound}, we know 
$lpRB(LOOP_{t_i}, \tpath)$ is the
bound for the number of $LOOP_{t_i}$'s execution iterations
%  will 
% bound the execution times of $LOOP_{t_0}$
% in each single execution of the $LOOP_{t_i}$ for every
such that, during these iterations, $LOOP_{t_0}$ will be executed. 
$LOOP_{t_i} \in lpch(\tpath)$.
%
\subcaseL{$rpLB(LOOP_i, \tpath) \neq \bot$}
        In this case, we have $lpRB(LOOP_i, \tpath) = rpLB(LOOP_i, \tpath)$.
        \\
        By computation of \emph{Local Repeat Chain Bound}, we know 
        $LOOP_i$ is the closest while loop containing transition path $\tpath$.
        \\
        Then we know $LOOP_i = LOOP_{t_0}$, and 
        \[
          r_0 = \counter(\trace_{t_0}, l)
        \]
    Since $\trace_{t_0}$ is the trace of a single execution of the while loop body $\rprog_{t_0}$ for $LOOP_{t_0}$,
    it is equivalent to show that $rpLB(LOOP_{t_0}, \tpath)$ is the local bound for $\tpath$, i.e.,
    % \\
        % It is sufficient to show:
        \[
            \forall v \in \mathbb{N} \st
            \config{
        %   \prod_{LOOP_i \in lpch(\tpath)}
        \max \left\{ \prod\limits_{\rprog_j \in ch}  rLB(\rprog_j) 
        ~\middle\vert~ ch \in rp\mathcal{C}(LOOP_{t_0}, \tpath) \right\}, \trace_0
          } \earrow v \implies  \counter(\vtrace_{t_0}, l) \leq v
        \]
    % i.e., since $\trace_{t_0}$ is the trace of single execution of $\rprog_{t_0}$ for $LOOP_{t_0}$,
    % it is equivalent to show that $rpLB(LOOP_{t_0}, \tpath)$ is the local bound for $\tpath$.
    % \\
    For each $ch \in rp\mathcal{C}(LOOP_i, \tpath)$ and 
    every $\rprog_j \in ch$, let $ v_j \in \mathbb{N}$  and 
            $\trace_j', \trace_j \in \mathcal{T}$ be arbitrary natural number
            and evaluation traces satisfying 
            \[
                \config{ c, \trace_0} \to^{*} 
                \config{\rprog_j, \trace_0 \tracecat \trace_j'} \to^*
                \config{\eskip, \trace_0 \tracecat \trace_j' \tracecat \trace_j}
                \land
                 \config{rLB(\rprog_j),\trace_0 } \earrow v_j.
            \]
        By the soundness of the computation of $rLB(\rprog_j)$, 
       i.e., the soundness of \textbf{Outside-In Algorithm} for the local reachability
                  bound of the $\rprog_j$, we know 
                  \[
                    \counter(\vtrace_j, l) \leq v_j
                    \]
Then we have 
\[
    \counter(\vtrace_{t_0}, l) 
    \leq \max_{ch \in rp\mathcal{C}(LOOP_i, \tpath)} 
    \prod\limits_{\rprog_j \in ch}
    \counter(\vtrace_j, l) \leq 
    \max \left\{ \prod\limits_{\rprog_j \in ch}  v_j 
        ~\middle\vert~ ch \in rp\mathcal{C}(LOOP_i, \tpath) \right\}
    \]
Since $ v_j \in \mathbb{N}$  and 
$\trace_j', \trace_j \in \mathcal{T}$ are arbitrary natural number
and evaluation traces satisfying the assumptions, we have this case proved.
%
        \subcaseL{$rpLB(LOOP_i, \tpath) = \bot$}
        In this case, we know 
        $lpRB(LOOP_i, \tpath) =
        % \prod\limits_{\rprog_i \in lpchain(\tpath)}
        \frac{lpInit(LOOP_i, \tpath) - rFinal(\tpath)}{lpInit(LOOP_i, \tpath) - lpNext(LOOP_i, \tpath)}$.
        \\
        By the computation of the operators $lpInit$ and $lpNext$, and soundness of path-insensitive reachability bound 
        in Theorem~\ref{thm:pathinsensitive_rb_soundness}, 
        we know 
        $lpRB(LOOP_i, \tpath)$ is the
        bound for the number of $LOOP_i$'s execution iterations
        %  will 
        % bound the execution times of $LOOP_{t_0}$
        % in each single execution of the $LOOP_{t_i}$ for every
        such that, during these iterations, $LOOP_{t_0}$ will be executed. 
        \\
        Since $LOOP_{t_0}$ is the closest loop containing $\tpath$, so we know 
        $lpRB(LOOP_i, \tpath)$  is also the sound bound for the number of $LOOP_i$'s execution iterations
        %  will 
        % bound the execution times of $LOOP_{t_0}$
        % in each single execution of the $LOOP_{t_i}$ for every
        such that, during these iterations, $\tpath$ will be executed. 
        \\
        Then we have
        \[ 
          r_i \leq lpRB(LOOP_i, \tpath) 
          \]
        Then this case is proved
        % $LOOP_{t_i} \in lpch(\tpath)$.
        \\
        By the alternative computation method 
        $\kw{BOUNDFINDERD(INIT(c, i, \absinit(\tpath)), NEXT(c, i, \absinit(\tpath)), V_{\ln})}$ from \cite{GulwaniJK09},
        we can also obtain a sound upper bound on
        % execution times of 
        % transition path $\tpath$ in while loop $\rprog_i$.
        the number of $LOOP_i$'s execution iterations
        %  will 
        % bound the execution times of $LOOP_{t_0}$
        % in each single execution of the $LOOP_{t_i}$ for every
        during which $\tpath$ is executed. 
        % \[
        %     \begin{array}{l}
        %     \forall (l, w_{t}) \in \exeRB(c),
        %     % (x^l, w_{p}) \in \progV, 
        %     \trace_0 \in \mathcal{T}_0(c), 
        %     v \in \mathbb{N} \st
        %     \config{psRB(l), \trace_0} \earrow v
        %     \implies
        %     % \right\} 
        %     w_{t}(\trace_0) \leq v
        %     \end{array}
        %   \]        
  \end{proof}

  \begin{lem}[Uniqueness of the Nested Loop Chain]
    \label{lem:lpch_unique}
    For every program $c$, let $\rprog$ be the corresponded refined program, 
    then for each of the transition path $\tpath \in \rprog$, there is at most one nested loop chain $lpch(\tpath) \in lp\mathcal{C}(\tpath)$.
    \[
      \forall c \in \cdom \st \rprog = refined(c) \land \tpath \in \rprog \implies 
      |lp\mathcal{C}(\tpath)| \leq 1\]
  \end{lem}
  Proof Summary:
  \\
  By induction on the program.
  \\
  Or by showing contradiction.
  % Taking an arbitrary program $c$, let $\rprog$ be its  refined program.
  % By the label consistency, for any simple path $\tpath$, $\tpath$ cannot be in two different while commands 
  % $c_1 = \ewhile \clabel{b_1}^{l_1} \edo c_{w1}$ and $c_2 = \ewhile \clabel{b_2}^{l_2} \edo c_{w2}$.
  % $c_1 \in c$ and $c_2 \in c$.
  % If $c_1 \not\in c_2$ and $c_2 \not\in c_1$, by label consistency, we know 
  % \\
  % $\lvar(c_{w1}) \cap \lvar(c_2) = \emptyset$.
  % By $\tpath \in c_{w1}$ and $\tpath \in c_{w2}$, we know 
  % \\
  % $\lvar(c_{w1}) \cap \lvar(c_2) = \lvar(\tpath) \neq \emptyset$.
  % \\
  % Then we have a contradiction and this Lemma is proved
  \begin{proof}
    Taking an arbitrary program $c$, let $\rprog$ be its  refined program.
    By the label consistency, for any transition path $\tpath$, $\tpath$ cannot be in two different while commands.
    \\
    It is sufficient to show the existence of a contradiction by assuming that 
    $\tpath$ is contained in two different while commands as follows,
    \\
    $c_1 = \ewhile \clabel{b_1}^{l_1} \edo c_{w1}$ and $c_2 = \ewhile \clabel{b_2}^{l_2} \edo c_{w2}$, 
    \\
    such that $c_1 \not\in c_2$ and $c_2 \not\in c_1$,
    $c_1 \in c$ and $c_2 \in c$.
    \\
    By $c_1 \not\in c_2$ and $c_2 \not\in c_1$ and the label consistency, we know 
    $\lvar(c_{w1}) \cap \lvar(c_2) = \emptyset$.
    \\
    By $\tpath \in c_{w1}$ and $\tpath \in c_{w2}$, we know 
    % \\
    $\lvar(c_{w1}) \cap \lvar(c_2) = \lvar(\tpath) \neq \emptyset$.
    \\
    Then we have a contradiction and this Lemma is proved.    
  \end{proof}