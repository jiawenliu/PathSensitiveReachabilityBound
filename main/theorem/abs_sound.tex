\begin{lemma}[Soundness of The Abstract Execution Trace]
    % \label{lem:abscfg_sound}
    For every program $c$ and
    a finite execution trace $\trace \in \ftdom$ which is generated w.r.t.
    an initial trace  $\trace_0 \in \ftdom_0(c)$,
    there is an abstract event $\absevent = (l, \_, \_) \in \absflow(c)$ 
    for every event $\event \in \trace$ having the same label $l$, i.e., $\event = (\_, l, \_)$.
    %
    \[
    \begin{array}{l}
      \forall c \in \cdom, \trace_0 \in \ftdom_0(c), \trace \in \ftdom ,  \event = (\_, l, \_) \in \eventset \st
      \big(
        \config{{c}, \trace_0} \to^{*} \config{c', \trace_0 \tracecat \trace} 
      % \lor 
      % \config{{c}, \trace_0} \uparrow^{\infty} \trace_0 \tracecat \trace 
      \big)
      \land \event \in \trace 
      \\
      \qquad \implies \exists \absevent = (l, \_, \_) \in (\ldom\times \dcdom^{\top} \times \ldom) \st 
      \absevent \in \absflow(c)
    \end{array}
    \]
  \end{lemma}
  \begin{proof}
    Taking an arbitrary initial trace $\trace_0 \in \ftdom_0(c)$, an arbitrary event $\event = (\_, l, \_) \in \eventset$ and an execution trace $\trace \in \tdom$
    such that
    \[
      \config{{c}, \trace_0} \to^{*} \config{c', \trace_0 \tracecat \trace} 
      \lor 
      \config{{c}, \trace_0} \to^{\infty} \config{c', \trace_0 \tracecat \trace}.
    \]
    Then it is sufficient to show,
    \[
      \event \in \trace  \implies
      \exists \absevent = (l, \_, \_) \in (\ldom\times \dcdom^{\top} \times \ldom) \st 
      \absevent \in \absflow(c)
    \]
    By induction on program $c$, we have the following cases:
    \caseL{$c = \clabel{\assign{x}{\expr}}^{l'}$}
    By the evaluation rule $\rname{assn}$, we have
    $
    {
    \config{\clabel{\assign{{x}}{\aexpr}}^{l'},  \trace_0 } 
    \xrightarrow{} 
    \config{\clabel{\eskip}^{l'}, \trace_0 :: ({x}, l', v)}
    }$, for some $v \in \mathbb{N}$.
    \\
    There are 2 cases, where $l' = l$ and $l' \neq l$.
    \\
    In case of $l' \neq l$, we know $\event \not\in [({x}, l', v)]$, then this Lemma is vacuously true.
    \\
    In case of $l' = l$, by the abstract execution trace computation in Definition~\ref{def:absevent_compute}, we have 
    \[
      \absflow(c) = \absflow'([x := \expr]^{l'}; \clabel{\eskip}^{\lex}) = \{(l', \absexpr(\expr), \lex)\}
    \]
    Let $\absevent = (l', \absexpr(\expr), \lex) $, then we have $\absevent \in \absflow(c)$ and this case is proved.
    \caseL{$\ewhile \clabel{b}^{l_w} \edo (c_w)$}
    By the first rule applied to $c$ in operational semantics, there are two cases as follows.
    \subcaseL{$\textbf{while-t}$}
    In this case, we have
    \[
      \config{{\ewhile \clabel{b}^{l_w} \edo (c_w), \trace_0}}
      \xrightarrow{} 
      \config{{
      c_w; \ewhile \clabel{b}^{l_w} \edo (c_w),
      \trace_0 :: (b, l_w, \etrue)}}.
    \]
      Then it is sufficient to show.
      \[
        \event \in [(b, l_w, \etrue)]
        \exists \absevent = (l, \_, \_) \in (\ldom\times \dcdom^{\top} \times \ldom) \st 
        \absevent \in \absflow(c)
      \]
      In case of $l \neq l_w$, we know $\event \not \in [(b, l_w, \etrue)]$, and this Lemma is vacuously true.
      \\
      Then, let $\event = (b, l_w, \etrue)$ and $\trace' =  [(b, l_w, \etrue)]$. 
      \\
      By abstract trace computation in Definition~\ref{def:absevent_compute}, we have
      \[
        \absflow(c) = \absflow'(c; \clabel{\eskip}^{\lex}) = \{(l_w, \top, \init(c_w))\} \cup \cdots.
      \]
      Let $\absevent = (l_w, \top, \init(c_w))$, then we have this case is proved.
%     Let $\trace_w \in \tdom$ satisfying following execution:
%     \\
%     $
%     \config{{
%     c_w,
%     \trace_0 \tracecat [(b, l_w, \etrue)]}}
%     \xrightarrow{*} 
%     \config{{
%     \eskip,
%     \trace_0 \tracecat [(b, l_w, \etrue)] \tracecat \trace_w}}
%   $
%   \\
%   Then we have the following execution
% \[
%   \begin{array}{l}
%     \config{{\ewhile \clabel{b}^{l_w} \edo (c_w), \trace}}
%   \xrightarrow{} 
%   \config{{
%   c_w; \ewhile \clabel{b}^{l_w} \edo (c_w),
%   \trace_0 \tracecat [(b, l_w, \etrue)]}} \\
%   \xrightarrow{*} 
%   \config{{
%     \ewhile \clabel{b}^{l_w} \edo (c_w),
%   \trace_0 \tracecat [(b, l_w, \etrue)] \tracecat \trace_w}}
%   \xrightarrow{*} 
%   \config{{
%   \eskip,
%   \trace_0 \tracecat [(b, l_w, \etrue)] \tracecat \trace_w \tracecat \trace_1}}
%   \end{array}
%  \]
%   for some $\trace_1 \in \tdom$ and $\trace = [(b, l_w, \etrue)] \tracecat \trace_w \tracecat \trace_1$.
  % \\
  % Then we have 3 cases: 
  % (1) $\event \eventeq (b, l_w, \etrue)$, 
  % (2) $\event \in \trace_w$ or 
  % (3) $\event \in \trace_1$.
  % \\
  % In case of (1). $\event \eventeq (b, l_w, \etrue)$, since $\absflow(c) = \absflow'(c;\clabel{\eskip}^{\lex}) = \{(l, \top, \init(c_w))\} \cup \cdots $, we have $\absevent = (l, \top, \init(c_w))$ and this case is proved.
  % \\
  % In case of (2). $\event \in \trace_w$,
  % by induction hypothesis on 
  % $c_w$ with the execution 
  %   $\config{{
  %   (c_w),
  %   \trace_0 \tracecat [(b, l_w, \etrue)]}}
  %   \xrightarrow{*} 
  %   \config{{
  %   \eskip,
  %   \trace_0 \tracecat [(b, l_w, \etrue)] \tracecat \trace_w}}$ and trace $\trace_w$, 
  %   we know there is an abstract event of the form 
  %   $\absevent' = (l, \_, \_ ) \in \absflow(c_w)$ where $\absflow(c_w) = \absflow'(c_w;\clabel{\eskip}^{\lex})$.
  %   \\
  %   Let $\absevent' = (l, dc, l')$ for some $dc$ and $l'$ such that $\absevent \in \absflow(c)$.
  %   \\
  %   By definition of $\absflow'$, we have 
  %   $ \absflow'(c_w;\clabel{\eskip}^{\lex}) = 
  %   \absflow'(c_w) \cup  \{ (l', dc, \lex) | (l', dc) \in \absfinal(c_w) \} $.
  %   \\
  %   There are 2 subcases: (2.1) $\absevent' \in \absflow'(c_w)$ or 
  %   $ (2.2) \absevent' \in \{ (l', dc, \lex) | (l', dc) \in \absfinal(c_w) \}$.
  %   \subcaseL{(2.1)}
  %   Since $\absflow(c) = \absflow'(c_w) \cup \{(l', dc, l_w)| (l', dc) \in \absfinal(c_w) \} \cup \cdots $, 
  %   we know the abstract event $\absevent' \in \absflow(c)$. 
  %   \\
  %   This case is proved.
  %   \subcaseL{(2.2) $\absevent' \in \{ (l', dc, \lex) | (l', dc) \in \absfinal(c_w) \}$ }
  %   In this case, we know $(l, dc) \in \absfinal(c_w)$.
  %   \\
  %   Since $\absflow(c) = \absflow'(c_w) \cup \{(l', dc, l_w)| (l', dc) \in \absfinal(c_w) \} \cup \cdots $, 
  %   we know $(l, dc, l_w) \in \{(l', dc, l_w)| (l', dc) \in \absfinal(c_w) \}$, 
  %    i.e., the abstract event $(l, dc, l_w) \in \absflow(c)$ and $(l, dc, l_w)$ has the form $(l, \_, \_)$.
  %   \\
  %   This case is proved.
  %   \\
  %   %
  % In case of (3). $\event \in \trace_1$, we know either $\event = (b, l_w, \_)$, or $\event \in \trace_w'$ where $\trace_w' \in \tdom$ is the trace of executing $c_w$ in an iteration.
  % \\
  % Then this case is proved by repeating the proof in case (1) and case (2).
  %   \\
    \subcaseL{$\rname{while-f}$}
    In this case, we have
    \[
      \config{{\ewhile \clabel{b}^{l_w} \edo (c_w), \trace_0}}
      \xrightarrow{}^\rname{while-f}
      \config{{
      \eskip,
      \trace_0 :: (b, l_w, \efalse)}}.
    \]
    Then it is sufficient to show
      \[
        \event \in [(b, l_w, \efalse)]
        \implies
        \exists \absevent = (l, \_, \_) \in (\ldom\times \dcdom^{\top} \times \ldom) \st 
        \absevent \in \absflow(c).
      \]
      In case of $l \neq l_w$, we know $\event \not\in [(b, l_w, \efalse)]$, and this Lemma is vacuously true.
      \\
      Then we prove the lemma in the case where $l = l_w$, 
      let $\trace = [(b, l_w, \efalse)]$ and $\event = (b, l_w, \efalse)$.
    \\
    By abstract trace computation in Definition~\ref{def:absevent_compute}, we have
    \[
      \absflow(c) = \absflow'(c; \clabel{\eskip}^{\lex}) = \{(l_w, \top, \init(c_w))\} \cup \cdots.
    \]
    Let $\absevent = (l_w, \top, \init(c_w))$, then we have this case is proved.
    \caseL{$\eif(\clabel{b}^l, c_t, c_f)$}
    By the first rule applied to $\eif(\clabel{b}^l, c_t, c_f)$ in operational semantics,
    we have two subcases \rname{if-t} and \rname{{if-f}}.
    \\
    Then both of the two cases will append the evaluation result of the guard $\clabel{b}^l$, i.e., $(b, l_w, \etrue)$ and $(b, l_w, \efalse)$ respectively.
    \\
    By abstract trace computation in Definition~\ref{def:absevent_compute}, we have
    \[
      \begin{array}{l}
        \absflow(c) = \absflow'(c; \clabel{\eskip}^{\lex}) 
        \\ \quad
        = \absflow'(\eif \clabel{b}^l \ethen c_t \eelse c_f) \cup \absflow'(\clabel{\eskip}^{\lex}) 
        \\ \quad \quad 
        \cup \{ (l, dc, \absinit(c_2)) | (l, dc) \in \absfinal(c_1) \} 
        \\ \quad  
        = \{(l, \absexpr(\bexpr, \top),  \absinit(c_t) ) ,  (l, \absexpr(\neg\bexpr, \top), \absinit(c_f)) \}
        \\ \quad \quad 
        \cup \absflow'(c_t) \cup \absflow'(c_f) 
        \cup \{ (l, dc, \lex) | (l, dc) \in \absfinal(c_1) \} 
     \end{array}
    \]
    Let $\absevent = (l, \absexpr(\bexpr, \top),  \absinit(c_t) ) $ or $ (l, \absexpr(\neg\bexpr, \top), \absinit(c_f)) $
    respectively in the two subcases.
    \\
    Then we have both of the two subcases proved.
    \caseL{$c = c_1;c_2$}
    By the first rule applied to $c$ in operational semantics, there are two cases as follows.
    \subcaseL{$\rname{seq1}$}
    In this case, we have the following where $\trace = \trace_1$.
    \[
      \inferrule
      {
      \config{{c_1, \trace_0}}
      \xrightarrow{}
      \config{{c_1',  \trace_0 \tracecat \trace_1}}
      }
      {
      \config{{c_1; c_2, \trace_0}} 
      \xrightarrow{} 
      \config{{c_1'; c_2, \trace_0 \tracecat \trace_1}}
      }
    \]
    By induction hypothesis on $c_1$, we know 
    \[
      \event \in \trace_1
      \implies
      \exists \absevent = (l, \_, \_) \in (\ldom \times \dcdom^{\top} \times \ldom) \st 
      \absevent \in \absflow(c_1)
    \]
    Let $\event = (\_, l_1, \_) \in \trace_1$ and $\absevent_1 = (l_1, \_, \_) \in \absflow(c_1)$,
    Then it is sufficient to show
    \[
      \exists \absevent = (l_1, \_, \_) \in (\ldom \times \dcdom^{\top} \times \ldom) \st 
      \absevent \in \absflow(c)
    \]
    By abstract trace computation in Definition~\ref{def:absevent_compute}, we have
    \[
      \begin{array}{l}
        \absflow(c_1) = \absflow'(c_1; \clabel{\eskip}^{\lex}) 
        \\ \quad
        = \absflow'(c_1) \cup \absflow'(\clabel{\eskip}^{\lex}) 
        \cup \{ (l, dc, \absinit(\clabel{\eskip}^{\lex}))) | (l, dc) \in \absfinal(c_1) \} 
        \\ \quad
        = \absflow'(c_1) \cup  \{ (l, dc, \lex) | (l, dc) \in \absfinal(c_1) \} 
     \end{array}
    \]  
    and 
    \[
      \begin{array}{l}
        \absflow(c) = \absflow'(c; \clabel{\eskip}^{\lex}) 
        \\ \quad
        = \absflow'(c_1;c_2) \cup \absflow'(\clabel{\eskip}^{\lex}) 
        \cup \{ (l, dc, \absinit(\clabel{\eskip}^{\lex})) | (l, dc) \in \absfinal(c_1; c_2) \} 
        \\ \quad  
        = \absflow'(c_1) \cup \absflow'(c_2) 
        \\ \quad \quad
        \cup \{ (l, dc, \absinit(c_2)) | (l, dc) \in \absfinal(c_1) \} 
        \cup \{ (l, dc, \lex) | (l, dc) \in \absfinal(c_1; c_2) \} 
     \end{array}
    \]
    Then we have two subcases where 
    $\absevent_1 \in \absflow'(c_1)$ or $\absevent_1 \in \{ (l, dc, \lex) | (l, dc) \in \absfinal(c_1) \}$.
    \subcaseL{$\absevent_1 \in \absflow'(c_1)$}
    Since $\absflow'(c_1) \subseteq \absflow(c)$, then we know $\absevent_1 \in \absflow(c)$.
    \\
    Let $\absevent = \absevent_1$, then we have $\absevent \in \absflow(c)$ and this subcase is proved.
    \subcaseL{$\absevent_1 \in \{ (l, dc, \lex) | (l, dc) \in \absfinal(c_1) \} $}
    In this case, we know $(l_1, \_) \in \absfinal(c_1)$.
    \\
    Let $\absevent = (l_1, \_, \absinit(c_2))$, then we know 
    \[
      \absevent \in   \{ (l, dc, \absinit(c_2)) | (l, dc) \in \absfinal(c_1) \}. 
    \]
    Since $\{ (l, dc, \absinit(c_2)) | (l, dc) \in \absfinal(c_1) \}  \subset\absflow(c)$,  then we have $\absevent \in \absflow(c)$ and this subcase is proved.
    \subcaseL{$\rname{seq2}$}
    In this case, we have the following where $\trace = \trace_2$.
    \[
      \inferrule
      {
      \config{{c_2, \trace_0}}
      \xrightarrow{}
      \config{{c_2',  \trace_0 \tracecat \trace_2}}
      }
      {
      \config{{\clabel{\eskip}^l; c_2, \trace_0}} 
      \xrightarrow{} 
      \config{{c_2, \trace_0 \tracecat \trace_2}}
      }
    \]
    By induction hypothesis on $c_2$, we know 
    \[
      \event \in \trace_2
      \implies
      \exists \absevent = (l, \_, \_) \in (\ldom \times \dcdom^{\top} \times \ldom) \st 
      \absevent \in \absflow(c_2)
    \]
    Let $\event = (\_, l_2, \_) \in \trace_2$ and $\absevent_2 = (l_2, \_, \_) \in \absflow(c_2)$,
    Then it is sufficient to show
    \[
      \exists \absevent = (l_2, \_, \_) \in (\ldom \times \dcdom^{\top} \times \ldom) \st 
      \absevent \in \absflow(c)
    \]
    By abstract trace computation in Definition~\ref{def:absevent_compute}, we have
    \[
      \begin{array}{l}
        \absflow(c_2) = \absflow'(c_2; \clabel{\eskip}^{\lex}) 
        \\ \quad
        = \absflow'(c_2) \cup \absflow'(\clabel{\eskip}^{\lex}) 
        \cup \{ (l, dc, \absinit(\clabel{\eskip}^{\lex}))) | (l, dc) \in \absfinal(c_2) \} 
        \\ \quad
        = \absflow'(c_2) \cup  \{ (l, dc, \lex) | (l, dc) \in \absfinal(c_2) \} 
     \end{array}
    \]  
    and 
    \[
      \begin{array}{l}
        \absflow(c) = \absflow'(c; \clabel{\eskip}^{\lex}) 
        \\ \quad
        = \absflow'(c_1;c_2) \cup \absflow'(\clabel{\eskip}^{\lex}) 
        \cup \{ (l, dc, \absinit(\clabel{\eskip}^{\lex})) | (l, dc) \in \absfinal(c_1; c_2) \} 
        \\ \quad  
        = \absflow'(c_1) \cup \absflow'(c_2) 
        \\ \quad \quad
        \cup \{ (l, dc, \absinit(c_2)) | (l, dc) \in \absfinal(c_1) \} 
        \cup \{ (l, dc, \lex) | (l, dc) \in \absfinal(c_1; c_2) \} 
     \end{array}
    \]
    Then we have two subcases where 
    $\absevent_2 \in \absflow'(c_2)$ or $\absevent_2 \in \{ (l, dc, \lex) | (l, dc) \in \absfinal(c_2) \}$.
    \subcaseL{$\absevent_2 \in \absflow'(c_2)$}
    Since $\absflow'(c_2) \subseteq \absflow(c)$, then we know $\absevent_2 \in \absflow(c)$.
    \\
    Let $\absevent = \absevent_2$, then we have $\absevent \in \absflow(c)$ and this subcase is proved.
    \subcaseL{$\absevent_2 \in \{ (l, dc, \lex) | (l, dc) \in \absfinal(c_2) \} $}
    In this case, we know $(l_2, \_) \in \absfinal(c_2)$. By abstract continuation state computation in Section~\ref{sec:abs_prog-edge}, we know
    \[
      \absfinal(c_1; c_2) =  \absfinal(c_2) 
    \]
    Then we know $(l_2, \_) \in \absfinal(c_1; c_2)$. Let $\absevent = \absevent_2$, then we know 
    \[
      \absevent = \absevent_2 \in   \{ (l, dc, \lex) | (l, dc) \in \absfinal(c_1; c_2) \}  \subset\absflow(c).
    \]
    Then we have $\absevent \in \absflow(c)$ and this subcase is proved.
  \end{proof}
  
  For every labeled variable in program $c$, $x^l \in \lvar_c$, there is a unique abstract event in program's abstract execution trace $\absevent \in \absflow(c)$ of form $(l, \_, \_)$. 
  \begin{lemma}[Uniqueness of the Abstract execution trace]
  % \label{lem:absevent_unique}
  For every program $c$ and
  a finite execution trace $\trace \in \ftdom$ which is generated w.r.t.
  an initial trace $\trace_0 \in \ftdom_0(c)$,
  there is a unique abstract event $\absevent = (l, \_, \_) \in \absflow(c)$ 
  for every assignment event $\event \in \eventset^{\asn}$ in the
  execution trace having the label $l$, i.e., $\event = (\_, l, \_)$ and $\event \in \trace$.
  %
  \[
    \begin{array}{l}
    \forall c \in \cdom, \trace_0 \in \ftdom_0(c), \trace \in \ftdom ,  \event = (\_, l, \_) \in \eventset^{\asn} \st
    \big(
      \config{{c}, \trace_0} \to^{*} \config{c', \trace_0 \tracecat \trace} 
    % \lor 
    % \config{{c}, \trace_0} \uparrow^{\infty} \trace_0 \tracecat \trace 
    \big)
    \land \event \in \trace 
    \\
    \qquad \implies \exists ! \absevent = (l, \_, \_) \in (\ldom\times \dcdom^{\top} \times \ldom) \st 
    \absevent \in \absflow(c)
    \end{array}
  \]
  \end{lemma}
  \begin{proof}
    This is proved trivially by induction on the program $c$.
  \end{proof}