\begin{lem}[Soundness of the Abstract Execution Trace]
    % \label{lem:abscfg_sound}
  Given a program ${c}$, we have:
  %
  \[
    \begin{array}{l}
      \forall \vtrace_0, \trace \in \mathcal{T} ,  \event = (\_, l, \_) \in \eventset \st
  \config{{c}, \trace_0} \to^{*} \config{\eskip, \trace_0 \tracecat \vtrace} 
  \land \event \in \trace 
  \\
  \qquad \implies \exists \absevent = (l, \_, \_) \in (\ldom\times \dcdom^{\top} \times \ldom) \st 
  \absevent \in \absflow(c)
  \end{array}
  \]
  \end{lem}
  \begin{proof}
    Taking an arbitrary initial trace $\trace_0 \in \tdom_0(c)$, an arbitrary event $\event = (\_, l, \_) \in \eventset$ and an execution trace $\trace \in \tdom$
    such that $\config{{c}, \trace} \to^{*} \config{\eskip, \trace_0 \tracecat \vtrace}$,
    it is sufficient to show,
    \[
      \exists \absevent = (l, \_, \_) \in (\ldom\times \dcdom^{\top} \times \ldom) \st 
      \absevent \in \absflow(c)
    \]
    By induction on program $c$, we have the following cases:
    \caseL{$c = [\assign{x}{\expr}]^{l'}$}
    By the evaluation rule $\rname{assn}$, we have
    $
    {
    \config{[\assign{{x}}{\aexpr}]^{l'},  \trace_0 \tracecat \trace } 
    \xrightarrow{} 
    \config{\eskip, \trace \tracecat [({x}, l', v) ]}
    }$, for some $v \in \mathbb{N}$ and $\trace = [({x}, l', v) ]$.
    \\
    There are 2 cases, where $l' = l$ and $l' \neq l$.
    \\
    In case of $l' \neq l$, we know $\event \not\in \trace$, then this Lemma is vacuously true.
      \\
      In case of $l' = l$, by the abstract Execution Trace computation, we know 
      $\absflow(c) = \absflow'([x := \expr]^{l}; \clabel{\eskip}^{\lex}) = \{(l, \absexpr(\expr), \lex)\}$  
      \\
    Then we have $\absevent = (l, \absexpr(\expr), \lex) $ and $\absevent \in \absflow(c)$.
    \\
    This case is proved.
    \caseL{$\ewhile [b]^{l_w} \edo c$}
    If the rule applied to is $\rname{while-t}$, we have
    \\
    $\config{{\ewhile [b]^{l_w} \edo c_w, \trace}}
      \xrightarrow{} 
      \config{{
      c_w; \ewhile [b]^{l_w} \edo c_w,
      \trace_0 \tracecat [(b, l, \etrue)]}}
    $.
    \\
    Let $\trace_w \in \mathcal{T}$ satisfying following execution:
    \\
    $
    \config{{
    c_w,
    \trace_0 \tracecat [(b, l_w, \etrue)]}}
    \xrightarrow{*} 
    \config{{
    \eskip,
    \trace_0 \tracecat [(b, l_w, \etrue)] \tracecat \trace_w}}
  $
  \\
  Then we have the following execution
\[
  \begin{array}{l}
    \config{{\ewhile [b]^{l_w} \edo c_w, \trace}}
  \xrightarrow{} 
  \config{{
  c_w; \ewhile [b]^{l_w} \edo c_w,
  \trace_0 \tracecat [(b, l_w, \etrue)]}} \\
  \xrightarrow{*} 
  \config{{
    \ewhile [b]^{l_w} \edo c_w,
  \trace_0 \tracecat [(b, l_w, \etrue)] \tracecat \trace_w}}
  \xrightarrow{*} 
  \config{{
  \eskip,
  \trace_0 \tracecat [(b, l_w, \etrue)] \tracecat \trace_w \tracecat \trace_1}}
  \end{array}
 \]
  for some $\trace_1 \in \mathcal{T}$ and $\trace = [(b, l_w, \etrue)] \tracecat \trace_w \tracecat \trace_1$.
  \\
  Then we have 3 cases: 
  (1) $\event \eventeq (b, l_w, \etrue)$, 
  (2) $\event \in \trace_w$ or 
  (3) $\event \in \trace_1$.
  \\
  In case of (1). $\event \eventeq (b, l_w, \etrue)$, since $\absflow(c) = \absflow'(c;\clabel{\eskip}^{\lex}) = \{(l, \top, \init(c_w))\} \cup \cdots $, we have $\absevent = (l, \top, \init(c_w))$ and this case is proved.
  \\
  In case of (2). $\event \in \trace_w$,
  by induction hypothesis on 
  $c_w$ with the execution 
    $\config{{
    c_w,
    \trace_0 \tracecat [(b, l_w, \etrue)]}}
    \xrightarrow{*} 
    \config{{
    \eskip,
    \trace_0 \tracecat [(b, l_w, \etrue)] \tracecat \trace_w}}$ and trace $\trace_w$, 
    we know there is an abstract event of the form 
    $\absevent' = (l, \_, \_ ) \in \absflow(c_w)$ where $\absflow(c_w) = \absflow'(c_w;\clabel{\eskip}^{\lex})$.
    \\
    Let $\absevent' = (l, dc, l')$ for some $dc$ and $l'$ such that $\absevent \in \absflow(c)$.
    \\
    By definition of $\absflow'$, we have 
    $ \absflow'(c_w;\clabel{\eskip}^{\lex}) = 
    \absflow'(c_w) \cup  \{ (l', dc, \lex) | (l', dc) \in \absfinal(c_w) \} $.
    \\
    There are 2 subcases: (2.1) $\absevent' \in \absflow'(c_w)$ or 
    $ (2.2) \absevent' \in \{ (l', dc, \lex) | (l', dc) \in \absfinal(c_w) \}$.
    \subcaseL{(2.1)}
    Since $\absflow(c) = \absflow'(c_w) \cup \{(l', dc, l_w)| (l', dc) \in \absfinal(c_w) \} \cup \cdots $, 
    we know the abstract event $\absevent' \in \absflow(c)$. 
    \\
    This case is proved.
    \subcaseL{(2.2) $\absevent' \in \{ (l', dc, \lex) | (l', dc) \in \absfinal(c_w) \}$ }
    In this case, we know $(l, dc) \in \absfinal(c_w)$.
    \\
    Since $\absflow(c) = \absflow'(c_w) \cup \{(l', dc, l_w)| (l', dc) \in \absfinal(c_w) \} \cup \cdots $, 
    we know $(l, dc, l_w) \in \{(l', dc, l_w)| (l', dc) \in \absfinal(c_w) \}$, 
     i.e., the abstract event $(l, dc, l_w) \in \absflow(c)$ and $(l, dc, l_w)$ has the form $(l, \_, \_)$.
    \\
    This case is proved.
    \\
    %
  In case of (3). $\event \in \trace_1$, we know either $\event = (b, l_w, \_)$, or $\event \in \trace_w'$ where $\trace_w' \in \mathcal{T}$ is the trace of executing $c_w$ in an iteration.
  \\
  Then this case is proved by repeating the proof in case (1) and case (2).
    \\
    If the rule applied to is $\rname{while-f}$, we have
    \\
    $
    {
      \config{{\ewhile [b]^{l_w} \edo c_w, \trace_0}}
      \xrightarrow{}^\rname{while-f}
      \config{{
      \eskip,
      \trace_0 \tracecat [(b, l_w, \efalse)]}}
    }$,
    In this case, we have $\trace = [(b, l_w, \efalse)]$ and $\event = (b, l_w, \efalse)$ (o.w., $\event \not\in \trace$ and this lemma is vacuously true) with $l = l_w$.
    \\
    By the abstract execution trace computation, $\absflow(c) = \{(l, \top, \init(c_w))\} \cup \cdots $, 
    we have $\absevent = (l, \top, \init(c_w))$  and $\absevent \in \absflow(c)$.
  \\
    This case is proved.
    \caseL{$\eif([b]^l, c_t, c_f)$}
    This case is proved in the same way as \textbf{case: $c = \ewhile [b]^{l} \edo c$}.
    \caseL{$c = c_{s1};c_{s2}$}
   By the induction hypothesis on $c_{s1}$ and $c_{s2}$ separately, and the same step as case (2). of \textbf{case: $c = \ewhile [b]^{l} \edo c$},
   we have this case proved.
  \end{proof}
  
  For every labeled variable in program $c$, $x^l \in \lvar_c$, there is a unique abstract event in program's abstract execution trace $\absevent \in \absflow(c)$ of form $(l, \_, \_)$. 
  \begin{lem}[Uniqueness of the Abstract Execution Trace]
  % \label{lem:absevent_unique}
  Given a program ${c}$, we have:
  %
  \[
  \begin{array}{l}
   \forall \vtrace_0, \trace \in \mathcal{T} ,  \event = (\_, l, \_, \_) \in \eventset^{\asn} \st
  \config{{c}, \trace_0} \to^{*} \config{\eskip, \trace_0 \tracecat \vtrace} 
  \land \event \in \trace 
  \\
  \qquad \implies \exists! \absevent = (l, \_, \_) \in (\ldom\times \dcdom^{\top} \times \ldom) \st 
  \absevent \in \absflow(c)
  \end{array}
  \]
  \end{lem}
  \begin{proof}
    This is proved trivially by induction on the program $c$.
  \end{proof}