The operator $\progl: \cdom \to \mathcal{P}(\ldom)$,
computes the set of all labels
in a labeled command $c$ defined as follows, we call it program points.
Every program point is unique.
\begin{defn}[Program Points ($\progl : \cdom \to \mathcal{P}(\ldom)$]
\label{def:progl}
{\small
\[
 \progl(c) \triangleq
 \left\{
 \begin{array}{ll}
 \{l\} 
 & {c} = \clabel{\assign x \expr}^{l} 
 \\
 \progl({c_1}) \cup \progl({{c_2}}) 
 & {c} = {c_1};{c_2}
 \\
 \progl(c_1) \cup \progl({{c_2}}) \cup \{l\} 
 & {c} =\eif(\clabel{\bexpr}^{l}, c_1, c_2) 
 \\
 \progl(c_w) \cup \{l\} 
 & {c} = \ewhile \clabel{\bexpr}^{l} \edo (c_w)
\end{array}
\right.
\]
}
\end{defn}
%
Every program point corresponds to a labeled command in a program (either a guard of if or while command, or an assignment command), and it is unique.
The Lemma below formally describes the uniqueness property of the program points
with proof in Appendix~\ref{apdx:lem_language}.
\begin{lem}[Uniqueness of The Program Point]
 \label{lem:label_unique}
 For every program $c \in \cdom$ and every two labels such that
 $i, j \in \progl(c)$, then $i \neq j$.
 \[
 \forall c \in \cdom, i, j \in \ldom \st i, j \in \progl(c)\implies i \neq j.
 \]
\end{lem}
%
\begin{defn}(An Event Belongs to A Trace)
 An event $\event \in \eventset$ belongs to a trace $\trace \in \tdom$, i.e., $\event \in \trace$ are defined as follows:
%
\begin{equation*}
 \event \in \trace 
 \triangleq \left\{
 \begin{array}{ll} 
 \etrue & \trace = [\event] \tracecat \trace'
 \land \event = \event' \\
 \event \in \trace' & \trace = [\event'] \tracecat \trace'
 \land \event \neq \event' \\ 
 \efalse & \trace = [] \lor \trace \in \inftdom
 \end{array}
 \right.
\end{equation*}
As usual, we denote by $\event \notin \trace$ that the event $\event$ doesn't belong to the trace $\trace$.
\end{defn}
%
% We denote by $\bot$ a value s.t. $\bot < n $ for all $n \in \mathbb{N}$.
\begin{defn}[Counter Notation for Program Point]
 \label{def:counter}
The counting operator $\counter : \tdom \to \ldom \to \mathbb{N}$
counts the appearance of a label in a trace.
% \[
% \begin{array}{llll}
% \counter([(x, l, v)] \tracecat \trace', l ) \triangleq \counter(\trace', l) + 1 & \text{if}~ l = l
% &
% \counter([(b, l, v)] \tracecat \trace', l) \triangleq \counter(\trace', l) + 1 & \text{if}~ l = l
% \\
% \counter([(x, l', v)] \tracecat \trace', l) \triangleq \counter(\trace', l) & \text{if}~ l' \neq l
% &
% \counter([(b, l', v)] \tracecat \trace', l) \triangleq \counter(\trace', l) & \text{if}~ l' \neq l
% \\
% \counter({[]}, l) \triangleq 0 & 
% &
% \counter(\trace, l) \triangleq \bot & \text{if }~ \trace \in \inftdom
% \end{array}
% \]
\[
\begin{array}{ll}
 \counter({[]}, l) \triangleq 0 
 \\
 \counter([(\_, l, \_)] \tracecat \trace', l ) \triangleq \counter(\trace', l) + 1 & \text{if}~ l = l
 \\
 \counter([(\_, l', \_)] \tracecat \trace', l) \triangleq \counter(\trace', l) & \text{if}~ l' \neq l
 \\
 \counter(\trace, l) \triangleq 0 & \text{if }~ \trace \in \inftdom
\end{array}
\]{When the input trace is infinite, $\counter(\trace, l)$ returns $0$ for any program label $l$.}
\end{defn}
\begin{defn}[Counter Notation for List of Program Point]
 \label{def:lcounter}
 The counting operator $\lcounter : \tdom \to \mathcal{P}(\ldom) \to \mathbb{N}$
 counts the appearance of a list of labels $[l_1, \ldots, l_n]$ as follows.
\[
 \begin{array}{ll}
\lcounter([], [l_1, \ldots, l_n] ) 
\triangleq 0 & 
\\ 
\lcounter(\trace'' \tracecat \trace', [l_1, \ldots, l_n] ) 
 \triangleq \lcounter(\trace', [l_1, \ldots, l_n]) + 1 & \text{if}~ \pi_2(\trace''[i]) = l_i, \forall i = 1, \ldots, n
 \\ 
 \lcounter([(\_, l, \_)] \tracecat \trace', [l_1, \ldots, l_n] ) 
 \triangleq \lcounter(\trace', [l_1, \ldots, l_n]) & \text{if}~ l \neq l_1
 \\ 
 \lcounter(\trace, [l_1, \ldots, l_n] ) 
 \triangleq 0 & \text{if }~ \trace \in \inftdom
\end{array}
\]
{When the input trace is infinite, $\lcounter(\trace, L)$ returns $0$ for any list of labels as well.}
\end{defn}
%
% We define the operator $\tracel : \tdom \to \mathcal{P}{(\ldom)}$ projects the label from every event in a trace as a list of program points,
% defined as follows,
% \[
% \tracel([(\_, l, \_)] \tracecat \trace') \triangleq [l] \tracecat \tracel(\trace')
% \qquad
% \tracel([ ]) \triangleq []
% \]
%
\begin{defn}[Reachability-bound]
 \label{def:rb}
 For a program ${c}$ and a location $l \in \progl(c)$ in this program,
a function $f(c, l) : \ftdom_0(c) \to (\mathbb{N} \cup \{\infty\})$ is a \emph{reachability-bound} for $l$ if and only if
\[
 \begin{array}{l}
 \forall \trace_0 \in \ftdom_0(c), c' \in \cdom, l, l' \in \progl(c), \trace \in \tdom \st 
 \\ \qquad
 \Big(
 \config{{c}, \trace_0} \to^{*} \config{\clabel{\eskip}^{l'}, \trace_0 \tracecat \trace} 
 \lor 
 \config{{c}, \trace_0} \uparrow^{\infty} \trace_0 \tracecat \trace 
 \Big)
 \implies f({c}, l)(\trace_0) \geq \counter(\trace, l) 
 \end{array}
 \]
\end{defn}
Given a program point $l$ in $c$, our algorithm (introduced in the following sections) computes a reachability-bound for $l$ in $c$.
It is easy to compute a trivial \emph{reachability-bound} $f(c, l): \ftdom_0(c) \to \infty$, but it is not interesting to us.
Ideally, we aim to compute a precise reachability-bound as below.

\begin{defn}[Precise Reachability-bound]
 \label{def:exe_rb}
 For a program ${c}$ and a location $l \in \progl(c)$ in this program,
$\exeRB({c}, l): \ftdom_0(c) \to (\mathbb{N} \cup \{\infty\})$ is a \emph{Precise Reachability-bound} if and only if,
\[
 \begin{array}{l}
 \forall \trace_0 \in \ftdom_0(c), c' \in \cdom, l, l' \in \progl(c), \trace \in \tdom \st 
 \\ \qquad
 \Big(
 \config{{c}, \trace_0} \to^{*} \config{\clabel{\eskip}^{l'}, \trace_0 \tracecat \trace} 
 \lor 
 \config{{c}, \trace_0} \uparrow^{\infty} \trace_0 \tracecat \trace 
 \Big)
 \implies f({c}, l)(\trace_0) = \counter(\trace, l) 
 \end{array}
 \]
\end{defn}

