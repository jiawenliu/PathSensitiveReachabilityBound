The operator $\lvar: \cdom \to \mathcal{P}(\ldom)$,
computes the set of all labels
in a labeled command $c$ defined as follows, we call it program points, 
$\loopl(c) \subseteq \lvar(c)$ is the set of all program points of the loop headers in program $c$.
Every the program point is unique.
\begin{defn}[Program Points ($\lvar : \cdom \to \mathcal{P}(\ldom)$]
\label{def:lvar}
{\small
$$
  \lvar(c) \triangleq
  \left\{
  \begin{array}{ll}
      \{l\}                  
      & {c} = [{\assign x e}]^{l} 
      \\
      \lvar({c_1}) \cup \lvar({{c_2}})  \cup \{l\} 
      & {c} = {c_1};{c_2}
      \\
      \lvar(c_1) \cup \lvar({{c_2}}) \cup \{l\} 
      & {c} =\eif([\bexpr]^{l}, c_1, c_2) 
      \\
      \lvar({{c}'}) \cup \{l\} 
      & {c}   = \ewhile ([\bexpr]^{l}, {c}')
\end{array}
\right.
$$
}
\end{defn}

\begin{defn}[Loop Headers ($\loopl : \cdom \to \mathcal{P}(\ldom)$)]
  \label{def:loopl}
  $$
  \loopl(c) \triangleq 
  \left\{
    \begin{array}{ll}
      \{\}  & {c} = [{\assign x e}]^{l} \\
      \loopl({c_1}) \cup \loopl({{c_2}})  & {c} = {c_1};{c_2} \\
      \loopl(c_1) \cup \loopl({{c_2}})   & {c} =\eif([\bexpr]^{l}, c_1, c_2) \\
  \loopl(c') \cup \{l\}, &  {c}   = \ewhile ([\bexpr]^{l}, {c}')
  \end{array}
\right.
  $$
  \end{defn}
%
Every label corresponds to a labeled command in a program, and it is unique.
The Lemma below formally describes the uniqueness property of the program labels
with proof in Appendix~\ref{apdx:lemma_sec123}.
\begin{lem}[Uniqueness of the Program Labels]
  \label{lem:label_unique}
  For every program $c \in \cdom$ and every two labels such that
  $i, j \in \lvar(c)$, then $i \neq j$.
  \[
    \forall c \in \cdom, i, j \in \ldom \st i, j \in \lvar(c)\implies i \neq j.
    \]
\end{lem}
%
The free variables
showing up in $c$, which aren't defined before be used, are actually the input variables of this program.
%
\begin{defn}[Reachability-Bound]
  \label{def:rb}
  For a program ${c}$ and a location $l \in \lvar(c)$ in this program,
% its 
% $\exeRB({c}, l)$ 
a function $f(c, l) : \tdom_0(c) \to (\mathbb{N} \cup \{\infty\})$ is a \emph{Reachability-Bound} for $l$ if and only if
\highlight{
\[
  \forall \trace_0 \in \tdom_0(c), c' \in \cdom, \trace \in \inftdom \st 
  \Big(
    \config{{c}, \trace_0} \to^{*} \config{c', \trace_0 \tracecat \vtrace} 
    % \lor 
    % \config{{c}, \trace_0} \to^{\infty} \config{\cdot, \trace_0 \tracecat \vtrace} 
  \Big)
  \implies f({c}, l)(\trace_0) \geq \counter(\vtrace, l) 
  \]
}
\end{defn}
\highlight{
Given a program point $l$ in $c$, our algorithm (defined below) computes a Reachability-Bound for it.
It is easy to compute a trivial \emph{Reachability-Bound} $f(c, l): \tdom_0(c) \to \infty$, but it is not interesting to us.
\\
% In the following sections, we only focus on computing a finite \emph{reachability-bound} for program's given location and considering the executions that terminates.
Ideally, we aim to compute a precise reachability-bound as follows.
}
\begin{defn}[Precise Reachability-Bound]
  \label{def:exe_rb}
  For a program ${c}$ and a location $l \in \lvar(c)$ in this program,
$\exeRB({c}, l): \tdom_0(c) \to (\mathbb{N} \cup \{\infty\})$ is a \emph{Precise Reachability-Bound}  if and only if,
\highlight{
\[
  \forall \trace_0 \in \tdom_0(c) \st \exists \trace \in \inftdom, c' \in \cdom \st 
  \Big(
    \config{{c}, \trace_0} \to^{*} \config{c', \trace_0 \tracecat \vtrace} 
    % \lor 
    % \config{{c}, \trace_0} \to^{\infty} \config{\cdot, \trace_0 \tracecat \vtrace} 
  \Big)
  \implies \exeRB({c}, l)(\trace_0) = \counter(\vtrace, l) 
  \]
}
\end{defn}
