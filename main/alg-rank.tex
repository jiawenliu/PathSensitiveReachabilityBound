Three steps:
\begin{enumerate}
    \item It first collects three edge sets for each variable,
  in which the variable increases, decreases and reset respectively.
  \item
  Then, it assigns a variable to the edge on which this variable decreases as this edge's ranking function.
  \item
  In the last step, it estimates the bound on the maximum value of each ranking function recursively.
  In the meantime, it also solves the loop bound and loop reachability bound.
  \end{enumerate}

  The algorithm in this step is inspired from the Algorithm.2 in paper~\cite{SinnZV14},
  % which assigns a variable to each edge on which this variable decrease as its ranking function.
  the Algorithm.3 in paper~\cite{ZulegerGSV11},
  and the Definition.25 in Section 4 of paper~\cite{sinn2017complexity}.
  Algorithm.3 in paper~\cite{ZulegerGSV11} assigns a set of variables to each transition in which these variables decrease as the local bound
  and estimates the maximum value each variable in this set.
  Algorithm.2 in paper~\cite{SinnZV14} assigns a variable to each edge on which this variable decrease as its ranking function
  and then estimates the maximum value for the ranking function.
  The Definition.25 in paper~\cite{sinn2017complexity}
  assigns each transition with a variable that decreases in this transition, as the local bound and computes the bound similarly.
  %
  \subsection{Collecting Variable Modifications}
  For each variable $x$ in a program $c$, this step computes three edge sets, $\inc(c, x)$, $\dec(c, x)$,
  and $\reset(c, x)$ for $x$.
  Every edge in a set corresponds to a transition in which $x$ is increased,
  %  $\inc(c, x)$,
  decreased
  % $\dec(c, x)$ and 
  or reset
  % $\reset(c, x)$, 
  respectively.
  \\
  $\inc: \cdom \to \mathcal{VAR} \to \mathcal{P}(\absevent) $
  is the set of the edges where the variable increase, 
  \\
  $\inc(c, x) = \{ \absevent | \absevent = (l, l', x' \leq x + v) \land \absevent \in \absflow(c)\}$
  \\
  $\dec: \mathcal{VAR} \to \mathcal{P}(\absevent) $
  is the set of abstract events where the variable decrease,
  \\
  $\dec(c, x) = \{\absevent| \absevent = (l, l', x' \leq x - v) \land \absevent \in \absflow(c)\}$
  \\
  $\reset: \cdom \to \mathcal{VAR} \to \mathcal{P}(\absevent) $
  is the set of the abstract events where the variable is reset,
  \\
  $\reset(c, x) = \{\absevent| \absevent = (l, l', x' \leq y - v) \land x \neq y \land \absevent \in \absflow(c)\}$
  \\
  $\resetchain: \cdom \to \mathcal{VAR} \to \mathcal{P}(\mathcal{P}(\absevent)) $
  is the set of the chain of abstract events where the variable is reset through the chain.
  \\
  In addition to
  collect the edge set that $x$ is reset on every edge in this set, i.e., compute the $\reset(c, x)$,
  we also compute a set, $\resetchain(c, x)$ contains sequences of edges for $x$
  based on the Definition.20 in \cite{sinn2017complexity}.
  In each sequence, $(e_0, \cdots, e_m) \in \resetchain(c, x)$
  a variable $x_i$ is reset by another variable $x_{i + 1}$ on edge $e_{i}$
  and $x_{i + 1}$ is reset on edge $e_{i + 1}$ recursively
  for every $i = 0, \cdots, m - 1$.
  $x$ is reset on the first edge $e_0$ of every sequence in $\resetchain(c, x)$.
  \highlight{Rephrase: Each edge $e_i$ in a sequence $(e_0, \cdots, e_m) \in \resetchain(c, x)$
  resets a variable $x_i$ by another variable $x_{i + 1}$ such that $x_{i + 1}$
  is reset on edge $e_{i + 1}$ recursively. The first edge $e_0$ of each sequence resets the variable $x$.}
  % 
  % Each chain in this set is  where a given variable is reset by the 
  % variables of the abstract events through the chain.
  %
  \\
  In the following steps, $c$ is omitted in $\inc(x)$,
  $\dec(x)$ and $\reset(x)$ for concise when the reference of a program $c$ is clear in the context.

  \subsection{Assigning The Ranking Function to An Edge}
  For each edge in the transition graph $\absG(c)$ of a program $c$,
  this step assigns the variable that decreases on this edge as the ranking function   of this edge.
  This step adopts the local bound computation method in Section 4 of \cite{sinn2017complexity} to assign the local bound to each edge,
  formally as follows.
  \begin{defn}[Ranking Function   Generatation]
    \label{def:ranking_gen}
  % Given a program $c$ with its abstract transition graph 
  % $\absG(c) = (\absV, \absE)$,
  For every edge $\absevent$ in the transition graph $\absG(c)$ of a program $c$,
  its \emph{ranking function/local bound}, $\locbound(\absevent, c)$
  is the variable that decreases on this edge, computed as follows,
  %
  \[ 
  \begin{array}{ll}
    \locbound(\absevent, c) \triangleq 1 
    & \absevent \notin SCC(\absG(c))
    \\
    \locbound(\absevent, c) \triangleq x
    & \absevent \in SCC(\absG(c)) \land \absevent \in \dec(x) \land  \absevent = (\_, \_ , x' \leq x - v) \\
    \locbound(\absevent, c) \triangleq x
    & \absevent \in SCC(\absG(c)) \land 
    \absevent  \notin \bigcup_{x \in \mathcal{VAR}} \dec(x)
    \land \absevent \notin SCC(\absG(c) \setminus \dec(x)).
  \end{array}
  \]
  $SCC(\absG(c))$ is the set of all the strong connected components of $\absG(c)$.
  \end{defn}
    The first case is straightforward. 
    For the label $l$ which is not in any while loop, 
    the labeled command with the label $l$ will be 
    evaluated at most once. 
    The second and third cases are guaranteed by the \emph{Discussion on Soundness} in Section 4 in~\cite{sinn2017complexity}.
    The formal soundness proof is in Lemma~\ref{lem:local_bound_sound} in Appendix~\ref{apdx:pathinsensitive_rb_soundness}.
  %
  \subsection{Ranking Function Estimation}
  This step estimates the upper bound, $\varinvar(x, c) \in \mathcal{A}_{\lin}$
  on the maximum value for each ranking function   $x \in  \mathcal{VAR} \cup \constdom$.
  \\
  For a program $c$, the \emph{ranking function bound},
  $\varinvar(\locbound(\absevent, c)) \in \mathcal{A}_{\lin}$ is 
  the bound on the maximum value of the ranking function  
  assigned to the edge $\absevent \in \absE(c)$, formally in Definition~\ref{def:ranking_bound}.
  \\
  In order to estimate the maximum value of $\locbound(\absevent, c)$ assigned to edge $\absevent \in \absE(c)$,
  % for each (ranking function's) maximum value,
  the bound on the iteration times of each corresponding edge, $\absclr(\absevent, c)$ 
  is computed interactively in a path-insensitive manner.
  % , the 
  \\ 
  $ \varinvar: (\mathcal{VAR} \cup \constdom  \times \cdom) \to \mathcal{A}_{\lin}$
  \\
  $\absclr: (\absevent \times \cdom) \to \mathcal{A}_{\lin}$
  \begin{defn}[Ranking Function Estimation]
    \label{def:ranking_bound}
  For a program $c$ and an edge $\absevent \in \absE(c)$,
  the \emph{ranking function bound}, $\varinvar(\locbound(\absevent, c))$ for the ranking function $\locbound(\absevent, c)$
  of this edge
  is computed as follows,
    \[ 
  \begin{array}{lll}
    \varinvar(x, c) & \triangleq x & x \in \constdom \\
    \varinvar(x, c) & \triangleq \incrs(x, c) + \max(\{\varinvar(y, c) + c ~\mid~ (l, x' \leq y + c, l') \in \reset(x)\}) & c \notin \constdom
  \end{array}
  \]
  %
  $\incrs(x, c) \triangleq \sum\limits_{\absevent \in \inc(v)}\{\absclr(\absevent, c) \times c ~\mid~ \absevent = (l, x' \leq x + c, l')\}$ where 
  $\absclr(\absevent, c) \in \mathcal{A}_{\lin}$  is computed as below,
\[ 
\begin{array}{lll}
  \absclr(\absevent, c) 
  & \triangleq \varinvar(\locbound(\absevent, c))  & \\
  & \quad \locbound(\absevent, c) \in \constdom & \\
  \absclr(\absevent, c) 
  & \triangleq \Big(
    \sum\limits_{y \in \{ y ~|~ 
    ch \in \resetchain(x), (l_1, x, y, v, l_2) \in ch \} } \incrs(y, c) & \\
    & \quad + 
  \sum\limits_{ch \in \resetchain(x, c)}
  \big( \min\left\{\absclr(\absevent', c) ~\mid~ \absevent' \in ch\right\} \times 
  \max\left\{\varinvar(in(ch), c) + \sum\limits_{(l_1, x, y, v, l_2) \in ch } v, 0 \right\}\big) \Big)  & \\
  &  \quad \locbound(\absevent, c) = x \land x \notin \constdom & ,
\end{array}
  \]
 where $in(ch)$ is the variable on the last edge on the reset chain $ch$.
\end{defn}
  %
$\absclr$ can also be used to estimate the loop bound, $BD(\tpath)$ for the transition path $\tpath$ as follows
where $\absevent \in \tpath$ denotes $\absevent$ is an edge on path $\tpath$.
\begin{equation}
  \label{eq:absBD}
  BD(\tpath) = \max\left\{ \absclr(\absevent, c) \middle\vert \absevent \in \tpath \right\}.
\end{equation}
  %