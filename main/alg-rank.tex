Three steps:
\begin{enumerate}
    \item It first collects three edge sets for each variable,
  in which the variable increases, decreases and reset respectively.
  \item
  Then, it assigns a variable to the edge on which this variable decreases as this edge's ranking function.
  \item
  In the last step, it estimates the bound on the maximum value of each ranking function recursively.
  In the meantime, it also solves the loop bound and loop reachability bound.
  \end{enumerate}

  The algorithm in this step is inspired from the Algorithm.2 in paper~\cite{SinnZV14},
  % which assigns a variable to each edge on which this variable decrease as its ranking function.
  the Algorithm.3 in paper~\cite{ZulegerGSV11},
  and the Definition.25 in Section 4 of paper~\cite{sinn2017complexity}.
  Algorithm.3 in paper~\cite{ZulegerGSV11} assigns a set of variables to each transition in which these variables decrease as the local bound
  and estimates the maximum value each variable in this set.
  Algorithm.2 in paper~\cite{SinnZV14} assigns a variable to each edge on which this variable decrease as its ranking function
  and then estimates the maximum value for the ranking function.
  The Definition.25 in paper~\cite{sinn2017complexity}
  assigns each transition with a variable that decreases in this transition, as the local bound and computes the bound similarly.
  %
  \subsection{Collecting Variable Modifications}
  For each variable $x$ in a program $c$, this step computes three edge sets, $\inc(c, x)$, $\dec(c, x)$,
  and $\reset(c, x)$ for $x$.
  Every edge in a set corresponds to a transition in which $x$ is increased,
  %  $\inc(c, x)$,
  decreased
  % $\dec(c, x)$ and 
  or reset
  % $\reset(c, x)$, 
  respectively.
  \\
  $\inc: \cdom \to \vardom \to \mathcal{P}(\absevent) $
  is the set of the edges where the variable increase, 
  %\\
  \[ \inc(c, x) = \left\{ \absevent | \absevent = (l, x' \leq x + v, l') \land \absevent \in \absflow(c) \right\} \]
  %\\
  $\dec: \vardom \to \mathcal{P}(\absevent) $
  is the set of abstract events where the variable decrease,
  %\\
  \[\dec(c, x) = \left\{\absevent| \absevent = (l,  x' \leq x - v, l') \land \absevent \in \absflow(c) \right\}\]
  %\\
  $\reset: \cdom \to \vardom \to \mathcal{P}(\absevent) $ is the set of the abstract events where the variable is reset,
%
  \[\reset(c, x) = \left\{ \absevent| \absevent = (l,  x' \leq y - v, l') \land x \neq y \land \absevent \in \absflow(c) \right\}\]
  Additionally,
  we also compute the reset graph $\resetG(c)$ and the reset chain, $\resetchain(c, x) \in \mathcal{P}(\mathcal{P}(\absevent))$ for every rank $x$.
  The $\resetchain(c, x)$ for every rank $x$ contains all the paths in $\resetG(c)$ that are end at $x$.
  The computation of $\resetG(c)$ and $\resetchain(c, x)$ follows the Definition~20 in~\cite{sinn2017complexity}.
  \[\resetG(c) = (\resetV(c), \resetE(c))\]
  \[\resetE(c) = \left\{ (x, \absevent, y) ~\vert~ \absevent \in \reset(x) \land \absevent = (l, x' \leq y + c, l') \right\} \]
  \[\resetV(c) = \left\{ x ~\vert~ (x, \_, \_) \in \resetE(c) \lor (\_, \_, x) \in \resetE(c) \right\} \]
  In a variable $x$'s reset chain set, $\resetchain(c, x)$, in each chain $(e_0, \ldots, e_m) \in \resetchain(c, x)$
  a variable $x_i$ is reset by another variable $x_{i + 1}$ on edge $e_{i}$
  and $x_{i + 1}$ is reset on edge $e_{i + 1}$ recursively
  for every $i = 0, \ldots, m - 1$.
  $x$ is reset on the first edge $e_0$ of every sequence in $\resetchain(c, x)$.
  {Each edge $e_i$ in a sequence $(e_0, \ldots, e_m) \in \resetchain(c, x)$
  resets a variable $x_i$ by another variable $x_{i + 1}$ such that $x_{i + 1}$
  is reset on edge $e_{i + 1}$ recursively. The first edge $e_0$ of each sequence resets the variable $x$.}
  % 
  \\
  In the following steps, $c$ is omitted in $\inc(x)$,
  $\dec(x)$ and $\reset(x)$ for concise when the reference of a program $c$ is clear in the context.

  \subsection{Assigning The Ranking Function to An Edge}
  For each edge in the transition graph $\absG(c)$ of a program $c$,
  this step assigns the variable that decreases on this edge as the ranking function of this edge.
  This step adopts the local bound computation method in Section 4 of~\cite{sinn2017complexity} to assign the local bound to each edge,
  formally as follows.
  \begin{defn}[Ranking Function Generatation]
  \label{def:ranking_gen}
  For every edge $\absevent$ in the transition graph $\absG(c)$ of a program $c$,
  its \emph{ranking function/local bound}, $\locbound(\absevent, c)$
  is the variable that decreases on this edge, computed as follows,
  %
  \[ 
  \begin{array}{ll}
    \locbound(\absevent, c) \triangleq 1 
    & \absevent \notin SCC(\absG(c))
    \\
    \locbound(\absevent, c) \triangleq x
    & \absevent \in SCC(\absG(c)) \land \absevent \in \dec(x) \land  \absevent = (\_, \_ , x' \leq x - v) \\
    \locbound(\absevent, c) \triangleq x
    & \absevent \in SCC(\absG(c)) \land 
    \absevent  \notin \bigcup_{x \in \vardom} \dec(x)
    \land \absevent \notin SCC(\absG(c) \setminus \dec(x))\\
    \locbound(\absevent, c) \triangleq \infty
    & o.w..
  \end{array}
  \]
  $SCC(\absG(c))$ is the set of all the strong connected components of $\absG(c)$.
  \end{defn}
    The first case is straightforward. 
    For the label $l$ which is not in any while loop, 
    the labeled command with the label $l$ will be 
    evaluated at most once. 
    The second and third cases are guaranteed by the \emph{Discussion on Soundness} in Section 4 in~\cite{sinn2017complexity}.
    The soundness is formalized in Lemma~\ref{lem:local_bound_sound} with proof in Appendix~\ref{apdx:pathinsensitive_rb_soundness}.
  %
  \subsection{Ranking Function Estimation}
  This step estimates the upper bound, $\varinvar(x, c) \in \inpexpr$
  on the maximum value for each ranking function   $x \in  \vardom \cup \constdom$.
  \\
  For a program $c$, the \emph{ranking function bound},
  $\varinvar(\locbound(\absevent, c)) \in \inpexpr$ is 
  the bound on the maximum value of the ranking function  
  assigned to the edge $\absevent \in \absE(c)$, formally in Definition~\ref{def:ranking_bound}.
  \\
  In order to estimate the maximum value of $\locbound(\absevent, c)$ assigned to edge $\absevent \in \absE(c)$,
  the bound on the iteration times of each corresponding edge, $\absclr(\absevent, c)$ 
  is computed interactively in a path-insensitive manner.
  \\ 
  $ \varinvar: (\vardom \cup \constdom  \times \cdom) \to \inpexpr$
  \\
  $\absclr: (\absevent \times \cdom) \to \inpexpr$
  \begin{defn}[Ranking Function Estimation]
    \label{def:ranking_bound}
  For a program $c$ and an edge $\absevent \in \absE(c)$,
  the \emph{ranking function bound}, 
  $\varinvar(\locbound(\absevent, c), c)$ for the ranking function $x = \locbound(\absevent, c)$
  of this edge
  is computed as follows,
    \[ 
  \begin{array}{lll}
    \varinvar(x, c) & \triangleq x & x \in \constdom \\
    \varinvar(x, c) & \triangleq \incrs(x, c) + \max\left\{\varinvar(y, c) + v ~\mid~ (l, x' \leq y + v, l') \in \reset(x) \right\} & x \notin \constdom
  \end{array}
  \]
  %
  where $\incrs(x, c) \triangleq \sum\limits_{\absevent \in \inc(v)}\{\absclr(\absevent, c) \times v ~\mid~ \absevent = (l, x' \leq x + v, l')\}$
  The path-insensitive bound, $\absclr(\absevent, c) \in \inpexpr$  on the execution times of the transition $\absevent$, is interactively computed as well as below,
\[ 
\begin{array}{lll}
  \absclr(\absevent, c) 
  & \triangleq \varinvar(\locbound(\absevent, c))  & \\
  & \quad \locbound(\absevent, c) \in \constdom & \\
  \absclr(\absevent, c) 
  & \triangleq
    \sum \left\{ \incrs(y, c) | ch \in \resetchain(x) \land y \in ch \right\} & \\
    & \quad + 
  \sum\limits_{ch \in \resetchain(x, c)}
  \min \left\{\absclr(\absevent', c) ~\mid~ \absevent' \in ch\right\} \times 
  \big(\varinvar(in(ch), c) 
  + \sum\limits_{(\_, (\_, x' \leq y + v, \_), \_) \in ch} v \big) & \\
  &  \quad \locbound(\absevent, c) = x \land x \notin \constdom & ,
\end{array}
  \]
 where $in(ch)$ is the first vertex of the reset chain $ch$.
\end{defn}
  %
We also have the soundness of this path-insensitive transition bound. For a program $c$ and an edge $\absevent \in \absE(c)$,
$\absclr(\absevent)$ is a sound upper bound
on the execution times of this transition by paper~\cite{sinn2017complexity}, formally below in Theorem~\ref{thm:pathinsensitive_rb_soundness} with proof in Appendix~\ref{apdx:pathinsensitive_rb_soundness}.
%
\begin{thm}[Soundness of the Path-insensitive Transition Bound]
  \label{thm:pathinsensitive_rb_soundness}
For each program ${c}$ and an edge $\absevent = (l, \_, \_) \in \absG(c)$, if $l$ is the label of an assignment command,
%  label $l \in \lvar(c)$,
then its \emph{path-insensitive transition bound} $\absclr(\absevent, c)$ 
 is a sound upper bound on 
the execution times of this assignment command in $c$.
  \[
    \begin{array}{l}
      \forall \trace_0 \in \ftdom_0(c), \trace \in \tdom, c \in \cdom, l, l' \in \lvar(c) \st
      \Big( \config{c, \trace_0} \rightarrow^{*} \config{\clabel{\eskip}^{l'}, \trace_0 \tracecat \trace} 
        \lor  \config{c, \trace_0} \uparrow^{\infty} \trace_0 \tracecat \trace \Big)
       \\ \qquad \qquad
       \implies
       \exists \absevent = (\_, l, \_) \in \absflow(c) \land
      \counter(\trace, l) \leq \econfig{\absclr(\absevent, c)}(\trace_0)
    \end{array}
  \]
\end{thm}
%
\subsection{Estimating the Loop Bound} 
The $BD(\rprog)$ for the refined program $\rprog$ can also be estimated by $\absclr(\absevent, c)$ by taking the
minimum value over all $\absevent \in \rprog$.
We use $\absevent \in \rprog$ to denote $\absevent$ is an edge on path $\rprog$.
\begin{equation}
  \label{eq:absBD}
  BD(\rprog) = \min\left\{ \absclr(\absevent, c) \middle\vert \absevent \in \rprog \right\}.
\end{equation}
%
This bound is accurate on the iteration numbers of a single loop, but neither tight on the iterations of different paths inside the loop, nor the nested loop.
%