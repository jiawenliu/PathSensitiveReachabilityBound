
\subsection{Loop Bound}
There are three methods for computing the $BD(\rprepeat(\rprog'))$:
\begin{enumerate}
  \item Estimated by Equation~\ref{eq:absBD} in Section~\ref{sec:rank} path-insensitively.
  \item Estimated by $BD(\rprepeat(\rprog')) = BOUND(\rprepeat(\rprog'))$ from paper\cite{GulwaniJK09}, which is precise for simple loop without nested loops. 
  We provide an alternative computation method based on \cite{sinn2017complexity} in Definition~\ref{def:loopbound}.
  \item Computing the rank functions' abstract states, $\rfinit(\rprog)$, $\rffinal(\rprog)$, $\rfnext(\rprog)$ and $\varGD(\rprog, \rprog')$ as in Definition~\ref{def:alg-absstate}.
  Then compute $BD(\rprepeat(\rprog'))$ by Definition~\ref{def:loopbound} using SMT solver.
\end{enumerate}
%
\begin{defn}[Rank Abstract States Computation]
  \label{def:alg-absstate}
  We compute four different abstract states for rank functions in a refined program, $\rfinit(\rprog)$, $\rffinal(\rprog)$, $\rfnext(\rprog)$ and $\varGD(\rprog, \rprog')$ as follows.
 \begin{itemize}
  \item The \emph{Initial State}, 
  $\rfinit(\rprog)$ is a set of equations $x = e$, where $e \in \mathcal{A}_{in}$ is a
  symbolic expression describing the initial value of $x$ before executing $\rprog$.
  Each $x$ is the ranking function of a simple transition path $\tpath$ in this program. 
  It contains the initial value for every ranking function of the simple transition path $\tpath$ in this program.
 \[
   \rfinit(\rprog) \triangleq 
   \bigcup_{\tpath \in \rprog, x = \locbound(\tpath)}
   \left \{ 
   x = \arg\max_{l_1}\left\{
     \varinvar(y) + v ~\middle\vert~ 
     \begin{array}{l} 
       (l_1, x' \leq y + v, l_2) \in \reset(x) 
       \\
     \land l_1 \leq \absinit(\rprog)
   \end{array}
   \right\}
   \right\}
   \]
 $\rfinit(\tpath)$ can also be computed using the function $\kw{INIT(c, l_0, \absinit(\rprog))}$ in \cite{GulwaniJK09}. 
 $\kw{INIT(c, l_0, \absinit(\rprog))}$ computes the equivalent results.
 %
 \item  The \emph{Final State}, $\rffinal(\rprog)$ a conjunction of boolean expressions.
 It is the post-condition
 after the execution of $\rprog$.
 % \\
 \[
   \rffinal(\rprog) \triangleq 
   \bigcup_{\tpath \in \rprog, x = \locbound(\tpath)}
   \left \{ 
   x = \arg\min_{l_1}\left\{
     v_{post} ~\middle\vert~ 
     \begin{array}{l} 
      l_1 \geq \absfinal(\rprog)[2]
     \land \bigwedge_{b \in \kw{Guard}(\rprog)} \neg b
   \end{array}
   \right\}
   \right\}
   \]
  $\kw{Guard}(\rprog)$ is the set of all the unique boolean expressions (i.e., the boolean constraints) on this program.
 \item The \emph{Next State}, $\rfnext(\tpath) \in \mathcal{A}_{in}$ 
 is a
 symbolic expression describing how much $\tpath$'s ranking function ($\locbound(\tpath)$) is changed after the first execution of $\rprog$ and before the second execution.
 \[
   \begin{array}{l}
   \rfnext(\tpath) \triangleq 
     \begin{array}{l}
  \sum\limits_{\absevent \in \inc(x) }
   \left\{ v ~\middle\vert~ \absevent = (l, x' \leq x + v, \_) \land l \in \tpath\right\}
   \\ \qquad 
   + \arg\max\limits_{l' }
      \left\{ \varinvar(y) + v ~\middle\vert~ (l, x' \leq y + v, l') \in \reset(x) \land l \in \tpath\right\}
      \\ \qquad 
     - \sum\limits_{ \absevent \in \dec(x) }\left\{ 
       v ~\middle\vert~ \absevent = (l, x' \leq x - v, \_) \land l \in \tpath 
       \right\}
     \end{array}
   \end{array}
   , x = \locbound(\tpath)
 \]
 Indeed we only compute the $\rfnext(\tpath)$ because that the recursion is exhausted into the base case, i.e. $\tpath$ when computing $\varGD(\rprog)$ as below.
 \item  The \emph{Variable Gradient Decent}, 
 $\varGD(\rprog)$
 is a set of equations $x = e$ for the same variables in the $\rfinit(\rprog)$.
 It computes the set of next states for every ranking function of the simple transition path $\tpath$ in $\rprog$,
 and recursively computes the loop bound, $BD$ in the meantime.
 \\
 {$\varGD(\rprepeat(\rprog)) =  BD(\rprog)  \times
{\varGD(\rprog)}$}
 \\
 $\varGD(\rprog_1;\rprog_2) =  \varGD(\rprog_1) \cup \varGD(\rprog_2)$
 \\
 $\varGD(\tpath) =  \left\{x = \rfnext(\tpath) ~|~ x = \locbound(\tpath) \right\} $  
\end{itemize}
\end{defn}
The computation of the \emph{Variable Gradient Decent}, 
$\varGD(\rprog)$ involves the computation of $BD(\rprog)$ recursively as follows.
\begin{defn}[Loop Bound Computation]
  \label{def:loopbound}
  \[
    \begin{array}{rcl}
      BD(\tpath) & \triangleq & 1 \\
      BD(\rprog_1;\rprog_2) & \triangleq & BD(\rprog_1) + BD(\rprog_2) \\
      BD(\rpchoose{\rprog_1, \cdots, \rprog_m }) & \triangleq 
      & \max\left\{ BD(\rprog_1), \cdots, BD(\rprog_m) \right\} \\
      BD(\rprepeat(\rprog')) & \triangleq 
      &
      \max\limits_{\tpath \in \rprog, x = \locbound(\tpath)}
      \Big\{ \highlight{\frac{a - b}{a'}} ~\vert~
      x = a \in \rfinit(\rprog')
      \\ & & \qquad \qquad
      \land x = b \in \rffinal(\rprog')
      \land x = a' \in \varGD(\rprog')
       \Big\} 
       \\
      BD(l: \rprog') & \triangleq & BD(\rprog')
    \end{array}
    \]
  \end{defn}

  \subsection{Path Local Reachability-Bound}
We first compute $\kw{enclosed}(\rprog, \tpath)$, which is \text{the closest loop} inside the $\rprog$, where $\tpath$ is located.
Then based on the loop bound computed above in either way, we compute the \emph{path local reachability-bound} of
the $\tpath$ in its closest nested loop as follows in Definition~\ref{def:pathlocalrb}.
\begin{defn}[Path Local Reachability-Bound]
    \label{def:pathlocalrb}
    Given a refined program $\rprog$ and a simple transition path $\tpath$ in this program, 
    let $l: \rprog_l = \kw{enclosed}(\rprog, \tpath)$ in this program,
    $\tpath$'s local reachability-bound w.r.t. $l: \rprog_l$
    is computed recursively as follows. 
  \[
    \begin{array}{rcl}
      \outinB(\tpath, \tpath) & \triangleq & 1 \\
      \outinB(\tpath', \tpath) & \triangleq & \highlight{0} \qquad \text{if } \tpath' \neq \tpath\\
      \outinB(\rprog_1;\rprog_2, \tpath) & \triangleq & \outinB(\rprog_1, \tpath) + \outinB(\rprog_2, \tpath) \\
      \outinB(\rpchoose{\rprog_1, \cdots, \rprog_m }, \tpath) & \triangleq 
      & \max\left\{ \outinB(\rprog_1, \tpath), \cdots, \outinB(\rprog_m, \tpath) \right\} \\
      \outinB(\rprepeat(\rprog'), \tpath) & \triangleq 
      & BD(\rprepeat(\rprog'))
       \times \outinB(\rprog', \tpath)
       \\
      \outinB(l': \rprog', \tpath) & \triangleq & \outinB(\rprog', \tpath) \\
      &  & \text{this case always equals to } 0 \\
    \end{array}
    \]
\end{defn}
We show the \emph{path local reachability-bound} of the simple transition path $\tpath$ in a refined program $\rprog$ is a sound upper bound of its execution times when executing its enclosed loop $\kw{enclosed}(\rprog, \tpath)$ in Lemma~\ref{lem:pathlocalrb-sound}, with the proof in \todo{Appendix~\ref{apdx:pathlocalrb-sound}}.
\begin{lem}[Soundness of the Path Local Reachability-Bound]
  \label{lem:pathlocalrb-sound}
  For any refined program $\rprog$ and a simple transition path $\tpath$ in this program,
  if $l: \rprog_l = \kw{enclosed}(\rprog, \tpath)$ is the closest loop where $\tpath$ is nested,
  then the execution times of $\tpath$ when executing the $\rprog_l$ is bounded by $\outinB(\rprog_l, \tpath)$.
\end{lem}