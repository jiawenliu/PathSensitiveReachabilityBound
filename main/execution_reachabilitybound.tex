The $\lvar_c \in \mathcal{P}(\ldom)$,
represents the set of labels
% assigned variables and labeles variables 
for a labeled command $c$ defined as follows,
% defined in Definition~\ref{def:lvar} and 
% \ref{def:avar}.
%
\begin{defn}[Program Labels
(
% $\lvar_{c} \subseteq \mathcal{VAR} \times \mathbb{N}$ or 
$\lvar : \cdom \to \mathcal{P}(\ldom)$]
\label{def:lvar}
{\small
$$
  \lvar(c) \triangleq
  \left\{
  \begin{array}{ll}
      \{l\}                  
      & {c} = [{\assign x e}]^{l} 
      \\
      \lvar({c_1}) \cup \lvar({{c_2}})  \cup \{l\} 
      & {c} = {c_1};{c_2}
      \\
      \lvar(c_1) \cup \lvar({{c_2}}) \cup \{l\} 
      & {c} =\eif([\bexpr]^{l}, c_1, c_2) 
      \\
      \lvar({{c}'}) \cup \{l\} 
      & {c}   = \ewhile ([\bexpr]^{l}, {c}')
\end{array}
\right.
$$
}
\end{defn}
%
Every label and every labeled command in a program is unique, formally as follows with proof in Appendix~\ref{apdx:lemma_sec123}.
\begin{lem}[Uniqueness of the Program Labels]
  \label{lem:label_unique}
  For every program $c \in \cdom$ and every two labels such that
  $i, j \in \lvar(c)$, then $i \neq j$.
  \[
    \forall c \in \cdom, i, j \in \ldom \st i, j \in \lvar(c)\implies i \neq j.
    \]
\end{lem}
%
The free variables
showing up in $c$, which aren't defined before be used, are actually the input variables of this program.
%
\begin{defn}[Reachability-Bound]
  \label{def:rb}
  For a program ${c}$ and a location $l \in \lvar(c)$ in this program,
% its 
% $\exeRB({c}, l)$ 
a function $f(c, l)$ is a \emph{Reachability-Bound} for $l$ if and only if
%  is a function satisfying
%  defined as follows,
% over all possible traces,
%
\highlight{
% \[
% \begin{array}{l}
%   % \text{Vertices} &
%   \exeRB({c}, l) \triangleq
%   \Big\{ 
%   (l, w) 
%   ~ \vert ~ 
%   w : \mathcal{T} \to \mathbb{N}
%   \land
%   l \in \lvar(c) 
%   \\ \qquad \qquad \qquad \qquad
%   \land
%   \forall \trace_0 \in \mathcal{T}_0(c), \trace \in \mathcal{T} \st 
%   \config{{c}, \trace} \to^{*} \config{\eskip, \trace_0 \tracecat \vtrace} 
%   \implies w(\trace_0) = \vcounter(\vtrace, l) 
% \Big\}
% \end{array}
% \]
\[
  \forall \trace_0 \in \mathcal{T}_0(c), \trace \in \mathcal{T} \st 
  \config{{c}, \trace} \to^{*} \config{\eskip, \trace_0 \tracecat \vtrace} 
  \implies f({c}, l)(\trace_0) \geq \vcounter(\vtrace, l) 
  \]
}
$\mathcal{T}_0(c)$ is the set of $c$'s initial traces in which all the input variables in
$c$ are initialized.
\end{defn}
\highlight{
% The $\exeRB(c)$  is a set of pairs, $(l, w) \in \ldom \times (\mathcal{T} \to \mathbb{N})$,
% with a labeled as the first component and
% its weight $w$ the second component.
% Weight $w$ for
% a labeled variable 
Given a program point $l$ in $c$, our algorithm (defined below) computes a Reachability-Bound for it.
It is easy to compute a trivial \emph{Reachability-Bound} $f(c, l)(\trace_0) = \infty$, but it is meaningless to us.
\\
We aim to compute an ideal reachability-bound as follows.
}
\begin{defn}[Ideal Reachability-Bound]
  \label{def:exe_rb}
  For a program ${c}$ and a location $l \in \lvar(c)$ in this program,
% its 
$\exeRB({c}, l)$ is an \emph{Ideal Reachability-Bound}  if and only if,
%  satisfying
%  defined as follows,
% over all possible traces,
%
\highlight{
% \[
% \begin{array}{l}
%   % \text{Vertices} &
%   \exeRB({c}, l) \triangleq
%   \Big\{ 
%   (l, w) 
%   ~ \vert ~ 
%   w : \mathcal{T} \to \mathbb{N}
%   \land
%   l \in \lvar(c) 
%   \\ \qquad \qquad \qquad \qquad
%   \land
%   \forall \trace_0 \in \mathcal{T}_0(c), \trace \in \mathcal{T} \st 
%   \config{{c}, \trace} \to^{*} \config{\eskip, \trace_0 \tracecat \vtrace} 
%   \implies w(\trace_0) = \vcounter(\vtrace, l) 
% \Big\}
% \end{array}
% \]
\[
  \forall \trace_0 \in \mathcal{T}_0(c), \trace \in \mathcal{T} \st 
  \config{{c}, \trace} \to^{*} \config{\eskip, \trace_0 \tracecat \vtrace} 
  \implies \exeRB({c}, l)(\trace_0) = \vcounter(\vtrace, l) 
  \]
}
\end{defn}
% In most data analysis programs c we are interested, there are usually some user input variables,
% such as $k$ in Example~\ref{ex:whileOdd}. We denote 
% There are two components in this definition. 
% \highlight{
% % The $\exeRB(c)$  is a set of pairs, $(l, w) \in \ldom \times (\mathcal{T} \to \mathbb{N})$,
% % with a labeled as the first component and
% % its weight $w$ the second component.
% % Weight $w$ for
% % a labeled variable 
% Given a program point $l$ in $c$, our algorithm (defined below) computes a Reachability-Bound for it.
% It is easy to compute a trivial \emph{Reachability-Bound} $f(c, l)(\trace_0) = \infty$, but it is meaningless to us.
% % $l$ is a function $w : \mathcal{T} \to \mathbb{N}$
% % mapping from an initial trace to a natural number.
% % When program executes under an initial trace $\trace_0$,
% % $\config{{c}, \trace_0} \to^{*} \config{\eskip, \trace_0\tracecat\vtrace} $, 
% % it generates an execution trace $\trace$.
% % % Overall possible execution traces generated under $\trace_0$,
% % This natural number is the
% % %  maximum 
% % evaluation times of the command with label $l$ in
% % % all possible execution traces generated under $\trace_0$. 
% % this execution trace, 
% % % This is 
% % computed by the counter operator $w(\trace_0) = \vcounter(\vtrace, l)$.
% }
% For example in the execution-based dependency graph of $\kw{twoRounds}$ in
%  Figure~3(b), the weights of vertices in the while loop are  
%  $w_k : \trace_0 \to \env(\trace) k$.
% $w_k$ is function return the value of the input variable $k$ from the initial trace $\trace_0$.
% It depends on the value of the user input $k$ specified in the initial trace $\trace_0$.