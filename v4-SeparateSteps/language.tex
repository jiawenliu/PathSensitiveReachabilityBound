%
%
\subsection{Labeled Language}
\[
\begin{array}{llll}
\mbox{Arithmetic Operators} 
& \oplus_a & ::= & + ~|~ - ~|~ \times 
%
~|~ \div ~|~ \emax ~|~ \emin
\\  
% ~|~ \div \\  
% \mbox{Boolean Operators} 
% & \oplus_b & ::= & \lor ~|~ \land
% \\
%
% \mbox{Relational Operators} 
% & \sim & ::= & < ~|~ \leq ~|~ == 
% \\  
%
\mbox{Arithmetic Expression} 
& \aexpr & ::= & 
n ~|~ {x} ~|~ \aexpr \oplus_a \aexpr  
 ~|~ \elog \aexpr  ~|~ \esign \aexpr
\\
%
\mbox{Boolean Expression} & \bexpr & ::= & 
%
\etrue ~|~ \efalse  ~|~ \neg \bexpr
 ~|~ \bexpr \land \bexpr
%
~|~ \bexpr \lor \bexpr
~|~ \aexpr \leq \aexpr 
~|~ \aexpr < \aexpr 
~|~ \aexpr = \aexpr 
\\
%
\mbox{Expression} & \expr & ::= & v ~|~ \aexpr ~|~ \bexpr ~|~ [\expr, \dots, \expr]
\\  
%
\mbox{Value} 
& v & ::= & { n ~|~ \etrue ~|~ \efalse ~|~ [] ~|~ [v, \dots, v]} \\
%
% \\%
\mbox{Label} 
& l & \in & (\mathbb{N} \cup \{\lin, \lex\}) 
% ~|~ (l, n)
\\ 
%
\mbox{Labeled Command} 
& {c} & ::= &  
\clabel{\assign{x}{\expr}}^l 
% ~|~ \clabel{\assign{x}{\query(\qexpr)}}^l
~|~  \clabel{\eskip}^l
~|~ \ewhile \clabel{\bexpr}^{l} \edo {c}
~|~ \eif(\clabel{\bexpr}^{l} , {c}, {c}) 
~|~ {c};{c}  
\\ 
% \\
\mbox{Event} 
& \event & ::= & 
% ~|~ ({x}, l, v, \qval)
({x}, l, v) ~ \mbox{Assignment Event} 
% \\
% &&& 
~|~(\bexpr, l, v) ~ \mbox{Testing Event}
\\
% \mbox{Trace} & \trace
% & ::= & [] ~|~ \event:: \trace ~|~ \trace \tracecat \trace  \\
\mbox{Trace} & \trace
& ::= & [] ~|~ \trace :: \event
\\
\end{array}
\]
% \todo{change trace notation into list, and update corresponding operator nations}
% \\
% \wqside{"$\cdot$" has two meanings? empty, delimit. Trace is list of event?}
We use following notations to represent the sets of corresponding terms:
\[
\begin{array}{lll}
\mathcal{VAR} & : & \mbox{Set of Variables}  
% \\ 
% %
% \mathcal{VAL} & : & \mbox{Set of Values} 
% \\ 
% %
% \mathcal{QVAL} & : & \mbox{Set of Query Values} 
\\ 
%
\cdom & : & \mbox{Set of Commands} 
\\ 
%
\eventset  & : & \mbox{Set of Events}  
\\
%
\eventset^{\asn}  & : & \mbox{Set of Assignment Events}  
\\
%
\eventset^{\test}  & : & \mbox{Set of Testing Events}  
\\
%
\ldom  & : & \mbox{Set of Labels}  
\\
% \\
%
\highlight{\tdom} & : & \mbox{\highlight{Set of All Finite Execution Traces}}
\\
\highlight{\tdom^{\infty}} & : & \mbox{\highlight{Set of All Finite Or Infinite  Execution Traces}}
\\
%
\tdom_0(c) & : & \mbox{Set of Initial Traces, which is finite and all the input variables of the program $c$ are initialized.
}
\end{array}
\]
$FV: \expr \to \mathcal{P}(\mathcal{VAR})$, computes the set of free variables in an expression. To be precise,
$FV(\aexpr)$ and $FV(\bexpr)$
%  and $FV(\qexpr)$ 
represent the sets of free variables in arithmetic
expression $\aexpr$ and boolean expression $\bexpr$
%  and query expression $\qexpr$ 
respectively.
%
\subsection{{Trace-based Operational Semantics}}
\label{sec:operational_semantics}
% \subsubsection{Event and Trace}
\paragraph*{Event}
An event is a triple.
% tracks useful information about each step of the evaluation, as a quadruple. 
An \emph{assignment event} corresponds to the evaluation of an assignment command $\clabel{\assign{x}{\expr}}^l$.
% the first element is either
% an assigned 
Its first element is the variable name $x$,
% (on the left hand of the command),
or a boolean expression (from the guard of if or while command), 
following by 
 the label, $l$ associated to this command and the value assigned to the variable.
 \\
 A \emph{testing event} corresponds to the evaluation of the boolean expression $b$ in the guard of a $\eif(\clabel{b}^l, c_1, c_2)$ command or $\ewhile \clabel{b}^l \edo c$.
 Its first element is the boolean expression $b$, following by 
 the label $l$ associated to the guard and the evaluated boolean result.
% or the boolean expression in the guard.
%
\[
\begin{array}{llll}
  \mbox{Event} 
  & \event & ::= & 
  % ~|~ ({x}, l, v, \qval)
  ({x}, l, v) ~ \mbox{Assignment Event} 
  % \\
  % &&& 
  ~|~(\bexpr, l, v) ~ \mbox{Testing Event}
\end{array}
\]
Event projection operators $\pi_i$ projects the $i$th element from an event: 
\\
$\pi_i : 
\eventset \to \mathcal{VAR}\cup \mbox{Boolean Expression}  \cup \mathbb{N} $
% \subsubsection{Trace}
\paragraph*{Trace}
%
A trace $\trace \in \tdom$ is a list of events, 
collecting the events generated during a specific program execution. 
\[
\begin{array}{llll}
\mbox{Trace} & \trace
& ::= & [] ~|~ \trace :: \event 
% ~|~ []^{\infty}
\end{array}
\]
A trace can be regarded as the program history, 
which records all the evaluations for assignment commands and guards in $\eif$ and $\ewhile$ command.
\\
\highlight{
A trace can be finite ($\trace \in \ftdom$) or infinite $\trace \in \inftdom$.
If a program doesn't terminate when executing under some initial trace,
it produces the infinite trace 
from $\inftdom$, which records a non-terminating computation.
So we denote by $\tdom$ the set of all traces, and $\tdom = \ftdom \cup \inftdom$.
The trace-based semantics with non-terminating execution is defined below following the maximal trace semantics in \cite{Cousot19}.}

We use list notation for traces, where $[]$ is the empty trace, the operator $\traceadd$ combines an event and a trace in a new event, 
and the operator $\tracecat$ concatenates two traces formally defined as follows. 

\begin{defn}[Trace Concatenation, $\tracecat: \tdom \to \tdom\to \tdom $]
  \label{def:trace_concate}
Given two traces $\trace_1 \in \tdom, \trace_2 \in \tdom$, the trace concatenation operator 
$\tracecat$ is defined as:
\[
\trace_1 \tracecat \trace_2 \triangleq
\left\{
\begin{array}{ll} 
  \trace_1 & \trace_2 = [] \lor \trace_1 \in \inftdom \\
  \trace_2 & \trace_1 = [] \lor \trace_2 \in \inftdom \\
  (\trace_1   \tracecat \trace_2'):: \event & \trace_1 \in \ftdom \land \trace_2 = \trace_2' :: \event
  % \trace_2 &  \trace_2 \in \inftdom \\
\end{array}
\right.
\]
\end{defn}

\begin{defn}(An Event Belongs to A Trace)
  An event $\event \in \eventset$ belongs to a trace $\trace \in \tdom$, i.e., $\event \in \trace$ are defined as follows:
%
\begin{equation*}
  \event \in \trace  
  \triangleq \left\{
  \begin{array}{ll} 
    \etrue                  & \trace =  [\event] \tracecat \trace'
     \land \event = \event' \\
    \event \in \trace' & \trace =  [\event'] \tracecat \trace'
      \land \event \neq \event' \\ 
    \efalse                 & \trace = [] \lor \trace \in \inftdom
  \end{array}
  \right.
\end{equation*}
As usual, we denote by $\event \notin \trace$ that the event $\event$ doesn't belong to the trace $\trace$.
\end{defn}
%
In the rest of the paper, we denote by $\bot$ a value s.t. $\bot < n $ for all $n \in \mathbb{N}$.
\begin{defn}[Counter Notation for Program Point]
  \label{def:counter}
The counting operator $\counter : \tdom \to \ldom \to (\mathbb{N} \cup \{\bot, \infty\})$
counts the appearance of a label in a trace.
\[
\begin{array}{llll}
\counter([(x, l, v)] \tracecat \trace', l ) \triangleq \counter(\trace', l) + 1 & \text{if}~ l = l
&
\counter([(b, l, v)] \tracecat \trace', l) \triangleq \counter(\trace', l) + 1 & \text{if}~ l = l
\\
\counter([(x, l', v)] \tracecat \trace', l) \triangleq \counter(\trace', l)   & \text{if}~ l' \neq l
&
\counter([(b, l', v)] \tracecat \trace', l) \triangleq \counter(\trace', l)   & \text{if}~ l' \neq l
\\
\counter({[]}, l) \triangleq 0 & 
&
\counter(\trace, l) \triangleq \bot & \text{if }~ \trace \in \inftdom
\end{array}
\]
{When the input trace is infinite, $\counter(\trace, l)$ returns $\bot$ for any program label $l$.}
\end{defn}
\begin{defn}[Counter Notation for List of Program Point]
  \label{def:lcounter}
  The counting operator $\lcounter : \tdom \to \mathcal{P}(\ldom) \to (\mathbb{N} \cup \{\infty\})$
  counts the appearance of a list of labels $[l_1, \ldots, l_n]$ as follows.
\[
  \begin{array}{ll}
  \lcounter(\trace'' \tracecat \trace', [l_1, \ldots, l_n] ) 
  \triangleq \lcounter(\trace', [l_1, \ldots, l_n]) + 1  & \text{if}~ \pi_2(\trace''[i]) = l_i, \forall i = 1, \ldots, n
  \\ 
  \lcounter([(\_, l, \_)] \tracecat \trace', [l_1, \ldots, l_n] ) 
  \triangleq \lcounter(\trace', [l_1, \ldots, l_n]) & \text{if}~ l \neq l_1
  \\ 
  \lcounter(\trace, [l_1, \ldots, l_n] ) 
  \triangleq \bot & \text{if }~ \trace \in \inftdom
\end{array}
\]
{When the input trace is infinite, $\lcounter(\trace, L)$ returns $\bot$ for any list of labels as well.}
\end{defn}
%
We define the operator $\tracel : \tdom \to \mathcal{P}{(\ldom)}$ projects the label from every event in a trace as a list of program points,
defined as follows,
\[
\tracel([(\_, l, \_)] \tracecat \trace') \triangleq [l] \tracecat \tracel(\trace')
\qquad
\tracel([ ]) \triangleq []
\]
%
\paragraph{Environment} $ \env : {\tdom}  \to \mathcal{VAR} \to( \mathbb{N} \cup \{\bot\})$
\[
\begin{array}{llll}
\env(\trace  \traceadd (x, l, v)) x \triangleq v
&
\env(\trace \traceadd (y, l, v)) x \triangleq \env(\trace) x, y \neq x
&
\env(\trace \traceadd (b, l, v)) x \triangleq \env(\trace) x
&
\env({[]} ) x \triangleq \bot
\end{array}
\]
\paragraph{Arithmetic Expression Semantics}
\begin{mathpar}
  \boxed{ \econfig{\aexpr}(\trace) = v \, : \, \mbox{Trace  $\times$ Arithmetic Expr $\Rightarrow$ Arithmetic Value} }
  \\
  \inferrule{ 
    \empty
  }{
   \econfig{n} (\trace)
   = n
  }
  \and
  \inferrule{ 
    \env(\trace) x = v
  }{
   \econfig{x} 
   = v
  }
  \and
  \inferrule{ 
    \econfig{\aexpr_1}(\trace) = v_1
    \and 
    \econfig{\aexpr_2}(\trace) = v_2
    \and 
     v_1 \oplus_a v_2 = v
  }{
   \econfig{\aexpr_1 \oplus_a \aexpr_2} 
   = v
  }
  \and
  \inferrule{ 
    \econfig{\aexpr}(\trace) = v'
    \and 
    \elog v' = v
  }{
   \econfig{\elog \aexpr}(\trace) 
   = v
  }
  \and
  \inferrule{ 
    \econfig{\aexpr}(\trace) = v'
    \and 
    \esign v' = v
  }{
   \econfig{\esign \aexpr} 
   = v
  }
   \end{mathpar}
\paragraph{Operational Semantics Rules}
The operational semantics rules for expression evaluation and command execution is presented below.

\highlight{
  Given an initial trace $\trace_0 \in \tdom_0(c)$ w.r.t. a program $c$,
we use $\to^*$ to represent the multiple-step execution. $\config{c, \trace_0} \to^{*} \config{\eskip, \trace_0 \tracecat \trace}$
represents that the program's execution terminates and produces a finite execution trace $\trace \in \tdom$.
When the program execution doesn't terminate under $\trace_0$, 
we use $\config{c, \trace_0} \to^{\infty} \config{\cdot, \trace_0 \tracecat \trace}$ to
represent the non-terminated execution w.r.t. an infinite trace $\trace \in \tdominf$.
\\
If we observe the operational semantics rules, we can find that no rule will shrink the trace. 
So we have the Lemma~\ref{lem:tracenondec} with proof in Appendix~\ref{apdx:lemma_sec123}, 
specifically the trace has the property that its length never decreases during the program execution.
\begin{lem}
[Trace Non-Decreasing]
\label{lem:tracenondec}
For any program $c \in \cdom$ and traces $\trace_0 \in \tdom_0(c), \trace \in \tdominf$, if 
$\config{c, \trace_0} \rightarrow^{*} \config{\eskip, \trace} $ or 
$\config{c, \trace_0} \rightarrow^{\infty} \config{\cdot, \trace}$,
then there exists a trace $\trace' \in \tdominf$ with $\trace_0 \tracecat \trace' = \trace$
%
$$
\forall \trace_0 \in \tdom_0(c), \trace \in \tdominf, c \st
\Big(
  \config{c, \trace_0} \rightarrow^{*} \config{\eskip, \trace} 
  \lor 
  \config{c, \trace_0} \rightarrow^{\infty} \config{\cdot, \trace}
\Big)
\implies \exists \trace' \in \tdominf \st \trace_0 \tracecat \trace' = \trace
$$
\end{lem}
}

{
\begin{mathpar}
\boxed{ \config{\trace, \bexpr} \barrow v \, : \, \mbox{Trace $\times$ Boolean Expr $\Rightarrow$ Boolean Value} }
\\% \\
\inferrule{ 
  \empty
}{
 \config{\trace,  \efalse} 
 \barrow \efalse
}
\and 
\inferrule{ 
  \empty
}{
 \config{\trace,  \etrue} 
 \barrow \etrue
}
\and 
\inferrule{ 
  \config{\trace, \bexpr} \barrow v'
  \and 
  \neg v' = v
}{
 \config{\trace,  \neg \bexpr} 
 \barrow v
}
\and 
\inferrule{ 
  \config{\trace, \bexpr_1} \barrow v_1
  \and 
  \config{\trace, \bexpr_2} \barrow v_2
  \and 
   v_1 \land v_2 = v
}{
 \config{\trace,  \bexpr_1 \land \bexpr_2} 
 \barrow v
}
\and 
\inferrule{ 
  \config{\trace, \bexpr_1} \barrow v_1
  \and 
  \config{\trace, \bexpr_2} \barrow v_2
  \and 
   v_1 \lor v_2 = v
}{
 \config{\trace,  \bexpr_1 \lor \bexpr_2} 
 \barrow v
}
\and 
\inferrule{ 
  \config{\trace, \aexpr_1} \aarrow v_1
  \and 
  \config{\trace, \aexpr_2} \aarrow v_2
  \and 
   v_1 \leq v_2 = v
}{
 \config{\trace,  \aexpr_1 \leq \aexpr_2} 
 \barrow v
}
\and 
\inferrule{ 
  \config{\trace, \aexpr_1} \aarrow v_1
  \and 
  \config{\trace, \aexpr_2} \aarrow v_2
  \and 
   v_1 < v_2 = v
}{
 \config{\trace,  \aexpr_1 < \aexpr_2} 
 \barrow v
}
\and 
\inferrule{ 
  \config{\trace, \aexpr_1} \aarrow v_1
  \and 
  \config{\trace, \aexpr_2} \aarrow v_2
  \and 
   v_1 = v_2 = v
}{
 \config{\trace,  \aexpr_1 = \aexpr_2} 
 \barrow v
}
\\
\boxed{ \config{\trace, \expr} \earrow v \, : \, \mbox{Trace $\times$ Expression $\Rightarrow$ Value} }
\\
\inferrule{ 
  \econfig{\aexpr}(\trace) = v
}{
 \config{\trace,  \aexpr} 
 \earrow v
}
\and
\inferrule{ 
  \config{\trace, \bexpr} \barrow v
}{
 \config{\trace,  \bexpr} 
 \earrow v
}
\and
\inferrule{ 
  \config{\trace, \expr_1} \earrow v_1
  \cdots
  \config{\trace, \expr_n} \earrow v_n
}{
 \config{\trace,  [\expr_1, \cdots, \expr_n]} 
 \earrow [v_1, \cdots, v_n]
}
\and
\inferrule{ 
  \empty
}{
 \config{\trace,  v} 
 \earrow v
}
 \end{mathpar}
%
The trace based operational semantics rules are defined as follows,
\begin{mathpar}
\boxed{
\mbox{Command $\times$ Trace}
\xrightarrow{}
\mbox{Command $\times$ Trace}
}
\and
\boxed{\config{{c, \trace}}
\xrightarrow{} 
\config{{c',  \trace'}}
}
\\
\inferrule
{
\empty
}
{
\config{\clabel{\eskip}^l,  \trace } 
\xrightarrow{} 
\config{\clabel{\eskip}^l, \trace}
}
~\textbf{skip}
%
\and
%
\inferrule
{
  \config{\expr, \trace} \earrow v
  \and
\event = ({x}, l, v)
}
{
\config{[\assign{{x}}{\expr}]^{l},  \trace } 
\xrightarrow{} 
\config{\clabel{\eskip}^l, \trace \traceadd \event}
}
~\textbf{assn}
\and
%
\inferrule
{
  \config{\trace, b} \earrow \etrue
 \and 
 \event = (b, l, \etrue)
}
{
\config{{\ewhile [b]^{l} \edo c, \trace}}
\xrightarrow{} 
\config{{
c; \ewhile [b]^{l} \edo c,
\trace \traceadd \event}}
}
~\textbf{while-t}
%
%
\and
%
\inferrule
{
  \config{\trace, b} \earrow \efalse
 \and 
 \event = (b, l, \efalse)
}
{
\config{{\ewhile [b]^{l}, \edo c, \trace}}
\xrightarrow{} 
\config{{
  \clabel{\eskip}^l,
\trace \traceadd \event}}
}
~\textbf{while-f}
%
%
\and
%
%
\inferrule
{
\config{{c_1, \trace}}
\xrightarrow{}
\config{{c_1',  \trace'}}
}
{
\config{{c_1; c_2, \trace}} 
\xrightarrow{} 
\config{{c_1'; c_2, \trace'}}
}
~\textbf{seq1}
%
\and
%
\inferrule
{
  \config{{c_2, \trace}}
  \xrightarrow{}
  \config{{c_2',  \trace'}}
}
{
\config{{\clabel{\eskip}^l; c_2, \trace}} \xrightarrow{} \config{{ c_2', \trace'}}
}
~\textbf{seq2}
%
\and
%
%
\inferrule
{
  \config{\trace, b} \earrow \etrue
 \and 
 \event = (b, l, \etrue)
}
{
 \config{{
\eif([b]^{l}, c_1, c_2), 
\trace}}
\xrightarrow{} 
\config{{c_1, \trace \traceadd \event}}
}
~\textbf{if-t}
%
\and
%
\inferrule
{
 \config{\trace, b} \earrow \efalse
 \and 
 \event = (b, l, \efalse)
}
{
\config{{\eif([b]^{l}, c_1, c_2), \trace}}
\xrightarrow{} 
\config{{c_2, \trace \traceadd \event}}
}
~\textbf{if-f}
%
\end{mathpar}
}