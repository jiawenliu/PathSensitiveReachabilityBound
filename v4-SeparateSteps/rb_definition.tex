The operator $\lvar: \cdom \to \mathcal{P}(\ldom)$,
computes the set of labels
% assigned variables and labeles variables 
for a labeled command $c$ defined as follows,
% defined in Definition~\ref{def:lvar} and 
% \ref{def:avar}.
%
\begin{defn}[Program Labels
(
% $\lvar_{c} \subseteq \mathcal{VAR} \times \mathbb{N}$ or 
$\lvar : \cdom \to \mathcal{P}(\ldom)$]
\label{def:lvar}
{\small
$$
  \lvar(c) \triangleq
  \left\{
  \begin{array}{ll}
      \{l\}                  
      & {c} = [{\assign x e}]^{l} 
      \\
      \lvar({c_1}) \cup \lvar({{c_2}})  \cup \{l\} 
      & {c} = {c_1};{c_2}
      \\
      \lvar(c_1) \cup \lvar({{c_2}}) \cup \{l\} 
      & {c} =\eif([\bexpr]^{l}, c_1, c_2) 
      \\
      \lvar({{c}'}) \cup \{l\} 
      & {c}   = \ewhile ([\bexpr]^{l}, {c}')
\end{array}
\right.
$$
}
\end{defn}
%
Every label and every labeled command in a program is unique, formally as follows with proof in Appendix~\ref{apdx:lemma_sec123}.
\begin{lem}[Uniqueness of the Program Labels]
  \label{lem:label_unique}
  For every program $c \in \cdom$ and every two labels such that
  $i, j \in \lvar(c)$, then $i \neq j$.
  \[
    \forall c \in \cdom, i, j \in \ldom \st i, j \in \lvar(c)\implies i \neq j.
    \]
\end{lem}
%
The free variables
showing up in $c$, which aren't defined before be used, are actually the input variables of this program.
%
\begin{defn}[Reachability-Bound]
  \label{def:rb}
  For a program ${c}$ and a location $l \in \lvar(c)$ in this program,
% its 
% $\exeRB({c}, l)$ 
a function $f(c, l) : \tdom_0(c) \to (\mathbb{N} \cup \{\infty\})$ is a \emph{Reachability-Bound} for $l$ if and only if
\highlight{
\[
  \forall \trace_0 \in \tdom_0(c), \trace \in \tdominf \st 
  \Big(
    \config{{c}, \trace_0} \to^{*} \config{\eskip, \trace_0 \tracecat \vtrace} 
    \lor 
    \config{{c}, \trace_0} \to^{\infty} \config{\cdot, \trace_0 \tracecat \vtrace} 
  \Big)
  \implies f({c}, l)(\trace_0) \geq \vcounter(\vtrace, l) 
  \]
}
\end{defn}
\highlight{
Given a program point $l$ in $c$, our algorithm (defined below) computes a Reachability-Bound for it.
It is easy to compute a trivial \emph{Reachability-Bound} $f(c, l): \tdom_0(c) \to \infty$, but it is not interesting to us.
\\
In the following sections, we only focus on computing a finite \emph{reachability-bound} for program's given location and considering the executions that terminates.
Ideally, we aim to compute a precise reachability-bound as follows.
}
\begin{defn}[Precise Reachability-Bound]
  \label{def:exe_rb}
  For a program ${c}$ and a location $l \in \lvar(c)$ in this program,
% its 
$\exeRB({c}, l): \tdom_0(c) \to (\mathbb{N} \cup \{\infty\})$ is a \emph{Precise Reachability-Bound}  if and only if,
%  satisfying
%  defined as follows,
% over all possible traces,
%
\highlight{
\[
  \forall \trace_0 \in \tdom_0(c), \trace \in \tdom \st 
  \Big(
    \config{{c}, \trace_0} \to^{*} \config{\eskip, \trace_0 \tracecat \vtrace} 
    \lor 
    \config{{c}, \trace_0} \to^{\infty} \config{\cdot, \trace_0 \tracecat \vtrace} 
  \Big)
  \implies \exeRB({c}, l)(\trace_0) = \vcounter(\vtrace, l) 
  \]
}
\end{defn}
