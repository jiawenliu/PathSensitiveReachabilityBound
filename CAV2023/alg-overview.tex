
\textbf{Step 1}
in Section~\ref{sec:progabs} first 
generates the \emph{Abstract Transition Graph} as in Figure~\ref{fig:relatedNestedWhileOdd-overview}(b).
Each edge $l \xrightarrow{dc} l'$ is an abstract transition $\absevent = (l, dc, l')$ annotated with a constraint $dc$ corresponding to the command of label $l$.

% \textbf{Step 2: Program Refinement}
\textbf{Step 2} in Section~\ref{sec:refine}
% The second step in Section~\ref{sec:refine}
computes the \emph{Refined Program}, $\rprog$ for a program $c$ based on 
its abstraction transition graph and transforms the multiple-paths loops
into multiple loops where
the interleaving of paths is explicit as in bottom part of Figure~\ref{fig:relatedNestedWhileOdd-overview}(c).
It has two interleaving patterns in the loop $L_1$.
%  denoted as $\rprog_1^1$ and $\rprog_1^2$.

% \textbf{Step 3: Ranking Function Estimation}
\textbf{Step 3} computes the \emph{Ranking Function}, $\locbound(\absevent, c)$ 
for every edge $\absevent$ 
and estimates an upper bound invariant, which is presented in Section~\ref{sec:rank}.

% \textbf{Step 4: Path-sensitive Reachability-bound Computation.}
\textbf{Step 4}
% The \emph{path-sensitive reachability-bound} 
computes the \emph{Reachability-bound}, $\psRB(l, c)$ for every program point $l$ using the $\rprog$ and the upper bound invariant of the $\locbound(\absevent, c)$ where $\absevent = (l,dc,l')$.
\todo{It requires computing the \emph{path reachability-bound}, $\inoutB(\rprog, \tpath, c)$ for every $\tpath$.
If there are nested loops, we need to compute the \emph{loop reachability-bound}, $\lpchB(l: \rprog, \tpath, c)$ w.r.t. every level of $\tpath$'s outer loop. }
Section~\ref{sec:psrb} introduces this algorithm and the following sections describe the computations. 
The soundness and proof of each algorithm is given in the Appendices and the input $c$ will be omitted in the rest of the paper if the context is clear.

In the first interleaving pattern, $\rprog_1^1 = 1: \rprepeat(\tpath_1; 4:\rprepeat(\tpath_3); \tpath_2; \tpath_4)$ in Figure~\ref{fig:relatedNestedWhileOdd-overview},
we first compute $\outinB(4:\rprepeat(\tpath_3), \tpath_3, c) = n - m$
 for $\tpath_3$ in its innermost loop $L_4$ as a local \emph{path reachability-bound} by Section~\ref{sec:pathlocalrb}.
Then we compute $\lpchB(\rprog_1^1, \tpath_3, c) = 1$ w.r.t. its outer loop $L_1$ in Section~\ref{sec:looprb}. In the second interleaving pattern $\rprog_1^2$, we compute $\outinB(4:\rprepeat(\tpath_3), \tpath, c) = n - m - 3$ and the same $\lpchB(\rprog_1^2, \tpath_3, c)$.
So Section~\ref{sec:pathrb} computes $\inoutB(\rprog, \tpath_3, c) = \max\{ 1 \times (n - m), 1 \times (n - m - 3) \} = n - m$ globally.
%
Then for every program point $l$, we sum up all the $\inoutB(\rprog, \tpath)$ over $\tpath$ that contains $l$ and get $\psRB(l, c)$.
Since point $5$ only shows up on $\tpath_3$, we compute \highlight{$\psRB(5, c) = n - m$}.
The points $0$ and $\lex$ are not in any loop, so $\psRB(0, c) = \psRB(\lex, c) = 1$,
The points $3, 6, 7$ and $8$ which only show up once on $\tpath_2$ and $\tpath_4$ are all equal to $\lfloor\frac{m}{4}\rfloor$, which are same as their $\inoutB$.
For the loop headers $1$ and $4$, we only count the $\tpath$ where they show up as start-point.
So $\psRB(4, c) =  \lfloor\frac{m}{4}\rfloor + n - m + 1$ and $\psRB(1,c) = 2 \times \lfloor\frac{m}{4}\rfloor + 1$ all as expected.