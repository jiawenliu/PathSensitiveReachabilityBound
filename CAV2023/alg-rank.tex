The algorithm in this step is inspired from the Algorithm.2 in paper~\cite{SinnZV14},
  % which assigns a variable to each edge on which this variable decrease as its ranking function.
  the Algorithm.3 in paper~\cite{ZulegerGSV11},
  and the Definition.25 in Section 4 of paper~\cite{sinn2017complexity}.
  Algorithm.3 in paper~\cite{ZulegerGSV11} assigns a set of variables to each transition in which these variables decrease as the local bound
  and estimates the maximum value each variable in this set.
  Algorithm.2 in paper~\cite{SinnZV14} assigns a variable to each edge on which this variable decrease as its ranking function
  and then estimates the maximum value for the ranking function.
  The Definition.25 in paper~\cite{sinn2017complexity}
  assigns each transition with a variable that decreases in this transition, as the local bound and computes the bound similarly.
\begin{enumerate}
\item 
For each variable $x$ in a program $c$, this algorithm first computes three edge sets, $\inc(c, x)$, $\dec(c, x)$,
and $\reset(c, x)$ for $x$.
Every edge in a set corresponds to a transition in which $x$ is increased, decreased or reset respectively.
  \item
  Then, 
  For each edge in the transition graph $\absG(c)$ of a program $c$,
  the second step assigns the variable that decreases on this edge as the ranking function   of this edge.
  This step adopts the local bound computation method in Section 4 of \cite{sinn2017complexity} to assign the local bound to each edge,
  formally as follows.
  \begin{defn}[Ranking Function   Generatation]
    \label{def:ranking_gen}
  % Given a program $c$ with its abstract transition graph 
  % $\absG(c) = (\absV, \absE)$,
  For every edge $\absevent$ in the transition graph $\absG(c)$ of a program $c$,
  its \emph{ranking function/local bound}, $\locbound(\absevent, c)$
  is the variable that decreases on this edge, computed as follows,
  %
  \[ 
  \begin{array}{ll}
    \locbound(\absevent, c) \triangleq 1 
    & \absevent \notin SCC(\absG(c))
    \\
    \locbound(\absevent, c) \triangleq x
    & \absevent \in SCC(\absG(c)) \land \absevent \in \dec(x) \land  \absevent = (\_, \_ , x' \leq x - v) \\
    \locbound(\absevent, c) \triangleq x
    & \absevent \in SCC(\absG(c)) \land 
    \absevent  \notin \bigcup_{x \in \mathcal{VAR}} \dec(x)
    \land \absevent \notin SCC(\absG(c) \setminus \dec(x)).
  \end{array}
  \]
  $SCC(\absG(c))$ is the set of all the strong connected components of $\absG(c)$.
  \end{defn}
  \item
  In the last step, it estimates the upper bound,
  $\varinvar(x, c) \in \mathcal{A}_{\lin}$ on the maximum value of each ranking function recursively as follows.
  \\
  For a program $c$, the \emph{ranking function bound},
  $\varinvar(\locbound(\absevent, c)) \in \mathcal{A}_{\lin}$ is 
  the bound on the maximum value of the ranking function  
  assigned to the edge $\absevent \in \absE(c)$, formally in Definition~\ref{def:ranking_bound}.
  \\
  In order to estimate the maximum value of $\locbound(\absevent, c)$ assigned to edge $\absevent \in \absE(c)$,
  % for each (ranking function's) maximum value,
  the bound on the iteration times of each corresponding edge, $\absclr(\absevent, c)$ 
  is computed interactively in a path-insensitive manner.
  % , the 
  \\ 
  $ \varinvar: (\mathcal{VAR} \cup \constdom  \times \cdom) \to \mathcal{A}_{\lin}$
  \\
  $\absclr: (\absevent \times \cdom) \to \mathcal{A}_{\lin}$
  \begin{defn}[Ranking Function Estimation]
    \label{def:ranking_bound}
  For a program $c$ and an edge $\absevent \in \absE(c)$,
  the \emph{ranking function bound}, $\varinvar(\locbound(\absevent, c))$ for the ranking function $\locbound(\absevent, c)$
  of this edge
  is computed as follows,
    \[ 
  \begin{array}{lll}
    \varinvar(x, c) & \triangleq x & x \in \constdom \\
    \varinvar(x, c) & \triangleq \incrs(x, c) + \max(\{\varinvar(y, c) + c ~\mid~ (l, x' \leq y + c, l') \in \reset(x)\}) & c \notin \constdom
  \end{array}
  \]
  %
  $\incrs(x, c) \triangleq \sum\limits_{\absevent \in \inc(v)}\{\absclr(\absevent, c) \times c ~\mid~ \absevent = (l, x' \leq x + c, l')\}$ where 
  $\absclr(\absevent, c) \in \mathcal{A}_{\lin}$  is computed as below,
\[ 
\begin{array}{lll}
  \absclr(\absevent, c) 
  & \triangleq \varinvar(\locbound(\absevent, c))  & \\
  & \quad \locbound(\absevent, c) \in \constdom & \\
  \absclr(\absevent, c) 
  & \triangleq \Big(
    \sum\limits_{y \in \{ y ~|~ 
    ch \in \resetchain(x), (l_1, x, y, v, l_2) \in ch \} } \incrs(y, c) & \\
    & \quad + 
  \sum\limits_{ch \in \resetchain(x, c)}
  \big( \min\left\{\absclr(\absevent', c) ~\mid~ \absevent' \in ch\right\} \times 
  \max\left\{\varinvar(in(ch), c) + \sum\limits_{(l_1, x, y, v, l_2) \in ch } v, 0 \right\}\big) \Big)  & \\
  &  \quad \locbound(\absevent, c) = x \land x \notin \constdom & ,
\end{array}
  \]
 where $in(ch)$ is the variable on the last edge on the reset chain $ch$.
\end{defn}
  \end{enumerate}

  