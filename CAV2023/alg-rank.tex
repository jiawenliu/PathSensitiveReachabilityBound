% The approach in this step is inspired mainly from the Alg.~2 in~\cite{SinnZV14},
% % which assigns a variable to each edge on which this variable decreases as its ranking function.
% the Alg.~3 in~\cite{ZulegerGSV11},
% and the Def.~25 in Section 4 from~\cite{SinnZV17}, and enriched with path-sensitivity.
\paragraph{Collecting Variable Modifications}
 For each variable $x$ in a program $c$, we first compute three transition sets, $\inc(x, c)$, $\dec(x, c)$,
 and $\reset(x, c)$ for $x$
 % Every edge in a set corresponds to a transition 
 in which $x$ is increased,
 % $\inc(x, c)$,
 decreased
 % $\dec(x, c)$ and 
 or reset
 % $\reset(x, c)$, 
 respectively.
 \begin{defn}[Variable Modifications]
 \label{def:var_modi}
 For each variable $x$ in a program $c$,
 $\inc, \dec : \vardom \to \cdom \to \mathcal{P}(\absevent) $
 is the set of edges where the variable increase and decreases respectively, 
 where $\inc(x, c) = \left\{ \absevent | \absevent = (l, x' \leq x + v, l') \land \absevent \in \absE(c) \right\}$,
 % $\dec: \vardom \to \cdom \to \mathcal{P}(\absevent) $
 % is the set of abstract events where the variable decrease where
 %\\
 and $\dec(x, c) = \left\{\absevent| \absevent = (l, x' \leq x - v, l') \land \absevent \in \absE(c) \right\}$;
 %\\
 $\reset: \vardom \to \cdom \to \mathcal{P}(\absevent) $ is the set of abstract events where the variable is assigned by other variables or symbols such that
 %
 $\reset(x, c) = \left\{ \absevent| \absevent = (l, x' \leq y - v, l') \land x \neq y \land \absevent \in \absE(c) \right\}$. 
 \end{defn}
 % We also compute the reset graph and the reset chain in Appendices~\ref{apdx:alg-resetgraph}. 
% \\
% $\inc: \cdom \to \vardom \to \mathcal{P}(\absevent) $
% is the set of edges where the variable increase, 
% %\\
% \[ \inc(x, c) = \left\{ \absevent | \absevent = (l, x' \leq x + v, l') \land \absevent \in \absE(c) \right\} \]
% %\\
% $\dec: \vardom \to \mathcal{P}(\absevent) $
% is the set of abstract events where the variable decrease,
% %\\
% \[\dec(x, c) = \left\{\absevent| \absevent = (l, x' \leq x - v, l') \land \absevent \in \absE(c) \right\}\]
% %\\
% $\reset: \cdom \to \vardom \to \mathcal{P}(\absevent) $ is the set of the abstract events where the variable is reset,
% %
% \[\reset(x, c) = \left\{ \absevent| \absevent = (l, x' \leq y - v, l') \land x \neq y \land \absevent \in \absE(c) \right\}\]
 % Additionally,
 % we compute the reset graph and the reset chain for higher precision in Appendices~\ref{apdx:alg-resetgraph}. 
 % The computation is mainly following the Definition~20 in~\cite{SinnZV17}.
 % \[\resetG(c) = (\resetV(c), \resetE(c))\]
 % \[\resetE(c) = \left\{ (x, \absevent, y) ~\vert~ \absevent \in \reset(x, c) \land \absevent = (l, x' \leq y + c, l') \right\} \]
 % \[\resetV(c) = \left\{ x ~\vert~ (x, \_, \_) \in \resetE(c) \lor (\_, \_, x) \in \resetE(c) \right\} \]
 % In a variable $x$'s reset chain set, $\resetchain(x, c)$, in each chain $(e_0, \ldots, e_m) \in \resetchain(x, c)$
 % a variable $x_i$ is reset by another variable $x_{i + 1}$ on edge $e_{i}$
 % and $x_{i + 1}$ is reset on edge $e_{i + 1}$ recursively
 % for every $i = 0, \ldots, m - 1$.
 % $x$ is reset on the first edge $e_0$ of every sequence in $\resetchain(x, c)$.
 % {Each edge $e_i$ in a sequence $(e_0, \ldots, e_m) \in \resetchain(x, c)$
 % resets a variable $x_i$ by another variable $x_{i + 1}$ such that $x_{i + 1}$
 % is reset on edge $e_{i + 1}$ recursively. The first edge $e_0$ of each sequence resets the variable $x$.}
 % % 
 % \\
 % In the following steps, $c$ is omitted in $\inc(x, c)$,
 % $\dec(x, c)$ and $\reset(x, c)$ for concise when the reference of a program $c$ is clear in the context.

 \paragraph{Assigning The Ranking Function}
 For each edge in the transition graph $\absG(c)$ of a program $c$,
 we assign the variable that decreases on this edge as the ranking function of this edge.
 It is similar to the local bound computation in~\cite{SinnZV17}.
 % to assign the local bound to each edge as follows.
 \begin{defn}[Ranking Function Generatation]
 \label{def:ranking_gen}
 For every edge $\absevent$ in the transition graph $\absG(c)$ of a program $c$,
 its \emph{ranking function}, $\locbound(\absevent, c)$
 is the symbol $x\in \scvar(c)$ that decreases on this edge as follows where $SCC(\absG(c))$ is the set of all the strong connected components of $\absG(c)$.
{\small
\[ 
\begin{array}{ll}
 \locbound(\absevent, c) \triangleq 1 
 & \absevent \notin SCC(\absG(c))
 \\
 \locbound(\absevent, c) \triangleq x
 & \absevent \in SCC(\absG(c)) \land \absevent \in \dec(x, c) \land \absevent = (\_, \_ , x' \leq x - v) \\
 \locbound(\absevent, c) \triangleq x
 & \exists \absevent' \in \absG(c), \tpath \in \paths( \absG(c)) \st \absevent, \absevent' \in \tpath \land \locbound(\absevent', c) = x \\
 % \in SCC(\absG(c)) \land \absevent \in \dec(x, c) \land \absevent = (\_, \_ , x' \leq x - v) \\
 \locbound(\absevent, c) \triangleq x
 & \absevent \in SCC(\absG(c)) \land 
 \absevent \notin \bigcup_{x \in \vardom} \dec(x, c)
 \land \absevent \notin SCC(\absG(c) \setminus \dec(x, c))\\
 \locbound(\absevent, c) \triangleq \infty
 & o.w.
\end{array}
\]
}
 \end{defn}
 %
 % \todo{Example of ranking function estimated}
 In Figure~\ref{fig:relatedNestedWhileOdd-overview}(b), we assign variable $i$ to edge $7 \xrightarrow{i' \leq i - 1} 1$ as ranking.
 If we remove this edge, the edges on $\tpath_2$ and $\tpath_1$ are not in any SCC, so they also have the same ranking.
In the same way, ranking for edges on $\tpath_4$ is $i$ and $\tpath_3$ is $k$.
 \paragraph{Estimating the Upper Bound Invariant}
Then we estimate the upper bound invariant, $\varinvar(\locbound(\absevent, c), c)$ for the ranking function of each abstract transition, $\locbound(\absevent, c)$. Meanwhile, we interactively compute an iteration upper bound, $\absclr(\absevent, c)$ for each $\absevent$ path-insensitively when estimating the ranking function invariant.
\begin{defn}[Ranking Function Estimation]
 \label{def:ranking_bound}
For a program $c$ and an edge $\absevent \in \absE(c)$,
the \emph{ranking function bound}, 
$\varinvar(\locbound(\absevent, c), c)$ for the ranking function $x = \locbound(\absevent, c)$
of this edge
is computed as follows,
{\small
 \[ 
\begin{array}{lll}
 \varinvar(x, c) & \triangleq x & x \in \scvar(c) \\
 \varinvar(x, c) & \triangleq \incrs(x, c) + \max\left\{\varinvar(y, c) + v ~\mid~ (l, x' \leq y + v, l') \in \reset(x, c) \right\} & x \notin \scvar(c)
\end{array}
\]
}
%
where $\incrs(x, c) \triangleq \sum\limits_{\absevent \in \inc(x, c)}\{\absclr(\absevent, c) \times v ~\mid~ \absevent = (l, x' \leq x + v, l')\}$
The path-insensitive bound, $\absclr(\absevent, c) \in \scexpr(c)$ on the execution times of the transition $\absevent$, is interactively computed as well as below,
% \footnote{We only give the computation based on the variable reset set ($\reset$) for easier understanding
% % instead of the reset chain ($\resetchain$), 
% and the approach based on the reset graph is in Appendices~\ref{apdx:alg-resetgraph}.
% }
{\small
% \[ 
% \begin{array}{lll}
% \absclr(\absevent, c) 
% & \triangleq \varinvar(\locbound(\absevent, c), c) & \\
% & \quad \text{if} ~ \locbound(\absevent, c) \in \scvar(c) & \\
% \absclr(\absevent, c) 
% & \triangleq
% \sum \left\{ \incrs(y, c) | ch \in \resetchain(x, c) \land y \in ch \right\} & \\
% & \quad + 
% \sum\limits_{ch \in \resetchain(x, c)}
% \min \left\{\absclr(\absevent', c) ~\mid~ \absevent' \in ch\right\} \times 
% \big(\varinvar(in(ch), c) 
% + \sum\limits_{(\_, (\_, x' \leq y + v, \_), \_) \in ch} v \big) & \\
% & \quad \text{if} ~\locbound(\absevent, c) = x \land x \notin \scvar(c) & ,
% \end{array}
% \]
\[ 
\begin{array}{lll}
 \absclr(\absevent, c) 
 & \triangleq \varinvar(\locbound(\absevent, c), c) \qquad \qquad \text{if} \quad \locbound(\absevent, c) \in \scvar(c) & \\
 \absclr(\absevent, c) 
 & \triangleq \incrs(x, c) 
 + 
 \sum\limits_{\absevent' \in \reset(x, c) \land \absevent' = (l, x \leq y + v, l') }
 \Big( \absclr(\absevent', c) \times \big( \varinvar(y, c) + v \big) \Big)
 & \\
 & \text{if} \quad \locbound(\absevent, c) = x \land x \notin \scvar(c) &.
\end{array}
 \]
}
% where $in(ch)$ is the first vertex of the reset chain $ch$.
\end{defn}
% \todo{Example of computed invariant for ranking function and the transition bound}
In Figure~\ref{fig:relatedNestedWhileOdd-overview}(a), the ranking $i$ is reset by $n$ at edge $0 \to 1$ and 
$k + m$ at edge $6 \to 7$ and $k$ is assigned value $i - m$ at edge $3 \to 4$. Through reset graph, 
% (which we do not fully present but can be found in the Definition~20 in~\cite{SinnZV17}), 
we estimate a symbolic value $n$ for both ranking variables $i$ and $k$.
Interactively, iteration bounds for all the edges in this loop are $n$ as well path-insensitively.