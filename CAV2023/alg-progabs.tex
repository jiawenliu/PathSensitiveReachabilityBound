This path-sensitive reachability-bound algorithm
is performed on basis of an \emph{Abstract Transition Graph}, $\absG(c)$
% $ =(\absV(c), \absE(c))$ 
for the program $c$.
We first introduce some components of this graph.
\begin{defn}[Symbolic Constant, Symbolic Expression]
 \label{def:symbolic_expr}
The universe of the \emph{symbolic constant} is denoted by $\scvardom \subseteq \mathbb{N} \cup \vardom \cup \{\infty \}$, in which a symbolic constant can be either a natural number, or $\infty$, or a program's input variable.
 The set of the \emph{symbolic expression} for a program $c$, $\scexpr(c) \subseteq \mathcal{A}$ is all the arithmetic expressions over $\mathbb{N} \cup \inpvar(c) \cup \{\infty \}$.
\end{defn}

\begin{defn}[Difference Constraints]
 A difference constraint $d$ is an inequality of
form $x' \leq y + v$ or $x' \leq v$ where $x, y \in \vardom $ and $v \in \scvardom$.
An inequality $x' \leq y + v$ describes that the value of $x$ in the current state is
at most the value of $y$ in the previous state plus the symbolic constant $v$, and $x' \leq v$ describes that the value of $x$ in the current state is
at most $v$.
$DC(\vardom \cup \scvardom)$ denotes the set of all difference constraints over $\vardom \cup \scvardom$.
\end{defn}

\begin{defn}[Constraints]
A constraint $dc \in \dcdom^{\top}$
is either a
difference constraint $d \in DC(\vardom \cup \scvardom)$, a boolean expression $\bexpr \in \booldom$
or $\top$ denotes always true.
\end{defn}
When the constraint in a transition is a difference constrain, $l \xrightarrow{x' \leq y + v} l'$,
% Then $x'$ 
it denotes that
the value of variable $x$
after executing the command at $l$ is at most
% and the right-hand side describes 
the value of variable $y$ plus $v$ before the execution,
and $l \xrightarrow{x' \leq v} l'$ respectively denotes
$x$ has at most
% and the right-hand side describes 
the value of the symbolic constant $v$.
% before the execution.
For example in Figure~\ref{fig:relatedNestedWhileOdd-overview}(b), constraint $i' \leq i - 1$ on the edge $7 \xrightarrow{i' \leq i - 1} 1$
describes the execution of
 the command at line $7$, 
$\clabel{\assign{i}{i - 1}}^{7}$, and value of $i$ is at most the previous $i$'s value minus $1$.
%
% For every expression in each of the label commands, it is computed in three steps via the program abstraction method adopted from Section~6 in~\cite{SinnZV17}. 
%
The boolean constraint, $\bexpr$ on an edge $l \xrightarrow{b} l'$ describes
that after evaluating the guard with label $l$,
$\bexpr$ holds to execute the command with label $l'$.
In Figure~\ref{fig:relatedNestedWhileOdd-overview}, $i \leq 0 $ on the edge $1 \xrightarrow{i \leq 0} \lex$, 
represents the negation of the testing guard $\clabel{i > 0}^1$
in the $\ewhile$ command, and $i \leq 0$ must hold in order to perform this transition from program point $1$ to
the program exit. 
$\top$ is preserved for $\eskip$ command or the commands that don't interfere with any guard variable.

\begin{defn}[Abstract Transition Graph]
 \label{def:abs_cfg}
 The \emph{abstract transition graph} of a program $c$ is $\absG(c) \triangleq (\absV(c), \absE(c))$, where
 $\absV(c) \triangleq \lvar(c)\cup\{\lex\}$
 and 
 % $\absE(c) \triangleq \{(l_1, dc, l_2) | (l_1, dc, l_2) \in \absflow(c)\}$,
 % and
 % .
 each edge $(l, dc, l') \in \absE(c)$ is an abstract transition
between two program points $l, l'$ if and only if
the command with label $l'$ can execute right after the execution of the command with label $l$.
% if and only if there is a control flow between two program points.
The constraint $dc \in \dcdom$ on each edge
describes the abstract execution of the command with label $l$.
\end{defn}
Again in Figure~\ref{fig:relatedNestedWhileOdd-overview}(b),
the edge $(0 \xrightarrow{i' \leq n} 1)$ on the top tells us the command 
$\clabel{\assign{i}{n}}^0$ is executed with a continuation point $1$, and the while loop with header at location $1$, $\ewhile \clabel{i > 0}^1 \edo \{\ldots\}$ will be executed right after.

\begin{defn}[Path]
 \label{def:abs_cfgpath} 
 A path on $\absG(c)$ is a sequence, $ l_0 \xrightarrow{dc_0} l_1 \xrightarrow{dc_1} \ldots $ with
 \begin{itemize}
 \item the vertices $(l_0, \ldots)$, where $l_i \in \absV(c)$ for every $i = 0, 1, \ldots$ and
 %
 \item the edges $(e_0, \ldots)$, where $e_i = (l_{i}, dc_i, l_{i + 1}) \in \absE(c)$ for every $i = 0, 1, \ldots$.
 \end{itemize}
 A path is cyclic if it has the same start- and end-point. A path is simple if it does not visit a location twice except for the start- and end-location. We use $\paths(\absG(c))$ to denote the set of all the paths on $\absG(c)$,
 and $\pathl(p)$ to represent the list of program points corresponding to the vertices sequence of this path $p \in \paths(\absG(c))$,
 where $\pathl: \paths(\absG(c)) \to \mathcal{P}{(\ldom)}$.
 \end{defn}
 In Figure~\ref{fig:relatedNestedWhileOdd-overview}(b), $1 \to 2 \to 3 \to 4 \to 5 \to 4$ is a \emph{path}, but it is not simple (the program points $4$ is visited twice). The path $4 \to 5 \to 4$ is both cyclic and simple.