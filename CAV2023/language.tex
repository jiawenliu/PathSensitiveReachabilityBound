We use a standard integer while language with labels to identify different program locations and equipped with trace-based operational semantics.

\emph{Labeled Command}
The commands are the typical ones from while languages with each annotated with a label $l \in (\mathbb{N} \cup \{\lin, \lex\})$ to uniquely identify each program point. The program is a labeled command $c$ having the following syntax. 
\\
$ \qquad \qquad
{c} ::= 
\clabel{\assign{x}{\expr}}^l 
~|~ \clabel{\eskip}^l
~|~ \ewhile \clabel{\bexpr}^{l} \edo ({c})
~|~ \eif(\clabel{\bexpr}^{l} , {c}, {c}) 
~|~ {c};{c} 
$
\\
Expressions include
integer arithmetic expression, $\aexpr \in \mathcal{A}$ (with the value $n \in \mathbb{N} \cup \{ \infty\}$ and the variable $x$ from a finite universe $\vardom$) and boolean expression, $\bexpr \in \booldom$, where we denote by $\infty$ a value s.t. $n < \infty $ for all $n \in \mathbb{N}$.
% The operator $\kw{FV}: \expr \to \mathcal{P}(\mathcal{V})$ computes the set of free variables in an expression.
We denote by $\cdom$ the universe of all labeled commands, $\ldom$ for all labels, and
$\inpvar(c) \subseteq \vardom$ the set of all input variables in a program $c$.


\emph{Event and Trace.}
An event, $\event \in \eventset$ is a triple $({x}, l, v)$--\emph{assignment event} or $(\bexpr, l, v)$--\emph{testing events} tracking the execution of an assignment command or evaluation of $b$ in
% if and while commands, specifically the evaluation of the boolean expression $b$ in 
the guard for a $\eif(\clabel{b}^l, c_1, c_2)$ command or $\ewhile \clabel{b}^l \edo (c)$ respectively.
Its first element is the variable name $x$
or the boolean expression $b$, 
following by 
the label, $l$ of this command, and the evaluated value.
Event projection operators $\pi_i$ projects the $i^{th}$ element from an event: 
$\pi_i : \eventset \to \vardom \cup \booldom \cup \ldom $.


A trace $\trace \in \tdom$ is a list of events, 
collecting the events generated in each program evaluation step during the execution. 
We use list notation for traces, where $[]$ is the empty trace,
% the operator $\traceadd$ combines an event and a trace in a new event, 
the operator $\tracecat$ concatenates two traces, and $\event \in \trace$ or $\event \notin \trace$ denotes an event $\event$ belongs or not to $\trace$.
{
A trace can be finite, $\trace \in \ftdom$ or infinite $\trace \in \inftdom$ if a program doesn't terminate,
and $\tdom = \ftdom \cup \inftdom$.
% The trace-based semantics with non-terminating execution is defined below following the maximal trace semantics in~\cite{Cousot19}.
}
We denote by $\bot$ a value s.t. $\bot < n $ for all $n \in \mathbb{N}$, and we use the operator $\env: {\ftdom} \to \vardom \to(\mathbb{N} \cup \{\bot\})$ to fetch the latest value assigned to a variable in the trace, which returns $\bot$ if the variable doesn't have a value in the trace.
% In the rest of the paper, We denote by $\bot$ a value s.t. $\bot < n $ for all $n \in \mathbb{N}$.
We use $\ftdom_0(c)$ to denote the set of all initial traces for the program $c$, and each input variable $x \in \inpvar(c)$ has an initial value in an initial trace $\trace_0 \in \ftdom_0(c)$.

\emph{Trace-based Semantics}
We evaluate the expression by algorithmic computation in math, denoted by
$\econfig{} : \mathcal{A} \cup \booldom \to \tdom \to \mathcal{V}$ where
% for the expression evaluation, and 
$\econfig{\expr}(\trace)$ evaluates $\expr$ under trace $\trace$.
Our symbolic bound expression is a subset of the arithmetic expressions, and we use the same evaluation notation for evaluating the bound expression.
The trace-based operational semantics is described in terms of a small step evaluation relation $\config{c, \trace} \to \config{c', \trace'}$ describing how a configuration program-trace evaluates to another
one.
The rules are standard and omitted here.
We use $\to^*$ for the reflexive and transitive closure of $\to$. 
If $\config{c, \trace_0} \rightarrow^{*} \config{\clabel{\eskip}^l, \trace_0 \tracecat \trace}$ under an initial trace 
$\trace_0 \in \ftdom_0(c)$,
then the program's execution terminates and produces a finite execution trace $\trace \in \ftdom$. We also have non-terminating execution. The complete rules and definition can be found in Appendices~\ref{apdx:language}.