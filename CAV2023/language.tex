We use a standard integer while language with labels to identify different program locations and equipped with a  trace-based operational semantics.
\paragraph{Labeled Command}
The commands are the typical ones from while languages. Each command is annotated with a label $l \in (\mathbb{N} \cup \{\lin, \lex\})$. The program is a labeled command $c$ having the following syntax. 
\[
{c} ::= 
\clabel{\assign{x}{\expr}}^l 
~|~  \clabel{\eskip}^l
~|~ \ewhile \clabel{\bexpr}^{l} \edo ({c})
~|~ \eif(\clabel{\bexpr}^{l} , {c}, {c}) 
~|~ {c};{c} 
\]
We use label to record
the location of each command, so that we can uniquely identify each program point.
We denote by $\cdom$ the universe of all labeled commands and $\ldom$ for all labels.
% \[
% \begin{array}{llll}
% % \mbox{Label} 
% & l & \in & (\mathbb{N} \cup \{\lin, \lex\}) 
% \\ 
% %
% % \mbox{Labeled Command} 
% & {c} & ::= &  
% \clabel{\assign{x}{\expr}}^l 
% ~|~  \clabel{\eskip}^l
% ~|~ \ewhile \clabel{\bexpr}^{l} \edo ({c})
% ~|~ \eif(\clabel{\bexpr}^{l} , {c}, {c}) 
% ~|~ {c};{c}  
% \end{array}
% \]
Expressions include
standard arithmetic expression, $\aexpr \in \mathcal{A}$ (with value $n \in \mathbb{N} \cup \{ \infty\}$) and boolean expression, $\bexpr \in \booldom$, where we denote by $\infty$ a value s.t. $n < \infty $ for all $n \in \mathbb{N}$.
% They are the typical ones from while languages. Each command is annotated with a label $l \in (\mathbb{N} \cup \{\lin, \lex\})$. We use label to record
% the location of each command, so that we can uniquely identify each program point.
% We denote by $\cdom$ the universe of all labeled commands and $\ldom$ for all labels.
The operator $\kw{FV}: \expr \to \mathcal{P}(\mathcal{V})$ computes the set of free variables in an expression.
% For example,
% $\kw{FV}(\aexpr)$ and $\kw{FV}(\bexpr)$ represent the set of free variables in arithmetic
% expression $\aexpr$ and boolean expression $\bexpr$ respectively.
All the free variables
showing up in $c$ such that they are never defined before using, are actually the input variables of this program.
We denote by $\mathcal{V}_{\lin}$ the universe of all input variables and $\kw{V_{\lin}}(c)$ the set of input variables in a given program $c$.


% We use following notations to represent the sets of corresponding terms:
% \[
% \begin{array}{lll}
% \vardom & : & \mbox{Set of Variables}  
% \\ 
% %
% \booldom & : & \mbox{Set of Boolean Expressions}  
% \\ 
% %
% \cdom & : & \mbox{Set of All Labeled Commands} 
% \\ 
% %
% \eventset  & : & \mbox{Set of Events}  
% \\
% %
% \eventset^{\asn}  & : & \mbox{Set of Assignment Events}  
% \\
% %
% \eventset^{\test}  & : & \mbox{Set of Testing Events}  
% \\
% %
% \ldom  & : & \mbox{Set of Labels}  
% \\
% %
% \highlight{\ftdom} & : & \mbox{\highlight{Set of All Finite Execution Traces}}
% \\
% \highlight{\inftdom} & : & \mbox{\highlight{Set of Infinite  Execution Traces}}
% \\
% \highlight{\tdom} & : & \mbox{\highlight{Set of All Finite Or Infinite  Execution Traces}}
% \\ 
% %
% \inpvar(c) & : & \mbox{Set of Program $c$'s Input Variables}  
% \\
% %
% \ftdom_0(c) & : & \mbox{Set of Program $c$'s Well-formed Initial Traces.}
% \\ & & \mbox{Each well formed initial trace $\trace_0 \in \ftdom_0(c)$ is finite and every input variable of the program $c$ has an initial value in $\trace_0$.}
% \end{array}
% \]
%
% \paragraph{Expression}
% The expression is standard, it can be
% % can be 
% either a standard arithmetic expression or a boolean expression, or a list of expressions.
% An arithmetic expression can be a constant $n$ denoting integer, a variable $x$ from some countable set $\vardom$, binary operation $\oplus_a$ such as addition, product, subtraction, etc., over arithmetic expressions, with also $\elog$ and $\esign$. 
% %
% A boolean expression can be either {\tt true} or {\tt false}, basic boolean connectives such as logical negation, logical and or denoted by $\oplus_b$, and basic comparison $sym$ between arithmetic expressions, e.g., $\leq, =, <$, etc.
% Additionally, I also introduce list in expression.

% \paragraph{Labeled Command}
% Commands are the typical ones from while languages. Each command is annotated with a label $l$, we will use natural numbers as labels and we will use them to record
% the location of each command, so that we can uniquely identify them.

% \paragraph{Free Variables and Input Variables}
%   We denote by $\kw{FV}$ the operator $\kw{FV}: \expr \to \mathcal{P}(\mathcal{V})$, which computes the set of free variables in an expression. For example,
%   $\kw{FV}(\aexpr)$ and $\kw{FV}(\bexpr)$ represent the set of free variables in arithmetic
%   expression $\aexpr$ and boolean expression $\bexpr$ respectively.
%   All the free variables
%   showing up in $c$ such that they are never defined before using, are actually the input variables of this program.
%   We denote by $\mathcal{V}_{\lin}$ the universe of all input variables and $\kw{V_{\lin}}(c)$ the set of input variables in a given program $c$.

% \paragraph{{Trace-based Operational Semantics}}
% \label{sec:operational_semantics}
\paragraph{Event}
An event, $\event \in \eventset$ is a triple.
Its first element is the variable name $x$,
or a boolean expression (from the guard of if or while command), 
following by 
 the label, $l$ associated to this command and the value assigned to the variable.
 \[
  \begin{array}{llll}
    \mbox{Event} 
    & \event & ::= & 
    ({x}, l, v) ~ \mbox{Assignment Event} 
    ~|~(\bexpr, l, v) ~ \mbox{Testing Event}
  \end{array}
  \]  
 We have two kinds of events: \emph{assignment events} and \emph{testing events},
 and we use $\eventset^{\asn}$ and $\eventset^{\test}$ to denote the set of all assignment events and testing events, respectively and $\eventset = \eventset^{\asn} \cup \eventset^{\test}$.
 An \emph{assignment event} tracks the execution of an assignment and consists of the assigned variable, the label of the command that generates it, the value assigned to the variable.
 A \emph{testing event} tracks the execution of if and while commands, specifically the evaluation of the boolean expression $b$ in the guard of a $\eif(\clabel{b}^l, c_1, c_2)$ command or $\ewhile \clabel{b}^l \edo (c)$.
 It consists of the boolean expression $b$ in the guard of the command, the label of the guard, the result of evaluating the guard.
%
Event projection operators $\pi_i$ projects the $i^{th}$ element from an event: 
$\pi_i : \eventset \to \vardom \cup \booldom \cup \ldom $

\paragraph{Trace.}
%
A trace $\trace \in \tdom$ is a list of events, 
collecting the events generated during a specific program execution. 
% \[
% \begin{array}{llll}
% \mbox{Trace} & \trace
% & ::= & [] ~|~ \trace :: \event 
% % ~|~ []^{\infty}
% \end{array}
% \]
% A trace 
It can be regarded as the program history, 
which records all the evaluations for assignment commands and guards in $\eif$ and $\ewhile$ command.
We use list notation for traces, where $[]$ is the empty trace, the operator $\traceadd$ combines an event and a trace in a new event, 
the operator $\tracecat$ concatenates two traces, and $\event \in \trace$ or $\event \notin \trace$ denotes an event $\event$ belongs or not to $\trace$.
\highlight{
A trace can be finite ($\trace \in \ftdom$) or infinite $\trace \in \inftdom$.
If a program doesn't terminate when executing under some initial trace,
it produces the infinite trace 
from $\inftdom$, which records a non-terminating computation.
So we denote by $\tdom$ the set of all traces, and $\tdom = \ftdom \cup \inftdom$.
The trace-based semantics with non-terminating execution is defined below following the maximal trace semantics in~\cite{Cousot19}.}
%  formally defined as follows. 
% \highlight{
% \begin{defn}[Trace Concatenation, $\tracecat: \tdom \to \tdom\to \tdom $]
%   \label{def:trace_concate}
% Given two traces $\trace_1 \in \tdom, \trace_2 \in \tdom$, the trace concatenation operator 
% $\tracecat$ is defined as:
% \[
% \trace_1 \tracecat \trace_2 \triangleq
% \left\{
% \begin{array}{ll} 
%   \trace_1 & \trace_2 = [] \lor \trace_1 \in \inftdom \\
%   ((\trace_1 :: \event)  \tracecat \trace_2') & \trace_1 \in \ftdom \land \trace_2 = [\event] \tracecat \trace_2'
%   % \trace_2 &  \trace_2 \in \inftdom \\
% \end{array}
% \right.
% \]
% \end{defn}
% }
% \begin{defn}(An Event Belongs to A Trace)
%   An event $\event \in \eventset$ belongs to a trace $\trace \in \tdom$, i.e., $\event \in \trace$ are defined as follows:
% %
% \begin{equation*}
%   \event \in \trace  
%   \triangleq \left\{
%   \begin{array}{ll} 
%     \etrue                  & \trace =  [\event] \tracecat \trace'
%      \land \event = \event' \\
%     \event \in \trace' & \trace =  [\event'] \tracecat \trace'
%       \land \event \neq \event' \\ 
%     \efalse                 & \trace = [] \lor \trace \in \inftdom
%   \end{array}
%   \right.
% \end{equation*}
% As usual, we denote by $\event \notin \trace$ that the event $\event$ doesn't belong to the trace $\trace$.
% \end{defn}
% %

We use the operator $\env : {\ftdom}  \to \vardom \to(\mathbb{N} \cup \{\bot\})$ to fetch the latest value assigned to a variable in the trace, which returns $\bot$ if the variable doesn't have a value in the trace.
In the rest of the paper, we denote by $\bot$ a value s.t. $\bot < n $ for all $n \in \mathbb{N}$.
%
% We define the operator $\tracel : \tdom \to \mathcal{P}{(\ldom)}$ projects the label from every event in a trace as a list of program points,
% defined as follows,
% \[
% \tracel([(\_, l, \_)] \tracecat \trace') \triangleq [l] \tracecat \tracel(\trace')
% \qquad
% \tracel([ ]) \triangleq []
% \]
%
% \paragraph{Environment.} $\env : {\ftdom}  \to \vardom \to(\mathbb{N} \cup \{\bot\})$.  
% We don't use separate data structure as the environment to record each variable's value or state. Instead, we use the operator $\env (\trace) x$ to fetch the latest value assigned to $x$ in the trace $\trace$. 
% \[
% \begin{array}{llll}
% \env(\trace  \traceadd (x, l, v)) x \triangleq v
% &
% \env(\trace \traceadd (y, l, v)) x \triangleq \env(\trace) x, y \neq x
% &
% \env(\trace \traceadd (b, l, v)) x \triangleq \env(\trace) x
% &
% \env({[]} ) x \triangleq \bot
% \end{array}
% \]
%
\begin{defn}[Well-formed Initial Traces]
  \label{def:initial_trace}
  Given a program $c$, $\trace \in \tdom$, then is a well-formed initial trace of $c$, i.e., , $\trace \in \ftdom_0(c)$ if and only if all the input variables of $c$ has an initial value in this trace.
  \[
    \forall c \in \cdom, \trace \in \ftdom \st \trace \in \ftdom_0(c) \iff 
    \forall x \in \inpvar(c) \st \env(\trace_0) x \neq \bot,
    \]
where $\ftdom_0(c)$ is the set of all well-formed initial traces for program $c$.
\end{defn}
%
% In the rest part of the paper, we implicitly consider $\trace$ as a well-formed initial trace of $c$ 
% when $\trace \in \ftdom_0(c)$, and $\ftdom_0(c)$ is used as the set of well-formed initial traces for program $c$.
%
\paragraph{Trace-based Semantics}
The evaluation for expression evaluation is standard, and 
% the notation is $\config{\bexpr, \trace} \earrow \etrue$.
we use notation for the expression evaluation $\econfig{} : \mathcal{A} \cup \booldom \to \tdom \to \mathcal{V}$, and $\econfig{\expr}(\trace)$ evaluates an expression $\expr$ under trace $\trace$.
Our symbolic bound language is also a subset of the expression, we will use the same evaluation notation and rules for evaluating the symbolic bound expression.
The trace based operational semantics is described in terms of a small step evaluation relation $\config{c, \trace} \to \config{c', \trace'}$ describing how a configuration program-trace evaluates to another configuration
program-state.
The rules are standard and omitted here.

% The $\econfig{\aexpr}(\trace)$ evaluates an arithmetic expression $\aexpr$ under trace $\trace$ following the arithmetic expression evaluation rules in Figure~\ref{fig:aexpr-eval}. 
% \begin{figure}
% \begin{mathpar}
%   \boxed{ \econfig{} \, : \, \mbox{Arithmetic Expression $\to$ Trace $\to$ Arithmetic Value}}
%   \\
%   \inferrule{ 
%     \empty
%   }{
%    \econfig{n} (\trace)
%    = n
%   }
%   \and
%   \inferrule{ 
%     \env(\trace) x = v
%   }{
%    \econfig{x} (\trace)
%    = v
%   }
%   \and
%   \inferrule{ 
%     \econfig{\aexpr_1}(\trace) = v_1
%     \and 
%     \econfig{\aexpr_2}(\trace) = v_2
%     \and 
%      v_1 \oplus_a v_2 = v
%   }{
%    \econfig{\aexpr_1 \oplus_a \aexpr_2} (\trace)
%    = v
%   }
%   \and
%   \inferrule{ 
%     \econfig{\aexpr}(\trace) = v'
%     \and 
%     \elog v' = v
%   }{
%    \econfig{\elog \aexpr}(\trace) 
%    = v
%   }
%   \and
%   \inferrule{ 
%     \econfig{\aexpr}(\trace) = v'
%     \and 
%     \esign v' = v
%   }{
%    \econfig{\esign \aexpr} (\trace)
%    = v
%   }
%    \end{mathpar}
%    \caption{Evaluation Rules of Arithmetic Expression}
%    \label{fig:aexpr-eval}
%    \end{figure}

 % The evaluation rules for boolean expression and standard expression are in Figure~\ref{fig:bexpr-eval} and Figure~\ref{fig:expr-eval}.
 % \begin{figure}
 %  \begin{mathpar}
 %  \boxed{ \barrow \, : \, \mbox{ Boolean Expression $\times$ Trace $\rightarrow$ Boolean Value} }
 %  \\
 %  \inferrule{ 
 %    \empty
 %  }{
 %   \config{\efalse, \trace} 
 %   \barrow \efalse
 %  }
 %  \and 
 %  \inferrule{ 
 %    \empty
 %  }{
 %   \config{\etrue, \trace} 
 %   \barrow \etrue
 %  }
 %  \and 
 %  \inferrule{ 
 %    \config{\bexpr, \trace} \barrow v'
 %    \and 
 %    \neg v' = v
 %  }{
 %   \config{\neg \bexpr, \trace} 
 %   \barrow v
 %  }
 %  \and 
 %  \inferrule{ 
 %    \config{\bexpr_1, \trace} \barrow v_1
 %    \and 
 %    \config{\bexpr_2, \trace} \barrow v_2
 %    \and 
 %     v_1 \land v_2 = v
 %  }{
 %   \config{\bexpr \land \bexpr_2, \trace} 
 %   \barrow v
 %  }
 %  \and 
 %  \inferrule{ 
 %    \config{\bexpr_1, \trace} \barrow v_1
 %    \and 
 %    \config{\bexpr_2, \trace} \barrow v_2
 %    \and 
 %     v_1 \lor v_2 = v
 %  }{
 %   \config{\bexpr_1 \lor \bexpr_2, \trace} 
 %   \barrow v
 %  }
 %  \end{mathpar}
 %  \caption{Evaluation Rules of Boolean Expression}
 %  \label{fig:bexpr-eval}
 %  \end{figure}
  
 %  \begin{figure}
 %    \begin{mathpar}
 %  \boxed{ \earrow \, : \, \mbox{Expression $\times$ Trace $\rightarrow$ Value} }
 %  \\
 %  \inferrule{ 
 %    \econfig{\aexpr}(\trace) = v
 %  }{
 %   \config{\aexpr, \trace} 
 %   \earrow v
 %  }
 %  \and
 %  \inferrule{ 
 %    \config{\bexpr, \trace} \barrow v
 %  }{
 %   \config{\bexpr, \trace} 
 %   \earrow v
 %  }
 %  \and
 %  \inferrule{ 
 %    \config{\expr_1, \trace} \earrow v_1
 %    \and 
 %    \ldots
 %    \and 
 %    \config{\expr_n, \trace} \earrow v_n
 %  }{
 %   \config{ [\expr_1, \ldots, \expr_n], \trace} 
 %   \earrow [v_1, \ldots, v_n]
 %  }
 %  \and
 %  \inferrule{ 
 %    \empty
 %  }{
 %   \config{v, \trace} 
 %   \earrow v
 %  }
 %   \end{mathpar}
 %   \caption{Evaluation Rules of Standard Expression}
 %   \label{fig:expr-eval}
 %   \end{figure}

% \paragraph{Operational Semantics Rules}
%
% The trace based operational semantics is described in terms of a small step evaluation relation $\config{c, \trace} \to \config{c', \trace'}$ describing how a configuration program-trace evaluates to another configuration
% program-state.
% The rules are standard and omitted here.
% presented in Figure~\ref{fig:command-os}.
% \begin{figure}
%   \begin{mathpar}
% \boxed{
% \mbox{Command $\times$ Trace}
% \xrightarrow{}
% \mbox{Command $\times$ Trace}
% }
% \and
% \boxed{\config{{c, \trace}}
% \xrightarrow{} 
% \config{{c',  \trace'}}
% }
% %
% \\
% %
% \inferrule
% {
% \config{\expr, \trace} \earrow v
%   \and
% \event = ({x}, l, v)
% }
% {
% \config{\clabel{\assign{{x}}{\expr}}^{l},  \trace } 
% \xrightarrow{} 
% \config{\clabel{\eskip}^l, \trace \traceadd \event}
% }
% ~\rname{assn}
% \and
% %
% \inferrule
% {
%   \config{\bexpr, \trace} \earrow \etrue
%  \and 
%  \event = (\bexpr, l, \etrue)
% }
% {
% \config{{\ewhile \clabel{\bexpr}^{l} \edo (c), \trace}}
% \xrightarrow{} 
% \config{{
% c; \ewhile \clabel{\bexpr}^{l} \edo (c),
% \trace \traceadd \event}}
% }
% ~\rname{while-t}
% %
% %
% \and
% %
% \inferrule
% {
%   \config{\bexpr, \trace} \earrow \efalse
%  \and 
%  \event = (\bexpr, l, \efalse)
% }
% {
% \config{{\ewhile \clabel{\bexpr}^{l} \edo (c), \trace}}
% \xrightarrow{} 
% \config{{
%   \clabel{\eskip}^l,
% \trace \traceadd \event}}
% }
% ~\rname{while-f}
% \and
% %
% %
% \inferrule
% {
% \config{{c_1, \trace}}
% \xrightarrow{}
% \config{{c_1',  \trace'}}
% }
% {
% \config{{c_1; c_2, \trace}} 
% \xrightarrow{} 
% \config{{c_1'; c_2, \trace'}}
% }
% ~\rname{seq1}
% %
% \and
% %
% \inferrule
% {
%   \config{{c_2, \trace}}
%   \xrightarrow{}
%   \config{{c_2',  \trace'}}
% }
% {
% \config{{\clabel{\eskip}^l; c_2, \trace}} \xrightarrow{} \config{{ c_2', \trace'}}
% }
% ~\rname{seq2}
% %
% \and
% %
% %
% \inferrule
% {
%   \config{\bexpr, \trace} \earrow \etrue
%  \and 
%  \event = (\bexpr, l, \etrue)
% }
% {
% \config{{
% \eif(\clabel{\bexpr}^{l}, c_1, c_2), 
% \trace}}
% \xrightarrow{} 
% \config{{c_1, \trace \traceadd \event}}
% }
% ~\rname{if-t}
% %
% \and
% %
% \inferrule
% {
%  \config{\bexpr, \trace} \earrow \efalse
%  \and 
%  \event = (\bexpr, l, \efalse)
% }
% {
% \config{{\eif(\clabel{\bexpr}^{l}, c_1, c_2), \trace}}
% \xrightarrow{} 
% \config{{c_2, \trace \traceadd \event}}
% }
% ~\rname{if-f}
% %
% \end{mathpar}
% \caption{Operational Semantics Rules}
% \label{fig:command-os}
% \end{figure}


Given an initial trace $\trace_0 \in \ftdom_0(c)$ of the program $c$,
we use $\to^*$ for the reflexive and transitive closure of $\to$. 
If $\config{c, \trace_0} \rightarrow^{*} \config{\clabel{\eskip}^l, \trace_0 \tracecat \trace}$,
then the program's execution terminates and produces a finite execution trace $\trace \in \ftdom$. We also have non-terminating execution defined as follows.
\begin{defn}[Non-terminating and Infinite Trace]
  \label{def:non-terminating}
  Given a program $c$ and an initial trace $\trace \in \ftdom_0(c)$,
  when $c$ executes with $\trace$,  we define the execution of $c$ under $\trace$ is non-terminating and produces an infinite trace $\trace' \in \inftdom$, as 
  $\config{c, \trace_0} \uparrow^{\infty} \trace' \in \lim(\uparrow)$
  where the limit is defined as follows.
  \[
    \begin{array}{l}
      \lim(\uparrow) 
      % \in \left( (\cdom \times \ftdom) \times (\cdom \times \inftdom) \right) 
      \triangleq 
    \\ \quad
    \Big\{
      (\config{c, \trace}, \trace') ~\vert~ 
      c\in \cdom, \trace \in \ftdom_0(c),
      \trace' \in \inftdom 
      \land \exists \trace_0 \in \ftdom, c_0 \in \cdom \st 
      \config{c, \trace} \to \config{c_0, \trace_0}
      \\ \qquad \qquad \qquad 
      \land \forall i \in \mathbb{N}, \exists \trace_i, \trace_{i + 1} \in \ftdom, \trace'' \in \inftdom, c_i, c_{i + 1} \in \cdom \st 
      \config{c_i, \trace_i} \to \config{c_{i + 1}, \trace_{i + 1}} 
      \land  \trace' = \trace_{i + 1} \tracecat \trace''
    \Big\}
    \end{array}
  \]
\end{defn}
This definition follows the intuition of the maximal trace semantics in Equation (12) from Section 2.5 of paper~\cite{Cousot19a}, and the the program model in~\cite{SinnZV17}.

%
% \begin{defn}[Non-terminating and Infinite Trace (alternative way)]
%   \label{def:non-terminating-2}
%   Given a program $c$ and an initial trace $\trace_0 \in \ftdom_0(c)$,
%   when $c$ executes with $\trace_0$,  we define $c$ is non-terminating under $\trace_0$, $\config{c, \trace_0} \uparrow^{\infty}$ if and only if there exists a function
%   $f : \mathbb{N} \to \cdom \times \tdom$ such that $f(0) = \config{c, \trace_0}$ and
%   for every $i \in \mathbb{N}$ there exist  $\trace_i, \trace_{i + 1}\in \tdom$, $c_i, c_{i + 1} \in \cdom$ such that  $f(i) = \config{c_i, \trace_i}$, $f(i + 1) =  \config{c_{i + 1}, \trace_{i + 1}}$ and
%   $\config{c_i, \trace_i} \to \config{c_{i + 1}, \trace_{i + 1}}$. 
%   \[
%     \begin{array}{l}
%     \forall \trace_0 \in \ftdom_0(c), c \in \cdom \st
%     \config{c, \trace_0} \uparrow^{\infty}
%     \\
%     \iff \exists f : \mathbb{N} \to \cdom \times \tdom \st 
%     f(0) = \config{c, \trace_0}
%     \\ \qquad \land
%     \forall i \in \mathbb{N}, \exists \trace_i, \trace_{i + 1} \in \tdom, c_i, c_{i + 1} \in \cdom\st 
%     \\ \qquad \quad
%     f(i) = \config{c_i, \trace_i}$, $f(i + 1) =  \config{c_{i + 1}, \trace_{i + 1}} \land \config{c_i, \trace_i} \to \config{c_{i + 1}, \trace_{i + 1}}
%     \end{array}
%   \]
%   Given a program $c$ and an initial trace $\trace_0 \in \ftdom_0(c)$, if $\config{c, \trace_0} \uparrow^{\infty}$, 
%   let $f$ be the function such that for every $i \in \mathbb{N}$,  $\trace_i, \trace_{i + 1}\in \tdom$, $c_i, c_{i + 1} \in \cdom$ where $\config{c_i, \trace_i} \to \config{c_{i + 1}, \trace_{i + 1}}$, we have $f(i) = \config{c_i, \trace_i}$, $f(i + 1) =  \config{c_{i + 1}, \trace_{i + 1}}$. 
%   Let $\pi_2 : (\cdom \times \tdom) \to \tdom$ be the projector which projects the trace from a configuration,
%   then we define $\config{c, \trace_0} \uparrow^{\infty} \trace'$ produces an infinite trace $\trace' = \pi_2(\lim\limits_{i \to \infty}(f(i))) \in \inftdom$.
%   \[ \trace' = \lim( \pi_2 \circ (f(i))). \]
% \end{defn}
% %
% \begin{defn}[Non-terminating and Infinite Trace (third way)]
%   \label{def:infinite-trace}
%   Given a program $c$ and an initial trace $\trace_0 \in \ftdom_0(c)$,
%   when $c$ executes with $\trace_0$,  we define $c$ is non-terminating under $\trace_0$, denoted as $\config{c, \trace_0} \uparrow^{\infty} \trace'$ and produce an infinite trace $\trace' \in \inftdom$ 
%   if and only if there exists a function
%   $f : \mathbb{N} \to \cdom \times \tdom$ such that $f(0) = \config{c, \trace_0}$ and
%   for every $i \in \mathbb{N}$ there exist  $\trace_i, \trace_{i + 1}\in \tdom$, $c_i, c_{i + 1} \in \cdom$ such that  $f(i) = \config{c_i, \trace_i}$, $f(i + 1) =  \config{c_{i + 1}, \trace_{i + 1}}$ and
%   $\config{c_i, \trace_i} \to \config{c_{i + 1}, \trace_{i + 1}}$. 
%   \[
%     \begin{array}{l}
%     \forall \trace_0 \in \ftdom_0(c), \trace' \in \inftdom, c \in \cdom \st
%     \config{c, \trace_0} \uparrow^{\infty} \trace'
%     \\
%     \iff \exists f : \mathbb{N} \to \cdom \times \tdom \st 
%     f(0) = \config{c, \trace_0}
%     \\ \qquad \land
%     \forall i \in \mathbb{N}, \exists \trace_i, \trace_{i + 1} \in \tdom, c_i, c_{i + 1} \in \cdom\st 
%     \\ \qquad \quad
%     f(i) = \config{c_i, \trace_i}$, $f(i + 1) =  \config{c_{i + 1}, \trace_{i + 1}} \land \config{c_i, \trace_i} \to \config{c_{i + 1}, \trace_{i + 1}}.
%     \end{array}
%   \]
%   Let $\pi_2 : (\cdom \times \tdom) \to \tdom$ be the projector which projects the trace from a configuration,
%   then the infinite trace $\trace'$ produced by $\config{c, \trace_0} \uparrow^{\infty} \trace'$ is
%   \[ \trace' = \pi_2(\lim\limits_{i \to \infty}(f(i))) \in \inftdom. \]
% \end{defn}
%
% This definition has the similar intuition as the maximal trace semantics in Equation (12) from Section 2.5 of paper~\cite{Cousot19a}, and the semantics of the program model in~\cite{SinnZV17}.
% \\
% If we observe the operational semantics rules, we can find that no rule will shrink the trace. 
% So we have the Lemma~\ref{lem:tracenondec} with proof in Appendix~\ref{apdx:lem_language}, 
% specifically the trace has the property that its length never decreases during the program execution.
% \begin{lem}
%   [Trace Non-Decreasing]
%   \label{lem:tracenondec}
%   For any program $c \in \cdom$ and initial trace $\trace_0 \in \ftdom_0(c)$,
%   if there exists $\trace \in \tdom$ and $c' \in \cdom $ such that $\config{c, \trace_0} \rightarrow^{*} \config{c', \trace} $ or 
%   $\config{c, \trace_0} \uparrow^{\infty} \trace$  
%   then there exists a trace $\trace' \in \tdom$ such that $\trace_0 \tracecat \trace' = \trace$ formally as follows.
%   %
%   \[
%     \begin{array}{l}
%     \forall \trace_0 \in \ftdom_0(c), \trace \in \tdom, c, c' \in \cdom \st
%     \Big( \config{c, \trace_0} \rightarrow^{*} \config{c', \trace} 
%     \lor  \config{c, \trace_0} \uparrow^{\infty} \trace \Big)
%     \\ \quad
%     \implies \exists \trace' \in \tdom \st \trace_0 \tracecat \trace' = \trace 
%     \end{array}
%     \]
%   \end{lem}