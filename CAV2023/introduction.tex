
Gulwani et al.~\cite{GulwaniZ10} first introduced \emph{reachability-bound analysis} as a way to provide upper bounds on the number of times a given program location 
is visited during program execution.
Obtaining tight reachability-bounds is useful in many applications.
For example from a privacy and security perspective,
how much secret information is leaked by a program may depend on the number of times a certain operation that leaks the data
is executed~\cite{Malacaria07};
from an efficiency perspective, when different program locations consume different resources, a precise reachability-bound for each location can help to estimate the resource cost more accurately than just computing the overall complexity.

%Existing works all have different limitations when inferring the \emph{reachability-bound}.
Gulwani et. al~\cite{GulwaniZ10}
give a two-step solution combining program abstraction with a set of proof-rules that are used to compute the actual reachability-bounds. This combination is effective at computing reachability-bounds for a rich class of programs, however, it also has limitations. This method 
computes complexity bounds for loops and over-approximate the reachability-bounds of the different locations inside the loop by using this bound. This result in an over-approximation of the reachability-bounds of several locations, especially when there are loops with multiple paths.

A similar approach to reachability-bounds can be also computed, in principle, using techniques from \emph{amortized program complexity analysis}~\cite{BradleyMS05,CookSZ13,Zuleger18,SinnZV14,SinnZV17,LuCT21,AliasDFG10}. Indeed, using techniques like ranking functions, counter increments, difference constraints, etc.
one can estimate 
precise bounds on loop iterations and execution time, and in turn obtain reachability bounds for complex programs. However, these techniques are designed to capture the worst-case or overall complexity, and again they may over-approximate the reachability-bounds of certain locations. This is again particularly evident when a loop contains multiple paths, since in this situation one usually obtains an overall complexity bound which is greater than the reachability-bounds on some locations of some of the paths. 

To overcome these limitations, 
we introduce a new path-sensitive reachability-bound analysis that combines ideas from amortized complexity analysis with ideas from 
\emph{path refinement complexity analysis}~\cite{GustafssonEL05,ManoliosV06,BalakrishnanSIG09,SharmaDDA11,Flores-MontoyaH14,HumenbergerJK18,CyphertBKR19,GulwaniJK09,ZulegerGSV11}.
% can help to summarize the loop paths and compute the program complexity path-sensitively. However, they still do not compute the  reachability-bound for every program location.
Specifically, we propose a new algorithm which by combining amortized  analysis with loop summarization-based multipath refinement mitigates the limitations faced by each technique individually. 
% \end{itemize}
%

More in details, our algorithm first uses the difference constraints program abstraction model~\cite{SinnZV17,SinnZV14} enriched with boolean expressions to generate an abstract transition graph.
 %in Section~\ref{sec:progabs}
Then the algorithm performs both path refinement and amortized complexity analysis through ranking estimation over this graph.
%in Section~\ref{sec:refine} and  Section~\ref{sec:rank}.
By computing the ranking function, we effectively alternate computation for the loop bound with the bound invariant of the ranking function. 
% $\locbound(\absevent, c)$ for each transition edge $\absevent$, 
% and estimating its bound invariant in Section~\ref{sec:rank}.
We also leverage the path-insensitivity of ranking function estimation through a lightweight path refinement algorithm adopted from~\cite{GulwaniJK09}.
Building on this combination, we then focus on two useful quantities which we call \emph{path reachability-bound} and \emph{loop reachability-bound}.
The first one bounds the evaluation times of each loop-free and interleave-free path in a refined program, and the second one bounds the number of iterations of an outer loop w.r.t. an inner loop, counting only the iterations when the inner loop is ``entered''. 
We present algorithms for estimating these two quantities. By using them we can then 
%Section~\ref{sec:looprb} and~\ref{sec:pathrb} 
compute the \emph{reachability-bound} for each program point in a path-sensitive way.
%in Section~\ref{sec:psrb}.
We have implemented our algorithm in a prototype and evaluated it on 290 program extracted from different benchmarks. Our evaluation %Section~\ref{sec:eval} 
shows that we can accurately estimate different bounds for different locations in programs with loops containing multiple paths. Based on this, we can also compute a tighter overall complexity compared to the state-of-art worst-case complexity analysis tools.

To summarize, our contributions are as follows.
\\
% \begin{itemize}
% \item 
1. We introduce a path-sensitive reachability-bound algorithm 
combining \emph{amortized bound analysis}  and \emph{path refinement}. The algorithm computes tight bounds on the times that each program point is evaluated.
\\
% \item 
% 2. The combination of \emph{amortized bound analysis} through ranking function estimation and the path refinement approach in our algorithm.
% % \item 
% \\
2. We identify two quantities,  \emph{path reachability-bound} and \emph{loop reachability-bound}, which are useful to measure precise information about the program execution.  We also propose two algorithms for soundly estimating these two quantities.
% \item 
\\
3. We develop a prototype implementation with evaluation over 290 programs.
 The evaluation shows improvement in precision compared to other tools for reachability-bound analysis and the worst-case complexity analysis.
% \end{itemize}

\section{Related Work.}
% \begin{itemize}
% \item 
\emph{Program Abstraction.}
Program abstraction is commonly used in program analysis as a preprocessing step to abstract program features and generates transition graphs or systems. For example, Gulwani et al.~\cite{GulwaniZ10} summarizes programs into some underlying abstract domains (the unified lattice~\cite{CousotH78}, polyhedra~\cite{CousotC77} or octagonal~\cite{Mine06})
and generates transition systems for loop counters. Kincaid et al.~\cite{KincaidCBR18} abstract  programs into the wedge domain and compute non-linear loop invariant.
% While it only works well for the specific targeting problem.
%Efficiency is the main bottleneck when generating and solving the constraint.
Sinn et al.~\cite{SinnZV17,SinnZV14}
introduce a program abstraction model based on difference constraints, which we also use in this paper. 
It is more accurate in the sense that it allow to represent program loops in a path-sensitive way.
 This representation is also comparatively lightweight.

% \item 
\emph{Amortized Complexity Analysis.}
This line of work follows 
% \emph{amortized complexity analysis}
% originated 
from Tarjan's seminal paper~\cite{Tarjan85}. It is usually combined with ranking functions~\cite{BradleyMS05,CookSZ13,Zuleger18} or counter increments~\cite{GulwaniMC09,ZulegerGSV11,AliasDFG10}.
% They do well in nested loops by alternating the loop bound computation with the ranking or counter estimation. This alternation is efficient without recursively unrolling the nested loops when composing the bound of different paths.
 % \\
 Techniques that estimate the counter increments or the ranking function invariants, such as the tools CofloCo~\cite{Montoya17,Flores-MontoyaH14,Flores-Montoya16} and KoAT~\cite{BrockschmidtEFFG16,BrockschmidtEFFG14,FalkeKS12,FalkeKS11}, or the algorithm in~\cite{LuCT21}, often ignore the interleaving between multiple paths in the same loop, on which we focus in this paper.
% Most of them 
%and etc.
% over-approximate the loop bound when the path interleaving affects loop execution.
%It is hard to repurpose their result as the reachability-bound on different points.
% loop bound path-insensitively as the reachability-bound on different points.
Several works have studied \emph{amortized complexity analysis} using type refinement and annotations~\cite{CraryW00,JostHLH10,CicekBG0H17,RajaniG0021,CarbonneauxHS15}. Most of these work also do not distinguish different paths in loops, resulting in over-approximation of the resource cost for different program points.
% We choose to use difference-constraints based approach and enrich it with boolean expressions and path-sensitivity according to the Alg.~2 in~\cite{SinnZV14},
% which assigns a variable to each edge on which this variable decrease as its ranking function.
% the Alg.~3 in~\cite{ZulegerGSV11},
% and the Def.~25 in Section 4 from~\cite{SinnZV17}.

\emph{Path Refinement Based Complexity Analysis}.
Another line of related works aims at analyzing loop bounds through loop summarization and path refinements~\cite{ManoliosV06,BalakrishnanSIG09,SharmaDDA11,Flores-MontoyaH14,HumenbergerJK18,CyphertBKR19}, and some of them also analyze the interleaving between different paths~\cite{GulwaniJK09,ZulegerGSV11}.
Kincaid et al.\cite{KincaidBCR19,KincaidCBR18,BreckCKR20} introduce some loop summarization techniques which can help to improve the accuracy of the path refinement algorithm for non-linear loops and with program recurrence.
% summarization techniques~\cite{KincaidCBR18} and the invariant generating algorithm considering recurrence in~\cite{BreckCKR20}. 
However, when composing the bound in nested loops, recursively unrolling the nested loops is heavy and in most cases non-terminating.
%
In contrast, our method simplifies the path refinement algorithm in~\cite{GulwaniJK09} using contextualization techniques based on~\cite{ZulegerGSV11,SinnZV14,ManoliosV06}.
We also limit the iterations of the refinement algorithm to a constant in our bound analysis algorithm.