% The programs' execution complexity affects our daily life from many perspectives.
% For example,
% from the privacy and security perspective,
% how much secret information is leaked by a program depends on the number of times a certain operation that leaks the data,
% is executed~\cite{Malacaria07};
% the amount of perturbation in the output data values resulting
% from a small perturbation or uncertainty in the input,
% values depend on the number of times additive error propagation operators are applied; etc.
% Estimating such quantitative properties requires us to know
% how many times is a given control location inside the program that performs certain operations executed?
% % \\
% From the performance perspective, it is important to give a precise estimation
% on the program's resource cost bound w.r.t. the program's inputs.
% For example, in memory-constrained environments such as embedded systems,
% it is important to bound the amount of memory required to run certain applications.
% In real-time systems, it is important to bound the worst-case execution time of the program.
% Applications running on low-power devices or low-bandwidth environments must use up little power or bandwidth respectively. 
% With the advent of cloud computing, where users would be charged per program execution,
% predicting resource usage characteristics would be a crucial component of accurate bid placement by cloud providers. 
% One of the challenges in bounding this cost precisely is that resource consumption is location-sensitive.
% In other words, 
% Different location has different resource cost as well as different execution times. 
% To give accurate estimation results on these execution properties,
% the fundamental questions that need to be addressed 
% is estimating the bound on the execution times
% a given control location inside the program that consumes these resources.
% For these reasons, this paper focuses on analyzing the bound on the execution times of a program's given control location.
% This bound is referred to as the reachability-bound in the program analysis area,
% which is firstly proposed by the paper~\cite{GulwaniZ10}.
% In this paper, finding a symbolic worst-case bound on this quantitative reachability property
% in terms of the inputs to that procedure
% is referred to as the \emph{reachability-bound problem}.

Computing a good solution to the \emph{reachability-bound problem} is challenging.
The paper~\cite{GulwaniZ10} that introduces this concept
gives a two-step solution by combining the abstract interpretation-based iterative technique
and the non-iterative proof-rules-based technique.
However, their solution
does not solve this problem in a path-sensitive manner.
It over-approximates the reachability-bounds on different paths inside a while loop.
% \\
There are also many works in analyzing the program complexity~\cite{GustafssonEL05,HumenbergerJK18},
or estimating the upper bound on a program's worst-case resource cost
~\cite{BrockschmidtEFFG16,AlbertAGP08,AliasDFG10,Flores-MontoyaH14}.
But their analysis
focus only on estimating 
the overall complexity 
by inferring the bounds on the loop iteration numbers,
or the worst-case running time and resource cost of the program's entire execution.
\\
None of them computes the reachability-bound on a given program control location directly or path-sensitively.
To leverage these limitations,
we introduce a path-sensitive reachability-bound analysis algorithm in this paper, which aims to solve 
the reachability-bounds problem efficiently and path-sensitively.
Our algorithm combines two lines of complexity analysis techniques.
\begin{itemize}
  \item One line of loop bound analysis works based on the \emph{amortized complexity analysis} originated from Tarjan's influential paper~\cite{PotechinP17} combined with \emph{ranking function} estimation from~\cite{BradleyMS05} and developed in~\cite{ZulegerGSV11,SinnZV14,SinnZV17,LuCT21,AliasDFG10}.
  They do well in nested loops by alternating the loop bound computation with the ranking function estimation. This alternation lines up with advantage of the \emph{amortized complexity analysis} is efficient and accurate without recursively unrolling the nested loops when composing the bound of different paths.
  \\
  But they have limitation, estimating the ranking function ignores the interleaving between multiple paths in the same loop.
  It over-approximates the bound of each single path as well the loop bound when the path interleaving affects the loop execution.
  \item 
  Another line of loop bound analysis through loop summarization and path refinement seek for precise loop path representation~\cite{ManoliosV06,BalakrishnanSIG09,SharmaDDA11,Flores-MontoyaH14,HumenbergerJK18,CyphertBKR19}, and compute loop bound over accurate loop summarization~\cite{GulwaniJK09,ZulegerGSV11}.
  They do well in summarizing the loop path and computing the interleaving between paths, as well as computing the precise bound considering the interleaving.
  \\
  The limitation occurs when composing the bound between nested loops. Recursively unrolling the nested loops and computing the interleaving between unrolled paths are heavy and non-terminating.
\end{itemize}
Our combination improves in both limitations, and computes 
a \emph{reachability-bound} for every program point $l$ in a program $c$ in a path sensitive manner as the follows.
% shown in Figure~\ref{fig:psRB-architecture}.
% Based on an abstract transition graph through program abstraction presented in Section~\ref{sec:abs_cfg},
% we perform both the path refinement and the ranking estimation over this graph.
% We utilize the effectiveness of the \emph{amortized complexity analysis} by computing the ranking function $\locbound(\absevent, c)$ for each transition edge $\absevent$, and then estimating the bound on each ranking function, $\varinvar(\locbound(\absevent, c), c)$ and transition edge, $\absclr(\absevent, c)$ alternatively and path-insensitively.
% In the meantime, we also take advantage from the accuracy of the path refinement techniques through a light-weight path refinement algorithm adapted from~\cite{GulwaniJK09} presented in Section~\ref{alg:alg-refine_rewrite}.

% Built on this, we compute the bounds over two novel quantities, the \emph{path reachability-bound} and \emph{loop reachability-bound}. These two bounds effectively combines the strength of the two different bound computation methods above.

% The first key novelty -- the \emph{path reachability-bound} $\inoutB(\rprog, \tpath)$ for a loop free path $\tpath$ within a loop program $\rprog$ bounds the evaluation times of each loop free path instead of the entire multipath loop.

% The second key idea combining two lines of works above is the \emph{loop reachability-bound}, $\lpchB(L:\rprog, \tpath)$.
% For each transition path $\tpath$ w.r.t each of the loops $L:\rprog$ in which $\tpath$ is nested,
% $\lpchB(L:\rprog, \tpath)$ bounds the iterations for
% the outside loop, $L:\rprog$ w.r.t. the innermost loop where $\tpath$ is enclosed,
% such that during these iterations of $L:\rprog$, the innermost loop is ``entered''. 
% Then by multiplication and summing over these two bounds where each program control point shows up, we compute each point's the \emph{reachability-bound} path-sensitively.
% \begin{figure}
% \centering
% \includegraphics[width=1.0\columnwidth]{psRB-architecture.png}
% \caption{Architecture of the path-sensitive reachability-bound algorithm.}
% \label{fig:psRB-architecture}
% \end{figure}
\paragraph{Main Steps of Path-sensitive Reachability-bound Analysis}
% \label{sec:static_rb}
In this section, we present our static program analysis algorithm for computing 
% an upper bound on the 
% execution-based reachability times 
the \emph{reachability-bound} for every program point $l$ in a program $c$ in a path sensitive manner.
% , as defined in last section.
%
% In order to have the upper bound of the reachability for every label of a program $c$, we design 
% a path sensitive reachability bound analysis algorithm {\THESYSTEM}.
The algorithm is summarized into the following steps,
% \begin{figure}
%   \centering    
% \includegraphics[width=1.0\columnwidth]{adapfun.png}
%   \vspace{-0.3cm}
%   \caption{The overview of {\THESYSTEM}}
%   \label{fig:adaptfun}
%   \vspace{-0.5cm}
% \end{figure}
%
\\
% \framebox{
    % \text{
    1. Compute Abstract Transition graph, $\absG(c)$ through program abstraction.
    % }
    \\
    % \text{
    2. Compute refined program by \cite{GulwaniJK09}. $\rprog = API(\absG(c))$
    \\
    3. Compute local bound $\outinB(\tpath, c)$
    \\
    4. Compute Reachability-Bound $\psRB = \inoutB(\tpath, c)$
    % }
% }
%
\begin{enumerate}
\item  In Section~\ref{sec:progabs}, we first construct an abstract transition graph based on $c$, by computing an abstract transition 
for every labeled command. 
This graph is used in the following sections
%  from Section~\ref{sec:refine} to Section~\ref{sec:psrbcompute} 
for computing the path-sensitive reachability-bound of a program location.
% see Section~\ref{sec:alg_vertexgen}
\item The second step in Section~\ref{sec:refine}
refines the multiple-paths loops in the program
% this program path sensitively, 
based on the abstract transition graph.
This step transforms the multiple-paths loops into multiple loops where
the interleaving of paths is explicit.
% \item Section~\ref{sec:lbcompute} computes the ranking function  
% \footnote{\textbf{ranking function} is the named used in \cite{SinnZV14}
% and \textbf{local bound} is the name used in \cite{ZulegerGSV11}, \cite{sinn2017complexity}.
% We refer to the two names as the same meaning in this paper.} for each edge in a program's abstract transition graph,
% and estimates the upper bounds on every ranking function's maximum value and every edge's execution times path-insensitively.
% path-insensitive reachability upper bound for every while loop command in $c$.
\item Section~\ref{sec:outinalg} performs the \textbf{Outside-In} algorithm and computes
the upper bound for the execution times for every path in a refined loop locally.
It first computes the ranking function  
\footnote{\textbf{ranking function} is the named used in \cite{SinnZV14}
and \textbf{local bound} is the name used in \cite{ZulegerGSV11}, \cite{sinn2017complexity}.
We refer to the two names as the same meaning in this paper.} 
for each edge in a program's abstract transition graph,
and estimates the upper bounds on every ranking function's maximum value and every path's execution times.
% , named \textbf{Outside-In} bound.
% path-sensitive local bounds.
\item Section~\ref{sec:inoutalg}
performs the \textbf{Inside-Out} algorithm and 
computes the path-sensitive reachability-bound for every program point,
It first computes the upper bound for the execution times of
every simple path in the refined program globally. 
Then it sums up the bounds of paths contains certain program point as its path-sensitive reachability-bound.
%  named \textbf{Inside-Out} bound.
% abstract transition graph.
% \item Section~\ref{sec:psrbcompute} computes the path-sensitive reachability-bound for every program point
% %  in this program 
% based on the above results.
%  by
%  ?summarizing 
% the path-sensitive reachabilitybound of each edge on the abstract transition graph.
% \item The Section~\ref{sec:reachabilitybound_algorithm} computes program's reachability bound in two steps as follows.
\end{enumerate}
The contributions of this work can be summarized as follows,
\begin{itemize}
  \item the main contribution is the combination of the \emph{amortized program analysis} through ranking function estimation and the path refinement and loop summarization based on \emph{algebraic program analysis} in bound analysis algorithm.
  \item A path-sensitive reachability-bound computation algorithm.
  This algorithm can compute the evaluation times of each program point accurately and path-sensitively.
  \item A prototype implementation of this algorithm and an evaluation over 5 different benchmarks.
  The evaluation results show that we can compute tight bound on the evaluation times of each program point in a program. For program with multi-path loop, we compute different bounds for the points on different paths.
\end{itemize}