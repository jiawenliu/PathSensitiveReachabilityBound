Skeleton and Plan: Background and Importance of analyzing the reachability bound.
\\
\textbf{Where it is required? How useful it can be? What impact it can bring?}
\\
\textbf{In one area: area-I}
"One of the fundamental questions that needs to be answered for computing such resource bounds is: 
How many times is a given control location inside the program that consumes these resources executed?"
\\
\textbf{In some other areas, II, III or ...}
How many times is a given control location inside the program that performs certain operations executed?
\\
\textbf{Short Summary} of existing works and limitations
\\
\textbf{Short Summary} of the new technique/algorithm: major steps/technique used,  major outcome
\\
The execution properties of the program running on mobile devices affect our daily life from many perspectives.
For example,
from the privacy and security perspective,
% protection area,
how much secret information is leaked by a program depends on the number of times a certain operation that leaks the data,
% either by direct or indirect information flow, 
is executed~\cite{Malacaria07};
% In the privacy
% protection area, 
the amount of perturbation in the output data values resulting
from a small perturbation or uncertainty in the input,
values depend on the number of times additive error propagation operators are applied; etc.
% This is the quantitative version of the boolean problem of continuity studied in~\cite{ChaudhuriGL10}. 
Estimating such quantitative properties requires us to know
% addressing a similar question as above:
how many times is a given control location inside the program that performs certain operations executed?
% \\
From the performance perspective, it is important to give a precise estimation
on the program's resource cost bound w.r.t. the program's inputs.
For example, in memory-constrained environments such as embedded systems,
it is important to bound the amount of memory required to run certain applications.
In real-time systems, it is important to bound the worst-case execution time of the program.
Applications running on low-power devices or low-bandwidth environments must use up little power or bandwidth respectively. 
With the advent of cloud computing, where users would be charged per program execution,
predicting resource usage characteristics would be a crucial component of accurate bid placement by cloud providers. 
One of the challenges in bounding this cost precisely is that resource consumption is location-sensitive.
In other words, different location has different resource cost as well as different execution times.
To give accurate estimation results on these execution properties,
% This brings me to one of 
the fundamental questions that need to be addressed 
% for computing such resource bounds:
is estimating the bound on the execution times
% How many times is 
a given control location inside the program that consumes these resources.
For these reasons, I'm interested in analyzing the bound on the execution times of a program's given control location.
This bound is referred to as the reachability-bound in the program analysis area,
which is firstly proposed by the paper~\cite{GulwaniZ10}.
% Motivated by this, this part aims to 
In this paper, finding a symbolic worst-case bound on this quantitative reachability property
in terms of the inputs to that procedure
is referred to as the \emph{reachability-bound problem}.

\highlight{
Providing a good solution to this problem is challenging.
The paper \cite{GulwaniZ10} that introduces this concept
gives a two-step solution by combining the abstract interpretation-based iterative technique
 and the non-iterative proof-rules-based technique.
 However, their solution
does not solve this problem in a path-sensitive manner.
It over-approximates the reachability-bounds on different paths inside a while loop.
% \\
 There are also many works in analyzing the program complexity \cite{GustafssonEL05}, \cite{HumenbergerJK18},
 or estimating the upper bound on a program's worst-case resource cost
 \cite{BrockschmidtEFFG16,AlbertAGP08,AliasDFG10,Flores-MontoyaH14}.
But their analysis
focus only on estimating 
the overall complexity 
by inferring the bounds on the loop iteration numbers,
 or the worst-case running time and resource cost of the program's entire execution.
 None of them computes the reachability-bound on a given program control location directly or path-sensitively.
To leverage these limitations, we design a path-sensitive reachability bound analysis algorithm.
\textbf{Introduce} each step of the new technique/algorithm:
\\
\textbf{Introduce} new technique/ results or experimental results. 
Summary of comparison with existing works. \cite{GulwaniJK09} \cite{Sumit2010rechability}, \cite{sinn2017complexity}
\\
\paragraph{Contributions}
\begin{itemize}
  \item New path-sensitive reachability-bound analysis algorithm, 
  solve the \emph{Reachability-Bound Program} more precise than existing methods.
  \item More efficient bound computation method than existing methods.
  \item Efficient implementation and good evaluation results.
\end{itemize}
Outlines}