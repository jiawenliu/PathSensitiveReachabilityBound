
There are two methods for computing the loop bound, $BD(\rprog_l, c)$ for every loop program $\rprog_l$ in $c$'s refined program.
%
Based only on the estimated ranking function from Section~\ref{sec:rank}, we can compute a loose $BD(\rprog', c)$ for each $\rprog'$
using the $\absclr(\absevent, c)$ as follows.
\begin{equation}
 \label{eq:absBD}
 BD(\rprog_l, c) = \min\left\{ \absclr(\absevent, c) \middle\vert \absevent \in \rprog_l \right\}.
\end{equation}
It is accurate only for simple loops, but not for multipath or nested loops.
For example in Example~\ref{ex:relatedNestedWhileOdd-overview},
it computes $BD(\rprog_1^1) = n$ and $BD(\rprog_1^2) = n $ for the two interleaving pattern of loop $L_1$.
% The bounds for $\rprog_1^1$ and $\rprog_1^2$ are both $n$ while 
But both of their iteration bounds are expected to be $\lfloor\frac{m}{4}\rfloor$.
The inaccuracy is caused by the nature of the path-insensitivity in the ranking function invariant estimation in Definition~\ref{def:ranking_bound}. 

We provide a new computation approach over the refined program.
It uses the ranking function invariant to compute a more precise progress invariant inspired from~\cite{GulwaniJK09}.
We first compute three abstract states for each ranking function in different subprograms of a refined program as below.
% Then we these precise abstract states, and estimate the loop bound path sensitively.
\begin{defn}[Rank Abstract States Computation]
 \label{def:alg-absstate}
 Given a program $c$ with its refined program $\rprog_c$ and a sub-program $\rprog$ in $\rprog_c$,
 we compute three abstract states, $\rfinit(\rprog, c)$, $\rffinal(\rprog, c)$ and $\rfnext(\rprog, c) \in \scexpr(c)$ w.r.t. the ranking functions in this sub program as follows.
 \begin{itemize}
 \item 
 We first compute the initial execution point and continuation execution points of $\rprog$ denoted by
 $\absinit(\rprog)$
 and 
 $\absfinal(\rprog)$ using standard control flow method.
 \item The \emph{Initial State}, 
 $\rfinit(\rprog, c)$ is a set of equations $x = e$, where $e \in \scexpr(c)$ is a
 symbolic expression bounding the initial value of $x$ before executing $\rprog$.
 Each $x$ is a ranking function of a transition edge in this program, and $\rfinit(\rprog, c)$ contains the equations for every ranking function in $\rprog$,
 $\rfinit(\rprog, c) \triangleq $
 {\small
 \[
 \bigcup_{x \in \left\{ \locbound(\absevent, c) | \tpath \in \rprog \land \absevent \in \tpath \right\} }
 \left \{ 
 x = \arg\max_{l_1}\left\{
 \varinvar(y, c) + v ~\middle\vert~ 
 \begin{array}{l} 
 (l_1, x' \leq y + v, l_2) \in \reset(x, c) 
 \\
 \land l_1 \leq \absinit(\rprog, c)
 \end{array}
 \right\}
 \right\}
 \]
 }
 $\rfinit(\tpath, c)$ can also be estimated by $\kw{INIT(c, 0, \absinit(\rprog))}$ in \cite{GulwaniJK09}. 
 %
 \item The \emph{Final State}, $\rffinal(\rprog, c)$ computes the set of minimum values for every ranking functions
 after executing $\rprog$.
 % $\rffinal(\rprog, c) \triangleq $
 % \\
 {\small
\[
 \begin{array}{l} 
 \bigcup\limits_{x \in \left\{ \locbound(\absevent, c) | \tpath \in \rprog \land \absevent \in \tpath \right\} }
 \left \{ 
 x = \min\limits_{ \left\{ l_1 ~|~ l_1 \geq \absfinal(\rprog, c) \right\} }
 \max\left\{
 v ~\middle\vert~ 
 \left( (x = v) \land \bigwedge\limits_{b \in \kw{B}(\rprog, c)} \neg b \right) \neq \efalse
 \right\}
 \right\}
 \end{array}
 \]
 }
 $\kw{B}(\rprog, c)$ is the set of all the boolean constraints in $\rprog$.
 \item The \emph{Next State}, $\rfnext(\rprog, c) \in \scexpr(c)$ 
 computes for the ranking functions on $\rprog$ the amount that each of them is changed after the first execution of $\rprog$ and before the second visit. This quantity is only required for $\tpath$.
 {\small
 \[
 \begin{array}{l}
 \rfnext(\tpath, c) \triangleq 
 \\
 \bigcup\limits_{x \in \left\{ \locbound(\absevent, c) | \absevent \in \tpath \right\}}
 \left\{ x = 
 \begin{array}{l}
 \sum\limits_{\absevent \in \inc(x, c) }
 \left\{ v ~\middle\vert~ \absevent = (l, x' \leq x + v, \_) \land l \in \tpath\right\}
 \\ \qquad 
 + \arg\max\limits_{l' }
 \left\{ \varinvar(y, c) + v ~\middle\vert~ (l, x' \leq y + v, l') \in \reset(x, c) \land l \in \tpath\right\}
 \\ \qquad 
 - \sum\limits_{ \absevent \in \dec(x, c) }\left\{ 
 v ~\middle\vert~ \absevent = (l, x' \leq x - v, \_) \land l \in \tpath 
 \right\}
 \end{array}
 \right\} 
 \end{array}
 \]
 }
\end{itemize}
\end{defn}
Now we compute $BD(\rprog, c)$ recursively by interactively computing another quantity, the \emph{Variable Gradient Decent} as follows.
\begin{defn}[Loop Bound Computation]
\label{def:loopbound}
Given a program $c$ with its refined program $\rprog$ and a subprogram $\rprog' = \rprepeat(\rprog_l) \in \rprog$,
the \emph{Variable Gradient Decent}, 
 $\varGD(\rprog', c)$ is an invariant of the minimum amount decreased in one execution of $\rprog_l$ for the ranking functions in $\rprog$,
% and recursively computes the loop bound, $BD(\rprog, c)$ in the meantime.
{\small
\[
 \varGD(\rprog', c) \triangleq
 \left\{
 \begin{array}{ll}
 BD(\rprog', c) \times {\varGD(\rprog_l', c)} & \rprog' = \rprepeat(\rprog_l') \\
 \varGD(\rprog_1, c) + \varGD(\rprog_2, c) & \rprog' = \rprog_1;\rprog_2 \\
 \min\left\{v ~|~ x = v \in \rfnext(\tpath, c) \right\} & \rprog' = \tpath, 
 \end{array}
 \right.
 \]
 which is interactively computed together with $BD(\rprepeat(\rprog_l), c) \triangleq$
 \[
 % BD(\rprepeat(\rprog_l), c) =
 \left\{ 
 \begin{array}{ll}
 \max\limits_{x \in \left\{ \locbound(\absevent, c) | \absevent \in \tpath \right\}} 
 \Big\{ \highlight{\frac{a - b}{\varGD(\tpath, c)}} & \rprog_l = \tpath
 \\ \qquad 
 ~\vert~
 x = a \in \rfinit(\tpath, c)
 \land x = b \in \rffinal(\tpath, c)
 \Big\} 
 \\
 \max\limits_{x \in \left\{ \locbound(\absevent, c) | \absevent \in \tpath \right\}} 
 \Big\{ \highlight{\frac{\min\{ a, a - \varGD(\rprog_1, c) \} - b}{\varGD(\rprog_1;\rprog_2, c)}} 
 & \rprog_l = \rprog_1;\rprog_2
 \\ \qquad 
 ~\vert~
 x = a \in \rfinit(\rprog_1, c)
 % \\ \qquad 
 \land x = b \in \rffinal(\rprog_2;\rprog_2, c)
 \Big\} \\
 0 & o.w.
 \end{array} 
 \right.
\]
}
\end{defn}
% This approach computes a sound loop bound for a subprogram in a refined program formally in \highlight{Appendix~\ref{apdx:loopbound-sound}}.
In Example~\ref{ex:relatedNestedWhileOdd-overview}, we first compute $\rfinit = \{k = n - m\}$, $\rffinal = \{k = 0\}$ and $\rfnext = 1$ for subprogram $\rprepeat(\tpath_3)$ and get $BD(\rprepeat(\tpath_3)) = n - m$ tighter than $n$ by Eq.~\ref{eq:absBD}.
For the first iteration pattern $\rprog_1^1$ in Figure~\ref{fig:relatedNestedWhileOdd-overview}(b), we compute 
$\rfinit = \{k = n - m, i = m - 1\}$, $\rffinal = \{k = 0, i = 0\}$, $\varGD(\rprog_1^1) = 4$, and get $BD(\rprog_1^1) = \lfloor \frac{m}{4} \rfloor $ as expected.

Using the \emph{loop bound} computed by our new approach, we compute the local \emph{path reachability-bound} for
each $\tpath$ w.r.t. its \textbf{innermost nested loop} as below.

\begin{defn}[Local Path Reachability-bound Computation]
 \label{def:pathlocalrb}
 Given program $c$ with its refined program $\rprog$ and a simple transition path $\tpath$ and its innermost located loop program, $\rprog_l = \kw{enclosed}(\rprog, \tpath)$ in $\rprog$,
 we compute $\tpath$'s \emph{local reachability-bound} $\outinB(\rprog_l, \tpath, c)$ w.r.t. $\rprog_l$ inductively as follows.
 {\small
 \[
 \left\{
 \begin{array}{ll}
 1 & \rprog = \tpath\\
 \highlight{0} & \rprog = \tpath' \neq \tpath\\
 \outinB(\rprog_1, \tpath, c) + \outinB(\rprog_2, \tpath, c) & \rprog = \rprog_1;\rprog_2 \\
 % \outinB(\rpchoose{\rprog_1, \ldots, \rprog_m }, \tpath) & \triangleq 
 % & 
 \max\left\{ \outinB(\rprog_1, \tpath, c), \ldots, \outinB(\rprog_m, \tpath, c) \right\} 
 & \rprog = \rpchoose{\rprog_1, \ldots, \rprog_m } \\
 % \outinB(\rprepeat(\rprog'), \tpath) & \triangleq 
 % & 
 BD(\rprepeat(\rprog'), c) \times \outinB(\rprog', \tpath, c) & \rprog = \rprepeat(\rprog')
 \end{array}
 \right.
 \]
 }
 \end{defn}
 Soundness of $\outinB(\rprog_l, \tpath, c)$ is in \highlight{Appendix~\ref{apdx:pathlocalrb-sound}}.

 For the program in the first interleaving pattern $\rprog_1^1$ in Example~\ref{ex:relatedNestedWhileOdd-overview}, $\rprog_1^1$ is the innermost loop for $\tpath_1$, $\tpath_2$ and $\tpath_4$, 
 so compute $\outinB(\rprog_1^1, \tpath_1, c) = BD(\rprog_1^1, c) \times \outinB(\tpath_1; 4:\rprepeat(\tpath_3); \tpath_2; \tpath_4, \tpath_1, c)
 = BD(\rprog_1^1, c) \times (1 + 0) = \lfloor \frac{m}{4} \rfloor $ and the same for $\tpath_2$ and $\tpath_4$.

 As discussed in Section~\ref{sec:psrb}, if there isn't a nested loop, we will skip the Section~\ref{sec:looprb} and use this local bound directly as the global \emph{path reachability-bound}, $\inoutB(\rprog, \tpath, c)$ as in Section~\ref{sec:pathrb}. 
 So we already had the $\psRB$ for every program point $l \in \lvar(c)$ by replacing the equation in Definition~\ref{def:point_psrb} with the following one,
 \begin{equation}
 \label{eq:psrb-sub}
 \psRB(l, c) = 
 \sum
 \left\{ \outinB(\rprog, \tpath, c) ~\vert~ \tpath \in \rprog \land 
 l \in \tpath \right\}.
 \end{equation}
 However, if there are nested loops, 
 by replacing the restriction, $\rprog_l$ in Definition~\ref{def:pathlocalrb} with the global program $\rprog$, $\outinB(\rprog, \tpath, c)$ is actually a loose global \emph{path reachability-bound} for $\tpath$. Though we can still compute a loose $\psRB$ just using $\outinB(\rprog, \tpath, c)$, it is not tight enough. 
It computes
$\outinB(\rprog, \tpath_3, c) = \frac{m}{4} \times (n - m)$ in Example~\ref{ex:relatedNestedWhileOdd-overview}.
It is loose because Eq.~\ref{eq:psrb-sub} assumes the inner loop $L_4$ is reached in every iteration of the outer loop $L_1$. In this sense, $\psRB(4, c) = \frac{m}{4} \times (n - m)$ is loose as well.
While the inner loop $L_4$ is only reached in the first iteration of the outer loop,
so we will compute a more accurate quantity, the \emph{loop reachability-bound} in the next section.

