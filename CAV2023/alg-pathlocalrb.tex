
\subsubsection{Loop Bound}
There are two methods for computing the loop bound, $BD(\rprepeat(\rprog'), c)$ for every sub program, specifically loop program $\rprepeat(\rprog')$ in a program $c$'s refined program $\rprog$,
% \begin{enumerate}
%   \item Estimated by $BD(\rprepeat(\rprog'), c) = BOUND(\rprepeat(\rprog'))$ from paper~\cite{GulwaniJK09}, which is precise for simple loop without nested loops but not for nested loops. We are not giving the computation detail in this paper, which can be found in paper~\cite{GulwaniJK09}.
%   \item Estimated by Equation~\ref{eq:absBD} path-insensitively, which is based only on the estimated ranking function from Section~\ref{sec:rank}. 
%   This bound is precise when there is only one path in this loop but not for loops with multiple paths.
%   \item We provide a new computation approach using both the variable invariant estimation idea and the progress invariant idea from~\cite{GulwaniJK09} presented in Definition~\ref{def:loopbound}.
% \end{enumerate}
%
\paragraph{Ranking Function Based Loop Bound Computation}
\label{sec:loopbound-rankbased}
Based only on the estimated ranking function from Section~\ref{sec:rank}, the $BD(\rprog, c)$ for the refined program $\rprog$ can be estimated by $\absclr(\absevent, c)$ by taking the
minimum value over all $\absevent \in \rprog$.
We use $\absevent \in \rprog$ to denote $\absevent$ is an edge on path $\rprog$.
\begin{equation}
  \label{eq:absBD}
  BD(\rprog, c) = \min\left\{ \absclr(\absevent, c) \middle\vert \absevent \in \rprog \right\}.
\end{equation}
%
This bound is accurate on the iteration numbers of a single loop, but neither tight on the iterations of different paths inside the loop, nor the nested loop.
%
% \paragraph{Walk through Example.}
% \todo{The walk through example}
In our walk through program in Example~\ref{ex:relatedNestedWhileOdd_abscfg} Figure~\ref{fig:relatedNestedWhileOdd_abscfg}(b), by using this ranking function-based approach, we compute the loose loop bound for each subprogram,
$BD(\tpath_0) = 1$, $BD(\tpath_5) = 1$, $BD(\rprepeat(\tpath_3)) = n - m$, $BD(\rprog_1^1) = n$, and $BD(\rprog_1^2) = n $.
The bounds for $\rprog_1^1$ and $\rprog_1^2$ are both $n$ while expected to be $\frac{m}{4}$.
They are loose because of the nature of the path-insensitivity in the ranking function invariant computation in Definition~\ref{def:ranking_bound}. 

\paragraph{New Approach of Computing The Loop Bound} 
We provide a new computation approach using both the variable invariant estimation idea and the progress invariant idea from~\cite{GulwaniJK09} presented in Definition~\ref{def:loopbound}.
This new approach also uses the ranking function, but combines with the refined program.
We first compute the refined abstract states for ranking function in different program point and different local program based on the refined program.
Then we these precise abstract state, and estimate the loop bound path sensitively.

\begin{defn}[Rank Abstract States Computation]
  \label{def:alg-absstate}
  Given a program $c$, with $\rprog$ as a sub-program in $c$'s refined program, where $\rprog \subseteq REFINE(\kw{Rewrite(c)})$, let $c_r$ be the while program such that $\rprog = \algrewrite(c_r)$,
  we compute four different abstract states for the ranking functions of this sub program, 
  $\rfinit(\rprog, c)$, $\rffinal(\rprog, c)$, $\rfnext(\rprog, c)$ and $\varGD(\rprog, c)$ as follows.
 \begin{itemize}
  \item We first compute the initial execution point and continuation execution points of $\rprog$ as
  $\absinit(\rprog) = \absinit(c_r)$
  and 
  $\absfinal(\rprog) = \absfinal(c_r)$ where $\rprog = \algrewrite(c_r)$.
  \item The \emph{Initial State}, 
  $\rfinit(\rprog, c)$ is a set of equations $x = e$, where $e \in \scexpr(c)$ is a
  symbolic expression describing the initial value of $x$ before executing $\rprog$.
  Each $x$ is the ranking function of a simple transition path $\tpath$ in this program. 
  It contains the initial value for every ranking function of the simple transition path $\tpath$ in this program.
 \[
   \rfinit(\rprog, c) \triangleq 
   \bigcup_{x \in \left\{ \locbound(\absevent, c) | \tpath \in \rprog \land \absevent \in \tpath \right\} }
   \left \{ 
   x = \arg\max_{l_1}\left\{
     \varinvar(y, c) + v ~\middle\vert~ 
     \begin{array}{l} 
       (l_1, x' \leq y + v, l_2) \in \reset(x, c) 
       \\
     \land l_1 \leq \absinit(\rprog, c)
   \end{array}
   \right\}
   \right\}
   \]
 $\rfinit(\tpath, c)$ can also be computed using the function $\kw{INIT(c, 0, \absinit(\rprog))}$ in \cite{GulwaniJK09}. 
 %
 \item  The \emph{Final State}, $\rffinal(\rprog, c)$ a conjunction of boolean expressions.
 It is the post-condition
 after the execution of $\rprog$.
 % \\
 \[
  \begin{array}{l} 
    \rffinal(\rprog, c) \triangleq 
    \\
   \bigcup\limits_{x \in \left\{ \locbound(\absevent, c) | \tpath \in \rprog \land \absevent \in \tpath \right\} }
   \left \{ 
   x = \min\limits_{ \left\{ l_1 ~|~ l_1 \geq \pi_2(\absfinal(\rprog, c)) \right\} }
   \max\left\{
     v ~\middle\vert~ 
     \left( (x = v) \land \bigwedge\limits_{b \in \kw{Guard}(\rprog, c)} \neg b \right) \neq \efalse
   \right\}
   \right\}
  \end{array}
  \]
  $\kw{Guard}(\rprog, c)$ is the set of all the unique boolean expressions (i.e., the boolean constraints) on this program.
 \item The \emph{Next State}, $\rfnext(\tpath, c) \in \scexpr(c)$ 
 is a
 symbolic expression describing how much $\tpath$'s ranking function ($\locbound(\tpath)$) is changed after the first execution of $\rprog$ and before the second execution.
 \[
   \begin{array}{l}
   \rfnext(\tpath, c) \triangleq 
   \\
   \bigcup\limits_{x \in \left\{ \locbound(\absevent, c) | \absevent \in \tpath \right\}}
   \left\{ x = \begin{array}{l}
  \sum\limits_{\absevent \in \inc(x, c) }
   \left\{ v ~\middle\vert~ \absevent = (l, x' \leq x + v, \_) \land l \in \tpath\right\}
   \\ \qquad 
   + \arg\max\limits_{l' }
      \left\{ \varinvar(y, c) + v ~\middle\vert~ (l, x' \leq y + v, l') \in \reset(x, c) \land l \in \tpath\right\}
      \\ \qquad 
     - \sum\limits_{ \absevent \in \dec(x, c) }\left\{ 
       v ~\middle\vert~ \absevent = (l, x' \leq x - v, \_) \land l \in \tpath 
       \right\}
     \end{array}
   \right\} 
   \end{array}
 \]
 Indeed we only compute the $\rfnext(\tpath)$ because that the recursion is exhausted into the base case, i.e. $\tpath$ when computing $\varGD(\rprog, c)$ as below.
 \item  The \emph{Variable Gradient Decent}, 
 $\varGD(\rprog, c)$
 is a set of equations $x = e$ for the same variables in the $\rfinit(\rprog, c)$.
 It computes the set of next states for every ranking function of the simple transition path $\tpath$ in $\rprog$,
 and recursively computes the loop bound, $BD(\rprog, c)$ in the meantime.
 \\
 {$\varGD(\rprepeat(\rprog), c) =  BD(\rprog, c)  \times
{\varGD(\rprog, c)}$}
 \\
 $\varGD(\rprog_1;\rprog_2, c) =  \varGD(\rprog_1, c) + \varGD(\rprog_2, c)$
 \\
 $\varGD(\tpath, c) =  \min\left\{v  ~|~ x = v \in \rfnext(\tpath, c) \right\}  $  
\end{itemize}
\end{defn}
The computation of the \emph{Variable Gradient Decent}, 
$\varGD(\rprog, c)$ involves the computation of $BD(\rprog, c)$ recursively as follows.
\begin{defn}[Loop Bound Computation]
\label{def:loopbound}
  \[
    \begin{array}{rcl}
      BD(\tpath, c) & \triangleq & 1 \\
      BD(\rprog_1;\rprog_2, c) & \triangleq & \min \left\{BD(\rprog_1, c), BD(\rprog_2, c) \right\} \\
      BD(\rpchoose{\rprog_1, \ldots, \rprog_m }, c) & \triangleq 
      & \max\left\{ BD(\rprog_1, c), \ldots, BD(\rprog_m, c) \right\} \\
      BD(\rprepeat(\rprog'), c) & \triangleq 
      &
      \max\limits_{x \in \left\{ \locbound(\absevent, c) | \tpath \in \rprog' \land \absevent \in \tpath \right\}}
      \Big\{ \highlight{\frac{a - b}{\varGD(\rprog', c)}} ~\vert~
      x = a \in \rfinit(\rprog', c)
      \\ & & \qquad \qquad \qquad \qquad \qquad \qquad \qquad 
      \land x = b \in \rffinal(\rprog', c)
      \Big\} 
      \\
      BD(l: \rprog_l, c) & \triangleq & BD(\rprog_l, c)
    \end{array}
  \]
\end{defn}
The Definition~\ref{def:loopbound} computes a sound loop bound for every loop in a refined program.
%  below in Lemma~\ref{lem:loopbound_sound} with proof in \highlight{Appendix~\ref{apdx:loopbound-sound}}.
% \begin{lem}[Soundness of Loop Bound]
%   \label{lem:loopbound_sound}
%   For every program $c$ with it refined program $\rprog$,
%   % for every loop $\rprepeat(\rprog')$ inside $\rprog$, 
%   $BD(\rprog, c)$ is a sound upper bound on the iterating times of this program.
% \end{lem}

% \paragraph{Walk through Example.}
% \todo{The walk through example}
Using our new approach, we compute a more precise loop bound $BD(\rprog)$ for each subprogram of our walk through example in Figure~\ref{fig:relatedNestedWhileOdd_abscfg}(b), in comparison to the purely ranking function-based approach in Equation~\ref{eq:absBD}.
Specifically for $\rprog_1^1$ and $\rprog_1^2$, we compute $BD(\rprog_1^1) = \frac{m}{4}$ and $BD(\rprog_1^2) = \frac{m}{4}$ as expected,
and compute the same bounds for the other sub-programs.
% $BD(\tpath_0) = 1$, $BD(\tpath_5) = 1$, $BD(\rprepeat(\tpath_3)) = n - m$, $BD(\rprog_1^1) = n$, and $BD(\rprog_1^2) = n $.
% The bounds for $\rprog_1^1$ and $\rprog_1^2$ are both $n$ while expected to be $\frac{m}{4}$.
% They are loose because of the nature of the path-insensitivity in the ranking function invariant computation in Definition~\ref{def:ranking_bound}. 

\subsubsection{Path Local Reachability-bound}
We first compute $\kw{enclosed}(\rprog, \tpath)$, which is \textbf{the innermost loop} inside $\rprog$ where $\tpath$ is located. Then by using the loop bound computed by our new approach, we compute the \emph{path local reachability-bound} of
the $\tpath$ in its closest nested loop as follows in Definition~\ref{def:pathlocalrb}.
\begin{defn}[Path Local Reachability-bound Computation]
    \label{def:pathlocalrb}
    Given program $c$ with its refined program $\rprog$ and a simple transition path $\tpath$ in this program, 
    let $l: \rprog_l = \kw{enclosed}(\rprog, \tpath)$ be a sub loop program in $\rprog$,
    then $\tpath$'s \emph{local reachability-bound} $\outinB(\rprog_l, \tpath, c)$ w.r.t. $l: \rprog_l$
    is computed inductively as follows.
    The input $c$ is omitted in the following equations for concise given the context is clear.
  \[
    \begin{array}{rcl}
      \outinB(\tpath, \tpath) & \triangleq & 1 \\
      \outinB(\tpath', \tpath) & \triangleq & \highlight{0} \qquad \text{if } \tpath' \neq \tpath\\
      \outinB(\rprog_1;\rprog_2, \tpath) & \triangleq & \outinB(\rprog_1, \tpath) + \outinB(\rprog_2, \tpath) \\
      \outinB(\rpchoose{\rprog_1, \ldots, \rprog_m }, \tpath) & \triangleq 
      & \max\left\{ \outinB(\rprog_1, \tpath), \ldots, \outinB(\rprog_m, \tpath) \right\} \\
      \outinB(\rprepeat(\rprog'), \tpath) & \triangleq 
      & BD(\rprepeat(\rprog'), c) \times \outinB(\rprog', \tpath)
       \\
       \outinB(l: \rprog_l, \tpath) & \triangleq & \outinB(\rprog_l, \tpath) \\
       \outinB(l': \rprog', \tpath) & \triangleq & 0  \qquad \text{if } l': \rprog' \neq \kw{enclosed}(\rprog, \tpath)
    \end{array}
    \]
\end{defn}
We show the \emph{path local reachability-bound} of the simple transition path $\tpath$ in a refined program $\rprog$ is a sound upper bound of its execution times when executing its enclosed loop $\kw{enclosed}(\rprog, \tpath)$ in Lemma~\ref{lem:pathlocalrb-sound}, with the proof in \highlight{Appendix~\ref{apdx:pathlocalrb-sound}}.
\begin{lem}[Soundness of the Path Local Reachability-bound]
  \label{lem:pathlocalrb-sound}
  For any program $c$ with its refined program $\rprog$ and a simple transition path $\tpath$ in $\rprog$,
  if $l: \rprog_l = \kw{enclosed}(\rprog, \tpath)$ is the closest loop where $\tpath$ is nested in this program,
  then the execution times of $\tpath$ when executing the $\rprog_l$ under initial trace $\trace_l \in \ftdom_0(c_l)$ is bounded by $\econfig{\outinB(\rprog_l, \tpath)}(\trace_0)$ with any initial trace $\trace_0 \in \ftdom_0(c)$.
  \[
    \begin{array}{l}
    \forall c, c_l \in \cdom, \tpath \in \absG(c), 
    \trace_l \in \ftdom_0(c_l), \trace_0 \in \ftdom_0(c), \trace \in \tdom, l, l' \in \ldom, \rprog \st 
    \\ \qquad
    \rprog = REFINE(c)
    \land
    l: \rprog_l = \kw{enclosed}(\rprog, \tpath)
    \land 
    \rprog_l = \algrewrite(c_l)
    \\ \qquad
    \land
    \Big(
    \config{c_l, \trace_l} \rightarrow^* \config{\clabel{\eskip}^{l'}, \trace_l \tracecat \trace}
    \lor \config{c_l, \trace_l} \uparrow^{\infty} \trace_l \tracecat \trace 
    \Big)
    \\ \qquad
    \implies
    \econfig{\outinB(\rprog_l, \tpath)}(\trace_0) \geq \lcounter(\trace, \pathl(\tpath)).
    \end{array}
  \]  
\end{lem}


% \paragraph{Example.}
% \todo{Walk through example}
Now based on the tighter loop bound,
%  we compute another loop bound for every subprogram in the Example~\ref{ex:relatedNestedWhileOdd_abscfg} as $BD(\tpath_0) = 1$, $BD(\tpath_5) = 1$, $BD(\rprepeat(\tpath_3)) = n - m $, $BD(\rprog_1^1) = \frac{m}{4} $, and $BD(\rprog_1^2) = \frac{m}{4} $.
% They are more precise comparing to purely ranking function-based approach in Equation~\ref{eq:absBD}.
% Using this precise loop bound, 
we compute local reachability-bound for every transition path w.r.t. its innermost loop as
$\outinB(\rprog, \tpath_0) = 1$,
$\outinB(\rprog, \tpath_4) = 1$,
$\outinB(1: \rprog_1^1, \tpath_1) = \frac{m}{4}$,
$\outinB(1: \rprog_1^1, \tpath_2) = \frac{m}{4}$,
$\outinB(1: \rprog_1^1, \tpath_4) = \frac{m}{4}$,
$\outinB(1: \rprog_1^2, \tpath_1) = \frac{m}{4}$,
$\outinB(1: \rprog_1^2, \tpath_2) = \frac{m}{4}$,
$\outinB(1: \rprog_1^2, \tpath_4) = \frac{m}{4}$, and
$\outinB(4: \rprepeat(\tpath_3), \tpath_3) = n - m$ 

% \todo{necessary?}
\highlight{
Notice here we can already compute the path reachability-bound simply use the Loop Bound $BD$ recursively. By simply remove the restriction, $l: \rprog_l = \kw{enclosed}(\rprog, \tpath)$ in Definition~\ref{def:pathlocalrb} and use $\rprog$
as the input. Then the same Definition~\ref{def:pathlocalrb} computes $\outinB(\rprog, \tpath)$ over the global program, $\rprog$ and use $\outinB(\rprog, \tpath)$  as the path global reachability-bound.
Given we have already computed the path global reachability-bound,
we are able to compute the \emph{Reachability-bound} for every program point  $l \in \lvar(c)$ by replacing the equation in Definition~\ref{def:point_psrb} with the following one,
\[ 
  \psRB(c, l) = 
  \sum
  \left\{ \outinB(\rprog, \tpath) ~\vert~ \tpath \in \rprog \land 
  l \in \tpath \right\}
\]
But this substitution will compute loose reachability-bound for every path.
\paragraph{Walk through Example.}
To show the imprecision of this simple substitution, we compute the global reachability-bound for every transition paths for our walk through example as
$\outinB(\rprog, \tpath_0) = 1$,
$\outinB(\rprog, \tpath_1) = \frac{m}{4}$,
$\outinB(\rprog, \tpath_2) = \frac{m}{4}$,
$\outinB(\rprog, \tpath_3) = \frac{m}{4} \times (n - m)$,
$\outinB(\rprog, \tpath_4) = \frac{m}{4}$, and
$\outinB(\rprog, \tpath_5) = 1$.
It is loose for the $\tpath_3$ because the substituted computation assumes the loop $4$ is reached in every iteration of its outside loop $1$. Naturally, the program point reachability-bound computed using $\outinB(\rprog, \tpath)$ will be loose as well for location $4$ and $5$.
In this sense, we compute the \emph{loop reachability-bound} in the next section for a tighter program point \emph{Reachability-bound}.
% \todo{Walk through example with a loose path and point reachability bound}
}