In this section, we present our static program analysis algorithm for computing 
% an upper bound on the 
% execution-based reachability times 
the \emph{reachability-bound} for every program point $l$ in a program $c$ in a path sensitive manner.
The algorithm is summarized into the following steps,
%
\begin{enumerate}
\item  The Section~\ref{sec:progabs} first 
computes the Abstract Transition graph, $\absG(c)$ for a program $c$ through program abstraction.
It computes the abstract transition 
for every labeled command in $c$ and each abstract transition is an edge in $\absG(c)$.
\item The second step in Section~\ref{sec:refine}
computes the refined program, $\rprog$ for a program $c$ based on 
its abstraction transition graph by the refine algorithm in \cite{GulwaniJK09}.
This algorithm transforms the multiple-paths loops
into multiple loops where
the interleaving of paths is explicit.
\item Section~\ref{sec:psrb} computes the path-sensitive reachability-bound for every program point.
It is based on computing the path reachability bound and the loop reachability bound
in Section~\ref{sec:looprb}.
The path reachability bound is an upper bound on the execution times of a transition path,
and
the loop reachability bound is an upper bound on the execution times of a outside loop 
w.r.t its nested loop. These two bounds also involve computing the ranking function  
\footnote{\textbf{ranking function} is the named used in \cite{SinnZV14}
and \textbf{local bound} is the name used in \cite{ZulegerGSV11}, \cite{sinn2017complexity}.
We refer to the two names as the same meaning in this paper.} 
for every transition path,
and estimating the upper bound on every ranking function's maximum value. The details are in Section~\ref{sec:rank}.
\end{enumerate}