Our path-sensitive reachability-bound algorithm relies on the \emph{Abstract Transition Graph}, $\absG(c)$, the \emph{Refined Program}, $\rprog$ and the upper bound invariant of the \emph{Ranking Function} computed previously for the program $c$.
It first requires to compute the new quantities, \emph{Path Reachability-bound} and the \emph{Loop Reachability-bound}, and this section introduces the definition and the following sections describe how we compute them.
%  of \emph{Path Reachability-bound} and the \emph{Loop Reachability-bound}. 

% Given a program $c$ with its \emph{Abstract Transition Graph}, $\absG(c)$ and refined program $\rprog$, our path-sensitive reachability-bound algorithm is presented as follows.

As pre-procedures, we first need to compute the loop bound, $BD(\rprog', c) \in \scexpr(c)$ for every subprogram $\rprog'$ of $c$ in $\rprog$, and use it to estimate the \emph{path local reachability-bound}, $\outinB(\rprog_l, \tpath, c) \in \scexpr(c)$ for each $\tpath$ w.r.t. the sub loop program $\rprog_l$.
\begin{defn}[Loop Bound]
  % \label{def:loopbound}
  For any program $c$ with it refined program $\rprog$,
  the loop bound $BD(\rprog', c) \in \scexpr(c)$ for a subprogram $\rprog'$ of $c$ in $\rprog$ is a upper bound on the iterating times of this program from its enter point to the exit point.
\end{defn}
% 
% Then we compute the \emph{path local reachability-bound}, $\outinB(\rprog_l, \tpath, c) \in \scexpr(c)$ for every sub loop program $\rprog_l$ of $c$ in $\rprog$.
\begin{defn}[Path Local Reachability-bound]
  % \label{def:pathlocalrb}
  Given program $c$ with its refined program $\rprog$ and a simple transition path $\tpath$ in this program, 
  let $l: \rprog_l = \kw{enclosed}(\rprog, \tpath)$ be a sub loop program in $\rprog$,
  then $\tpath$'s \emph{path local reachability-bound} w.r.t. $l: \rprog_l$,  $\outinB(\rprog_l, \tpath, c) \in \scexpr(c)$
  is an upper bound on the execution times of $\tpath$ when executing program $\rprog$.
\end{defn}
Intuitively,
% the local reachability-bound of a \emph{simple transition path},
$\outinB(\rprog_l, \tpath, c)$ bounds the execution times of $\tpath$ when executing its innermost loop program $\rprog_l$.
% and $\rprog$ is the closest loop where $\tpath$ is nested.
For example in the first interleaving pattern $\rprog_1^1$ in Example~\ref{ex:relatedNestedWhileOdd-overview}, 
$4:\rprepeat(\tpath_3)$ is the innermost loop program of $\tpath_3$. So we first compute $m - n$ as its \emph{path local reachability-bound} to bound its iteration times in its closest loop by Section~\ref{sec:pathlocalrb}.
% $4:\rprepeat(\tpath_3)$ is the innermost loop program of $\tpath_1$, $\tpath_2$ and $\tpath_4$.
% We compute the $\frac{m}{4}$ for all the three path $\tpath_1$, $\tpath_2$ and $\tpath_4$ w.r.t. $\rprog_1^1$, as their \emph{path local reachability-bound}.
% $\outinB(1: \rprog_1^1, \tpath_1) = \frac{m}{4}$,
% $\outinB(1: \rprog_1^1, \tpath_2) = \frac{m}{4}$,
% $\outinB(1: \rprog_1^1, \tpath_4) = \frac{m}{4}$,

Next, we define the \emph{loop reachability-bound},
$\lpchB(l: \rprog_l, \tpath, c) \in \scexpr(c)$ for every outer loop program $\rprog_l$ of $\tpath$ in $\rprog$. This quantity aims to precisely bound the iteration numbers of the outer loop $l$,
such that,
during these iterations, the innermost loop $l' = \kw{enclosed(\tpath)}$ is executed, i.e., reached.
\begin{defn}[Loop Reachability-bound]
% \label{def:looprb}
For a program with its refined program $\rprog$ and a simple transition path $\tpath$ in this program, 
let $l: \rprog_l$ be a loop program in $\rprog$,
then $l: \rprog_l$'s \emph{loop reachability-bound} w.r.t. $\tpath$,  $\lpchB(l: \rprog_l, \tpath, c) \in \scexpr(c)$
is the upper bound on iteration numbers of the outside loop $l$,
such that,
during these iterations, the nested loop $l' = \kw{enclosed(\tpath)}$ is entered.
\end{defn}
As introduced in Section~\ref{sec:overview} for Example~\ref{ex:relatedNestedWhileOdd-overview}, $\tpath_3$ has an outer loop program $\rprog_1^1$. Since $L_4$ will be ``entered'' only in the first iteration of $\rprog_1^1$,
we aim to compute $1$ for $\lpchB(\rprog_1^1, \tpath_3, c)$ in Section~\ref{sec:looprb}.
  % and we compute $\lpchB(\rprog_1^1, \tpath_3, c) = 1$. It is tight because the innermost loop of $\tpath_3$ will be ``entered'' only in the first iteration of $\rprog_1^1$.

% Intuitively $\lpchB(l: \rprog_l, \tpath, c) \in \scexpr(c)$
% is the bound on iteration numbers of the outside loop $l$,
% such that,
% during these iterations, the nested loop $l' = \kw{enclosed(\tpath)}$ is executed, i.e., reached.
The \emph{path reachability-bound}, $\inoutB(\rprog, \tpath)$ for each $\tpath$
aims to bound the execution times of $\tpath$ globally during the execution of $c$ and Section~\ref{sec:pathrb} presents the estimating algorithm.
%
\begin{defn}[Path Reachability-bound]
% \label{def:pathrb}
For a program $c$ with its refined program $\rprog$ and a simple transition path $\tpath$ in this program, 
$\tpath$'s reachability-bound, $\inoutB(\rprog, \tpath) \in \scexpr(c)$ is the upper bound on the
execution times of $\tpath$ when executing the $\rprog$.
\end{defn}
% Intuitively, $\inoutB(\rprog, \tpath)$ bounds the execution times of $\tpath$ globally during the execution of $c$.

For our running program in Example~\ref{ex:relatedNestedWhileOdd-overview}, since there isn't nested loop for $\tpath_1, \tpath_2$ and $\tpath_3$, we compute their $\inoutB(\rprog, \tpath) = \frac{m}{4} $, which is the same as their local bound.
While for $\tpath_3$ we compute  the $\inoutB(\rprog, \tpath_3) = \frac{m}{4} \times 1$ by multiplying its local bound with loop reachability bound.
% path is $\tpath_3 = \frac{m}{4} \times 1 = \frac{m}{4} $.

% \paragraph{Program Points Reachability-bound Computation}
% \label{sec:point-psrb}
The \emph{Reachability-bound} for program points is finally computed as follows.
For each program point in a program $c$, $l \in \lvar(c)$,
%  in a program $c$,
its \emph{reachability-bound}, $\psRB(c, l)$ during the execution of $c$ is computed as follows.
%
\begin{defn}[Program Point Reachability-bound Computation]
\label{def:point_psrb}
Given a program $c$ with its \emph{Abstract Transition Graph}, $\absG(c)$ and refined program $\rprog$,
the \emph{reachability bound} of each program point $l \in \lvar(c)$, $\psRB(c, l)$ 
sums up all the path reachability bounds, $\inoutB(\rprog, \tpath)$ over all simple transition paths $\tpath$ that contains the program point $l$.
\[ 
  \psRB(c, l) = 
  \sum
  \left\{ \inoutB(\rprog, \tpath) ~\vert~ \tpath \in \rprog \land 
  l \in \tpath \right\}\footnotemark
\]
$l \in \tpath$ denotes that the program point $l$ is a vertex on $\tpath$ 
and $\tpath \in \rprog$ denotes $\tpath$ is a simple transition path in $\absG(c)$.
\footnotetext{$l \in \tpath$ and $\tpath \in \rprog$, the $\in$ notation is abused to denote
the program point $l$ is a vertex on this path and $\tpath$ is a simple transition path on $\absG(c)$ respectively.}
\end{defn}
$\econfig{\psRB(c, l)}$ is a \emph{reachability-bound} for every program point $l$ in a program $c$.
\begin{thm}[Soundness of the Path-sensitive Reachability-bound Estimation]
\label{thm:pathsensitive_rb_soundness}
For every program ${c}$ and every label $l$ in this program,
$\econfig{\psRB(c, l)}$ is a \emph{Reachability-bound} for $l$ in $c$.
%
{\small
\[
  \begin{array}{l}
    \forall c, c_r \in \cdom, \tpath \in \absG(c), \trace_0 \in \ftdom_0(c),  \trace_r \in \ftdom_0(c_r), \trace \in \tdom, l, l' \in \ldom, \rprog \st 
    \\ \qquad
    \rprog = REFINE(\algrewrite(c))
    \land 
    \rprog = \algrewrite(c_r)
    \land
    \\ \qquad
    \land
    \Big(
    \config{c_r, \trace_0} \rightarrow^* \config{\clabel{\eskip}^{l'}, \trace_0 \tracecat \trace}
    \lor \config{c_r, \trace_0} \uparrow^{\infty} \trace_0 \tracecat \trace 
    \Big)
    \\ \qquad
    \implies \econfig{\psRB(c, l)}(\trace_0) \geq \counter(\trace, l)
  \end{array}.
\]
}
\end{thm}