We implemented {\THESYSTEM} as a tool which takes a labeled command as input  
and outputs the reachability-bound on each program point.
This implementation consists of the 
program abstraction from Section~\ref{sec:progabs} in OCaml and the program refinement, ranking function estimation and the path-sensitive reachability-bound algorithm in Python.
We evaluated it on our self-extracted programs from literatures, and XX programs extracted from four benchmarks. 
Including 75 {\tt .c} programs from different benchmarks\footnote{ABC: 15, C4B: 15, KoAT~\cite{BrockschmidtEFFG14,FalkeKS12,FalkeKS11}: 3, Loopus: 10, Rank10: 3, SPEED: 27, WTC: 37} and 37 challenging {\tt .c} programs 
from the \hyperlink{https://forsyte.at/static/people/sinn/loopusJAR/index.html}{LoopusJAR}~\cite{SinnZV17},
37 {\tt .java} programs from~\hyperlink{https://zenodo.org/record/5140586\#.Y5pBoC-B1QI}{Tianhan}\cite{LuCT21}
and XXX {\tt .c} programs from~\hyperlink{https://github.com/icra-team/icra}{Icra}~\cite{KincaidBCR19,CyphertBKR19}.
We translated the programs from different benchmarks into our language model.
As the comparison, we also evaluated four bound analysis tools, 
\hyperlink{https://forsyte.at/software/loopus/}{LoopusJAR}~\cite{SinnZV17},
\hyperlink{https://github.com/aeflores/CoFloCo/tree/master/src}{CoFloCo}~\cite{Montoya17,Flores-Montoya16,Flores-MontoyaH14}
and \hyperlink{https://zenodo.org/record/5140586\#.Y5pBoC-B1QI}{Tianhan}~\cite{LuCT21}
and self-implemented SPEEDi based on paper~\cite{GulwaniJK09}
over the same benchmarks.

Tab.~\ref{tb:accuracy-eval} shows the evaluation results including the theoretical bounds, and empirical bounds computed.
% on different path locations.
We are the only one that computes different bounds on different path points,
and we present only the computed numbers by other tools. In column $4$, for almost all the multipath loops, the bounds on different path location are significantly different, which never be computed be any of the existing tools.
For the challenging loops without multipath, we do slightly worse but still comparable to existing tools.
% As in Tab.~\ref{tb:performance-eval}, our tool suffers the efficiency problem because of the implementation and the possible non-terminating in REFINE algorithm and the ranking function estimating algorithm.
Our tool suffers the efficiency problem because of the possible non-terminating in program refining and the ranking function estimating.
The performance evaluation is in Appendices~\ref{apdx:eval-performance}.
%  Tab.~\ref{tb:performance-eval}.
% \begin{table}[H]
%     \caption{The Performance Evaluation of {\THESYSTEM}}
%     \label{tb:performance-eval}
%     \centering
%         {\footnotesize
%         \begin{tabular}{ >{\small}c | c | c | c | c | c | c | c | c | c }
%         \multirow{2}{*}{Benchmark} & \multirow{2}{*}{\# P}  & \multirow{2}{*}{M.P.L \#} & Nested  & \multirow{2}{*}{Bounded} & {Success} & \multirow{2}{*}{Failed} & Time  & Total\\
%          &  &  & Loop \# & & Rate &  & Outs &   Runtime \\
%         \hline
%             {Loopus} & {110}  & 53  & 52  & 98 & 89.1\% & 11 & 7 & 7min42sec \\
%             \hline
%             Loopus-C & 23  & 20 & 20 & {20} & {86\%}  & 2 & 3 & {12min39sec} \\
%             \hline
%             \todo{Icra} & 72 & 72 & 6 & {55} + \todo{-} & + \todo{-} & 0 & 0 & \todo{1min58sec} \\
%             \hline
%             Tianhan & 37 & 2 & 3 & 37 & 100.0\% & 1 & 0 & 1min03sec \\
%             \hline
%         \end{tabular}
%         }
%     \end{table}
    \begin{table}[ht]
        \vspace{-1cm}
        \caption{The Accuracy Evaluation of {\THESYSTEM}}
        \label{tb:accuracy-eval}
        \centering
        {\scriptsize
        \begin{tabular}{ >{\scriptsize}c | >{\scriptsize}c | >{\scriptsize}c | >{\scriptsize}c | c | c | c | c | c | c }
        {Benchmark} &  {Theory} & & {\THESYSTEM} & \multicolumn{5}{c}{Computed}  \\
        \cline{5-8}
         Suit & Worst & \# & on Loop Paths  & {\tiny \THESYSTEM} & {\tiny Loopus} & {\tiny CoFloCo} & {\tiny SPEED} & {\tiny Tianhan} \\
        %  & {\tiny Icra} \\
        \hline
        \multirow{5}{*}{Loopus} 
        & $O(1)$        &   4   & $O(1)$  & 4   & 2 & 3 & 2 & 1 \\
        \cline{2-9}
        & $O(n)$        &  59   & $O(1), O(n)$  & 48  & 51 & 45 & 46 & 40 \\
        \cline{2-9}
        & $O(n^2)$      &  31   & $O(1), O(n), O(n^2)$ & 37  & 29 & 34 & 37 & 49 \\
        \cline{2-9}
        & $O(n\log(n))$ &  6   & $-$ & 0  & 0 & 0 & 0 & 0 \\
        \cline{2-9}
        & $O(n^3)$      &  2  & $O(n^3)$     & 4  & 1 & 2 & 5 & 7 \\
        \cline{2-9}
        & $O(n^{4})$    &  3  & $O(n^4)$  & 4  & 5 & 3 & 5 & 5 \\
        \hline \hline
        % \multirow{2}{*}{Loopus} 
        & $O(1)$      & 1     & $-$  & 0  & 3 & 1 & 0 & 0 \\
        \cline{2-9}
        Loopus & $O(n)$  & 13   & $O(1), O(n)$   & 14 & 17 & 17 & 15 & 11 \\
        \cline{2-9}
        & $O(n^2)$      & 4    &$O(n), O(n^2)$ & 3 & 14 & 15 & 16 & 21 \\
        \cline{2-9}
        Challenging
        & $O(n^3)$     & 4     &  $O(1), O(n), O(n^3)$ & 2 & 1 & 0 & 2 & 2 \\
        \cline{2-9}
        & $O(n^{4})$    & 1    & $O(n^4), O(n^3)$  & 1 & 5 & 3 & 5 & 5 \\
        \hline \hline
        \multirow{3}{*}{Icra} 
        & $O(1)$       & 3     & $O(1)$  & - &  &  & - & \\
        \cline{2-9}
        & $O(n)$       & 3     &  $ O(1), O(n)$ & 51  & 51 & - & - & - & \\
        \cline{2-9}
        & $O(n^2)$     & 3     &  $O(1), O(n), O(n^2)$ & 4 & - & - & - & - \\
        % \hline
        % \multirow{3}{*}{Icra-SV} 
        % & $O(1)$            & $O(1)$ & $O(1)$  & - &  &  & -\\
        % \cline{2-9}
        % & $O(n)$            &  $-$ & $-$  & - &  &  & -  \\
        % \cline{2-9}
        % & $O(n^2)$          &  $-$ & $ - $ & - &  &  & - &  \\
        \hline \hline
        \multirow{3}{*}{Tianhan} 
        & $O(n)$       & 34     & $O(n) $ & 34 & 35 & 35 & 35 & 35 \\
        \cline{2-9}
        & $O(n^2)$      & 3    &  $O(n^2)$  & 3 & 2 & 2 & 2 & 2 \\
        \hline
        \end{tabular}
        }
        \vspace{-1cm}
    \end{table}
    