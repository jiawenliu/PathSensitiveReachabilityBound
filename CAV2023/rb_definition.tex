% The operator $\lvar: \cdom \to \mathcal{P}(\ldom)$,
% computes the set of all labels
% in a labeled command $c$ defined as follows, we call it program points.
% Every the program point is unique.
% \begin{defn}[Program Points ($\lvar : \cdom \to \mathcal{P}(\ldom)$]
% \label{def:lvar}
% {\small
% \[
%   \lvar(c) \triangleq
%   \left\{
%   \begin{array}{ll}
%       \{l\}                  
%       & {c} = \clabel{\assign x \expr}^{l} 
%       \\
%       \lvar({c_1}) \cup \lvar({{c_2}}) 
%       & {c} = {c_1};{c_2}
%       \\
%       \lvar(c_1) \cup \lvar({{c_2}}) \cup \{l\} 
%       & {c} =\eif(\clabel{\bexpr}^{l}, c_1, c_2) 
%       \\
%       \lvar(c_w) \cup \{l\} 
%       & {c}   = \ewhile \clabel{\bexpr}^{l} \edo (c_w)
% \end{array}
% \right.
% \]
% }
% \end{defn}
% %
% Every program point corresponds to a labeled command in a program (either a guard of if or while command, or an assignment command), and it is unique.
% The Lemma below formally describes the uniqueness property of the program points
% with proof in Appendix~\ref{apdx:lem_language}.
% \begin{lem}[Uniqueness of The Program Point]
%   \label{lem:label_unique}
%   For every program $c \in \cdom$ and every two labels such that
%   $i, j \in \lvar(c)$, then $i \neq j$.
%   \[
%     \forall c \in \cdom, i, j \in \ldom \st i, j \in \lvar(c)\implies i \neq j.
%   \]
% \end{lem}
\paragraph{The Reachability-bound}
We use a set $\lvar(c) \subseteq \ldom$ collecting all the program points in a program $c$, which are simply all the unique labels of $c$.
% , and we denote by $\lvar(c) \subseteq \ldom$ the set of all program points in a program $c$.  
%
And we also define the operator $\tracel : \tdom \to \mathcal{P}{(\ldom)}$ projects the label from every event in a trace as a list of program points as follows,
{\small
\[
\tracel([(\_, l, \_)] \tracecat \trace') \triangleq [l] \tracecat \tracel(\trace')
\qquad
\tracel(\trace) \triangleq []~~ \mbox{if}~~ \trace = [] \lor \trace \in \inftdom
\]
}
Now we introduce two important counter notations counting the occurrence of a program point and a list of program point respectively.
\begin{defn}[Program Point Counter]
  \label{def:counter}
The counting operator $\counter : \tdom \to \ldom \to (\mathbb{N} \cup \{\bot\})$
counts the appearance of a label in a trace.
{\small
\[
\begin{array}{llll}
\counter([(\_, l, v)] \tracecat \trace', l ) \triangleq \counter(\trace', l) + 1 & \text{if}~ l = l
&
\counter({[]}, l) \triangleq 0 & 
\\
\counter([(\_, l', v)] \tracecat \trace', l) \triangleq \counter(\trace', l)   & \text{if}~ l' \neq l
&
\counter(\trace, l) \triangleq \bot & \text{if }~ \trace \in \inftdom
\end{array}
\]
}
% {When the input trace is infinite, $\lcounter(\trace, l)$ returns $\bot$ for any $l$.}
\end{defn}
\begin{defn}[Point List Counter]
  \label{def:lcounter}
  The counting operator $\lcounter : \tdom \to \mathcal{P}(\ldom) \to (\mathbb{N} \cup \{\infty\})$
  counts the appearance of a list of labels $L = [l_1, \ldots, l_n]$ as
{\small
  \[
    \begin{array}{ll}
    \lcounter(\trace'' \tracecat \trace', L ) 
    \triangleq \lcounter(\trace', L) + 1  & \text{if}~ \pi_2(\trace''[i]) = l_i, \forall i = 1, \ldots, n
    \\ 
    \lcounter([(\_, l, \_)] \tracecat \trace', L ) 
    \triangleq \lcounter(\trace', L) & \text{if}~ l \neq l_1
    \\ 
    \lcounter(\trace, L ) 
    \triangleq \bot & \text{if }~ \trace \in \inftdom.
    \end{array}
  \]
}
% {When the input trace is infinite, $\lcounter(\trace, L)$ returns $\bot$ for any $L$.}
\end{defn}
%
{Notice here in both Definition~\ref{def:lcounter} and~\ref{def:counter}, when the input trace is infinite, both $\counter(\trace, l)$ and  $\lcounter(\trace, L)$ returns $\bot$ for any $l$ or $L$.}
% {When the input trace is infinite, $\lcounter(\trace, L)$ returns $\bot$ for any $L$.}

\begin{defn}[Reachability-bound]
  \label{def:rb}
  For a program ${c}$ and a location $l \in \lvar(c)$ in this program,
a function $f(c, l) : \ftdom_0(c) \to (\mathbb{N} \cup \{\infty\})$ is a \emph{Reachability-Bound} for $l$ if and only if
{\small
\[
\begin{array}{l}
  \forall \trace_0 \in \ftdom_0(c), c' \in \cdom, l, l' \in \lvar(c), \trace \in \tdom \st 
  \\ \qquad
  \Big(
    \config{{c}, \trace_0} \to^{*} \config{\clabel{\eskip}^{l'}, \trace_0 \tracecat \trace} 
    \lor 
    \config{{c}, \trace_0} \uparrow^{\infty} \trace_0 \tracecat \trace 
  \Big)
  \implies f({c}, l)(\trace_0) \geq \counter(\trace, l) 
  \end{array}
  \]
}
\end{defn}
Given a program point $l$ in $c$, our algorithm (defined below) computes a Reachability-Bound for it.
It is easy to compute a trivial \emph{Reachability-Bound} $f(c, l): \ftdom_0(c) \to \infty$, but it is not interesting to us.
% In the following sections, we only focus on computing a finite \emph{reachability-bound} for program's given location and considering the executions that terminates.
% Ideally, we aim to compute a precise reachability-bound as below.
% \begin{defn}[Precise Reachability-Bound]
%   \label{def:exe_rb}
%   For a program ${c}$ and a location $l \in \lvar(c)$ in this program,
% $\exeRB({c}, l): \ftdom_0(c) \to (\mathbb{N} \cup \{\infty\})$ is a \emph{Precise Reachability-Bound}  if and only if,
% \highlight{
% \[
% \begin{array}{l}
%   \forall \trace_0 \in \ftdom_0(c), c' \in \cdom, l, l' \in \lvar(c), \trace \in \tdom \st 
%   \\ \qquad
%   \Big(
%     \config{{c}, \trace_0} \to^{*} \config{\clabel{\eskip}^{l'}, \trace_0 \tracecat \trace} 
%     \lor 
%     \config{{c}, \trace_0} \uparrow^{\infty} \trace_0 \tracecat \trace 
%   \Big)
%   \implies \exeRB({c}, l)(\trace_0) = \counter(\trace, l) 
%   \end{array}
%   \]
% }
% \end{defn}
