\emph{The Reachability-bound}
We use a set $\lvar(c) \subseteq \ldom$ collecting all the program points in a program $c$, which are simply all the unique labels of $c$.
And we also define the operator $\tracel: \tdom \to \mathcal{P}{(\ldom)}$ projects the label from every event in a trace as a list of program points as follows,
{\small
\[
\tracel([(\_, l, \_)] \tracecat \trace') \triangleq [l] \tracecat \tracel(\trace')
\qquad
\tracel(\trace) \triangleq []~~ \mbox{if}~~ \trace = [] \lor \trace \in \inftdom
\]
}
Now we introduce two important counter notations counting the occurrence of a program point and a list of program points respectively.
\begin{defn}[Program Point Counter]
 \label{def:counter}
The counting operator $\counter : \tdom \to \ldom \to (\mathbb{N} \cup \{\bot\})$
counts the appearance of a label in a trace.
\begin{center}
{\small
$
\begin{array}{llll}
\counter([(\_, l, v)] \tracecat \trace', l ) \triangleq \counter(\trace', l) + 1 & \text{if}~ l = l
&
\counter({[]}, l) \triangleq 0 & 
\\
\counter([(\_, l', v)] \tracecat \trace', l) \triangleq \counter(\trace', l) & \text{if}~ l' \neq l
&
\counter(\trace, l) \triangleq \bot & \text{if }~ \trace \in \inftdom
\end{array}
$
}
\end{center}
\end{defn}
\begin{defn}[Point List Counter]
 \label{def:lcounter}
 The counting operator $\lcounter : \tdom \to \mathcal{P}(\ldom) \to (\mathbb{N} \cup \{\infty\})$
 counts the appearance of a list of labels $L = [l_1, \ldots, l_n]$ as
{\small
\begin{center}
 $
 \begin{array}{ll}
 \lcounter(\trace'' \tracecat \trace', L ) 
 \triangleq \lcounter(\trace', L) + 1 & \text{if}~ \pi_2(\trace''[i]) = l_i, \forall i = 1, \ldots, n
 \\ 
 \lcounter([(\_, l, \_)] \tracecat \trace', L ) 
 \triangleq \lcounter(\trace', L) & \text{if}~ l \neq l_1
 \\ 
 \lcounter(\trace, L ) 
 \triangleq \bot & \text{if }~ \trace \in \inftdom.
 \end{array}
 $
 \end{center}
 }
% {When the input trace is infinite, $\lcounter(\trace, L)$ returns $\bot$ for any $L$.}
\end{defn}
%
% {Notice here in both Definition~\ref{def:lcounter} and~\ref{def:counter}, w
When the input trace is infinite, both $\counter(\trace, l)$ and $\lcounter(\trace, L)$ returns $\bot$.
% {When the input trace is infinite, $\lcounter(\trace, L)$ returns $\bot$ for any $L$.}

\begin{defn}[Reachability-bound]
 \label{def:rb}
 For a program ${c}$ and a location $l \in \lvar(c)$ in $c$,
a function $f(c, l) : \ftdom_0(c) \to (\mathbb{N} \cup \{\infty\})$ is a \emph{reachability-bound} for $l$ if and only if
{\small
\begin{center}
 $
\begin{array}{l}
 \forall \trace_0 \in \ftdom_0(c), c' \in \cdom, l, l' \in \lvar(c), \trace \in \tdom \st 
 \\ \qquad
 \Big(
 \config{{c}, \trace_0} \to^{*} \config{\clabel{\eskip}^{l'}, \trace_0 \tracecat \trace} 
 \lor 
 \config{{c}, \trace_0} \uparrow^{\infty} \trace_0 \tracecat \trace 
 \Big)
 \implies f({c}, l)(\trace_0) \geq \counter(\trace, l) 
 \end{array}
 $
\end{center}
}
\end{defn}
It is easy to compute a trivial \emph{reachability-bound} $f(c, l): \ftdom_0(c) \to \infty$, and 
our algorithm computes a non-trivial one