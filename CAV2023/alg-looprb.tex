The loop reachability bound is formally defined below. 

\begin{defn}[Loop Reachability Bound]
    \label{def:looprb}
    Loop Reachability Bound
    \begin{equation}
        \label{eq:looprb}
      \lpchB(l: \rprog, \tpath) \triangleq
        \frac{\lpinit(l: \rprog, \tpath) - \rffinal(\tpath)}{\lpnext(l: \rprog, \tpath)}
    \end{equation}
  \end{defn}

  We first compute each component in Equation.~\ref{eq:looprb}
  and solve the computation by SMT solver as follows.
  \\
  1. Compute the ranking function for every transition, $\locbound(\tpath)$ and estimate its maximum value,
  $\varinvar(\locbound(\tpath))$.
  This computation are adopted from \cite{sinn2017complexity} with more details in Section~\ref{sec:rank}.
  \\
  2. Compute the abstract states,  $\lpinit(l: \rprog, \tpath)$,
  $\rffinal(\rprog)$, and $\lpnext(l: \rprog, \tpath)$ using ranking function,
  as in Definition~\ref{def:alg-absstate}.
  \\
  3. Solve the equation using abstract states above and SMT solver.
%
\begin{defn}[Abstract States Computation]
\label{def:alg-absstate}
The abstract states $\lpinit(l: \rprog, \tpath)$, $\lpnext(l: \rprog, \tpath) \in \mathcal{A}_{in}$,
and $\rffinal(\rprog)$ are computed as follows.
\begin{itemize}
   \item 
The loop initial state 
$\lpinit(l: \rprog, \tpath) \in \mathcal{A}_{\lin}$ is symbolic expression as well. 
It describes the abstract initial value of $\tpath$'s ranking function before
any visit of $\tpath$ and during the first execution of $l: \rprog$.
\[
  \lpinit(l: \rprog, \tpath) \triangleq 
  \arg\max_{l_1}\left\{
       \varinvar(y) + v ~\middle\vert~ 
       \begin{array}{l} 
         (l_1, x' \leq y + v, l_2) \in \reset(x) 
         \\
         \land \absinit(\rprog) \leq l_1 \leq \absinit(\tpath)
       \end{array}
     \right\}
    , x = \locbound(\tpath)
  \]
\item
The loop next state 
$\lpnext(l: \rprog, \tpath) \in \mathcal{A}_{\lin}$ 
describes how much $\tpath$'s ranking function
is modified before
the second visit of $\tpath$ but during the second execution of $l: \rprog$.
\footnote{$l' \in \rprog$: the $\in$ notation is abused to denote
the program point $l'$ is a vertex on a path in the program $\rprog$.}
%
\[
  \begin{array}{l}
  \lpnext(l: \rprog, \tpath) \triangleq 
    \begin{array}{l}
  \sum\limits_{(x, \absevent) \in \inc(x) }
  \left\{ 
    \varinvar(y) + v ~\middle\vert~ \absevent = (l', x' \leq y + v, \_) \land  l' \in \rprog 
    \land l' \notin \tpath \right\}
    \\ \qquad 
    - \sum\limits_{(x, \absevent) \in \dec(x) }\left\{ 
      \varinvar(y) + v 
      ~\middle\vert~ \absevent = (l', x' \leq y + v, \_) \land l' \in \rprog \land l' \notin \tpath \right\}
    \end{array}
  \end{array}
  , x = \locbound(\tpath)
  \]
  \item  The \emph{Final State}, $\rffinal(\rprog)$ a conjunction of boolean expressions.
  It is the post-condition
  after the execution of $\rprog$.
\[
    \rffinal(\rprog) \triangleq 
    \bigwedge_{b \in \kw{Guard}(\rprog)}
    \neg b
\]
   $\kw{Guard}(\rprog)$ is the set of all the unique boolean expressions (i.e., the boolean constraints) on this program.
\end{itemize}
\end{defn}
