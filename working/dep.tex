%
%
%
\subsection{Dependency}
  
 To define the may dependency relation on two labeled variables, we rely on the limited information at hand - the trace generated by the operational semantics. In this end, we first define the \emph{May-Dependency} between events, and use it as a foundation of the variable may-dependency relation.
 
%  In order to distinguish if a query's choice is affected by previous values, 
% % \jl{we need to be able to identify whether two queries are the equivalent or not, so that when we change the result of one query, whether another query is affected. For the equivalence of queries, } 
% we need to be able to identify whether two queries are the equivalent or not,
%  by quantifying over all values returned from database on a certain form of query value and check the equivalence of the query value of different quantification, formally as follows.
% \begin{defn}[Equivalence of Query Expression]
% %
% \label{def:query_equal}
% % \mg{Two} \sout{2} 
% Two query expressions $\qexpr_1$, $\qexpr_2$ are equivalent, denoted as $\qexpr_1 =_{q} \qexpr_2$, if and only if
% % $$
% %  \begin{array}{l} 
% %   \exists \qval_1, \qval_2 \in \mathcal{QVAL} \st \forall \trace \in \mathcal{T} \st
% %     (\config{\trace,  \qexpr_1} \qarrow \qval_1 \land \config{\trace,  \qexpr_2 } \qarrow \qval_2) 
% %     \\
% %     \quad \land (\forall D \in \dbdom, r \in D \st 
% %     \exists v \in \mathcal{VAL} \st 
% %           \config{\trace, \qval_1[r/\chi]} \aarrow v \land \config{\trace,  \qval_2[r/\chi] } \aarrow v)  
% %   \end{array}.
% % $$
% $$
%  \begin{array}{l} 
%   \forall \trace \in \mathcal{T} \st \exists \qval_1, \qval_2 \in \mathcal{QVAL} \st
%     (\config{\trace,  \qexpr_1} \qarrow \qval_1 \land \config{\trace,  \qexpr_2 } \qarrow \qval_2) 
%     \\
%     \quad \land (\forall D \in \dbdom, r \in D \st 
%     \exists v \in \mathcal{VAL} \st 
%           \config{\trace, \qval_1[r/\chi]} \aarrow v \land \config{\trace,  \qval_2[r/\chi] } \aarrow v)  
%   \end{array}.
% $$
% % \mg{$$
% %  \begin{array}{l} 
% %    \forall \trace \in \mathcal{T} \st \exists \qval_1, \qval_2 \in \mathcal{QVAL} \st
% %     (\config{\trace,  \qexpr_1} \qarrow \qval_1 \land \config{\trace,  \qexpr_2 } \qarrow \qval_2) 
% %     \\
% %     \quad \land (\forall D \in \dbdom, r \in D \st 
% %     \exists v \in \mathcal{VAL} \st 
% %           \config{\trace, \qval_1[r/\chi]} \aarrow v \land \config{\trace,  \qval_2[r/\chi] } \aarrow v)  
% %   \end{array}.
% % $$
% % }
%  %
%  where $r \in D$ is a record in the database domain $D$. 
%   We denote by $\qexpr_1 \neq_{q} \qexpr_2$  the negation of the equivalence relation.
% % \\ 
% % where $r \in D$ is a record in the database domain $D$,
% % \mg{is  $FV(\qexpr)$ being defined here? If yes, I suggest to put it in a different place, rather than in the middle of another definition.} 
% % $FV(\qexpr)$ is the set of free variables in the query expression $\qexpr$.
% % \sout{$\qexpr_1 \neq_{q}^{\trace} \qexpr_2$  is defined vice versa.}
% % \mg{As usual, we will denote by $\qexpr_1 \neq_{q}^{\trace} \qexpr_2$  the negation of the equivalence.}
% %
% \end{defn}
%

% \mg{In the next definition you don’t need the subscript e, it is clear that it is equivalence of events by the fact that the elements on the two sides of = are events. That is also true for query expressions. Also, I am confused by this definition. What happen for two query events?}
% \\
% \jl{The last component of the event is equal based on Query equivalence, $\pi_{4}(\event_1) =_q \pi_{4}(\event_2)$.
% In the previous version, the query expression is in the third component and I defined $v \neq \qexpr$ for all $v$ that isn't a query value.}
% \begin{defn}[Event Equivalence $\eventeq$]
% Two events $\event_1, \event_2 \in \eventset$ \mg{are equivalent, \sout{is in \emph{Equivalence} relation,}} denoted as $\event_1 \eventeq \event_2$ if and only if:
% \[
% \pi_1(\event_1) = \pi_1(\event_2) 
% \land  
% \pi_2(\event_1) = \pi_2(\event_2) 
% \land
% \pi_{3}(\event_1) = \pi_{3}(\event_2)
% \land 
% \pi_{4}(\event_1) =_q \pi_{4}(\event_2)
% \]
% %
% % \sout{The $\event_1 \eventneq \event_2$ is defined as vice versa.}
% % \mg{As usual, we will denote by $\event_1 \eventneq \event_2$  the negation of the equivalence.}
% \end{defn}
% \wq{Now we can compare two events by defining the event equivalence and difference relation.}
 We compare two events by defining the $\diff(\event_1, \event_2)$, we use $\qexpr_1 =_{q} \qexpr_2$ and $\qexpr_1 \neq_{q} \qexpr_2$ to notate query expression equivalence and inquivalence. 
% by defining the event equivalence and difference relation based on the query equivalence.
% \begin{defn}[Event Equivalence]
% \label{def:event_eq}
%   Two events $\event_1, \event_2 \in \eventset$ are equivalent, 
%   % denoted as $\event_1 \eventeq \event_2$ 
%   denoted as $\event_1 = \event_2$ 
%   if and only if:
%   \[
%   \pi_1(\event_1) = \pi_1(\event_2) 
%   \land  
%   \pi_2(\event_1) = \pi_2(\event_2) 
%   \land
%   \pi_{3}(\event_1) = \pi_{3}(\event_2)
%   \land 
%   \pi_{4}(\event_1) =_q \pi_{4}(\event_2)
%   \]
%   %
%   As usual, we will denote by $\event_1 \neq \event_2$  the negation of the equivalence.
%   % As usual, we will denote by $\event_1 \eventneq \event_2$  the negation of the equivalence.
%   % When it is clear from the context, we omit the subscript $\kw{e}$ and use 
%   % $\event_1 = \event_2$ (and $\event_1 \neq \event_2$) for event equivalent
% \end{defn}
%
%
% \begin{defn}[Signature Equivalence of Events $\sigeq$]

% Two events $\event_1, \event_2 \in \eventset$ is in \emph{signature equivalence} relation, denoted as $\event_1 \sigeq \event_2$ if and only if:
% \[
% \forall i \in \{1, 2, 3\} \st \pi_{\sig}(\event_1) = \pi_{\sig}(\event_2) 
% \]
% The $\event_1 \signeq \event_2$ is defined as vice versa.
% \end{defn}
%
% \begin{defn}[Events Different up to Value ($\diff$)]
% Two events $\event_1, \event_2 \in \eventset$ \mg{are \sout{is}} \emph{Different up to Value}, 
% denoted as $\diff(\event_1, \event_2)$ if and only if:
% \[
% \pi_1(\event_1) = \pi_1(\event_2) 
% \land  
% \pi_2(\event_1) = \pi_2(\event_2) 
% \land  
% \pi_3(\event_1) \neq_q \pi_3(\event_2)
% \]
% \end{defn}
\begin{defn}[Events Different up to Value ($\diff$)]
  Two events $\event_1, \event_2 \in \eventset$ are  \emph{Different up to Value}, 
  denoted as $\diff(\event_1, \event_2)$ if and only if:
  \[
    \begin{array}{l}
  \pi_1(\event_1) = \pi_1(\event_2) 
  \land  
  \pi_2(\event_1) = \pi_2(\event_2) \\
  \land  
  \big(
    (\pi_3(\event_1) \neq \pi_3(\event_2)
  \land 
  \pi_{4}(\event_1) = \pi_{4}(\event_2) = \bullet )
  % \qquad \qquad 
  \lor 
  (\pi_4(\event_1) \neq \bullet
  \land 
  \pi_4(\event_2) \neq \bullet
  \land 
  \pi_{4}(\event_1) \neq_q \pi_{4}(\event_2)) 
  \big)
  \end{array}
  \]
  \end{defn}
 %
For a program, its labeled variables and assigned variables are sub set of 
the labeled variables $\mathcal{LV}$.
% annotated by a label. 
We use  
%$\mathcal{LVAR} = \mathcal{VAR} \times \mathcal{L} $ 
% $\mathcal{LV}$ represents the universe of all the labeled variables and 
$\avar(c) \in \mathcal{P}(\mathcal{VAR} \times \mathbb{N}) \subset \mathcal{LV}$ and 
$\lvar(c) \in \mathcal{P}(\mathcal{VAR} \times \mathcal{L}) \subseteq \mathcal{LV}$ for them. $FV: \expr \to \mathcal{P}(\mathcal{VAR})$, computes the set of free variables in an expression. We also define the set of query variables for a program $c$, $\qvar: \cdom \to 
\mathcal{P}(\mathcal{LV})$.
% defined in Definition~\ref{def:lvar}.
% \begin{defn}[Assigned Variables (
% % $\avar_{c} \subseteq \mathcal{VAR} \times \mathbb{N}$ or 
% $\avar : \cdom \to \mathcal{P}(\mathcal{VAR} \times \mathbb{N})$,
% labelled Variables 
% (
% % $\lvar_{c} \subseteq \mathcal{VAR} \times \mathbb{N}$ or 
% $\lvar : \cdom \to \mathcal{P}(\mathcal{VAR} \times \mathcal{L})$]
% \label{def:avar}
% {\footnotesize
% $$ \avar_{c} \triangleq
%   \left\{
%   \begin{array}{ll}
%       \{{x}^l\}                   
%       & {c} = [{\assign x e}]^{l} 
%       \\
%       \{{x}^l\}                   
%       & {c} = [{\assign x \query(\qexpr)}]^{l} 
%       \\
%       \avar_{{c_1}} \cup \avar_{{c_2}}  
%       & {c} = {c_1};{c_2}
%       \\
%       \avar_{{c}} \cup \avar_{{c_2}} 
%       & {c} =\eif([\bexpr]^{l}, c_1, c_2) 
%       \\
%       \avar_{{c}'}
%       & {c}   = \ewhile ([\bexpr]^{l}, {c}')
% \end{array}
% \right.
% $$
% }
% \end{defn}
% %
% \begin{defn}[labelled Variables 
% (
% % $\lvar_{c} \subseteq \mathcal{VAR} \times \mathbb{N}$ or 
% $\lvar : \cdom \to \mathcal{P}(\mathcal{VAR} \times \mathcal{L})$]
% \label{def:lvar}
% {\footnotesize
% $$
%   \lvar_{c} \triangleq
%   \left\{
%   \begin{array}{ll}
%       \{{x}^l\} \cup FV(\expr)^{in}                  
%       & {c} = [{\assign x e}]^{l} 
%       \\
%       \{{x}^l\}   \cup FV(\qexpr)^{in}                
%       & {c} = [{\assign x \query(\qexpr)}]^{l} 
%       \\
%       \lvar_{{c_1}} \cup \lvar_{{c_2}}  
%       & {c} = {c_1};{c_2}
%       \\
%       \lvar_{{c}} \cup \lvar_{{c_2}} \cup FV(\bexpr)^{in}
%       & {c} =\eif([\bexpr]^{l}, c_1, c_2) 
%       \\
%       \lvar_{{c}'} \cup FV(\bexpr)^{in}
%       & {c}   = \ewhile ([\bexpr]^{l}, {c}')
% \end{array}
% \right.
% $$
% }
% \end{defn}
% \begin{defn}[
% Assigned Variables
% % ($\avar:\cdom \to \mathcal{P}(\mathcal{VAR} \times \mathbb{N})$,
% and Labeled Variables 
% % ($\lvar : \cdom \to \mathcal{P}(\mathcal{VAR} \times \mathcal{L})$
% ]
% \label{def:lvar}
% {\footnotesize
% $$ \avar_{c} \triangleq
%   \left\{
%   \begin{array}{ll}
%       \{{x}^l\}                   
%       & {c} = [{\assign x e}]^{l} 
%       \\
%       \{{x}^l\}                   
%       & {c} = [{\assign x \query(\qexpr)}]^{l} 
%       \\
%       \avar_{{c_1}} \cup \avar_{{c_2}}  
%       & {c} = {c_1};{c_2}
%       \\
%       \avar_{{c}} \cup \avar_{{c_2}} 
%       & {c} =\eif([\bexpr]^{l}, c_1, c_2) 
%       \\
%       \avar_{{c}'}
%       & {c}   = \ewhile ([\bexpr]^{l}, {c}')
% \end{array}
% \right.
% ~~
%   \lvar_{c} \triangleq
%   \left\{
%   \begin{array}{ll}
%       \{{x}^l\} \cup FV(\expr)^{in}                  
%       & {c} = [{\assign x e}]^{l} 
%       \\
%       \{{x}^l\}   \cup FV(\qexpr)^{in}                
%       & {c} = [{\assign x \query(\qexpr)}]^{l} 
%       \\
%       \lvar_{{c_1}} \cup \lvar_{{c_2}}  
%       & {c} = {c_1};{c_2}
%       \\
%       \lvar_{{c}} \cup \lvar_{{c_2}} \cup FV(\bexpr)^{in}
%       & {c} =\eif([\bexpr]^{l}, c_1, c_2) 
%       \\
%       \lvar_{{c}'} \cup FV(\bexpr)^{in}
%       & {c}   = \ewhile ([\bexpr]^{l}, {c}')
% \end{array}
% \right.
% $$
% }
% \end{defn}
% Free Variables:
% To be precise,
% $FV(\aexpr)$, $FV(\bexpr)$ and $FV(\qexpr)$ represent the set of free variables in arithmetic
% expression $\aexpr$, boolean expression $\bexpr$ and query expression $\qexpr$ respectively.
% Labeled variables in $c$ is the set of assigned variables and all the free variables
% showing up in $c$ with a default label $in$. 
% The free variables
% showing up in $c$, which aren't defined before be used, are actually the input variables of this program.
% \\
% It is easy to see that for any program $c$ in the {\tt Query While} language, its every labeled variable is unique.
% %
% \begin{lem}[Uniqueness of the Labeled Variables]
%   For every program $c \in \cdom$ and every two labeled variables such that
%   $x^i, y^j \in \lvar(c)$, then $x^i \neq y^j$.
% \end{lem}
%
% formally in Definition~\ref{def:qvar}.
% \mg{In the next definition, why do you call it a vector? It seems that you define it as a set.}\\
% \jl{fixed}\\
%
% \begin{defn}[Query Variables ($\qvar_{c} \subseteq \mathcal{VAR} \times \mathbb{N}$)].
  % \\
% \begin{defn}[Query Variables ($\qvar: \cdom \to \mathcal{P}(\mathcal{LV})$)] 
%   \label{def:qvar}
% Given a program $c$, its query variables 
% % \mg{it seems you are missing the $_c$ subscript. Also, this is a minor point but I don't think it is a good idea to use a subscript, cannot you just use $\qvar(c)$.}
% $\qvar(c)$ is the set of variables set to the result of a query in the program.
% % \jl{fixed}
% It is defined as follows:
% {\footnotesize
% $$
%   % \qvar_{{c}} \triangleq
%   \qvar(c) \triangleq
%   \left\{
%   \begin{array}{ll}
%       \{\}                  
%       & {c} = [{\assign x \expr}]^{l} 
%       \\
%       \{{x}^l\}                  
%       & {c} = [{\assign x \query(\qexpr)}]^{l} 
%       \\
%       \qvar(c_1) \cup \qvar(c_2)  
%       & {c} = {c_1};{c_2}
%       \\
%       \qvar(c_1) \cup \qvar(c_2) 
%       & {c} =\eif([\bexpr]^{l}, c_1, c_2) 
%       \\
%       \qvar(c')
%       & {c}   = \ewhile ([\bexpr]^{l}, {c}')
% \end{array}
% \right.
% $$
% }
% \end{defn}
%
It is easy to see that a program $c$'s query variables is a subset of 
its labeled variables, $\qvar(c) \subseteq \lvar(c)$. We have the operator $\tlabel : \mathcal{T} \to \ldom$, which gives the set of labels in every event belonging to a trace.
Then we introduce a counting operator $\vcounter : \mathcal{T} \to \mathbb{N} \to \mathbb{N}$, 
% \wq{which counts the occurrence of of a variable in the trace,} 
which counts the occurrence of of a labeled variable in the trace,
whose behavior is defined as follows,
% \[
% \begin{array}{lll}
% \vcounter(\trace :: (x, l, v, \bullet) ) l \triangleq \vcounter(\trace) l + 1
% &
% \vcounter(\trace  ::(b, l, v, \bullet) ) l \triangleq \vcounter(\trace) l + 1
% &
% \vcounter(\trace  :: (x, l, v, \qval) ) l \triangleq \vcounter(\trace) l + 1
% \\
% \vcounter(\trace  :: (x, l', v, \bullet) ) l \triangleq \vcounter(\trace ) l, l' \neq l
% &
% \vcounter(\trace  :: (b, l', v, \bullet) ) l \triangleq \vcounter(\trace ) l, l' \neq l
% &
% \vcounter(\trace  :: (x, l', v, \qval)) l \triangleq \vcounter(\trace ) l, l' \neq l
% \\
% \vcounter({[]}) l \triangleq 0
% &&
% \end{array}
% \]
\[
\begin{array}{lll}
\vcounter(\trace :: (\_, l, \_, \_), l ) \triangleq \vcounter(\trace, l) + 1
&
% \vcounter(\trace  ::(b, l, v, \bullet), l) \triangleq \vcounter(\trace, l) + 1
% &
% \vcounter(\trace  :: (x, l, v, \qval), l) \triangleq \vcounter(\trace, l) + 1
% \\
\vcounter(\trace :: (\_, l', \_, \_), l ) \triangleq \vcounter(\trace, l), l' \neq l 
&
% \vcounter(\trace, l) + 1
% \vcounter(\trace  :: (x, l', v, \bullet), l) \triangleq \vcounter(\trace, l), l' \neq l
% &
% \vcounter(\trace  :: (b, l', v, \bullet), l) \triangleq \vcounter(\trace, l), l' \neq l
% &
% \vcounter(\trace  :: (x, l', v, \qval), l) \triangleq \vcounter(\trace, l), l' \neq l
% \\
\vcounter({[]}, l) \triangleq 0
\end{array}
\]
The full definitions of these above operators can be found in the appendix.
%
\highlight{
\begin{defn}[Value Sequence $\seq(\trace, x^l)$]
  \label{def:vseq}
  \[
\begin{array}{l}
  \seq(\trace :: (x, l, v, \bullet), x^l) \triangleq \seq(\trace)::v  \qquad
  \seq(\trace :: (x, l, v, \qval), x^l) \triangleq \seq(\trace):: \qval \qquad
  \seq([]) \triangleq []\\
  \seq(\trace :: (y, j, \_, \_), x^l) \triangleq \seq(\trace) \quad y \neq x \lor j \neq l 
\end{array}
\]
\end{defn}
%
\begin{defn}[Difference Sequence $\sdiff(\trace_1, \trace_2, x^l )$]
  \label{def:diffseq}
  Let $ s_1 = \seq(\trace_1, x^l) \land s_2 = \seq(\trace_2, x^l)$ be the value sequence of $x^l$ 
  on $\trace_1$ and $\trace_2$, and $s^l$ be the sequence with longer length and $s^t$ the 
  shorter one,
  then their difference sequence is defined as follows,
  \[
    \sdiff(\trace_1, \trace_2, x^l) \triangleq
    \begin{array}{l}
      \{ (s^t[k], s^l[k]) ~|~ 
      % \land 
      % \land 
      s^t[k] \neq s^l[k], k = 0, \ldots, len(s^t)
      \}
      \\
      \cup 
      \{ (\cdot, s^l[k]) ~|~ 
      \len(s^t) \leq \len(s^l)k = len(s^t), \ldots \len(s^l)
      \}
    \end{array}
    \]
\end{defn}
}
%
%  based on the trace-based operational semantics,
% by considering all the possible execution traces.
%   as well as the program's \emph{Adaptivity} based on this \emph{May-Dependency} relation. \wq{   }
% \\
% \mg{Also, is $\event_2$ or $\event_2'$ in the last line of the first block?}
% \jl{$\event_2'$, fixed}

\begin{defn}[Variable May-Dependency].
  \label{def:var_dep}
  \\
  A variable ${x}_2^{l_2} \in \lvar(c)$ is in the \emph{variable may-dependency} relation with another
  variable ${x}_1^{l_1} \in \lvar(c)$ in a program ${c}$, denoted as 
  %
  $\vardep({x}_1^{l_1}, {x}_2^{l_2}, {c})$, if an only if.
  %
\[
  \begin{array}{l}
\exists 
\event_1, \event_2 \in \eventset^{\asn}, \trace \in \mathcal{T} , 
D \in \dbdom \st
% (\pi_{1}{(\event_1)}, \pi_{2}{(\event_1)}) = ({x}_1, l_1)
% \land
% (\pi_{1}{(\event_2)}, \pi_{2}{(\event_2)}) = ({x}_2, l_2)
\pi_{1}{(\event_1)}^{\pi_{2}{(\event_1)}} = {x}_1^{l_1}
\land
\pi_{1}{(\event_2)}^{\pi_{2}{(\event_2)}} = {x}_2^{l_2}% \\ \quad 
\land 
\eventdep(\event_1, \event_2, \trace, c, D) 
  \end{array}
\]  
\highlight{
  A labeled variable $y^j \in \lvar(c)$ is in the \emph{may-dependency} relation with another
  labeled variable $x^i \in \lvar(c)$ in a program ${c}$, w.r.t. an initial trace $\trace_0 \in \mathcal{T}_0(c)$
  and two witness traces $\trace_1, \trace_2 \in \mathcal{T}$,
  denoted as 
  %
  $\dep(x^i, y^j, \trace_1, \trace_2, \trace_0, {c})$, if an only if
  \[
    \begin{array}{l}
  \exists 
  % \event_1, \event_2 \in \eventset^{\asn}, \trace \in \mathcal{T} , 
  D \in \dbdom, 
  \trace_0' \in \mathcal{T} \st
  % \event_1' \in \eventset^{\asn}, {c}_1  \in \cdom \st
  % \\ \quad 
  % (\pi_{1}{(\event_1)}, \pi_{2}{(\event_1)}) = ({x}_1, l_1)
  % \land
  % (\pi_{1}{(\event_2)}, \pi_{2}{(\event_2)}) = ({x}_2, l_2)
  % \pi_{1}{(\event_1)}^{\pi_{2}{(\event_1)}} = {x}_1^{l_1}
  % \land
  % \pi_{1}{(\event_2)}^{\pi_{2}{(\event_2)}} = {x}_2^{l_2}% 
  % \land 
  % % \exists \vtrace_0' \in \mathcal{T},\event_1' \in \eventset^{\asn}, {c}_1  \in \cdom  \st
  % \diff(\event_1, \event_1') 
  % \env(\trace_0, x^i) \neq   \env(\trace_0', x^i) 
  % \land 
  (\forall z \neq x \st   \env(\trace_0 ) z =   \env(\trace_0') z )
  \\ \quad \land 
  % \\ \quad \land 
   \config{{c}, \vtrace_0} \rightarrow^{*} 
  % \config{{c}_1, \vtrace_0' \tracecat [\event_1]}  \rightarrow^{*} 
    \config{\clabel{\eskip}^l, \vtrace_0  \tracecat \trace_1 } 
    % \vtrace_0' \tracecat [\event_1] \tracecat \trace_1 } 
    % 
    % \\ \quad 
    \land 
    % \config{{c}_1, \vtrace_0' \tracecat [\event_1']}  \rightarrow^{*} 
    % \config{\clabel{\eskip}^l,  
    % \vtrace_0' \tracecat [\event_1] \tracecat \trace_2 }   
    \config{{c}, \vtrace_0'} \rightarrow^{*} 
    % \config{{c}_1, \vtrace_0' \tracecat [\event_1]}  \rightarrow^{*} 
      \config{\clabel{\eskip}^l, \vtrace_0'  \tracecat \trace_2} 
    \land 
      \sdiff(\trace_1, \trace_2, y^j ) \neq \emptyset
    \end{array}
  \]  
}%
% , where $\eventdep$ is defined in Definition~\ref{def:event_dep}.
  % , where $\eventdep^{val}$ and $\eventdep^{\test}$ is defined in \ref{def:event_valdep} and \ref{def:event_ctldep}.
  % %
  %
  \end{defn}
%
In most data analysis programs $c$ we are interested, there are usually some user input variables, such as $k$ in $\kw{twoRounds}$. 
We denote $\mathcal{T}_0(c)$ as the set of initial traces in which all the input variables in $c$ are initialized.
% We denote $\mathcal{T}_0(c)$ as the set of initial traces in which all the input variables in $c$ are initialized.
%
\subsection{Execution Based Dependency Graph}
\label{sec:execution-base-graph-def}
%
%
%
%
The variable \emph{May-Dependency} relation gives us the edges, we define the execution based dependency graph.
% \wq{Just a few sentences here, some overview of this subsection. See 4.2 for instance.}
\begin{defn}[Execution Based Dependency Graph]
\label{def:trace_graph}
Given a program ${c}$,
its \emph{Execution-Base Dependency Graph} 
$\traceG({c}) = (\traceV({c}), \traceE({c}), \traceW({c}), \traceF({c}))$ is defined as follows,
% over all possible traces,
%
\highlight{\small
\[
\begin{array}{lcl}
  % \text{Vertices} &
  \traceV({c}) & := & 
  \{ 
  (x^l, w) 
  % \in \mathcal{LV} \times \mathbb{N}
  ~ \vert ~ 
  w : \mathcal{T} \to \mathbb{N}
  \land
  x^l \in \lvar(c) 
  \\ & &
  \land
  % n = \max \left\{ 
    % ~ \middle\vert~
  \forall \trace \in \mathcal{T}_0(c), \trace' \in \mathcal{T} \st \config{{c}, \trace} \to^{*} \config{\eskip, \trace\tracecat\vtrace'} 
  \implies w(\trace) = \vcounter(\vtrace', l) 
  %  \right\}
%   \right
\}
  \\
  % \text{Edges} &
  \traceE({c}) & := & 
  \{ 
  (x^i, w, y^j) 
%   \in \mathcal{LV} \times \mathcal{LV}
  ~ \vert ~
  x^i, y^j \in \lvar(c)
  \land w \in \mathcal{P}( \mathcal{T}_0(c) \to \mathbb{N})
  \land 
  \exists \trace \in \mathcal{T}_0(c), 
  \trace_1, \trace_2 \in \mathcal{T} \st \dep(x^i, y^j,\trace_1, \trace_2, \trace_0, c)
  \\ & &
  \land \forall \trace_0 \in \mathcal{T}_0(c) \st
  w (\trace_0) = \max \left\{ | \sdiff(\trace_1, \trace_2, y)|
  ~\middle\vert~
  \forall \trace_1, \trace_2 \in \mathcal{T} \st \dep(x^i, y^j,\trace_1, \trace_2, \trace_0, c) \right\}
  \}
\end{array}
\]
}
\end{defn}
There are two components of the execution-based dependency graph. 
\\
\highlight{
The vertices $\traceV(c)$  is a set of pairs, $(x^l, w) \in \mathcal{LV} \times (\mathcal{T} \to \mathbb{N})$,
with a labeled variable as first component and
its weight $w$ the second component.
Weight $w$ for
% a labeled variable 
$x^l$ is a function $w : \mathcal{T} \to \mathbb{N}$
mapping from a starting trace to a natural number.
When program executes under this starting trace $\trace$,
$\config{{c}, \trace} \to^{*} \config{\eskip, \trace\tracecat\vtrace'} $, it generates an execution trace $\trace'$.
This natural number is the evaluation times of the labeled command corresponding to the vertex, 
computed by the counter operator $w(\trace) = \vcounter(\vtrace', l)$.
We can see in the execution-based dependency graph of $\kw{twoRounds}$ in
 Figure~3(b) in main paper, the weight of vertices in the while loop is  $\env(\trace) k$, which depends on the value of the user input $k$ specified in the starting trace $\tau$.
\\
The directed edges $\traceE({c})$ is a set of triples $ (x^i, w, y^j) \in \mathcal{LV} \times \mathcal{P}(\mathcal{T}_0(c) \to \mathbb{N}) \times \mathcal{LV}$,
 with two labeled variables (from $x^i$ pointing to $y^j$) and a weight $w$ for this edge.
% comes 
The edges are constructed directly from our variable may-dependency relation. 
For any two vertices $x^{i}$ and $y^{j}$ in $\traceV(c)$, if there exists two witness traces $\trace_1, \trace_2$ and an initial trace $\trace_0 \in \mathcal{T}_0$ such that,
they satisfy the variable may-dependency relation 
$\dep(x^i, y^j, \trace_1, \trace_2, \trace_0, c)$ , 
there is a direct edge. 
The weight of the edge is a function $w: \mathcal{T}_0(c) \to \mathbb{N}$,
where given an initial trace $\trace_0$,
it is the maximum length of the difference sequence between all pairs of the witness traces $\trace_1, \trace_2$ 
satisfying the dependency relation.
}
% it is also reflected in $\traceW({c})$.    

\subsection{Variable Reachability}
