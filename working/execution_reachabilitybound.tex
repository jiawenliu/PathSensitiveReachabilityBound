%
\begin{defn}[Execution Based Reachability Bound]
\label{def:trace_graph}
Given a program ${c}$,
its \emph{Execution-Base Reachability Bound} 
$\exeRB({c})$ is defined as follows,
% over all possible traces,
%
\highlight{
\[
\begin{array}{lcl}
  % \text{Vertices} &
  \exeRB({c}) & := & 
  \{ 
  (x^l, w) 
  % \in \mathcal{LV} \times \mathbb{N}
  ~ \vert ~ 
  w : \mathcal{T} \to \mathbb{N}
  \land
  x^l \in \lvar(c) 
  \\ & &
  \land
  \forall \trace \in \mathcal{T}_0(c), \trace' \in \mathcal{T} \st \config{{c}, \trace} \to^{*} \config{\eskip, \trace\tracecat\vtrace'} 
  \implies w(\trace) = \vcounter(\vtrace', l) 
\}
\end{array}
\]
}
\end{defn}
There are two components.
\\
In most data analysis programs c we are interested, there are usually some user input variables,
such as $k$ in twoRounds. We denote $\mathcal{T}_0(c)$ as the set of initial traces in which all the input variables in
$c$ are initialized
\highlight{
The $\exeRB(c)$  is a set of pairs, $(x^l, w) \in \mathcal{LV} \times (\mathcal{T} \to \mathbb{N})$,
with a labeled variable as first component and
its weight $w$ the second component.
Weight $w$ for
% a labeled variable 
$x^l$ is a function $w : \mathcal{T} \to \mathbb{N}$
mapping from a starting trace to a natural number.
When program executes under this starting trace $\trace$,
$\config{{c}, \trace} \to^{*} \config{\eskip, \trace\tracecat\vtrace'} $, it generates an execution trace $\trace'$.
This natural number is the evaluation times of the labeled command corresponding to the vertex, 
computed by the counter operator $w(\trace) = \vcounter(\vtrace', l)$.
We can see in the execution-based dependency graph of $\kw{twoRounds}$ in
 Figure~3(b) in main paper, the weight of vertices in the while loop is  $\env(\trace) k$, 
 which depends on the value of the user input $k$ specified in the starting trace $\tau$.
}
% it is also reflected in $\traceW({c})$.   